\documentclass{report}
   \usepackage[utf8x]{inputenc}
   \usepackage{longtable}
   \usepackage{verbatim}
   \usepackage{amsfonts}
   \usepackage{amssymb}
   \usepackage{amsmath}
   \usepackage{array}
   \usepackage{hhline}
   \usepackage{indentfirst}
   \usepackage{mathtext}
\usepackage[russian]{babel}
%\usepackage[dvips]{graphicx}%
\usepackage[pdftex]{graphicx}
\usepackage{ccaption}
\usepackage[dvipsnames,usenames]{color}
%\usepackage{color}
   \usepackage{hyperref}
\hypersetup{unicode,breaklinks=true}
%   \usepackage{syntax}
   \usepackage{multicol}
\textheight 250mm
\textwidth 171mm
\hoffset -25mm
\voffset -30mm
%\renewcommand{\baselinestretch}{1.347}
\newcommand{\No}{${\cal N}{\!}\underline{\circ}$}
\righthyphenmin=2
\hfuzz=12.5pt
\makeatletter
\@addtoreset{equation}{section}
\renewcommand{\theequation}{\thesection.\arabic{equation}}
\@addtoreset{table}{section}
\@addtoreset{figure}{section}
\renewcommand{\thetable}{\thesection.\arabic{table}}
\renewcommand{\thefigure}{\thesection.\arabic{figure}}

\newlength{\chaprule}
\newlength{\ATchapskip}
\setlength{\chaprule}{0.4\p@} \setlength{\ATchapskip}{10\p@} \advance \ATchapskip by -1\chaprule
\renewcommand{\@makechapterhead}[1]{%
\ifdim\chaprule>6\p@ \setlength{\chaprule}{6\p@}\fi
\vspace*{\ATchapskip}%
%\noindent\rule{\textwidth}{\chaprule}\par%
%\vskip 10\p@
{\parindent \z@ \normalfont \ifnum \c@secnumdepth >\m@ne {\huge\bfseries \@chapapp\space \thechapter.}
%\par\nobreak
%\vskip 8\p@
\fi \interlinepenalty\@M \huge\bfseries #1\par\nobreak \vskip 10\p@
%\noindent\rule{\textwidth}{\chaprule}\par%
\vskip\ATchapskip }}

\makeatother
\footskip 8mm
%\setcounter{page}{3}
\setcounter{section}{0}
\newcommand{\dom}{\rm dom}
\newcounter{rem}[section]
\renewcommand{\therem}{\thesection.\arabic{rem}}
\newenvironment{Remark}{\par\refstepcounter{rem} \bf Замечание \therem. \it}{\rm\par}

\renewcommand{\theenumi}{\arabic{enumi}}
\renewcommand{\labelenumi}{\theenumi)}
\newcommand{\udc}[1]{УДК #1}

\newcounter{lem}[section]
\renewcommand{\thelem}{\thesection.\arabic{lem}}
\newenvironment{Lemma}{\par\refstepcounter{lem}\bf Лемма \thelem. \it}{\rm\par}

\newcounter{cor}[section]
\renewcommand{\thecor}{\thesection.\arabic{cor}}
\newenvironment{Corrolary}{\par\refstepcounter{cor}\bf Следствие \thecor. \it}{\rm\par}
\newcounter{theor}[section]
\renewcommand{\thetheor}{\thesection.\arabic{theor}}
\newenvironment{Theorem}{\par\refstepcounter{theor}\bf Теорема \thetheor. \it}{\rm\par}
%\let \varkappa=\ae
\newcommand{\diag}{\mathop{\rm diag}}
\newcommand{\epi}{\mathop{\rm epi}}
\newenvironment{Proof}{\par\noindent\bf Доказательство.\rm}{ $\blacksquare$\par}

\newcounter{exam}[section]
\renewcommand{\theexam}{\thesection.\arabic{exam}}
\newenvironment{Example}{\par\refstepcounter{exam}\bf Пример \theexam. \rm}{\rm\par}

\newenvironment{scExample}{\par\refstepcounter{exam}\tt ; Пример \theexam. }{\rm\par}

\newenvironment{bslExample}{\par\refstepcounter{exam}\tt // Пример \theexam. }{\rm\par}

\newcounter{defin}[section]
\renewcommand{\thedefin}{\thesection.\arabic{defin}}
 \newenvironment{Definition}{\par\refstepcounter{defin}\bf Определение
\thedefin.\it}{\rm\par}
\newcommand{\ljoq}{<<}
\newcommand{\rjoq}{>>}
\newcommand{\vraisup}{\mathop{\rm vraisup}}
\newcommand{\sgn}{\mathop{\rm sgn}}
\newcommand{\diam}{\mathop{\rm diam}\nolimits}
\newcommand{\ByMeasureUniformly}{\mathop{\Rrightarrow}}
\newcommand{\pr}{\mathop{\rm pr}\nolimits}
\newcommand{\meas}{\mathop{\rm meas}\nolimits}
\newcommand{\Var}{\mathop{\mathbf{V}}\nolimits}
\newcommand{\supp}{\mathop{\rm supp}\nolimits}
\newcommand{\osc}{\mathop{\rm osc}\nolimits}
\newcommand{\lin}{\mathop{\rm lin}\nolimits}

\newcommand{\myalpha}{\text{\mathversion{bold}$\alpha$\mathversion{normal}}}
\newcommand{\mydelta}{\text{\mathversion{bold}$\delta$\mathversion{normal}}}
\newcommand{\sign}{\mathop{\rm sign}\nolimits}


\captiondelim{. }

\newcommand{\mytext}[1]{\rotatebox[origin=lb]{90}{#1}}

%\renewcommand{\syntleft}{\normalfont\itshape}
%\renewcommand{\syntright}{}

%\newcommand{\myarrow}{\textcolor{ForestGreen}{$\to$}}

\makeatletter
\def\@seccntformat#1{\csname the#1\endcsname.\quad}
\makeatother

\begin{document}

\title{\textbf{(Суб)оптимальное управление гиперболическими уравнениями дивергентного вида\\[10mm] Том 1\\[10mm] Теория уравнений}}
\author{\textbf{В.С. Гаврилов}}
\maketitle
\tableofcontents
    \chapter*{Введение и обозначения}\addcontentsline{toc}{chapter}{Введение и обозначения}
        \section*{Введение}\addcontentsline{toc}{section}{Введение}
В конце 40--х годов XXв. в работах Ладыженской О.А. было предложено определять обобщённые решения краевых и начально--краевых задач для эллиптических, параболических и гиперболических
уравнений с помощью интегральных тождеств, заменяющих собой уравнение, а иногда и часть начальных и граничных условий. Было также отмечено, что для каждой задачи можно вводить различные
классы обобщённых решений. Тем самым определение обобщённого решения задачи было отделено от какого--либо способа его получения (в отличие от предшествоваших работ Фридрихса К. и
Соболева С.Л.) и от каких бы то ни было аналитических представлений решения (в отличие от работ Гюнтера Н.М. и Лерэ Ж.). Более того, для ряда классов обобщённых решений классических
краевых и начально--краевых задач были доказаны теоремы единственности, использующие лишь свойства исследуемых уравнений, вытекающие из их определения.

Вначале Ладыженская О.А доказала существование обобщённых решений с помощью метода конечных разностей. Тем же методом было исследовано и увеличение гладкости этих обобщённых решений
по мере увеличения гладкости исходных данных и коэффициентов задачи. Полученные результаты для случая гиперболических уравнений с начально--краевыми условиями были отражены в
работе~\cite{lad1953}. Впоследствии в главе IV работы \cite{lad} те же результаты доказаны с помощью метода Галёркина.

Затем Ладыженская О.А. \cite{Lad1954,Lad1956,Lad1958} и Ворович И.И. \cite{Vorovich} предложили так называемый \glqq функциональный
метод\grqq\footnote{Термин, по--видимому, предложен Ладыженской О.А.}\ исследования начально--краевых задач для гиперболических уравнений\footnote{В работах \cite{Lad1954,Lad1956,Lad1958}
речь шла также о параболических уравнениях и уравнениях типа уравнения Шрёдингера.}.

Более точно, в работах \cite{Lad1954,Lad1956,Lad1958} предложено сводить начально--краевые задачи для гиперболических уравнений к задаче Коши для уравнения вида
\begin{gather}\label{Lad.abstr}
S_1(t)\frac{d^2z}{dt^2}+S_2(t)\frac{dz}{dt}+S_3(t)z=f(t),
\end{gather}
и доказаны существование и единственность решений рассматриваемых абстрактных задач Коши. Здесь  $S_i(t)$ --- неограниченные линейные операторы в гильбертовом пространстве $\mathcal{H}$,
зависящие от времени $t$ и обладающие некоторыми общими свойствами.


Что касается работы \cite{Vorovich}, то в ней рассматривалась абстрактная постановка начально--краевых задач для гиперболических уравнений с локально--липшицевыми нелинейностями по младшим
производным и с симметричной автономной главной частью. А именно, рассматривалась задача Коши для абстрактного уравнения вида
\begin{gather}\label{vor}
\omega_{tt}+A_1\omega+A_2\omega+K\omega_t=f(t),
\end{gather}
где $A_1$ --- симметричный положительно определённый оператор, $A_2$ --- некоторый нелинейный оператор, действущий из энергетического пространства оператора $A_1$ в пространство непрерывных
функций, а $K$ --- линейный ограниченный оператор, действующий в пространстве суммируемых с квадратом по области $\Omega$ функций. При этом в работе \cite{Vorovich} доказана лишь теорема
существования решения задачи Коши для уравнения \eqref{vor}.

Затем в работах Лионса и Мадженеса \cite{Lions1972,LionsMajenes} было предложено сводить начально--краевые задачи для гиперболических уравнений к задаче Коши для уравнения вида
\begin{gather}\label{Lions.abstr}
\frac{d^2z}{dt^2}+A(t)z=f(t),\,\,\,t\in[0,T],\\
z(0)=\varphi,\,\,\,\dot{z}(0)=\psi,
\end{gather}
где $A(t)\in\mathcal{L}(V,V^*)$, $t\in[0,T]$, --- симметричный оператор, $\varphi\in V$, $\psi\in H$, а $V$, $H$ --- гильбертовы пространства, такие, что $V\subset H$, и вложение $V$ в $H$
--- непрерывно и компактно. При этом предполагается, что при некотором вещественном $\lambda$ оператор $A(t)$ --- положительно определён, в том смысле, что
\begin{gather*}
\langle A(t)w,w\rangle+\lambda\|w\|^2_H\geqslant\|w\|^2_V.
\end{gather*}
Также в работах \cite{Lions1972,LionsMajenes} рассматривались некоторые случаи нелинейности в младших членах уравнений и в главной части.

После работ Лионса и Ладыженской появилось много работ, использующих методы Лионса и Ладыженской для изучения разных аспектов теории гиперболических уравнений дивергентного вида.

Например, в работе Якубова \cite{Yakubov} с помощью схемы, предложенной в работах \cite{Lad1954,Lad1956,Lad1958} доказана однозначная разрешимость задачи Коши
\begin{gather}\label{Yak.abstr}
\frac{d^2z}{dt^2}+A(t)\frac{dz}{dt}+B(t)z=f(t),\,\,\,t\in[0,T],\\
z(0)=\varphi,\,\,\,\dot{z}(0)=\psi,
\end{gather}
в классе функций, сильно непрерывных в норме энергетического пространства оператора $A(t)$, имеющих сильно непрерывную в пространстве типа пространства $L_2$ первую производную по времени.

Далее, в работах Железовского \cite{ZhelBuk2}, \cite{Zhel1995}--\cite{ZhelLyashko}, Ласеки \cite{LasiekaSokolowski,BalesLasiecka}, и Ишмухаметова обоснован метод
Бубнова--Галёркина приближённого решения гиперболических уравнений дивергентного вида, с локально липшицевой по фазовой переменной и по её младшим производным правой частью. Доказаны
локальные теоремы существования и единственности решений, получены оценки скорости сходимости метода Бубнова--Галёркина.

В работе Ишмухаметова \cite{Ishmuhametov} для абстрактной задачи Коши вида \eqref{Yak.abstr} обосновывается результат об аппроксимации, дающий возможность получать оценки скорости
сходимости в энергетической норме разностных схем для начально--краевых задач для гиперболических уравнений дивергентного вида.

Достаточное большое число работ (см., например, работы \cite{Del'Santo.Mitidieri,Pohozhaev.Mitidieri} и библиографию к ним) посвящено вопросам разрушения решений начально--краевых задач
для гиперболических уравнений дивергентного вида. Имеются работы (см., например, \cite{LasiekaLionsTriggiani,LasiekaTriggiani1,LasiekaTriggiani2}), посвящённые получению тонких свойств
регулярности решений; работы \cite{Il'inKuleshov2012a,Il'inKuleshov2012a,Nikitin}, в которых решение начально--краевой задачи для одномерного волнового решения ищется в пространствах $L_p$
и $W^1_p$; работы (см., например, \cite{Ramm,KaraStavr}), в которых изучается асимптотическое поведение решений начально--краевых задач для гиперболических уравнений.

Кроме того, в работе \cite{KozhanovLar'kin} для одномерного волнового уравнения, рассматриваемого в непрямоугольной области, изучается разрешимость некоторых начально--краевых задач,
а в работе \cite{KolesovRozov} изучается возможность многопараметрического резонанса в  краевой задаче для одномерного волнового уравнения.

Наконец, в работе \cite{Lomovcev} изучается существование, единственность, и гладкость решений абстрактной задачи Коши, являющейся абстрактной формулировкой начально--краевых задач для
гиперболических уравнений дивергентного вида, для случая, когда главная часть разрывна по времени и имеет разные области определения в разные моменты времени.

Отметим, однако, что во всех этих работах рассматриваются уравнения с симметричной главной частью, а при выписывании сопряжённых уравнений принципа максимума для задач оптимизации систем,
динамика которых описывается гиперболическими уравнениями дивергентного вида, в случае наличия в исходном уравнении младших производных главная часть сопряжённого уравнения сразу же
становится несимметричной. Поясним, о чём идёт речь, для чего рассмотрим следующую простейшую задачу оптимального управления:
\begin{gather}\label{Primitive.OptControl}
I[\pi]\to\min,\,\,\,\pi\in\mathcal{D},
\end{gather}
где $\mathcal{D}\equiv\{\pi\equiv(u,v)\in\mathcal{D}_1\times\mathcal{D}_2\}$, $\mathcal{D}_1\equiv\{u\in L_\infty^m(Q_T):u(x,t)\in U\text{ п.в. в $Q_T$}\}$, $\mathcal{D}_2\equiv\{v\in
L_\infty(\Omega):v(x)\in V\text{ п.в. в $\Omega$}\}$, $U\subset \mathbb{R}^m$ --- компакт, $V\subset\mathbb{R}$ --- отрезок числовой оси, а функционал $I$ задаётся равенством
\begin{gather*}
I[\pi]\equiv\int\limits_\Omega G(x,z[\pi](x,T))dx.
\end{gather*}
Здесь $z[\pi]$ --- отвечающее паре $\pi\equiv(u,v)\in\mathcal{D}$ единственное обобщённое решение начально--краевой задачи
\begin{gather}\label{Primitive.OptControl!main.eq}
z_{tt}-\frac{\partial}{\partial x_i}(a_{ij}(x,t)z_{x_j})+a(x,t)z+b_i(x,t)z_{x_i}=\langle f(x,t),u(x,t)\rangle,\,\,\,(x,t)\in Q_T;\\
\notag z(x,0)=\varphi(x),\,\,\,z_t(x,0)=v(x),\,\,\,x\in\Omega;\,\,\,z(x,t)=0,\,\,\,(x,t)\in S_T.
\end{gather}
Тогда необходимое условие оптимальности в этой задаче, называемое принципом максимума Л.С.Понтрягина, формулируется следующим образом (вывод принципа максимума, см., например, в работе
\cite{variation_2008}).

\begin{Theorem}
Пусть управление $\pi_0\equiv(u_0,v_0)\in\mathcal{D}$ --- оптимально в задаче \eqref{Primitive.OptControl}, в том смысле, что $I[\pi_0]=\inf\limits_{\pi\in\mathcal{D}}I[\pi]$. Тогда при
почти всех $(x,t)\in Q_T$ справедливо равенство
\begin{gather}\label{Primitive.OptControl!maximum.principle}
H(x,t,z[\pi_0](x,t),u_0(x,t),\eta[\pi_0](x,t))=\max\limits_{w\in U}H(x,t,z[\pi_0](x,t),w,\eta[\pi_0](x,t)),
\end{gather}
где $H(x,t,z,u,\eta)\equiv\eta[a(x,t)z-\langle f(x,t),u\rangle]$, а $\eta[\pi_0]$ --- решение при $\pi\equiv\pi_0$ сопряжённой начально--краевой задачи
\begin{gather}\label{Primitive.OptControl!adjoint.problem}
\eta_{tt}-\frac{\partial}{\partial x_i}(a_{ij}(x,t)\eta_{x_j}+b_i(x,t)\eta)+a(x,t)\eta=0,\,\,\,(x,t)\in Q_T;\\
\notag \eta(x,T)=0,\,\,\,\eta_t(x,T)=\nabla_zG(x,z[\pi](x,T)),\,\,\,x\in\Omega;\,\,\,\eta(x,t)=0,\,\,\,(x,t)\in S_T.
\end{gather}
\end{Theorem}

Таким образом, важной задачей является изучение начально--краевых задач для гиперболических уравенений дивергентного вида с несимметричной главной частью.

В настоящей монографии мы изучаем именно такие уравнения.

Отметим также, что при получении необходимых условий оптимальности для задач оптимального управления с поточечными фазовыми ограничениями множитель Лагранжа, отвечающий оператору,
задающему поточечные фазовые ограничения, является мерой Радона, и эта мера Радона появляется в правой части сопряжённого уравнения, отвечающего оператору поточечных фазовых ограничений.
Поясним это на примере следующей задачи оптимального управления с поточечными ограничениями:
\begin{gather}\label{State.OptControl}
I_0[\pi]\to\min,\,\,\,I_1[\pi](t)\leqslant0\,\,\,\forall\,t\in[0,T],\,\,\,\pi\in\mathcal{D},
\end{gather}
где $\mathcal{D}\equiv\{\pi\equiv(u,v)\in\mathcal{D}_1\times\mathcal{D}_2\}$, $\mathcal{D}_1\equiv\{u\in L_\infty^m(Q_T):u(x,t)\in U\text{ п.в. в $Q_T$}\}$, $\mathcal{D}_2\equiv\{v\in
L_\infty(\Omega):v(x)\in V\text{ п.в. в $\Omega$}\}$, $U\subset \mathbb{R}^m$ --- компакт, $V\subset\mathbb{R}$ --- отрезок числовой оси, а функционал $I_0$ и оператор $I_1$ задаются
равенствами
\begin{gather*}
I_0[\pi]\equiv\int\limits_\Omega G(x,z[\pi](x,T))dx,\,\,\,I_1[\pi](t)\equiv\int\limits_\Omega\Phi(x,t,z[\pi](x,t))dx.
\end{gather*}
Здесь $z[\pi]$ --- отвечающее паре $\pi\equiv(u,v)\in\mathcal{D}$ единственное обобщённое решение начально--краевой задачи \eqref{Primitive.OptControl!main.eq}.

В этом случае необходимые условия оптимальности выглядят следующим образом\footnote{Подробный вывод этого принципа максимума можно найти, например, в работах
\cite{GavrilovSumin2008,GavrilovSumin2013-1,GavrilovSumin2013-2}.}:
\begin{Theorem}
Пусть управление $\pi_0\equiv(u_0,v_0)\in\mathcal{D}$ --- оптимально в задаче \eqref{State.OptControl}. Тогда найдутся неотрицательное вещественное число $\lambda$ и неотрицательная мера
Радона $\mu\in\mathbf{M}[0,T]$, $\lambda+\|\mu\|=1$, такие, что при почти всех $(x,t)\in Q_T$ справедливо равенство
\begin{gather}\label{State.OptControl!maximum.principle}
H(x,t,z[\pi_0](x,t),u_0(x,t),\eta[\pi_0,\lambda,\mu](x,t))=\max\limits_{w\in U}H(x,t,z[\pi_0](x,t),w,\eta[\pi_0,\lambda,\mu](x,t)),
\end{gather}
где $H(x,t,z,u,\eta)\equiv\eta[a(x,t)z-\langle f(x,t),u\rangle]$, а $\eta[\pi_,\lambda,\mu0]$ --- решение при $\pi\equiv\pi_0$ сопряжённой начально--краевой задачи
\begin{gather}\label{State.OptControl!adjoint.problem}
\eta_{tt}-\frac{\partial}{\partial x_i}(a_{ij}(x,t)\eta_{x_j}+b_i(x,t)\eta)+a(x,t)\eta=\nabla_z\Phi(x,t,z[\pi](x,t))\mu(dt)\,\,\,(x,t)\in Q_T;\\
\notag \eta(x,T)=0,\,\,\,\eta_t(x,T)=\lambda\nabla_zG(x,z[\pi](x,T)),\,\,\,x\in\Omega;\,\,\,\eta(x,t)=0,\,\,\,(x,t)\in S_T.
\end{gather}
\end{Theorem}

Поэтому представляется важным изучение свойств решений линейных гиперболических уравнений дивергентного вида с присутствующей в правой части уравнения мерой Радона. Из результатов в этой
области нам известны лишь работы \cite{MR1,MR2}. В отличие от них, в данной монографии рассматриваются более общие уравнения и более общие граничные условия.

Используемый в работах \textbf{[13–16]} метод вывода необходимых условий подразумевает аппроксимацию исходной задачи с ПФО задачами, каждая из которых «эквивалентна» задаче с
конечным числом функциональных ограничений--неравенств. Далее в каждой аппроксимирующей задаче выводится \glqq аппроксимирующий\grqq\ принцип максимума», после чего в семействе
этих принципов максимума совершается предельный переход при стремлении числа ограничений к бесконечности. Подобный же подход применяется в \textbf{[13–16]} и при получении результатов,
связанных с регулярностью, нормальностью и с чувствительностью.

Однако при таком подходе возникает проблема \glqq склейки\grqq\ сопряжённых уравнений аппроксимирующих принципов максимума в одно результирующее сопряжённое уравнение, отвечающее исходному
фазовому ограничению и содержащее меру Радона в своей правой части. Кроме того, при выводе \glqq аппроксимирующих\grqq\ принципов максимума возникает необходимость
\glqq подравнивания\grqq\ их сопряжённых уравнений с распространением решений этих уравнений на весь цилиндр, в котором рассматривается исходное уравнение. В результате такого
\glqq подравнивания\grqq\ левая часть сопряжённого уравнения не изменяется, а начальное условие для производной по времени \glqq перекачивается\grqq\ в правую часть сопряжённого уравнения,
так что в правой части получающегося уравнения оказывается $\delta$--мера Радона.

В связи с этим представляется важным изучение свойств решений гиперболических уравнений дивергентного вида с $\delta$--мерой Радона в правой части и изучение возможности представления
решения линейного уравнения с произвольной мерой Радона в правой части в виде интеграла от решения уравнения с $\delta$--мерой Радона в правой части.

Поэтому указанные вопросы подробно рассматриваются в данной монографии. Заметим также, что, насколько нам известно, применительно к гиперболическим уравнениям дивергентного вида никто
подобные вопросы ранее не рассматривал.

        \section*{Обозначения}\addcontentsline{toc}{section}{Обозначения}
Здесь и всюду ниже мы используем следующие обозначения:

$\mathbb{R}^m$ --- $m$--мерное пространство векторов--столбцов $x=(x_1,\dots,x_m)$ с евклидовой нормой
$$|x|\equiv\sqrt{\sum\limits_{i=1}^mx_i^2};$$

$\text{Ш}^m_\varepsilon(x^0)\equiv\{x\in\mathbb{R}^m:|x-x^0|<\varepsilon\}$;

$\mathbb{R}^{m\times n}$ --- $mn$--мерное пространство $(m\times n)$--матриц $A=\{a_{ij}\}$ со скалярным произведением
\begin{gather*}
\langle A,B\rangle\equiv\sum\limits_{i=1}^m\sum\limits_{j=1}^na_{ij}b_{ij};
\end{gather*}
этому скалярному произведению соответствует евклидова норма
\begin{gather*}
|A|\equiv\sqrt{\sum\limits_{i=1}^m\sum\limits_{j=1}^na_{ij}^2};
\end{gather*}

$\mathbf{M}(\mathcal{P})$ --- множество всех мер Радона на компакте $\mathcal{P}$, $\|\mu\|$ --- полная вариация меры
$\mu\in \mathbf{M}(\mathcal{P})$;

$\mathbf{M}_+(\mathcal{P})$ --- множество всех неотрицательных мер Радона на компакте $\mathcal{P}$;

$\Omega\subset \mathbb{R}^n$ --- ограниченная область с кусочно--гладкой границей $S$;

$T>0$ --- константа;

$S_T\equiv S\times(0,T)$;

${\myalpha}_i(s,t)$ --- угол между единичным вектором внешней нормали к $S_T$ в точке $(s,t)\in S_T$ и осью $Ox_i$, $i=\overline{1,n}$;

$cl X$ --- замыкание множества $X$;

$Q_T\equiv\Omega\times(0,T)$;

$S'\subseteq S$ --- множество, имеющее положительную поверхностную меру;

$S''\equiv S\setminus S'$ --- множество, имеющее положительную поверхностную меру;

$S'_T\equiv S'\times(0,T)$;

$S''_T\equiv S''\times(0,T)$;

$Q_{(t_1,t_2)} \equiv\Omega\times(t_1,t_2)$, где $t_1$, $t_2\in[0,T]$;

$Q_{[t_1,t_2]} \equiv\Omega\times[t_1,t_2]$, где $t_1$, $t_2\in[0,T]$;

$S_{(t_1,t_2)}\equiv S\times(t_1,t_2)$, где $t_1$, $t_2\in[0,T]$;

$S'_{(t_1,t_2)}\equiv S'\times(t_1,t_2)$, где $t_1$, $t_2\in[0,T]$;

$S''_{(t_1,t_2)}\equiv S''\times(t_1,t_2)$, где $t_1$, $t_2\in[0,T]$;

$L_p^m(G)$, где $G\subset \mathbb{R}^k$  --- ограниченная область, --- пространство $m$--мерных вектор--функций $z(x)\equiv(z_1(x),\dots,z_k(x))$, $x\in G$, с нормой
\begin{gather*}
\|z\|_{p,G}\equiv\left[\int\limits_G|z(x)|^pdx\right]^{1/p},\,\,\,1\leqslant p<\infty;\,\,\,
\|z\|_{\infty,G}=\vraisup\limits_{x\in G}|z(x)|;\,\,\, L^1_p(G)\equiv L_p(G);
\end{gather*}

$L_p([0,T],X)$, где $X$ --- сепарабельное банахово пространство с нормой $\|\cdot\|_X$, --- банахово пространство слабо измеримых на $[0,T]$ функций $\xi\colon[0,T]\to X$, для которых
функция $[0,T]\ni t\mapsto\|\xi(t)\|_X$ является элементом $L_p[0,T]$; норма в $L_p([0,T],X)$ определяется так:
\begin{gather*}
\|\xi\|_{p,[0,T],X}\equiv\left[\int\limits_0^T\|\xi(t)\|_X^pdt\right]^{1/p},\,\,\,1\leqslant p<\infty;\,\,\,
    \|\xi\|_{p,[0,T],X}\equiv\vraisup\limits_{t\in[0,T]}\|\xi(t)\|_X,\,\,\,p=\infty;
\end{gather*}

$L_{p,1}(Q_T)$ --- банахово пространство измеримых по Лебегу на $Q_T$ функций $\xi\colon Q_T\to \mathbb{R}$ с конечной нормой
\begin{gather*}
\|\xi\|_{p,1,Q_T}\equiv\int\limits_0^T\left[\int\limits_\Omega|\xi(x,t)|^pdx\right]^{1/p}dt\,\,\,(1\leqslant p<\infty);\,\,\,
\|\xi\|_{p,1,Q_T}\equiv\int\limits_0^T\vraisup\limits_{x\in\Omega}|\xi(x,t)|dt\,\,\,(p=\infty);
\end{gather*}

$L_{p,1}(Q_{[t_1,t_2]})$, где $t_1$, $t_2\in[0,T]$, $t_1< t_2$,  --- банахово пространство измеримых по Лебегу на $Q_{[t_1,t_2]}$
функций $\xi\colon Q_{[t_1,t_2]}\to \mathbb{R}$ с конечной нормой
\begin{gather*}
\|\xi\|_{p,1,Q_{[t_1,t_2]}}\equiv\int\limits_{t_1}^{t_2}\left[\int\limits_\Omega|\xi(x,t)|^pdx\right]^{1/p}dt\,\,\,(1\leqslant p<\infty);
\,\,\,\|\xi\|_{p,1,Q_{[t_1,t_2]}}\equiv\int\limits_{t_1}^{t_2}\vraisup\limits_{x\in\Omega}|\xi(x,t)|dt\,\,\,(p=\infty);
\end{gather*}

$L_{p,1}(S_T)$ --- банахово пространство измеримых по Лебегу на $S_T$ функций $\xi\colon S_T\to \mathbb{R}$ с конечной нормой
\begin{gather*}
\|\xi\|_{p,1,S_T}\equiv\int\limits_0^T\left[\int\limits_S|\xi(s,t)|^pds\right]^{1/p}dt\,\,\,(1\leqslant p<\infty);\,\,\,
\|\xi\|_{p,1,S_T}\equiv\int\limits_0^T\vraisup\limits_{s\in S}|\xi(s,t)|dt\,\,\,(p=\infty);
\end{gather*}

$L_{p,1}(S'_T)$ --- банахово пространство измеримых по Лебегу на $S'_T$ функций $\xi\colon S'_T\to \mathbb{R}$ с конечной нормой
\begin{gather*}
\|\xi\|_{p,1,S'_T}\equiv\int\limits_0^T\left[\int\limits_{S'}|\xi(s,t)|^pds\right]^{1/p}dt\,\,\,(1\leqslant p<\infty);\,\,\,
\|\xi\|_{p,1,S'_T}\equiv\int\limits_0^T\vraisup\limits_{s\in S'}|\xi(s,t)|dt\,\,\,(p=\infty);
\end{gather*}


$L_{p,1}(S_T,\mathbb{R}^n)$ --- банахово пространство измеримых по Лебегу на $S_T$ функций $\xi\colon S_T\to\mathbb{R}^n$ с конечной нормой
\begin{gather*}
\|\xi\|_{p,1,S_T}\equiv\int\limits_0^T\left[\int\limits_S|\xi(s,t)|^pds\right]^{1/p}dt\,\,\,(1\leqslant p<\infty);\,\,\,
\|\xi\|_{p,1,S_T}\equiv\int\limits_0^T\vraisup\limits_{s\in S}|\xi(s,t)|dt\,\,\,(p=\infty);
\end{gather*}

$L_{p,1}(S'_T,\mathbb{R}^n)$ --- банахово пространство измеримых по Лебегу на $S'_T$ функций $\xi\colon S'_T\to\mathbb{R}^n$ с конечной
нормой
\begin{gather*}
\|\xi\|_{p,1,S'_T}\equiv\int\limits_0^T\left[\int\limits_{S'}|\xi(s,t)|^pds\right]^{1/p}dt\,\,\,(1\leqslant p<\infty);\,\,\,
\|\xi\|_{p,1,S'_T}\equiv\int\limits_0^T\vraisup\limits_{s\in S'}|\xi(s,t)|dt\,\,\,(p=\infty);
\end{gather*}

$W^1_p[0,T]$, где $1\leqslant p\leqslant\infty$, --- множество всех функций $z\in L_p[0,T]$ с обобщённой производной $z'$ из $L_p[0,T]$; норма в $W^1_p[0,T]$ определяется соотношением
\begin{gather*}
\|z\|^{(1)}_{p,[0,T]}\equiv\left[\|z\|_{p,[0,T]}^p+\|z'\|_{p,[0,T]}^p\right]^{1/p},\,\,\,1\leqslant p<\infty;\,\,\,
\|z\|^{(1)}_{p,[0,T]}\equiv\|z\|_{\infty,[0,T]}+\|z'\|_{\infty,[0,T]},\,\,\,p=\infty;
\end{gather*}

$W^2_1[0,T]$ --- множество всех функций $z\in L_1[0,T]$ с суммируемыми первой и второй обобщёнными производными; норма в $W^2_1[0,T]$ определяется соотношением
\begin{gather*}
\|z\|^{(2)}_{1,[0,T]}\equiv\|z\|_{1,[0,T]}+\|z'\|_{1,[0,T]}+\|z''\|_{1,[0,T]};
\end{gather*}

$W^1_2(G)$, где $G\subset \mathbb{R}^k$  --- ограниченная область, --- гильбертово пространство функций $z\in L_2(G)$, у которых все первые обобщённые производные принадлежат $L_2(G)$, со
скалярным произведением
\begin{gather*}
\langle z_1,z_2\rangle=\int\limits_G\left[z_1z_2+\sum\limits_{i=1}^kz_{1x_i}z_{2x_i}\right]dx;
\end{gather*}
соответствующую норму в этом пространстве обозначим через $\|\cdot\|^{(1)}_{2,G}$;

$W^2_2(G)$, где $G\subset \mathbb{R}^k$  --- ограниченная область, --- гильбертово пространство функций $z\in L_2(G)$, у которых все обобщённые производные до второго порядка включительно
принадлежат $L_2(G)$, со скалярным произведением
\begin{gather*}
\langle z_1,z_2\rangle=\int\limits_G\left[z_1z_2+\sum\limits_{i=1}^kz_{1x_i}z_{2x_i}+\sum\limits_{i,j=1}^kz_{1x_ix_j}z_{2x_ix_j}\right]dx;
\end{gather*}
соответствующую норму в этом пространстве обозначим через $\|\cdot\|^{(2)}_{2,G}$;

$W^{0,1}_{p,1}(Q_T)$ --- банахово пространство функций $\xi\in L_{p,1}(Q_T)$, для которых $\xi_t\in L_{p,1}(Q_T)$; норма в  $W^{0,1}_{p,1}(Q_T)$ задаётся равенством
$$
\|\xi\|^{(0,1)}_{p,1,Q_T}\equiv\|\xi\|_{p,1,Q_T}+\|\xi_t\|_{p,1,Q_T};
$$

$W^{0,1}_{p,1}(S_T)$ --- банахово пространство функций $\xi\in L_{p,1}(S_T)$, для которых $\xi_t\in L_{p,1}(S_T)$; норма в  $W^{0,1}_{p,1}(S_T)$ задаётся равенством
$$
\|\xi\|^{(0,1)}_{p,1,S_T}\equiv\|\xi\|_{p,1,S_T}+\|\xi_t\|_{p,1,S_T};
$$

$W^{0,1}_{p,1}(S'_T)$ --- банахово пространство функций $\xi\in L_{p,1}(S'_T)$, для которых $\xi_t\in L_{p,1}(S'_T)$; норма в  $W^{0,1}_{p,1}(S'_T)$ задаётся равенством
$$
\|\xi\|^{(0,1)}_{p,1,S'_T}\equiv\|\xi\|_{p,1,S'_T}+\|\xi_t\|_{p,1,S'_T};
$$

$W^{0,1}_{p,1}(S_T,\mathbb{R}^n)$ --- банахово пространство функций $\xi\in L_{p,1}(S_T,\mathbb{R}^n)$, для которых $\xi_t\in L_{p,1}(S_T,\mathbb{R}^n)$; норма в
$W^{0,1}_{p,1}(S_T,\mathbb{R}^n)$ задаётся равенством
$$
\|\xi\|^{(0,1)}_{p,1,S_T}\equiv\|\xi\|_{p,1,S_T}+\|\xi_t\|_{p,1,S_T};
$$

$W^{0,1}_{p,1}(S'_T,\mathbb{R}^n)$ --- банахово пространство функций $\xi\in L_{p,1}(S'_T,\mathbb{R}^n)$, для которых $\xi_t\in L_{p,1}(S'_T,\mathbb{R}^n)$; норма в
$W^{0,1}_{p,1}(S'_T,\mathbb{R}^n)$ задаётся равенством
$$
\|\xi\|^{(0,1)}_{p,1,S'_T}\equiv\|\xi\|_{p,1,S_T}+\|\xi_t\|_{p,1,S'_T};
$$

$C^\infty_0(\Omega)$ --- множество всех бесконечно дифференцируемых финитных в $\Omega$ функций

$\stackrel{\circ}{W}\!\!^1_2(\Omega)$ --- замыкание в норме пространства $W^1_2(\Omega)$ множества $C^\infty_0(\Omega)$; норма в $\stackrel{\circ}{W}\!\!^1_2(\Omega)$ определяется так же,
как и в $W^1_2(\Omega)$;

$\stackrel{\circ}{W}\!\!^1_2(\Omega|S'')$ --- замыкание в норме пространства $W^1_2(\Omega)$ множества всех бесконечно дифференцируемых в $\Omega$ и финитных вблизи $S''$ функций; норма в
$\stackrel{\circ}{W}\!\!^1_2(\Omega|S'')$ определяется так же, как и в $W^1_2(\Omega)$;


$\stackrel{\circ}{W}\!^{-1}_2(\Omega)$ --- пространство, сопряжённое к $\stackrel{\circ}{W}\!\!^1_2(\Omega)$;

$W^{-1}_2(\Omega)$ --- пространство, сопряжённое к $W\!^1_2(\Omega)$;

$\stackrel{\circ}{W}\!^{-1}_2(\Omega|S'')$--- пространство, сопряжённое к $\stackrel{\circ}{W}\!\!^1_2(\Omega|S'')$;


$\stackrel{\circ}{W}\!\!^2_2(\Omega)$ --- замыкание в норме пространства $W^2_2(\Omega)$ множества $C^\infty_0(\Omega)$;
норма в $\stackrel{\circ}{W}\!\!^2_2(\Omega)$ определяется так же, как и в $W^2_2(\Omega)$;

$\stackrel{\circ}{W}\!\!^2_2(\Omega|S'')$ --- замыкание в норме пространства $W^2_2(\Omega)$ множества всех бесконечно дифференцируемых в $\Omega$ и финитных вблизи $S''$ функций; норма в
$\stackrel{\circ}{W}\!\!^2_2(\Omega|S'')$ определяется так же, как и в $W^2_2(\Omega)$;


$\stackrel{\circ}{W}\!^{-2}_2(\Omega)$ --- пространство, сопряжённое к $\stackrel{\circ}{W}\!\!^2_2(\Omega)$;

$W^{-2}_2(\Omega)$ --- пространство, сопряжённое к $W\!^2_2(\Omega)$;

$\stackrel{\circ}{W}\!^{-2}_2(\Omega|S'')$--- пространство, сопряжённое к $\stackrel{\circ}{W}\!\!^2_2(\Omega|S'')$;



$C^{\infty,0}(Q_T)$ --- множество всех бесконечно дифференцируемых в $Q_T$ финитных вблизи $S_T$ функций;

$C^{\infty,0}(Q_{[t_1,t_2]})$, где $t_1$, $t_2\in[0,T]$, $t_1< t_2$, --- множество всех бесконечно дифференцируемых в $Q_T$
финитных вблизи $S_{[t_1,t_2]}$ функций;

$W^1_{2,0}(Q_T)$ --- замыкание множества $C^{\infty,0}(Q_T)$ в норме $W^1_2(Q_T)$; норма в $W^1_{2,0}(Q_T)$ задаётся равенством
\begin{gather*}
\|z\|^{(1)}_{2,Q_T}\equiv\left[\int\limits_{Q_T}[z^2+z_t^2+|\nabla_x z|^2]dxdt\right]^{1/2};
\end{gather*}

$W^1_{2,0}(Q_{[t_1,t_2]})$, где $t_1$, $t_2\in[0,T]$, $t_1< t_2$, --- замыкание множества $C^{\infty,0}(Q_{[t_1,t_2]})$ в
норме $W^1_2(Q_{[t_1,t_2]})$; норма в $W^1_{2,0}(Q_{[t_1,t_2]})$ задаётся равенством
\begin{gather*}
\|z\|^{(1)}_{2,Q_{[t_1,t_2]}}\equiv\left[\int\limits_{Q_{[t_1,t_2]}}[z^2+z_t^2+|\nabla_x z|^2]dxdt\right]^{1/2};
\end{gather*}

$W^1_{2,0}(Q_T|S''_T)$ --- замыкание в норме $W^1_2(Q_T)$ множества бесконечно дифференцируемых в $Q_T$ финитных вблизи $S''_T$ функций; норма в $W^1_{2,0}(Q_T|S''_T)$ задаётся равенством
\begin{gather*}
\|z\|^{(1)}_{2,Q_T}\equiv\left[\int\limits_{Q_T}[z^2+z_t^2+|\nabla_x z|^2]dxdt\right]^{1/2};
\end{gather*}

$W^1_{2,0}(Q_{[t_1,t_2]}|S''_{[t_1,t_2]})$, где $t_1$, $t_2\in[0,T]$, $t_1< t_2$, --- замыкание множества  бесконечно дифференцируемых в $Q_{[t_1,t_2]}$ финитных вблизи $S''_{[t_1,t_2]}$
функций; норма в  $W^1_{2,0}(Q_{[t_1,t_2]}|S''_{[t_1,t_2]})$ задаётся равенством
\begin{gather*}
\|z\|^{(1)}_{2,Q_{[t_1,t_2]}}\equiv\left[\int\limits_{Q_{[t_1,t_2]}}[z^2+z_t^2+|\nabla_x z|^2]dxdt\right]^{1/2};
\end{gather*}


$W^{2;1}_2(Q_T)$ --- гильбертово пространство, состоящее из всех функций $z\in L_2(Q_T)$, у которых $z_{x_i}$, $z_{x_ix_j}$, $z_t\in
L_2(Q_T)$, $i,\,j=\overline{1,n}$; скалярное произведение в $W^{2;1}_2(Q_T)$ задаётся формулой
\begin{gather*}
\langle z_1,z_2\rangle=
\int\limits_{Q_T}\left[z_1z_2+\sum\limits_{i=1}^kz_{1x_i}z_{2x_i}+\sum\limits_{i,j=1}^kz_{1x_ix_j}z_{2x_ix_j}+
z_{1t}z_{2t}\right]dx;
\end{gather*}
норму, отвечающую этому скалярному произведению, обозначим через $\|z\|^{(2;1)}_{2,Q_T}$;



$W^{2;1}_{2,0}(Q_T)$ --- замыкание множества $C^{\infty,0}(Q_T)$ в норме $W^{2;1}_2(Q_T)$; норма в $W^{2;1}_{2,0}(Q_T)$ задаётся так же, как и в $W^{2;1}_2(Q_T)$;

$W^{2;1}_{2,0}(Q_{[t_1,t_2]})$, где $t_1$, $t_2\in[0,T]$, $t_1< t_2$, --- замыкание множества $C^{\infty,0}(Q_{[t_1,t_2]})$ в
норме $W^{2;1}_2(Q_{[t_1,t_2]})$; норма в $W^1_{2,0}(Q_{[t_1,t_2]})$ задаётся так же, как и в $W^{2;1}_2(Q_{[t_1,t_2]})$;



$W^{2;1}_{2,0}(Q_T|S''_T)$ --- замыкание в норме $W^1_2(Q_T)$ множества бесконечно дифференцируемых в $Q_T$ финитных вблизи $S''_T$ функций;
норма в $W^{2;1}_{2,0}(Q_T|S''_T)$ задаётся так же, как и в $W^{2;1}_2(Q_T)$;

$W^{2;1}_{2,0}(Q_{[t_1,t_2]}|S''_{[t_1,t_2]})$, где $t_1$, $t_2\in[0,T]$, $t_1< t_2$, --- замыкание множества  бесконечно дифференцируемых
в $Q_{[t_1,t_2]}$ финитных вблизи $S''_{[t_1,t_2]}$ функций; норма в $W^{2;1}_{2,0}(Q_{[t_1,t_2]}|S''_{[t_1,t_2]})$ задаётся так же, как и
в $W^{2;1}_2(Q_{[t_1,t_2]})$;


$C(\mathcal{P})$, где $\mathcal{P}$ --- компактное топологическое пространство, --- пространство непрерывных на $\mathcal{P}$ функций $z\colon\mathcal{P}\to \mathbb{R}$, с нормой
$$
\pmb{|}z\pmb{|}^{(0)}_{\mathcal{P}}\equiv\max\limits_{p\in\mathcal{P}}|z(p)|;
$$

$C(\mathcal{P},X)$, где $(\mathcal{P},\tau)$ --- компактное топологическое пространство, а  $X$ --- банахово пространство с нормой $\|\cdot\|_X$, --- пространство сильно непрерывных на
$\mathcal{P}$ функций $\xi\colon\mathcal{P}\to X$, т.е. таких, что
\begin{gather*}
\forall\,p\in\mathcal{P}\,\,\forall\,\varepsilon>0\,\,\exists\,U=U(\varepsilon,p)\in\tau,\,\,p\in U\,\,\forall\,p'\in U:
\|\xi(p)-\xi(p')\|_X\leqslant\varepsilon;
\end{gather*}
норма в этом пространстве задаётся равенством
\begin{gather*}
\pmb{|}\xi\pmb{|}^{(0)}_{\mathcal{P},X}\equiv\max\limits_{p\in\mathcal{P}}\|\xi(p)\|_X;
\end{gather*}


$C^k(\mathcal{P},X)$, где  $X$ --- банахово пространство с нормой $\|\cdot\|_X$, а  $\mathcal{P}\subset \mathbb{R}^d$ --- компакт,--- пространство $k$ раз сильно непрерывно
дифференцируемых функций  $\xi:\mathcal{P}\to X$, с нормой
\begin{gather*}
\pmb{|}\xi\pmb{|}^{(k)}_{\mathcal{P},X}\equiv\sum\limits_{\substack{0\leqslant i_1+\dots+i_k\leqslant k\\
 i_1,\dots,i_k\geqslant0}}^k\max\limits_{p\in\mathcal{P}}\left\|
 \frac{\partial^{i_1+\dots+i_k}\xi(p)}{\partial p^{i_1}_1\dots\partial p^{i_k}_k}\right\|_X;\,\,\,C^0(\mathcal{P},X)\equiv C(\mathcal{P},X);
\end{gather*}

$\Gamma\equiv[0,T]\times[0,T]$;

$\mathbb{K}^1_m(\Gamma)$ --- множество функций $h(t,\tau)\in \mathbb{R}^m$, $(t,\tau)\in\Gamma$, таких, что $h$, $h_\tau\in C(\Gamma,\mathbb{R}^m)$; норма в этом пространстве задаётся
 равенством
$$
\|h\|_{\mathbb{K}^1_m(\Gamma)}\equiv\pmb{|}h\pmb{|}_{\Gamma,\mathbb{R}^m}^{(0)}+ \pmb{|}h_\tau\pmb{|}_{\Gamma,\mathbb{R}^m}^{(0)};
$$

$C_s([0,T],X)$, где $X$ --- банахово пространство, --- пространство функций $\xi\colon[0,T]\to X$, слабо непрерывных на $[0,T]$, т.е. таких, что
$$
\forall\,x^*\in X^*\,\forall\,\tau\in[0,T]:\lim\limits_{t\to\tau}\langle\xi(t),x^*\rangle=\langle\xi(\tau),x^*\rangle,
$$
где $X^*$ обозначает сопряжённое к $X$ пространство, а $\langle x,x^*\rangle$ --- значение линейного непрерывного функционала $x^*\in X^*$ в точке $x\in X$ (пространство $C_s([0,T],X)$
введено в \cite{LionsMajenes});

$C^\infty[0,T]$ --- множество всех бесконечно дифференцируемых на отрезке $[0,T]$ функций;

$\mathcal{L}(X,Y)$, где $X$ и $Y$ --- банаховы пространства с нормами $\|\cdot\|_X$ и $\|\cdot\|_Y$ соответственно, --- пространство
линейных ограниченных операторов, действующих из $X$ в $Y$, наделённое стандартной нормой
$$
\|A\|_{X\to Y}\equiv\sup\limits_{\|x\|_X\leqslant1}\|Ax\|_Y;
$$

$\textrm{Э}^{1}_{2}(Q_T)$ --- \glqq энергетический класс первого рода\grqq, состоящий из измеримых по Лебегу на $Q_T$ функций $z$, удовлетворяющих следующим условиям:

при всех $t\in[0,T]$ справедливы включения $z(\cdot,t)\in{W}\!^1_2(\Omega)$, $z_t(\cdot,t)\in L_2(\Omega)$;

функция $[0,T]\ni t\mapsto z(\cdot,t)$ --- элемент пространства $C_s([0,T],W\!^1_2(\Omega))$;

функция  $[0,T]\ni t\mapsto z_t(\cdot,t)$ --- элемент $L_\infty([0,T],L_2(\Omega))$;

\noindent норма в  $\textrm{Э}^{1}_2(Q_T)$ задаётся равенством
$$
\|z\|_{\textrm{Э}^{1}_2(Q_T)}\equiv\sup\limits_{t\in[0,T]}\|z(\cdot,t)\|^{(1)}_{2,\Omega}+\vraisup\limits_{t\in[0,T]}\|z_t(\cdot,t)\|_{2,\Omega};
$$

$\hat{\textrm{Э}}{}^{1}_{2}(Q_T)\equiv\{z\in \textrm{Э}^{1}_{2}(Q_T):z(x,T)=0,\,\,\,x\in\Omega\}$;


$\textrm{Є}^{1}_{2}(Q_T)$ --- \glqq энергетический класс второго рода\grqq, состоящий из функций $z\in \textrm{Э}^{1}_{2}(Q_T)$, у которых $z_t\in C_s([0,T],L_2(\Omega))$; норма в
$\textrm{Є}^{1}_{2}(Q_T)$ задаётся равенством
$$
\|z\|_{\textrm{Є}^{1}_{2}(Q_T)}\equiv\sup\limits_{t\in[0,T]}\left(\int\limits_\Omega[z^2(x,t)+|\nabla_xz(x,t)|^2+z_t^2(x,t)]dx\right)^{1/2};
$$


$\textbf{Э}^{1}_{2}(Q_T)$ --- \glqq энергетический класс третьего рода\grqq, т.е. множество всех измеримых по Лебегу функций $z\colon Q_T\to \mathbb{R}$, таких, что при всех $t\in[0,T]$
справедливы включения $z(\cdot,t)\in{W}^1_2(\Omega)$, $z_t(\cdot,t)\in L_2(\Omega)$, причём $z(\cdot,t)$ и $z_t(\cdot,t)$ непрерывно зависят от $t\in[0,T]$ в норме пространств
$W^1_2(\Omega)$ и $L_2(\Omega)$ соответственно; норма в $\textbf{Э}^{1}_{2}(Q_T)$ определяется равенством
$$
\|z\|_{\textbf{Э}^{1}_{2}(Q_T)}\equiv \max\limits_{t\in[0,T]}\left(\int\limits_\Omega [|z(x,t)|^2+|\nabla_x z(x,t)|^2+|z_t(x,t)|^2]\,dx\right)^{1/2};
$$

$\hat{\textbf{Э}}^{1}_{2}(Q_T)\equiv\{z\in\textbf{Э}^{1}_{2}(Q_T):z(\cdot,T)=0\}$;


$\textrm{Э}^{1}_{2,0}(Q_T)$ --- \glqq энергетический класс первого рода\grqq, состоящий из измеримых по Лебегу на $Q_T$ функций $z$, удовлетворяющих следующим условиям:

при всех $t\in[0,T]$ справедливы включения $z(\cdot,t)\in\stackrel{\circ}{W}\!\!^1_2(\Omega)$, $z_t(\cdot,t)\in L_2(\Omega)$;

функция $[0,T]\ni t\mapsto z(\cdot,t)$ --- элемент пространства $C_s([0,T],\stackrel{\circ}{W}\!\!^1_2(\Omega))$;

функция  $[0,T]\ni t\mapsto z_t(\cdot,t)$ --- элемент $L_\infty([0,T],L_2(\Omega))$;

\noindent норма в  $\textrm{Э}^{1}_{2,0}(Q_T)$ задаётся так же, как и в $\textrm{Э}^{1}_{2}(Q_T)$;


$\hat{\textrm{Э}}{}^{1}_{2,0}(Q_T)\equiv\{z\in \textrm{Э}^{1}_{2,0}(Q_T):z(x,T)=0,\,\,\,x\in\Omega\}$;


$\textrm{Є}^{1}_{2,0}(Q_T)$ --- \glqq энергетический класс второго рода\grqq, состоящий из функций $z\in \textrm{Э}^{1}_{2,0}(Q_T)$, у которых
$z_t\in C_s([0,T],L_2(\Omega))$; норма в $\textrm{Є}^{1}_{2,0}(Q_T)$ задаётся так же, как и в $\textrm{Є}^{1}_{2}(Q_T)$;


$\textbf{Э}^{1}_{2,0}(Q_T)$ --- \glqq энергетический класс третьего рода\grqq, т.е. множество всех измеримых по Лебегу функций $z\colon Q_T\to \mathbb{R}$, таких, что при всех $t\in[0,T]$
справедливы включения $z(\cdot,t)\in\stackrel{\circ}{W}\!\!\!^1_2(\Omega)$, $z_t(\cdot,t)\in L_2(\Omega)$, причём $z(\cdot,t)$ и $z_t(\cdot,t)$ непрерывно зависят от $t\in[0,T]$ в норме
пространств $\stackrel{\circ}{W}\!\!\!^1_2(\Omega)$ и $L_2(\Omega)$ соответственно; норма в $\textbf{Э}^{1}_{2,0}(Q_T)$ определяется так же, как и в $\textbf{Э}^{1}_{2}(Q_T)$;


$\hat{\textbf{Э}}{}^{1}_{2,0}(Q_T)\equiv\{z\in\textbf{Э}^{1}_{2,0}(Q_T):z(\cdot,T)=0\}$;


$\textrm{Э}^{1}_{2,0}(Q_T|S''_T)$ --- \glqq энергетический класс первого рода\grqq, состоящий из измеримых по Лебегу на $Q_T$ функций $z$, таких, что выполнены следующие условия:


при всех $t\in[0,T]$ имеют место включения $z(\cdot,t)\in\stackrel{\circ}{W}\!\!^1_2(\Omega|S'')$, $z_t(\cdot,t)\in L_2(\Omega)$;

функция $[0,T]\ni t\mapsto z(\cdot,t)$ --- элемент пространства $C_s([0,T],\stackrel{\circ}{W}\!\!^1_2(\Omega|S''))$;

функция  $[0,T]\ni t\mapsto z_t(\cdot,t)$ --- элемент $L_\infty([0,T],L_2(\Omega))$;

\noindent норма в  $\textrm{Э}^{1}_{2,0}(Q_T|S''_T)$ задаётся так же, как и в $\textrm{Э}^{1}_{2}(Q_T)$;


$\hat{\textrm{Э}}{}^{1}_{2,0}(Q_T|S''_T)\equiv\{z\in \textrm{Э}^{1}_{2,0}(Q_T|S''_T):z(x,T)=0,\,\,\,x\in\Omega\}$;


$\textrm{Є}^{1}_{2,0}(Q_T|S''_T)$ --- \glqq энергетический класс второго рода\grqq, состоящий из функций $z\in \textrm{Э}^{1}_{2,0}(Q_T|S''_T)$, у которых
$z_t\in C_s([0,T],L_2(\Omega))$; норма в $\textrm{Є}^{1}_{2,0}(Q_T|S''_T)$ задаётся так же, как и в $\textrm{Є}^{1}_{2}(Q_T)$;


$\textbf{Э}^{1}_{2,0}(Q_T|S''_T)$ --- \glqq энергетический класс третьего рода\grqq, т.е. множество всех измеримых по Лебегу функций $z\colon Q_T\to \mathbb{R}$, таких, что при всех
$t\in[0,T]$ справедливы включения $z(\cdot,t)\in\stackrel{\circ}{W}\!\!^1_2(\Omega|S'')$, $z_t(\cdot,t)\in L_2(\Omega)$, причём $z(\cdot,t)$ и $z_t(\cdot,t)$ непрерывно зависят от
$t\in[0,T]$ в норме пространств $\stackrel{\circ}{W}\!\!^1_2(\Omega|S'')$ и $L_2(\Omega)$ соответственно; норма в $\textbf{Э}^{1}_{2,0}(Q_T|S''_T)$ определяется
тем же соотношением, что и в $\textbf{Э}^{1}_{2}(Q_T)$;


$\hat{\textbf{Э}}{}^{1}_{2,0}(Q_T|S''_T)\equiv\{z\in\textbf{Э}^{1}_{2,0}(Q_T|S''_T):z(\cdot,T)=0\}$;
%%%%%%%%%%%%%%%%%%%%%%%%%%%%%%%%%%%%%%%%%%%%%%%%%%%%%%%%%%%%%%%%%%%%%%%%%%%%%%%%%%%%%%%%%%%%%%%%%%%%%%%%%%%%%%%%%%%%%%%%%%%%%%%%%%%%%%%%%%%%

$\textrm{Э}^{2}_{2}(Q_T)$ --- \glqq энергетический класс первого рода\grqq, состоящий из измеримых по Лебегу на $Q_T$ функций $z$, удовлетворяющих следующим условиям:

при всех $t\in[0,T]$ справедливы включения $z(\cdot,t)\in{W}\!^2_2(\Omega)$, $z_t(\cdot,t)\in L_2(\Omega)$;

функция $[0,T]\ni t\mapsto z(\cdot,t)$ --- элемент пространства $C_s([0,T],W\!^2_2(\Omega))$;

функция  $[0,T]\ni t\mapsto z_t(\cdot,t)$ --- элемент $L_\infty([0,T],L_2(\Omega))$;

\noindent норма в  $\textrm{Э}^{2}_2(Q_T)$ задаётся равенством
$$
\|z\|_{\textrm{Э}^{2}_2(Q_T)}\equiv\sup\limits_{t\in[0,T]}\|z(\cdot,t)\|^{(2)}_{2,\Omega}+\vraisup\limits_{t\in[0,T]}\|z_t(\cdot,t)\|_{2,\Omega};
$$

$\hat{\textrm{Э}}{}^{2}_{2}(Q_T)\equiv\{z\in \textrm{Э}^{2}_{2}(Q_T):z(x,T)=0,\,\,\,x\in\Omega\}$;


$\textrm{Є}^{2}_{2}(Q_T)$ --- \glqq энергетический класс второго рода\grqq, состоящий из функций $z\in \textrm{Э}^{2}_{2}(Q_T)$, у которых
$z_t\in C_s([0,T],L_2(\Omega))$; норма в $\textrm{Є}^{2}_{2}(Q_T)$ задаётся равенством
$$
\|z\|_{\textrm{Є}^{2}_{2}(Q_T)}\equiv\sup\limits_{t\in[0,T]}\left(\int\limits_\Omega[z^2(x,t)+|\nabla_xz(x,t)|^2+|\nabla^2_xz(x,t)|^2+z_t^2(x,t)]dx\right)^{1/2};
$$


$\textbf{Э}^{2}_{2}(Q_T)$ --- \glqq энергетический класс третьего рода\grqq, т.е. множество всех измеримых по Лебегу функций $z\colon Q_T\to \mathbb{R}$, таких, что при всех $t\in[0,T]$
справедливы включения $z(\cdot,t)\in{W}^2_2(\Omega)$, $z_t(\cdot,t)\in L_2(\Omega)$, причём $z(\cdot,t)$ и $z_t(\cdot,t)$ непрерывно зависят от $t\in[0,T]$ в норме пространств
$W^2_2(\Omega)$ и $L_2(\Omega)$ соответственно; норма в $\textbf{Э}^{2}_{2}(Q_T)$ определяется равенством
$$
\|z\|_{\textbf{Э}^{2}_{2}(Q_T)}\equiv \max\limits_{t\in[0,T]}\left(\int\limits_\Omega [|z(x,t)|^2+|\nabla_x z(x,t)|^2+|\nabla^2_xz(x,t)|^2+|z_t(x,t)|^2]\,dx\right)^{1/2};
$$

$\hat{\textbf{Э}}{}^{2}_{2}(Q_T)\equiv\{z\in\textbf{Э}^{2}_{2}(Q_T):z(\cdot,T)=0\}$;


$\textrm{Э}^{2}_{2,0}(Q_T)$ --- \glqq энергетический класс первого рода\grqq, состоящий из измеримых по Лебегу на $Q_T$ функций $z$, удовлетворяющих следующим условиям:

при всех $t\in[0,T]$ справедливы включения $z(\cdot,t)\in\stackrel{\circ}{W}\!\!^2_2(\Omega)$, $z_t(\cdot,t)\in L_2(\Omega)$;

функция $[0,T]\ni t\mapsto z(\cdot,t)$ --- элемент пространства $C_s([0,T],\stackrel{\circ}{W}\!\!^2_2(\Omega))$;

функция  $[0,T]\ni t\mapsto z_t(\cdot,t)$ --- элемент $L_\infty([0,T],L_2(\Omega))$;

\noindent норма в  $\textrm{Э}^{2}_{2,0}(Q_T)$ определяется тем же равенством, что и в $\textrm{Э}^{2}_{2}(Q_T)$;


$\hat{\textrm{Э}}{}^{2}_{2,0}(Q_T)\equiv\{z\in \textrm{Э}^{2}_{2,0}(Q_T):z(x,T)=0,\,\,\,x\in\Omega\}$;


$\textrm{Є}^{2}_{2,0}(Q_T)$ --- \glqq энергетический класс второго рода\grqq, состоящий из функций $z\in \textrm{Э}^{2}_{2,0}(Q_T)$, у которых
$z_t\in C_s([0,T],L_2(\Omega))$; норма в $\textrm{Є}^{2}_{2,0}(Q_T)$ задаётся так же, как и в $\textrm{Є}^{2}_{2}(Q_T)$;


$\textbf{Э}^{2}_{2,0}(Q_T)$ --- \glqq энергетический класс третьего рода\grqq, т.е. множество всех измеримых по Лебегу функций $z\colon Q_T\to \mathbb{R}$, таких, что при всех $t\in[0,T]$
справедливы включения $z(\cdot,t)\in\stackrel{\circ}{W}\!\!\!^2_2(\Omega)$, $z_t(\cdot,t)\in L_2(\Omega)$, причём $z(\cdot,t)$ и $z_t(\cdot,t)$ непрерывно зависят от $t\in[0,T]$ в норме
пространств $\stackrel{\circ}{W}\!\!\!^2_2(\Omega)$ и $L_2(\Omega)$ соответственно; норма в $\textbf{Э}^{2}_{2,0}(Q_T)$ определяется так же, как и в $\textbf{Э}^{2}_{2}(Q_T)$;


$\hat{\textbf{Э}}^{2}_{2,0}(Q_T)\equiv\{z\in\textbf{Э}^{2}_{2,0}(Q_T):z(\cdot,T)=0\}$;


$\textrm{Э}^{2}_{2,0}(Q_T|S''_T)$ --- \glqq энергетический класс первого рода\grqq, состоящий из измеримых по Лебегу на $Q_T$ функций $z$, таких, что выполнены следующие условия:

при всех $t\in[0,T]$ имеют место включения $z(\cdot,t)\in\stackrel{\circ}{W}\!\!^2_2(\Omega|S'')$, $z_t(\cdot,t)\in L_2(\Omega)$;

функция $[0,T]\ni t\mapsto z(\cdot,t)$ --- элемент пространства $C_s([0,T],\stackrel{\circ}{W}\!\!^2_2(\Omega|S''))$;

функция  $[0,T]\ni t\mapsto z_t(\cdot,t)$ --- элемент $L_\infty([0,T],L_2(\Omega))$;

\noindent норма в  $\textrm{Э}^{2}_{2,0}(Q_T|S''_T)$ задаётся так же, как и в  $\textrm{Э}^{2}_{2}(Q_T)$;


$\hat{\textrm{Э}}{}^{2}_{2,0}(Q_T|S''_T)\equiv\{z\in \textrm{Э}^{2}_{2,0}(Q_T|S''_T):z(x,T)=0,\,\,\,x\in\Omega\}$;


$\textrm{Є}^{2}_{2,0}(Q_T|S''_T)$ --- \glqq энергетический класс второго рода\grqq, состоящий из функций $z\in \textrm{Э}^{2}_{2,0}(Q_T|S''_T)$, у которых
$z_t\in C_s([0,T],L_2(\Omega))$; норма в $\textrm{Є}^{2}_{2,0}(Q_T|S''_T)$ задаётся так же, как и в $\textrm{Є}^{2}_{2}(Q_T)$;


$\textbf{Э}^{2}_{2,0}(Q_T|S''_T)$ --- \glqq энергетический класс третьего рода\grqq, т.е. множество всех измеримых по Лебегу функций $z\colon Q_T\to \mathbb{R}$, таких, что при всех
$t\in[0,T]$ справедливы включения $z(\cdot,t)\in\stackrel{\circ}{W}\!\!^2_2(\Omega|S'')$, $z_t(\cdot,t)\in L_2(\Omega)$, причём $z(\cdot,t)$ и $z_t(\cdot,t)$ непрерывно зависят от
$t\in[0,T]$ в норме пространств $\stackrel{\circ}{W}\!\!^2_2(\Omega|S'')$ и $L_2(\Omega)$ соответственно; норма в $\textbf{Э}^{2}_{2,0}(Q_T|S''_T)$ определяется
тем же равенством, что и в $\textbf{Э}^{2}_{2}(Q_T)$;


$\hat{\textbf{Э}}{}^{2}_{2,0}(Q_T|S''_T)\equiv\{z\in\textbf{Э}^{2}_{2,0}(Q_T|S''_T):z(\cdot,T)=0\}$;
%%%%%%%%%%%%%%%%%%%%%%%%%%%%%%%%%%%%%%%%%%%%%%%%%%%%%%%%%%%%%%%%%%%%%%%%%%%%%%%%%%%%%%%%%%%%%%%%%%%%%%%%%%%%%%%%%%%%%%%%%%%%%%%%%%%%%%%%%%%%

$\Var^{T}_{0}[\varphi]$ --- полное изменение функции~$\varphi\colon[0,T]\to \mathbb{R}$, т.е. точная верхняя грань по всевозможным разбиениям~$0=\tau_0<\tau_1<\tau_2<\dots<\tau_{k-1} <
\tau_k=T$, $k=1,2,\dots$, сумм вида~$\sum\limits_{j=1}^k|\varphi(\tau_j)-\varphi(\tau_{j-1})|$;

$\mathbf{BV}[0,T]$ --- множество всех функций~$\varphi\colon[0,T]\to \mathbb{R}$ ограниченной вариации, т.е. таких, что $\Var^{T}_{0}[\varphi]<+\infty$;

$\mathbf{BV}^0[0,T]$ --- множество всех функций~$\varphi\in \mathbf{BV}[0,T]$, непрерывных справа в каждой точке полуинтервала~$(0,T]$ и равных нулю в точке~$t=0$, наделённое
нормой~$\|\varphi\|_{\mathbf{BV}^0}\equiv\Var^{T}_{0}[\varphi]$, $\mathbf{BV}^0[0,T]\equiv (C[0,T])^*$;


$\textbf{Э}_1([0,T];V,H)$, где $V$ и $H$, $V\subset H$, --- гильбертовы пространства, причём вложение $V\subset H$ --- компактно, --- множество функций $\textsf{z}\in C([0,T],V)$, у
которых $\dot{\textsf{z}}\in C([0,T],H)$; норма в пространстве $\textbf{Э}_1([0,T];V,H)$ задаётся равенством
$$
\|\textsf{z}\|_{\textbf{Э}_1([0,T];V,H)}\equiv\max\limits_{t\in[0,T]}\sqrt{\|\textsf{z}(t)\|^2_V+\|\dot{\textsf{z}}(t)\|^2_H};
$$

$\textbf{Э}_2([0,T];V,H)$ --- множество функций $\textsf{z}\in \textbf{Э}_1([0,T];V,H)$, у которых $\ddot{\textsf{z}}\in L_1([0,T],V^*)$; норма в пространстве $\textbf{Э}_2([0,T];V,H)$
задаётся равенством
$$
\|\textsf{z}\|_{\textbf{Э}_2([0,T];V,H)}\equiv\max\limits_{t\in[0,T]}\sqrt{\|\textsf{z}(t)\|^2_V+\|\dot{\textsf{z}}(t)\|^2_H}+\|\ddot{\textsf{z}}\|_{1,[0,T],V^*};
$$


$\osc(f;E)$, где функция $f$ определена на некотором множестве $E$ и принимает значения в метрическом пространстве $X$ с метрикой $d$ --- колебание функции $f$ на множестве $E$, то есть
величина
\begin{gather*}
\sup\limits_{t',\,t''\in E}d(f(t'),f(t''));
\end{gather*}

$\osc(f;t_0)$, где функция $f$ определена на некотором множестве $E\subset\mathbb{R}$ и принимает значения в метрическом пространстве $X$ с метрикой $d$, а $t_0\in E$, --- колебание
функции $f$ в точке $t_0$, то есть величина
\begin{gather*}
\inf\limits_{r>0}\osc(f;E\cap(t_0-r,t_0+r));
\end{gather*}

через $\chi(t,\tau)$ обозначается функция, задаваемая соотношениями
\begin{gather*}
\chi(t,\tau)=
\begin{cases} 1,\mbox{ при $0\leqslant t\leqslant\tau\leqslant T$;}\cr
0,\mbox{ при $0\leqslant\tau<t\leqslant T$;}
\end{cases}
\,\,\,(t,\tau)\in[0,T]\times[0,T];
\end{gather*}

$\chi_E(t)$, $t\in[0,T]$, --- характеристическая функция измеримого по Лебегу множества $E\subseteq[0,T]$;

$\mydelta_\tau$ --- мера Дирака, сосредоточенная в точке $\tau\in\mathbb{R}$;

$\Omega^*\equiv\Omega\times[0,1]$;

$Q_T^*\equiv Q_T\times[0,1]$;

$(\textrm{Б})\int\limits_E$ --- интеграл Бохнера по множеству $E$, $(\textrm{Л})\int\limits_E$ --- интеграл Лебега по множеству $E$, $(\textrm{Р})\int\limits_E$ ---
интеграл Римана по множеству $E$; если не оговорено иное, то все интегралы понимаются в смысле Лебега;

%%%%%%%%%%%%%%%%%%%%%%%%%%%%%%%%%%%%%%%%%%%%%%%%%%%%%%%%%%%%%%%%%%%%%%%%%%%%%%%%%%%%%%%%%%%%%%%%%%%%%%%%%%%%%%%%%%%%%%%%%%%%%%%%%%%%%%%%%%

если $\Omega\equiv(l_1,l_2)$, где $l_1$, $l_2$, $l_1<l_2$, --- некоторые вещественные числа, то
{\renewcommand{\theenumi}{\arabic{enumi}}
\renewcommand{\theenumii}{\asbuk{enumii}}
\renewcommand{\labelenumii}{\theenumii)}
\begin{enumerate}
    \item
через $\stackrel{\circ}{W}{\!\!\!}^1_{2[\textrm{л}]}(\Omega)$ обозначим банахово пространство, являющееся замыканием в норме пространства
$W^1_2[l_1,l_2]$ множества всех бесконечно дифференцируемых на $[l_1,l_2]$ вещественнозначных функций, равных нулю вблизи точки $x=l_1$;

    \item
через $\stackrel{\circ}{W}{\!\!\!}^1_{2[\textrm{\textrm{п}}]}(\Omega)$ обозначим банахово пространство, являющееся замыканием в норме пространства
$W^1_2[l_1,l_2]$ множества всех бесконечно дифференцируемых на $[l_1,l_2]$ вещественнозначных функций, равных нулю вблизи точки $x=l_2$;

    \item
под $\textrm{Э}^1_{2,0[\textrm{л}]}(Q_T)$ будем понимать множество функций $z:Q_T\to\mathbb{R}$, таких, что
\begin{enumerate}
    \item
при всех $t\in[0,T]$ справедливы включения $z(\cdot,t)\in\stackrel{\circ}{W}\!\!^1_{2[\textrm{л}]}(\Omega)$, $z_t(\cdot,t)\in L_2(\Omega)$;
    \item
функция $[0,T]\ni t\mapsto z(\cdot,t)$ --- элемент пространства $C_s([0,T],\stackrel{\circ}{W}\!\!^1_{2[{\textrm{л}}]}(\Omega))$;
    \item
функция  $[0,T]\ni t\mapsto z_t(\cdot,t)$ --- элемент $L_\infty([0,T],L_2(\Omega))$;
\end{enumerate}
норма в  $\textrm{\textrm{Э}}^1_{2,0[\textrm{л}]}(Q_T)$ задаётся как в $\textrm{Э}^1_{2}(Q_T)$;

    \item
под $\textrm{Э}^1_{2,0[\textrm{п}]}(Q_T)$ будем понимать множество функций $z:Q_T\to\mathbb{R}$, таких, что
\begin{enumerate}
    \item
при всех $t\in[0,T]$ справедливы включения $z(\cdot,t)\in\stackrel{\circ}{W}\!\!^1_{2[\textrm{п}]}(\Omega)$, $z_t(\cdot,t)\in L_2(\Omega)$;
    \item
функция $[0,T]\ni t\mapsto z(\cdot,t)$ --- элемент пространства $C_s([0,T],\stackrel{\circ}{W}\!\!^1_{2[\textrm{п}]}(\Omega))$;
    \item
функция  $[0,T]\ni t\mapsto z_t(\cdot,t)$ --- элемент $L_\infty([0,T],L_2(\Omega))$;
\end{enumerate}
норма в  $\textrm{Э}^1_{2,0[\textrm{п}]}(Q_T)$ задаётся как в $\textrm{Э}^1_{2}(Q_T)$;

    \item
под $\textrm{Є}^1_{2,0[\textrm{л}]}(Q_T)$ будем понимать множество функций $z:Q_T\to\mathbb{R}$, таких, что
\begin{enumerate}
    \item
при всех $t\in[0,T]$ справедливы включения $z(\cdot,t)\in\stackrel{\circ}{W}\!\!^1_{2[\textrm{л}]}(\Omega)$, $z_t(\cdot,t)\in L_2(\Omega)$;
    \item
функция $[0,T]\ni t\mapsto z(\cdot,t)$ --- элемент пространства $C_s([0,T],\stackrel{\circ}{W}\!\!^1_{2[\textrm{л}]}(\Omega))$;
    \item
функция  $[0,T]\ni t\mapsto z_t(\cdot,t)$ --- элемент $C_s([0,T],L_2(\Omega))$;
\end{enumerate}
норма в  $\textrm{Є}^1_{2,0[\textrm{л}]}(Q_T)$ задаётся как в $\textrm{Є}^1_{2}(Q_T)$;

    \item
под $\textrm{Є}^1_{2,0[\textrm{п}]}(Q_T)$ будем понимать множество функций $z:Q_T\to\mathbb{R}$, таких, что
\begin{enumerate}
    \item
при всех $t\in[0,T]$ справедливы включения $z(\cdot,t)\in\stackrel{\circ}{W}\!\!^1_{2[\textrm{п}]}(\Omega)$, $z_t(\cdot,t)\in L_2(\Omega)$;
    \item
функция $[0,T]\ni t\mapsto z(\cdot,t)$ --- элемент пространства $C_s([0,T],\stackrel{\circ}{W}\!\!^1_{2[\textrm{п}]}(\Omega))$;
    \item
функция  $[0,T]\ni t\mapsto z_t(\cdot,t)$ --- элемент $C_s([0,T],L_2(\Omega))$;
\end{enumerate}
норма в  $\textrm{Є}^1_{2,0[\textrm{п}]}(Q_T)$ задаётся как в $\textrm{Є}^1_{2}(Q_T)$;

    \item
через $C^{\infty,0}_{\textrm{л}}(Q_T)$ обозначается множество всех бесконечно дифференцируемых в $Q_T$ вещественнозначных функций, равных нулю вблизи левой стороны (стороны $x=l_1$)
прямоугольника $Q_T$;

    \item
через $C^{\infty,0}_{\textrm{п}}(Q_T)$ обозначается множество всех бесконечно дифференцируемых в $Q_T$ вещественнозначных функций, равных нулю вблизи
правой стороны (стороны $x=l_2$) прямоугольника $Q_T$;

    \item
под $C^{\infty,0}_{\textrm{л}}(Q_{[t_1,t_2]})$, где $t_1$, $t_2\in[0,T]$, $t_1<t_2$, понимается множество всех бесконечно дифференцируемых в $Q_T$
функций, финитных вблизи левой стороны (стороны $x=l_1$) прямоугольника $Q_T$;

    \item
под $C^{\infty,0}_{\textrm{п}}(Q_{[t_1,t_2]})$, где $t_1$, $t_2\in[0,T]$, $t_1<t_2$, понимается множество всех бесконечно дифференцируемых в $Q_T$
функций, финитных вблизи правой стороны (стороны $x=l_2$) прямоугольника $Q_T$;

    \item
$W^1_{2,0[\textrm{л}]}(Q_T)$ --- замыкание в норме $W^1_2(Q_T)$ множества $C^{\infty,0}_{\textrm{л}}(Q_T)$; норма в $W^1_{2,0[\textrm{л}]}(Q_T)$ задаётся так же, как и в $W^1_2(Q_T)$;

    \item
$W^1_{2,0[\textrm{п}]}(Q_T)$ --- замыкание в норме $W^1_2(Q_T)$ множества $C^{\infty,0}_{\textrm{п}}(Q_T)$; норма в $W^1_{2,0[\textrm{п}]}(Q_T)$ задаётся так же, как и в $W^1_2(Q_T)$;

    \item
положим $\hat{\textrm{Э}}{}^1_{2,0[\textrm{л}]}(Q_T)\equiv\{z\in \textrm{Э}^1_{2,0[\textrm{л}]}(Q_T):z(\cdot,T)=0\}$,
$\hat{\textrm{Э}}{}^1_{2,0[\textrm{п}]}(Q_T)\equiv\{z\in \textrm{Э}^1_{2,0[\textrm{п}]}(Q_T): z(\cdot,T)=0\}$,
$\hat{\textrm{Є}}{}^1_{2,0[\textrm{л}]}(Q_T)\equiv\{z\in \textrm{Є}^1_{2,0[\textrm{л}]}(Q_T):z(\cdot,T)=0\}$,
$\hat{\textrm{Є}}{}^1_{2,0[\textrm{п}]}(Q_T)\equiv\{z\in \textrm{Є}^1_{2,0[\textrm{п}]}(Q_T):z(\cdot,T)=0\}$;

    \item
под $\textbf{Э}^1_{2,0[\textrm{л}]}(Q_T)$ будем понимать множество функций $z:Q_T\to\mathbb{R}$, таких, что
\begin{enumerate}
    \item
при всех $t\in[0,T]$ справедливы включения $z(\cdot,t)\in\stackrel{\circ}{W}\!\!^1_{2[\textrm{л}]}(\Omega)$, $z_t(\cdot,t)\in L_2(\Omega)$;
    \item
функция $[0,T]\ni t\mapsto z(\cdot,t)$ --- элемент пространства $C([0,T],\stackrel{\circ}{W}\!\!^1_{2[\textrm{л}]}(\Omega))$;
    \item
функция  $[0,T]\ni t\mapsto z_t(\cdot,t)$ --- элемент $C([0,T],L_2(\Omega))$;
\end{enumerate}
норма в  $\textbf{Э}^1_{2,0[\textrm{л}]}(Q_T)$ задаётся как в $\textbf{Э}^1_{2}(Q_T)$;

    \item
под $\textbf{Э}^1_{2,0[\textrm{п}]}(Q_T)$ будем понимать множество функций $z:Q_T\to\mathbb{R}$, таких, что
\begin{enumerate}
    \item
при всех $t\in[0,T]$ справедливы включения $z(\cdot,t)\in\stackrel{\circ}{W}\!\!^1_{2[\textrm{п}]}(\Omega)$, $z_t(\cdot,t)\in L_2(\Omega)$;
    \item
функция $[0,T]\ni t\mapsto z(\cdot,t)$ --- элемент пространства $C([0,T],\stackrel{\circ}{W}\!\!^1_{2[\textrm{п}]}(\Omega))$;
    \item
функция  $[0,T]\ni t\mapsto z_t(\cdot,t)$ --- элемент $C([0,T],L_2(\Omega))$;
\end{enumerate}
норма в  $\textbf{Э}^1_{2,0[\textrm{п}]}(Q_T)$ задаётся как в $\textbf{Э}^1_{2}(Q_T)$;

    \item
положим $\hat{\textbf{Э}}{}^{1}_{2,0[\textrm{л}]}(Q_T)\equiv\{z\in {\textbf{Э}}{}^{1}_{2,0[\textrm{л}]}(Q_T):z(\cdot,T)=0\}$,
$\hat{\textbf{Э}}{}^{1}_{2,0[\textrm{п}]}(Q_T)\equiv\{z\in {\textbf{Э}}{}^{1}_{2,0[\textrm{п}]}(Q_T):z(\cdot,T)=0\}$;

    \item
через $\stackrel{\circ}{W}{\!\!\!}^2_{2[\textrm{л}]}(\Omega)$ обозначим банахово пространство, являющееся замыканием в норме пространства
$W^2_2[l_1,l_2]$ множества всех бесконечно дифференцируемых на $[l_1,l_2]$ вещественнозначных функций, равных нулю вблизи точки $x=l_1$;

    \item
через $\stackrel{\circ}{W}{\!\!\!}^2_{2[\textrm{п}]}(\Omega)$ обозначим банахово пространство, являющееся замыканием в норме пространства
$W^2_2[l_1,l_2]$ множества всех бесконечно дифференцируемых на $[l_1,l_2]$ вещественнозначных функций, равных нулю вблизи точки $x=l_2$;

    \item
под $\textrm{Э}^2_{2,0[\textrm{л}]}(Q_T)$ будем понимать множество функций $z:Q_T\to\mathbb{R}$, таких, что
\begin{enumerate}
    \item
при всех $t\in[0,T]$ справедливы включения $z(\cdot,t)\in\stackrel{\circ}{W}\!\!^2_{2[\textrm{л}]}(\Omega)$, $z_t(\cdot,t)\in L_2(\Omega)$;
    \item
функция $[0,T]\ni t\mapsto z(\cdot,t)$ --- элемент пространства $C_s([0,T],\stackrel{\circ}{W}\!\!^2_{2[\textrm{л}]}(\Omega))$;
    \item
функция  $[0,T]\ni t\mapsto z_t(\cdot,t)$ --- элемент $L_\infty([0,T],L_2(\Omega))$;
\end{enumerate}
норма в  $\textrm{Э}^2_{2,0[\textrm{л}]}(Q_T)$ задаётся как в $\textrm{Э}^2_{2}(Q_T)$;

    \item
под $\textrm{Э}^2_{2,0[\textrm{п}]}(Q_T)$ будем понимать множество функций $z:Q_T\to\mathbb{R}$, таких, что
\begin{enumerate}
    \item
при всех $t\in[0,T]$ справедливы включения $z(\cdot,t)\in\stackrel{\circ}{W}\!\!^2_{2[\textrm{п}]}(\Omega)$, $z_t(\cdot,t)\in L_2(\Omega)$;
    \item
функция $[0,T]\ni t\mapsto z(\cdot,t)$ --- элемент пространства $C_s([0,T],\stackrel{\circ}{W}\!\!^2_{2[\textrm{п}]}(\Omega))$;
    \item
функция  $[0,T]\ni t\mapsto z_t(\cdot,t)$ --- элемент $L_\infty([0,T],L_2(\Omega))$;
\end{enumerate}
норма в  $\textrm{Э}^2_{2,0[\textrm{п}]}(Q_T)$ задаётся как в $\textrm{Э}^2_{2}(Q_T)$;

    \item
под $\textrm{Є}^2_{2,0[\textrm{л}]}(Q_T)$ будем понимать множество функций $z:Q_T\to\mathbb{R}$, таких, что
\begin{enumerate}
    \item
при всех $t\in[0,T]$ справедливы включения $z(\cdot,t)\in\stackrel{\circ}{W}\!\!^2_{2[\textrm{л}]}(\Omega)$, $z_t(\cdot,t)\in L_2(\Omega)$;
    \item
функция $[0,T]\ni t\mapsto z(\cdot,t)$ --- элемент пространства $C_s([0,T],\stackrel{\circ}{W}\!\!^2_{2[\textrm{л}]}(\Omega))$;
    \item
функция  $[0,T]\ni t\mapsto z_t(\cdot,t)$ --- элемент $C_s([0,T],L_2(\Omega))$;
\end{enumerate}
норма в  $\textrm{Є}^2_{2,0[\textrm{л}]}(Q_T)$ задаётся как в $\textrm{Є}^2_{2}(Q_T)$;

    \item
под $\textrm{Є}^2_{2,0[\textrm{п}]}(Q_T)$ будем понимать множество функций $z:Q_T\to\mathbb{R}$, таких, что
\begin{enumerate}
    \item
при всех $t\in[0,T]$ справедливы включения $z(\cdot,t)\in\stackrel{\circ}{W}\!\!^2_{2[\textrm{п}]}(\Omega)$, $z_t(\cdot,t)\in L_2(\Omega)$;
    \item
функция $[0,T]\ni t\mapsto z(\cdot,t)$ --- элемент пространства $C_s([0,T],\stackrel{\circ}{W}\!\!^2_{2[\textrm{п}]}(\Omega))$;
    \item
функция  $[0,T]\ni t\mapsto z_t(\cdot,t)$ --- элемент $C_s([0,T],L_2(\Omega))$;
\end{enumerate}
норма в  $\textrm{Є}^2_{2,0[\textrm{п}]}(Q_T)$ задаётся как в $\textrm{Є}^2_{2}(Q_T)$;

    \item
$W^{2;1}_{2,0[\textrm{л}]}(Q_T)$ --- замыкание в норме $W^{2;1}_2(Q_T)$ множества $C^{\infty,0}_{\textrm{л}}(Q_T)$; норма в $W^{2;1}_{2,0[\textrm{л}]}(Q_T)$ задаётся
так же, как и в $W^{2;1}_2(Q_T)$;

    \item
$W^{2;1}_{2,0[\textrm{п}]}(Q_T)$ --- замыкание в норме $W^{2;1}_2(Q_T)$ множества $C^{\infty,0}_{\textrm{п}}(Q_T)$; норма в $W^{2;1}_{2,0[\textrm{п}]}(Q_T)$ задаётся
так же, как и в $W^{2;1}_2(Q_T)$;

    \item
положим $\hat{\textrm{Э}}{}^2_{2,0[\textrm{л}]}(Q_T)\equiv\{z\in \textrm{Э}^2_{2,0[\textrm{л}]}(Q_T):z(\cdot,T)=0\}$,
$\hat{\textrm{Э}}{}^2_{2,0[\textrm{п}]}(Q_T)\equiv\{z\in \textrm{Э}^2_{2,0[\textrm{п}]}(Q_T):z(\cdot,T)=0\}$,
$\hat{\textrm{Є}}{}^2_{2,0[\textrm{л}]}(Q_T)\equiv\{z\in \textrm{Є}^2_{2,0[\textrm{л}]}(Q_T):z(\cdot,T)=0\}$,
$\hat{\textrm{Є}}{}^2_{2,0[\textrm{п}]}(Q_T)\equiv\{z \in \textrm{Є}^2_{2,0[\textrm{п}]}(Q_T):z(\cdot,T)=0\}$;

    \item
под $\textbf{Э}^2_{2,0[\textrm{л}]}(Q_T)$ будем понимать множество функций $z:Q_T\to\mathbb{R}$, таких, что
\begin{enumerate}
    \item
при всех $t\in[0,T]$ справедливы включения $z(\cdot,t)\in\stackrel{\circ}{W}\!\!^2_{2[\textrm{л}]}(\Omega)$, $z_t(\cdot,t)\in L_2(\Omega)$;
    \item
функция $[0,T]\ni t\mapsto z(\cdot,t)$ --- элемент пространства $C([0,T],\stackrel{\circ}{W}\!\!^2_{2[\textrm{л}]}(\Omega))$;
    \item
функция  $[0,T]\ni t\mapsto z_t(\cdot,t)$ --- элемент $C([0,T],L_2(\Omega))$;
\end{enumerate}
норма в  $\textbf{Э}^2_{2,0[\textrm{л}]}(Q_T)$ задаётся как в $\textbf{Э}^2_{2}(Q_T)$;

    \item
под $\textbf{Э}^2_{2,0[\textrm{п}]}(Q_T)$ будем понимать множество функций $z:Q_T\to\mathbb{R}$, таких, что
\begin{enumerate}
    \item
при всех $t\in[0,T]$ справедливы включения $z(\cdot,t)\in\stackrel{\circ}{W}\!\!^2_{2[\textrm{п}]}(\Omega)$, $z_t(\cdot,t)\in L_2(\Omega)$;
    \item
функция $[0,T]\ni t\mapsto z(\cdot,t)$ --- элемент пространства $C([0,T],\stackrel{\circ}{W}\!\!^2_{2[\textrm{п}]}(\Omega))$;
    \item
функция  $[0,T]\ni t\mapsto z_t(\cdot,t)$ --- элемент $C([0,T],L_2(\Omega))$;
\end{enumerate}
норма в  $\textbf{Э}^2_{2,0[\textrm{п}]}(Q_T)$ задаётся как в $\textbf{Э}^2_{2}(Q_T)$;

    \item
положим $\hat{\textbf{Э}}{}^{2}_{2,0[\textrm{л}]}(Q_T)\equiv\{z\in {\textbf{Э}}{}^{2}_{2,0[\textrm{л}]}(Q_T):z(\cdot,T)=0\}$,
$\hat{\textbf{Э}}{}^{2}_{2,0[\textrm{п}]}(Q_T)\equiv\{z\in {\textbf{Э}}{}^{2}_{2,0[\textrm{п}]}(Q_T):z(\cdot,T)=0\}$;

\end{enumerate}}



$\blacksquare$ --- знак окончания доказательства.

\part{Сведения из теории функций и функционального анализа}
    \chapter{Функции со значениями в банаховых пространствах}
        \section{Предел, непрерывность и дифференцируемость}
Изложение материала настоящего раздела следует \cite{ZorichTomI, Trenogin}.
            \subsection{Функции одного вещественного переменного}
Пусть $X$ --- банахово пространство с нормой $\|\cdot\|_X$, и пусть $\mathcal{T}\subset\mathbb{R}$ --- некоторое множество, а $t_0\in\mathbb{R}$ --- его предельная точка.

Пусть $f\colon\mathcal{T}\to X$ --- некоторая функция.

\begin{Definition}\label{Definition:limit_one_var_func::Cauchy} (Определение предела по Коши) Говорят, что $a\in X$ --- предел функции $f$ в норме пространства $X$ при $t$, стремящемся к
$t_0$, и пишут $\lim\limits_{t\to t_0}^Xf(t)=a$, если
\begin{gather*}
\lim\limits_{t\to t_0}\|f(t)-a\|_X=0,
\end{gather*}
или, что то же самое,
\begin{gather*}
\forall\,\varepsilon>0\,\,\exists\,\delta=\delta(\varepsilon)>0\,\,\forall\,t\in\mathcal{T},\,\,0<|t-t_0|<\delta:\|f(t)-a\|_X<\varepsilon.
\end{gather*}
Если не возникает недоразумений, то вместо $\lim\limits_{t\to t_0}^Xf(t)=a$ пишем $\lim\limits_{t\to t_0}f(t)=a$.
\end{Definition}


\begin{Definition}\label{Definition:limit_one_var_func::Geine} (Определение предела по Гейне) Говорят, что $a\in X$ --- предел функции $f$ в норме пространства $X$ при $t$, стремящемся к
$t_0$, и пишут $\lim\limits_{t\to t_0}^Xf(t)=a$, если для любой последовательности точек $t_i\in\mathcal{T}$, $i=1,2,\dots$, сходящейся к точке $t_0$, последовательность $f(t_i)$,
$i=1,2,\dots$, сходится в $X$ к точке $a$.
\end{Definition}

Из определения предела вещественной функции одного вещественного переменного вытекает
\begin{Theorem}
Определения \ref{Definition:limit_one_var_func::Cauchy} и \ref{Definition:limit_one_var_func::Geine} эквивалентны.
\end{Theorem}

\begin{Definition}
Пусть $X$, $Y$ и $Z$ --- банаховы пространства с нормами $\|\cdot\|_X$, $\|\cdot\|_Y$ и $\|\cdot\|_Z$ соответственно. Отображение $\Phi\colon X\times Y\to Z$ называется \textbf{умножением}
элементов пространств $X$ и $Y$, принимающим значения в пространстве $Z$, если
\begin{enumerate}
    \item
$\Phi(x_1+x_2,y)=\Phi(x_1,y)+\Phi(x_2,y)$ для всех $x_1$, $x_2\in X$, $y\in Y$;
    \item
$\Phi(\alpha x,y)=\alpha\Phi(x,y)$ для всех $x\in X$, $\alpha\in\mathbb{R}$, $y\in Y$;
    \item
$\Phi(x,y_1+y_2)=\Phi(x,y_1)+\Phi(x,y_2)$ для всех $y_1$, $y_2\in Y$, $x\in X$;
    \item
$\Phi(x,\alpha y)=\alpha\Phi(x,y)$ для всех $x\in X$, $\alpha\in\mathbb{R}$, $y\in Y$;
    \item
найдётся постоянная $K>0$, такая, что для всех $x\in X$, $y\in Y$ выполнено неравенство $\|\Phi(x,y)\|_Z\leqslant K\|x\|_X\|y\|_Y$.
\end{enumerate}
Далее положим $x\bullet y\equiv\Phi(x,y)$.
\end{Definition}

Приведём примеры отображений, удовлетворяющих данному определению.

\begin{Example}
Пусть $X=Y=H$ --- гильбертово пространство со скалярным произведением $\langle\cdot,\cdot\rangle_H$, $Z=\mathbb{R}$, и пусть для всех
$x\in X$, $y\in Y$
$$
\Phi(x,y)\equiv\langle x,y\rangle_H.
$$
Тогда $\Phi$ --- умножение элементов $X$ и $Y$, принимающее значения в пространстве $Z$.
\end{Example}

\begin{Example}
Пусть $X$ --- банахово пространство, $Y=X^*$, $Z=\mathbb{R}$, и пусть для всех $x\in X$, $y\in Y$
$$
\Phi(x,y)\equiv\langle x,y\rangle.
$$
Тогда $\Phi$ --- умножение элементов $X$ и $Y$, принимающее значения в пространстве $Z$.
\end{Example}

\begin{Example}
Пусть $X$, $Z$ --- банаховы пространство, $Y={\mathcal{L}}(X,Z)$, и пусть для всех $x\in X$, $A\in Y$
$$
\Phi(x,A)\equiv Ax.
$$
Тогда $\Phi$ --- умножение элементов $X$ и $Y$, принимающее значения в пространстве $Z$.
\end{Example}

\begin{Example}
Пусть $X={\mathcal{L}}(V_1,V_2)$, $Y={\mathcal{L}}(V_0,V_1)$, $Z={\mathcal{L}}(V_0,V_2)$, где $V_0$, $V_0$ и $V_0$, --- банаховы
пространства, и пусть для всех $A\in X$, $B\in Y$
$$
\Phi(A,B)\equiv AB,
$$
где $(AB)(v)\equiv A(B(v))$ для всех $v\in V_0$. Тогда $\Phi$ --- умножение элементов $X$ и $Y$, принимающее значения в пространстве $Z$.
\end{Example}

Перейдём к свойствам предела функции.

\begin{Theorem}
Если $\lim\limits_{t\to t_0}^Xf(t)=a$, то найдётся функция $\alpha\colon\mathcal{T}\to X$, такая, что $\lim\limits_{t\to t_0}^X\alpha(t)=0_X$ и
$$
f(t)=a+\alpha(t)\,\,\,\forall\,t\in\mathcal{T}.
$$
\end{Theorem}
\begin{Proof}
Достаточно взять $\alpha(t)\equiv f(t)-a$.
\end{Proof}

\begin{Theorem}
Если $a\in X$, то $\lim\limits_{t\to t_0}^Xa=a$.
\end{Theorem}

\begin{Theorem}
Пусть существует предел $\lim\limits_{t\to t_0}^Xf(t)$, равный $a$. Тогда $\lim\limits_{t\to t_0}\|f(t)\|_X=\|a\|_X$.
\end{Theorem}
\begin{Proof}
В самом деле,
\begin{gather*}
|\,\|f(t)\|_X-\|a\|_X|\leqslant\|f(t)-a\|_X\to0,\,\,\,t\to t_0,
\end{gather*}
что и даёт утверждение теоремы.
\end{Proof}

Из данной теоремы и теории пределов вещественных функций одного вещественного переменного вытекает

\begin{Theorem}
Если функция $f\colon\mathcal{T}\to X$ имеет предел в точке $t_0$, то она ограничена в некоторой окрестности этой точки, то есть
найдутся $\delta>0$ и $L>0$, такие, что
\begin{gather*}
\|f(t)\|_X\leqslant L\,\,\,\forall\,t\in\mathcal{T}\cap(t_0-\delta,t_0+\delta).
\end{gather*}
\end{Theorem}

\begin{Theorem} Пусть функции $f\colon\mathcal{T}\to X$,  $g\colon\mathcal{T}\to X$, таковы, что существуют пределы $\lim\limits_{t\to t_0}^Xf(t)$ и $\lim\limits_{t\to t_0}^Xg(t)$. Тогда
для любых вещественных чисел $\alpha$, $\beta$ существует предел функции $\alpha f(t)+\beta g(t)$, $t\in\mathcal{T}$, при $t\to t_0$, причём
\begin{gather*}
\lim\limits_{t\to t_0}^X[\alpha f(t)+\beta g(t)]=\alpha\lim\limits_{t\to t_0}^Xf(t)+\beta\lim\limits_{t\to t_0}^Xg(t).
\end{gather*}
\end{Theorem}
\begin{Proof}
Обозначим предел $\lim\limits_{t\to t_0}^Xf(t)$ через $a$, а предел $\lim\limits_{t\to t_0}^Xg(t)$ через $b$. Тогда
\begin{gather*}
\|[\alpha f(t)+\beta g(t)]-[\alpha a+\beta b]\|_X=\|\alpha[f(t)-a]+\beta[g(t)-b]\|_X\leqslant\\
\leqslant|\alpha|\|f(t)-a\|_X+|\beta|\|g(t)-b\|_X\to0,\,\,\,t\to t_0.
\end{gather*}
Это и означает, что
\begin{gather*}
\lim\limits_{t\to t_0}^X[\alpha f(t)+\beta g(t)]=\alpha\lim\limits_{t\to t_0}^Xf(t)+\beta\lim\limits_{t\to t_0}^Xg(t).
\end{gather*}
Теорема доказана.
\end{Proof}

\begin{Theorem} Пусть функции $f\colon\mathcal{T}\to X$,  $g\colon\mathcal{T}\to Y$, таковы, что существуют пределы $\lim\limits_{t\to t_0}^Xf(t)$ и $\lim\limits_{t\to t_0}^Yg(t)$. Тогда
существует предел функции $f(t)\bullet g(t)$, $t\in\mathcal{T}$, при $t\to t_0$, причём
\begin{gather*}
\lim\limits_{t\to t_0}^Z[f(t)\bullet g(t)]=\left[\lim\limits_{t\to t_0}^Xf(t)\right]\bullet\left[\lim\limits_{t\to t_0}^Yg(t)\right].
\end{gather*}
\end{Theorem}
\begin{Proof}
Обозначим предел $\lim\limits_{t\to t_0}^Xf(t)$ через $a$, а предел $\lim\limits_{t\to t_0}^Yg(t)$ через $b$. Тогда
\begin{gather*}
f(t)\bullet g(t)-a\bullet b=[(f(t)-a)+a]\bullet g(t)-a\bullet b=[f(t)-a]\bullet g(t)+a\bullet g(t)-a\bullet b=[f(t)-a]\bullet g(t)+\\
+a\bullet[g(t)-b]=[f(t)-a]\bullet[(g(t)-b)+b]+a\bullet[g(t)-b]=[f(t)-a]\bullet[g(t)-b]+\\
+[f(t)-a]\bullet b+a\bullet[g(t)-b].
\end{gather*}
Поэтому
\begin{gather*}
\|f(t)\bullet g(t)-a\bullet b\|_Z=\|[f(t)-a]\bullet[g(t)-b]+[f(t)-a]\bullet b+a\bullet[g(t)-b]\|_Z\leqslant\\
\leqslant\|[f(t)-a]\bullet[g(t)-b]\|_Z+\|[f(t)-a]\bullet b\|_Z+\|a\bullet[g(t)-b]\|_Z\leqslant K\|f(t)-a\|_X\|g(t)-b\|_Y+\\
+K\|f(t)-a\|_X\|b\|_Y+K\|a\|_X\|g(t)-b\|_Y\to0,\,\,\,t\to t_0.
\end{gather*}
Последнее означает, что
\begin{gather*}
\lim\limits_{t\to t_0}^Z[f(t)\bullet g(t)]=\left[\lim\limits_{t\to t_0}^Xf(t)\right]\bullet\left[\lim\limits_{t\to t_0}^Yg(t)\right].
\end{gather*}
Теорема доказана.
\end{Proof}

\begin{Theorem}
Если функция $f\colon\mathcal{T}\to X$ имеет равный нулю предел при $t\to t_0$, а функция  $g\colon\mathcal{T}\to Y$ ограничена в некоторой окрестности этой точки, то предел функции
$f(t)\bullet g(t)$, $t\in\mathcal{T}$, при $t\to t_0$ равен нулю.
\end{Theorem}

Из определения \ref{Definition:limit_one_var_func::Cauchy} и определения непрерывности функции, определённой на метрическом пространстве и принимающей значения в другом метрическом
пространстве, следует
\begin{Theorem}
Функция $f\colon\mathcal{T}\to X$ непрерывна в точке $t_0\in\mathcal{T}$ тогда и только тогда, когда
\begin{gather*}
\lim\limits_{t\to t_0}^Xf(t)=f(t_0).
\end{gather*}
\end{Theorem}

Из данной теоремы и приведённых выше свойств предела вытекают следующие свойства непрерывных функций.
\begin{Theorem}
1) Пусть функции $f\colon\mathcal{T}\to X$,  $g\colon\mathcal{T}\to X$, непрерывны в точке $t_0\in\mathcal{T}$. Тогда для любых
вещественных чисел $\alpha$, $\beta$ функция $\alpha f(t)+\beta g(t)$, $t\in\mathcal{T}$, непрерывна в точке $t_0$.

2) Если функция $f\colon\mathcal{T}\to X$ непрерывна в точке $t_0\in\mathcal{T}$, то функция $[0,T]\ni t\mapsto\|f(t)\|_X$ также непрерывна в этой точке.

3) Пусть функции $f\colon\mathcal{T}\to X$,  $g\colon\mathcal{T}\to Y$, непрерывны в точке $t_0\in\mathcal{T}$. Тогда функция  $f(t)\bullet g(t)$, $t\in\mathcal{T}$, тоже непрерывна в
этой точке.
\end{Theorem}

Пусть $\mathcal{T}\subset\mathbb{R}$ --- некоторый промежуток, и задана функция $f\colon\mathcal{T}\to X$.

\begin{Definition}
Элемент $A\in X$ называется \textbf{сильной производной} функции $f$ \textbf{в точке} $t_0\in\mathcal{T}$ \textbf{в норме пространства} $X$, если
\begin{gather*}
\lim\limits_{\Delta t\to 0}^X\frac{f(t_0+\Delta t)-f(t_0)}{\Delta t}=A.
\end{gather*}
Этот элемент $A$ называется \textbf{сильной производной функции} $f$ \textbf{в точке} $t_0$, и обозначается $f'(t_0)$, или $\dot f(t_0)$, или $\displaystyle\frac{df(t_0)}{dt}$.
\end{Definition}

Приведём теперь некоторые свойства производных.
\begin{Theorem}
Если $A\in X$ --- константа, то $(A)'=0_X$.
\end{Theorem}
\begin{Theorem}
Если функции $f\colon\mathcal{T}\to X$ и $g\colon\mathcal{T}\to X$ имеют сильные производные в точке $t_0\in\mathcal{T}$, то для всех вещественных чисел $\alpha$, $\beta$ функция
$\alpha f(t)+\beta g(t)$, $t\in\mathcal{T}$, имеет сильную производную в точке $t_0$, причём
\begin{gather*}
(\alpha f+\beta g)'(t_0)=\alpha f'(t_0)+\beta g'(t_0).
\end{gather*}
\end{Theorem}
\begin{Proof}
В самом деле, пусть $h(t)\equiv \alpha f(t)+\beta g(t)$, $\Delta h\equiv h(t_0+\Delta t)-h(t_0)$. Тогда
\begin{gather*}
\frac{\Delta h}{\Delta t}=\frac{h(t_0+\Delta t)-h(t_0)}{\Delta t}=\frac{[\alpha f(t_0+\Delta t)+\beta g(t_0+\Delta t)]-[\alpha f(t_0)+\beta g(t_0)]}{\Delta t}=\\
=\frac{\alpha[f(t_0+\Delta t)-f(t_0)]+\beta[g(t_0+\Delta t)-g(t_0)]}{\Delta t}=\alpha\frac{f(t_0+\Delta t)-f(t_0)}{\Delta t}+\beta\frac{g(t_0+\Delta t)-g(t_0)}{\Delta t}.
\end{gather*}
Пользуясь затем свойствами предела, получаем утверждение теоремы.
\end{Proof}

\begin{Definition}
Функция $f$ называется \textbf{сильно дифференцируемой в точке} $t_0\in\mathcal{T}$, если для приращения функции $f$ в точке $t_0$ справедливо представление
\begin{gather}\label{strong_diff_1D}
\Delta f\equiv f(t_0+\Delta t)-f(t_0)=A\Delta t+\alpha(\Delta t)\Delta t,
\end{gather}
где $A\in X$ --- некоторая константа, а принимающая значения в пространстве $X$ функция $\alpha$ такова, что $\|\alpha(\Delta t)\|_X\to0$ при $\Delta t\to0$.
\end{Definition}

\begin{Theorem}
Представление вида (\ref{strong_diff_1D}) определяется однозначно.
\end{Theorem}
\begin{Proof}
В самом деле, пусть для приращения $\Delta f$ функции $f$ в точке $t_0\in\mathcal{T}$ справедливы два представления вида (\ref{strong_diff_1D}), т.е. найдутся $A_1$, $A_2\in X$, и функции
$\alpha_1(\Delta t)$, $\alpha_2(\Delta t)$, $\|\alpha_1(\Delta t)\|_X\to0$, $\|\alpha_2(\Delta t)\|_X\to0$, $\Delta t\to0$, такие, что
\begin{gather*}
\Delta f\equiv f(t_0+\Delta t)-f(t_0)=A_1\Delta t+\alpha_1(\Delta t)\Delta t,\,\,\,\Delta f\equiv f(t_0+\Delta t)-f(t_0)=A_2\Delta t+\alpha_2(\Delta t)\Delta t.
\end{gather*}
Вычитая из первого равенства второе, будем иметь
\begin{gather*}
0_X=[A_1-A_2]\Delta t+[\alpha_1(\Delta t)-\alpha_2(\Delta t)]\Delta t.
\end{gather*}
Поделив на $\Delta t$, получим, что
\begin{gather*}
0_X=[A_1-A_2]+[\alpha_1(\Delta t)-\alpha_2(\Delta t)].
\end{gather*}
Переходя затем к пределу при $\Delta t\to0$, заключаем, что $A_1=A_2$. Поэтому и $\alpha_1(\Delta t)\equiv\alpha_2(\Delta t)$. Теорема доказана.
\end{Proof}

\begin{Theorem}
Функция $f\colon\mathcal{T}\to X$ сильно дифференцируема в точке $t_0\in\mathcal{T}$  в том и только том случае, если она имеет в этой точке сильную производную. При этом $A=f'(t_0)$.
\end{Theorem}
\begin{Proof} 1) Пусть $f$ сильно дифференцируема в точке $t_0$. Тогда для $\Delta f\equiv f(t_0+\Delta t)-f(t_0)$ справедливо представление (\ref{strong_diff_1D}). Поделив на $\Delta t$,
выводим, что
\begin{gather*}
\frac{f(t_0+\Delta t)-f(t_0)}{\Delta t}=A+\alpha(\Delta t).
\end{gather*}
Поскольку же $\lim\limits_{\Delta t\to 0}^X\alpha(\Delta t)=0_X$, то существует предел $\lim\limits_{\Delta t\to 0}^X\frac{f(t_0+\Delta t)- f(t_0)}{\Delta t}$ и равен $A$. Это и означает,
что $f$ имеет в точке $t_0$ сильную производную $f'(t_0)$, причём $A=f'(t_0)$.

2) Пусть функция $f$ имеет в точке $t_0$ сильную производную $f'(t_0)$. Тогда найдётся принимающая значения в $X$ функция $\alpha(\Delta t)$, такая, что
$\lim\limits_{\Delta t\to 0}^X\alpha(\Delta t)=0_X$ и
\begin{gather*}
\frac{f(t_0+\Delta t)-f(t_0)}{\Delta t}=f'(t_0)+\alpha(\Delta t).
\end{gather*}
Отсюда следует, что
\begin{gather*}
\Delta f\equiv f(t_0+\Delta t)-f(t_0)=f'(t_0)\Delta t+\alpha(\Delta t)\Delta t.
\end{gather*}
Таким образом, функция $f$ сильно дифференцируема в точке $t_0$ и справедливо равенство $A=f'(t_0)$. Теорема полностью доказана.
\end{Proof}

\begin{Theorem}
Если функция $f\colon\mathcal{T}\to X$ сильно дифференцируема в точке $t_0\in\mathcal{T}$, то она непрерывна в этой точке в норме пространства $X$.
\end{Theorem}
\begin{Proof}
В самом деле, поскольку функция $f$ сильно дифференцируема в точке $t_0$, то
\begin{gather*}
\Delta f\equiv f(t_0+\Delta t)-f(t_0)=f'(t_0)\Delta t+\alpha(\Delta t)\Delta t.
\end{gather*}
Поэтому
\begin{gather*}
\|\Delta f\|_X\equiv\|f(t_0+\Delta t)-f(t_0)\|_X=\|f'(t_0)\Delta t+\alpha(\Delta t)\Delta t\|_X\leqslant|\Delta t|[\|f'(t_0)\|_X+\|\alpha(\Delta t)\|_X]\to0,\,\,\,\Delta t\to0,
\end{gather*}
что и доказывает непрерывность $f$ в точке $t_0$.
\end{Proof}

\begin{Theorem}
Пусть функции $f\colon\mathcal{T}\to X$ и  $g\colon\mathcal{T}\to Y$ сильно дифференцируемы в точке $t_0\in\mathcal{T}$. Тогда функция $f(t)\bullet g(t)$, $t\in\mathcal{T}$, также сильно
дифференцируема в этой точке, причём
\begin{gather*}
(f\bullet g)'(t_0)=f'(t_0)\bullet g(t_0)+f(t_0)\bullet g'(t_0).
\end{gather*}
\end{Theorem}
\begin{Proof}
Действительно, пусть $h(t)\equiv f(t)\bullet g(t)$, $\Delta f\equiv f(t_0+\Delta t)-f(t_0)$, $\Delta g\equiv g(t_0+\Delta t)-g(t_0)$, $\Delta h\equiv h(t_0+\Delta t)-h(t_0)$. Тогда
\begin{gather*}
\frac{\Delta h}{\Delta t}=\frac{h(t_0+\Delta t)-h(t_0)}{\Delta t}=\frac{f(t_0+\Delta t)\bullet g(t_0+\Delta t)-f(t_0)\bullet g(t_0)}{\Delta t}=\\
=\frac{[f(t_0+\Delta t)-f(t_0)]\bullet g(t_0+\Delta t)+f(t_0)\bullet[g(t_0+\Delta t)-g(t_0)]}{\Delta t}=
\frac{\Delta f}{\Delta t}\bullet g(t_0+\Delta t)+f(t_0)\bullet\frac{\Delta g}{\Delta t}.
\end{gather*}
Поскольку $f$ сильно дифференцируема в точке $t_0$, то
\begin{gather*}
\lim\limits_{\Delta t\to0}^X\frac{\Delta f}{\Delta t}=f'(t_0).
\end{gather*}
Так как $g$ сильно дифференцируема в точке $t_0$, то она непрерывна в этой точке, и, кроме того,
\begin{gather*}
\lim\limits_{\Delta t\to0}^X\frac{\Delta g}{\Delta t}=g'(t_0).
\end{gather*}
Поэтому существует предел $\lim\limits_{\Delta t\to0}^X\frac{\Delta h}{\Delta t}$, который равен $f'(t_0)\bullet g(t_0)+f(t_0)\bullet g'(t_0)$. Теорема доказана.
\end{Proof}

\begin{Definition}
Определённую на промежутке $\mathcal{T}\subset\mathbb{R}$ функцию $f\colon\mathcal{T}\to X$ назовём \textbf{сильно непрерывно дифференцируемой в точке} $t_0\in\mathcal{T}$, если она сильно
дифференцируема в каждой точке некоторой окрестности точки $t_0$, и сильная производная $f'(t)$ сильно непрерывна в этой точке в норме пространства $X$. Функция $f$ называется
\textbf{сильно непрерывно дифференцируемой на промежутке} $\mathcal{T}$, если она сильно дифференцируема в каждой точке этого промежутка и производная $f'(t)$, $t\in\mathcal{T}$, сильно
непрерывна на $\mathcal{T}$ в норме пространства $X$. Множество всех сильно непрерывно дифференцируемых на промежутке $\mathcal{T}$ функций обозначается $C^1(\mathcal{T},X)$.
\end{Definition}

\begin{Definition}
Пусть функция $f\colon\mathcal{T}\to X$ определена и сильно дифференцируема на промежутке $\mathcal{T}\subset\mathbb{R}$. Тогда функция $f'(t)$, $t\in\mathcal{T}$, также определена на этом
промежутке. Если функция  $f'(t)$, $t\in\mathcal{T}$, имеет сильную производную в точке $t_0\in\mathcal{T}$, то эту производную называют сильной второй производной функции $f$ в точке
$t_0$ и обозначают $f''(t_0)$, $\ddot{f}(t_0)$, $f^{(2)}(t_0)$ или $\displaystyle\frac{d^2f(t_0)}{dt^2}$. При этом говорят, что функция $f$ \textbf{дважды сильно дифференцируема в точке}
$t_0$.

Если функция $f$ дважды сильно дифференцируема в каждой точке промежутка $\mathcal{T}$, а функция $f''(t)$, $t\in\mathcal{T}$, сильно непрерывна на $\mathcal{T}$ в норме пространства $X$,
то говорят, что функция $f$ \textbf{дважды сильно непрерывно дифференцируема на промежутке} $\mathcal{T}$. Множество всех дважды сильно непрерывно дифференцируемых на промежутке
$\mathcal{T}$ функций обозначается  $C^2(\mathcal{T},X)$.
\end{Definition}

Аналогично определяются производные более высоких порядков и множества $C^m(\mathcal{T},X)$ при $m\geqslant2$.

\begin{Lemma}\label{Cs[0,T]X::finiteness.of.sup.of.normX}
Если $f\in C_s([0,T],X)$, то $\sup\limits_{t\in[0,T]}\|f(t)\|_X<+\infty$.
\end{Lemma}
\begin{Proof}
Поскольку $f\in C_s([0,T],X)$, то при всех $x^*\in X^*$ функция
\begin{gather*}
[0,T]\ni t\mapsto \langle f(t),x^*\rangle
\end{gather*}
непрерывна на $[0,T]$. Поэтому найдётся зависящая от $x^*\in X^*$ постоянная $C=C(x^*)>0$, такая, что
\begin{gather*}
|\langle f(t),x^*\rangle|\leqslant C(x^*)\,\,\,\forall\,t\in[0,T].
\end{gather*}
Следовательно, в силу вложения $X\subset X^{**}$ и теоремы о резонансе \cite[следствие 1 на стр.104]{iosida}, найдётся постоянная $C_1>0$, такая, что
\begin{gather*}
\|f(t)\|_{X^{**}}\leqslant C_1\,\,\,\forall\,t\in[0,T],
\end{gather*}
откуда, в силу изометричности вложения $X\subset X^{**}$, следует, что
\begin{gather*}
\sup\limits_{t\in[0,T]}\|f(t)\|_X\leqslant C_1.
\end{gather*}
Лемма доказана.
\end{Proof}

Предположим теперь, что пространство $X$ рефлексивно и наделим пространство $C_s([0,T],X)$ нормой
\begin{gather*}
\|f\|_{C_s([0,T],X)}\equiv\sup\limits_{t\in[0,T]}\|f(t)\|_X\,\,\,\forall\,f\in C_s([0,T],X).
\end{gather*}
которую в даленейшем будем называть \textbf{сильной нормой} пространства $C_s([0,T],X)$, сооответствующую топологию будем называть $X$--топологией пространства $C_s([0,T],X)$, а сходящиеся
в этой норме последовательности --- $X$--сходящимися.

\begin{Lemma}\label{completness_of_Cs([0,T],X)_with_strong_norm}
Пространство $C_s([0,T],X)$, наделённое нормой $\|\cdot\|_{C_s([0,T],X)}$, является банаховым пространством.
\end{Lemma}
\begin{Proof}
Пусть последовательность $f_i\in C_s([0,T],X)$, $i=1,2,\dots$, фундаментальна в норме $\|\cdot\|_{C_s([0,T],X)}$, то есть
\begin{gather*}
\forall\,\varepsilon>0\,\,\exists\,i_0=i_0(\varepsilon)\geqslant1\,\,\forall\,i,\,\,j\geqslant i_0(\varepsilon):\|f_i-f_j\|_{C_s([0,T],X)}\leqslant\varepsilon.
\end{gather*}
С другой стороны, для всех $i,\,j=1,2,\dots$, $x^*\in X^*$ и всех $t\in[0,T]$
\begin{gather*}
|\langle f_i(t)-f_j(t),x^*\rangle|\leqslant\|f_i(t)-f_j(t)\|_X\|x^*\|_{X^*}.
\end{gather*}
Поэтому
\begin{gather}\label{metka_Cs2}
\forall\,\varepsilon>0\,\,\exists\,i_0=i_0(\varepsilon)\geqslant1\,\,\forall\,i,\,\,j\geqslant i_0(\varepsilon)\,\,\forall\,x^*\in X^*\,\,\forall\,t\in[0,T]:
|\langle f_i(t)-f_j(t),x^*\rangle|\leqslant\varepsilon\|x^*\|_{X^*}.
\end{gather}
Это означает, что при каждом $x^*\in X^*$ последовательность функций
\begin{gather*}
[0,T]\ni t\mapsto \langle f_i(t),x^*\rangle,\,\,\,i=1,2,\dots,
\end{gather*}
фундаментальна в норме банахова пространства $C[0,T]$, а, значит, равномерно сходится к некоторой непрерывной на $[0,T]$ при каждом фиксированном $x^*\in X^*$  функции
$F(t,x^*)\in\mathbb{R}$, $t\in[0,T]$. В частности,
\begin{gather*}
\langle f_i(t),x^*\rangle\to F(t,x^*),\,\,\,i\to\infty,\,\,\,\forall\,t\in[0,T].
\end{gather*}
Выберем теперь $t\in[0,T]$ и зафиксируем. Поскольку элемент $f_i(t)\in X$ можно рассматривать как элемент пространства $X^{**}$, при всех $i=1,2,\dots$, то мы имеем поточечную сходимость
линейных непрерывных функционалов $f_i(t)$, $i=1,2,\dots$ над пространством $X^*$. Как известно, это означает, что $F(t,\cdot)$ также является линейным непрерывным функционалом над
пространством $X^*$, т.е. найдётся функция $f\colon[0,T]\to X^{**}$, такая, что $F(t,x^*)\equiv\langle f(t),x^*\rangle$. Поскольку пространство $X$ рефлексивно, то $X^{**}$ можно
отождествить с $X$, и, как следствие, рассматривать функцию $f$ как функцию со значениями в $X$.

Таким образом, мы доказали, что найдётся функция $f\in C_s([0,T],X)$, такая, что
\begin{gather*}
\lim\limits_{j\to\infty}\max\limits_{t\in[0,T]}|\langle f_j(t)-f(t),x^*\rangle|=0\,\,\,\forall\,x^*\in X^*.
\end{gather*}
Устремляя затем в (\ref{metka_Cs2}) $j$ к бесконечности, получим, что
\begin{gather*}
\forall\,\varepsilon>0\,\,\exists\,i_0=i_0(\varepsilon)\geqslant1\,\,\forall\,i\geqslant i_0(\varepsilon)\,\,
\forall\,x^*\in X^*\,\,\forall\,t\in[0,T]:|\langle f_i(t)-f(t),x^*\rangle|\leqslant\varepsilon\|x^*\|_{X^*}.
\end{gather*}
Переходя здесь к точной верхней грани по $x^*\in X^*$, у которых $\|x^*\|_{X^*}\leqslant1$, в силу изометричного вложения
$X\subset X^{**}$ будем иметь
\begin{gather*}
\forall\,\varepsilon>0\,\,\exists\,i_0=i_0(\varepsilon)\geqslant1\,\,\forall\,i\geqslant i_0(\varepsilon) \,\,\forall\,t\in[0,T]:\|f_i(t)-f(t)\|_X\leqslant\varepsilon,
\end{gather*}
откуда следует, что
\begin{gather*}
\forall\,\varepsilon>0\,\,\exists\,i_0=i_0(\varepsilon)\geqslant1\,\,\forall\,i\geqslant i_0(\varepsilon):\|f_i-f\|_{C_s([0,T],X)}\leqslant\varepsilon,
\end{gather*}
что и означает сходимость последовательности $f_i$, $i=1,2,\dots$, к $f$ в норме $\|\cdot\|_{C_s([0,T],X)}$. Лемма полностью доказана.
\end{Proof}

Пусть $Y$ --- банахово пространство с нормой $\|\cdot\|_Y$, $X\subset Y$. Пусть вложение $X\subset Y$ непрерывно и компактно. Иными словами, найдётся постоянная $\nu>0$, такая, что
\begin{gather*}
\|x\|_Y\leqslant\nu\|x\|_X\,\,\,\forall\,x\in X,
\end{gather*}
и любое ограниченное в норме пространства $X$ множество предкомпактно в пространстве $Y$.
\begin{Lemma}\label{Cs([0,T],X).is.subsetC([0,T],Y)} Справедливо вложение $C_s([0,T],X)\subset C([0,T],Y)$, причём
\begin{gather*}
\max\limits_{t\in[0,T]}\|f(t)\|_Y\leqslant\nu\sup\limits_{t\in[0,T]}\|f(t)\|_X\,\,\,\forall\,f\in C_s([0,T],X).
\end{gather*}
\end{Lemma}
\begin{Proof}
В самом деле, пусть $f\in C_s([0,T],X)$ --- произвольна. Тогда для любого $t\in[0,T]$ и любой последовательности $t_i\in[0,T]$, $i=1,2,\dots$, $t_i\to t$, $i\to\infty$, справедливо
предельное соотношение $f(t_i)\to f(t)$, $i\to\infty$, слабо в $X$. Поскольку же вложение $X\subset Y$ компактно, то $f(t_i)\to f(t)$, $i\to\infty$, сильно в $Y$. Таким образом,
$f\in C([0,T],Y)$. Далее,
\begin{gather*}
\|f(t)\|_Y\leqslant\nu\|f(t)\|_X\leqslant\nu\sup\limits_{\xi\in[0,T]}\|f(\xi)\|_X\,\,\,\forall\,t\in[0,T],
\end{gather*}
откуда и следует требуемая оценка. Лемма полностью доказана.
\end{Proof}

Далее нам потребуется ещё одна топология в $C_s([0,T],X)$. Введём эту топологию. А именно, следуя \cite[теорема 1.2 на стр.149]{Gaevskij:Greger:Zaharias}, введём на $C_s([0,T],X)$
полунормы
\begin{gather*}
\mathbf{p}_{x^*}(f)\equiv\max\limits_{t\in[0,T]}|\langle x^*,f(t)\rangle|,\,\,\,x^*\in X^*,
\end{gather*}
задав затем в качестве системы окрестностей нуля семейство множеств
\begin{gather*}
\{f\in C_s([0,T],X):\mathbf{p}_{x^*_j}(f)<\varepsilon_j,\,\,\,j=\overline{1,m}\},\,\,\,x^*_j\in X^*,\,\,\,\varepsilon_j>0,\,\,\,j=\overline{1,m},\,\,\,m\geqslant1.
\end{gather*}
Эту топологию в дальнейшем будем называть $X^*$--топологией пространства $C_s([0,T],X)$, а сходящиеся последовательности в этой топологии будем называть $X^*$--сходящимися.

\begin{Lemma}\label{boundness.of.X-convergence.sequence}
Пусть $f_k$, $f\in C_s([0,T],X)$, $k=1,2,\dots$, и
\begin{gather}\label{fk.is.X-convergent.to.f}
f_k\to f,\,\,\,k\to\infty,\,\,\,\text{в $X^*$--топологии пространства $C_s([0,T],X)$}.
\end{gather}
Тогда
$$
\sup\limits_{k\geqslant1}\sup\limits_{t\in[0,T]}\|f_k(t)\|_X<+\infty.
$$
\end{Lemma}
\begin{Proof}
В силу (\ref{fk.is.X-convergent.to.f}) для каждого фиксированного $x^*\in X^*$
\begin{gather*}
\lim\limits_{k\to\infty}\max\limits_{t\in[0,T]}|\langle x^*,f_k(t)\rangle-\langle x^*,f(t)\rangle|=0.
\end{gather*}
Поэтому найдётся постоянная $C=C(x^*)>0$, зависящая лишь от $x^*\in X^*$, такая, что
\begin{gather*}
\max\limits_{t\in[0,T]}|\langle x^*,f_k(t)\rangle|\leqslant C(x^*)\,\,\,\forall\,k=1,2,\dots
\end{gather*}
Следовательно, в силу вложения $X\subset X^{**}$ и теоремы о резонансе \cite[следствие 1 на стр.104]{iosida}, найдётся постоянная $C_1>0$, такая, что
\begin{gather*}
\|f_k(t)\|_{X^{**}}\leqslant C_1\,\,\,\forall\,t\in[0,T],\,\,\,k=1,2,\dots,
\end{gather*}
откуда, в силу изометричности вложения $X\subset X^{**}$, следует утверждение леммы. Лемма доказана.
\end{Proof}

\begin{Lemma}\label{X*convergence.criteria[in.Cs([0,T],X)]}
Для сходимости последовательности $f_k\in C_s([0,T],X)$, $k=1,2,\dots$, к элементу $f\in C_s([0,T],X)$ в $X^*$--топологии пространства $C_s([0,T],X)$ необходимо и достаточно одновременное
выполнение следующих условий:

1) для некоторого числа $C>0$ имеет место неравенство
\begin{gather}\label{X*convergence.criteria[in.Cs([0,T],X)]!m1}
\sup\limits_{k\geqslant1}\sup\limits_{t\in[0,T]}\|f_k(t)\|_X\leqslant C;
\end{gather}

2) для некоторого всюду плотного (в смысле нормы) в $X^*$ множества $Y$ справедливо соотношение
\begin{gather}\label{X*convergence.criteria[in.Cs([0,T],X)]!m2}
\lim\limits_{k\to\infty}\max\limits_{t\in[0,T]}|\langle x^*,f_k(t)\rangle-\langle x^*,f(t)\rangle|=0\,\,\,\forall\,x^*\in Y.
\end{gather}
\end{Lemma}
\begin{Proof}
Необходимость следует из определения $X^*$--сходимости в пространстве $C_s([0,T],X)$ и из леммы \ref{Cs[0,T]X::finiteness.of.sup.of.normX}. Поэтому докажем лишь достаточность.

В самом деле, пусть $y\in X^*$ --- произвольно. Выберем затем произвольно $\varepsilon>0$ и зафиксируем. Наконец, положим $P=C+\sup\limits_{t\in[0,T]}\|f(t)\|_X$. Так как $Y$ всюду плотно в
$X^*$, то найдётся элемент $y_\varepsilon\in Y$, такой, что $\|y_\varepsilon-y\|_{X^*}\leqslant\frac\varepsilon{2P}$. Поэтому для всех $t\in[0,T]$ и всех номеров $k\geqslant1$
\begin{gather*}
|\langle y,f_k(t)\rangle-\langle y,f(t)\rangle|=|\langle y-y_\varepsilon,f_k(t)\rangle+\langle y_\varepsilon,f_k(t)-f(t)\rangle+
\langle y_\varepsilon-y,f(t)\rangle|\leqslant\frac\varepsilon{2P}\|f_k(t)\|_X+\\
+|\langle y_\varepsilon,f_k(t)-f(t)\rangle|+\frac\varepsilon{2P}\|f(t)\|_X\leqslant\frac\varepsilon{2P}C+
|\langle y_\varepsilon,f_k(t)-f(t)\rangle|+\frac\varepsilon{2P}\sup\limits_{\xi\in[0,T]}\|f(\xi)\|_X=\frac\varepsilon2+\\
+|\langle y_\varepsilon,f_k(t)-f(t)\rangle|.
\end{gather*}

Таким образом,
\begin{gather*}
\max\limits_{t\in[0,T]}|\langle y,f_k(t)\rangle-\langle y,f(t)\rangle|\leqslant\frac\varepsilon2+\max\limits_{t\in[0,T]}|\langle y_\varepsilon,f_k(t)-f(t)\rangle|.
\end{gather*}

В силу (\ref{X*convergence.criteria[in.Cs([0,T],X)]!m2}) найдётся номер $k_0=k_0(\varepsilon)\geqslant1$, такой, что при всех $k\geqslant
k_0(\varepsilon)$
\begin{gather*}
\max\limits_{t\in[0,T]}|\langle y_\varepsilon,f_k(t)-f(t)\rangle|\leqslant\frac\varepsilon2.
\end{gather*}
Это означает, что для любого $\varepsilon>0$ найдётся номер $k_0=k_0(\varepsilon)\geqslant1$, такой, что при всех $k\geqslant k_0(\varepsilon)$
\begin{gather*}
\max\limits_{t\in[0,T]}|\langle y,f_k(t)\rangle-\langle y,f(t)\rangle|\leqslant\varepsilon.
\end{gather*}
В силу произвольности выбора $y\in X^*$ отсюда вытекает, что
\begin{gather*}
f_k\to f,\,\,\,k\to\infty,\,\,\,\text{в $X^*$--топологии пространства $C_s([0,T],X)$}.
\end{gather*}
Лемма полностью доказана.
\end{Proof}

\begin{Lemma}\label{Cs([0,T],X).podposledovatelnost'}
Пусть $X$ --- рефлексивно, $X^*$ --- сепарабельно, $\Delta=\{x_1^*,\dots,x^*_j,\dots\}$ --- счётное всюду плотное в $X^*$ подмножество, и пусть $f_k\in C_s([0,T],X)$, $k=1,2,\dots$, ---
такая последовательность, что

1) для некоторого положительного числа $C>0$
\begin{gather}\label{Cs([0,T],X).podposledovatelnost'!m1}
\sup\limits_{k\geqslant1}\sup\limits_{t\in[0,T]}\|f_k(t)\|_X\leqslant C;
\end{gather}

2) при любом фиксированном $x^*\in X^*$ семейство функций
\begin{gather}\label{Cs([0,T],X).podposledovatelnost'!m2}
[0,T]\ni t\mapsto\langle x^*,f_k(t)\rangle,\,\,\,k=1,2,\dots,
\end{gather}
равностепенно непрерывно.

Тогда найдутся подпоследовательность $f_{k_m}$, $m=1,2,\dots$, последовательности  $f_k$, $k=1,2,\dots$, и элемент $f\in C_s([0,T],X)$, такие, что
\begin{gather}\label{Cs([0,T],X).podposledovatelnost'!m3}
\sup\limits_{k\geqslant1}\sup\limits_{t\in[0,T]}\|f(t)\|_X\leqslant C;
\end{gather}
и
\begin{gather}\label{Cs([0,T],X).podposledovatelnost'!m4}
f_{k_m}\to f,\,\,\,m\to\infty,\,\,\,\text{в $X^*$--топологии пространства $C_s([0,T],X)$}.
\end{gather}
\end{Lemma}
\begin{Proof}
Положим
\begin{gather*}
\varphi_k(t,x^*)\equiv\langle x^*,f_k(t)\rangle,\,\,\,(t,x^*)\in[0,T]\times X^*,\,\,\,k=1,2,\dots
\end{gather*}
В силу условий леммы и теоремы Арцела--Асколи \cite[стр.125]{iosida} найдутся функция $F_1\in C[0,T]$ и подпоследовательность $k_{r,1}$, $r=1,2,\dots$, последовательности $k=1,2,\dots$,
такие, что
\begin{gather*}
\lim\limits_{r\to\infty}\max\limits_{t\in[0,T]}|\varphi_{k_{r,1}}(t,x^*_1)-F_1(t)|=0.
\end{gather*}
Далее, из условий леммы и теоремы Арцела--Асколи вытекает существование таких функции $F_2\in C[0,T]$ и подпоследовательности $k_{r,2}$, $r=1,2,\dots$, последовательности $k_{r,1}$,
$r=1,2,\dots$, что
\begin{gather*}
\lim\limits_{r\to\infty}\max\limits_{t\in[0,T]}|\varphi_{k_{r,2}}(t,x^*_2)-F_2(t)|=0.
\end{gather*}
Продолжая рассуждения, получаем семейство последовательностей $\{\varphi_{k_{r,p}}\}_{r=1}^\infty$, $p=1,2,\dots$, и последовательность функций $F_p\in C[0,T]$, $p=1,2,\dots$, такие, что
\begin{gather*}
\{\varphi_{k_{r,p+1}}\}_{r=1}^\infty\subset\{\varphi_{k_{r,p}}\}_{r=1}^\infty,\,\,\,p=1,2,\dots;\,\,\,
\lim\limits_{r\to\infty}\max\limits_{t\in[0,T]}|\varphi_{k_{r,p}}(t,x^*_p)-F_p(t)|=0,\,\,\,p=1,2,\dots
\end{gather*}
Определив затем функцию $G:[0,T]\times\Delta\to\mathbb{R}$ равенством
\begin{gather*}
G(t,x^*_p)=F_p(t),\,\,\,p=1,2,\dots,
\end{gather*}
выводим, что
\begin{gather*}
\lim\limits_{r\to\infty}\max\limits_{t\in[0,T]}|\varphi_{k_{r,p}}(t,x^*_p)-G(t,x^*_p)|=0,\,\,\,p=1,2,\dots
\end{gather*}
Положив $k_m\equiv k_{m,m}$, $m=1,2,\dots$, будем иметь
\begin{gather}\label{Cs([0,T],X).podposledovatelnost'!m5}
\lim\limits_{m\to\infty}\max\limits_{t\in[0,T]}|\varphi_{k_m}(t,x^*)-G(t,x^*)|=0,\,\,\,\forall\,x^*\in\Delta.
\end{gather}
Для любых $\lambda_j\in\mathbb{R}$, $x^*_j\in\Delta$, $j=\overline{1,l}$, $l\geqslant1$, положим
\begin{gather*}
G\left(t,\sum\limits_{j=1}^l\lambda_jx^*_j\right)=\sum\limits_{j=1}^l\lambda_jG(t,x^*_j).
\end{gather*}
Тогда из (\ref{Cs([0,T],X).podposledovatelnost'!m5}) вытекает, что
\begin{gather}\label{Cs([0,T],X).podposledovatelnost'!m6}
\lim\limits_{m\to\infty}\max\limits_{t\in[0,T]}|\varphi_{k_m}(t,x^*)-G(t,x^*)|=0\,\,\,\forall\,x^*\in\lin\Delta,
\end{gather}
где через $\lin\Delta$ обозначено множество всевозможных конечных линейных комбинаций элементов множества $\Delta$. При этом, поскольку
$\Delta$ --- всюду плотно в $X^*$, то $\lin\Delta$ --- тоже всюду плотно в $X^*$.

Таким образом, при каждом фиксированном $t\in[0,T]$ функционал $G(t,\cdot)$ линеен на $\lin\Delta$. Далее, при всех $t\in[0,T]$,
$x^*\in\lin\Delta$
\begin{gather*}
|\varphi_{k_m}(t,x^*)|=|\langle x^*,f_{k_m}(t)\rangle|\leqslant\|x^*\|_{X^*}\|f_{k_m}(t)\|_X\leqslant C\|x^*\|_{X^*}.
\end{gather*}
Переходя здесь, с учётом (\ref{Cs([0,T],X).podposledovatelnost'!m6}), к пределу при $m\to\infty$, получаем, что
\begin{gather*}
|G(t,x^*)|\leqslant\|x^*\|_{X^*}\|f_{k_m}(t)\|_X\leqslant C\|x^*\|_{X^*}.
\end{gather*}
Итак, при каждом фиксированном $t\in[0,T]$ функционал $G(t,\cdot)$ является линейным непрерывным функционалом на $\lin\Delta$, наделённом
той же нормой, что и пространство $X^*$. Поэтому в силу теоремы Хана--Банаха функционал $G(t,\cdot)$ можно продолжить до линейного
непрерывного функционала $\tilde G(t,\cdot)$, определённого на всём $X^*$.

Так как $\tilde G(t,\cdot)\in X^{**}$, а $X$ --- рефлексивно, то найдётся $f(t)\in X$, такое, что
$$
G(t,x^*)=\langle x^*,f(t)\rangle.
$$
Следовательно, соотношение (\ref{Cs([0,T],X).podposledovatelnost'!m6}) можно переписать в виде
\begin{gather}\label{Cs([0,T],X).podposledovatelnost'!m7}
\lim\limits_{m\to\infty}\max\limits_{t\in[0,T]}|\langle x^*,f_{k_m}(t)-f(t)\rangle|=0\,\,\,\forall\,x^*\in\lin\Delta.
\end{gather}
Отсюда следует, что при всех $x^*\in\lin\Delta$ функция
\begin{gather}\label{Cs([0,T],X).podposledovatelnost'!m8}
[0,T]\ni t\mapsto\langle x^*,f(t)\rangle
\end{gather}
непрерывна на отрезке $[0,T]$, как равномерный предел неперерывных на этом отрезке числовых функций. Покажем, что функция (\ref{Cs([0,T],X).podposledovatelnost'!m8}) непрерывна на отрезке
$[0,T]$ для всех $x^*\in X^*$. В самом деле, пусть $x^*\in X^*$, --- произвольно. Пусть $t_0\in[0,T]$ --- некоторая точка. Выберем произвольно $\varepsilon>0$ и зафиксируем.  Наконец,
положим $P\equiv\sup\limits_{\xi\in[0,T]}\|f(\xi)\|_X$. Тогда найдётся $y_\varepsilon\in X^*$, такое, что $\|y_\varepsilon-x^*\|_{X^*}\leqslant\frac\varepsilon{4P+2}$. Тогда для всех
$t\in[0,T]$
\begin{gather*}
|\langle x^*,f(t)\rangle-\langle x^*,f(t_0)\rangle|\leqslant|\langle x^*-y_\varepsilon,f(t)-f(t_0)\rangle|+|\langle y_\varepsilon,
f(t)-f(t_0)\rangle|\leqslant\frac\varepsilon{4P+2}2P+|\langle y_\varepsilon,f(t)-f(t_0)\rangle|=\\
=\frac\varepsilon2+|\langle y_\varepsilon,f(t)-f(t_0)\rangle|.
\end{gather*}
Далее, в силу доказанной выше непрерывности функции (\ref{Cs([0,T],X).podposledovatelnost'!m8}) при всех $x^*\in\lin\Delta$, найдётся $\delta=\delta(\varepsilon)>0$, такое, что при всех
$t\in[0,T]$, $|t-t_0|<\delta$ выполнено неравенство
\begin{gather*}
|\langle y_\varepsilon,f(t)-f(t_0)\rangle|\leqslant\frac\varepsilon2.
\end{gather*}
Таким образом, для любого $\varepsilon>0$ найдётся $\delta=\delta(\varepsilon)>0$, такое, что при всех $t\in[0,T]$, $|t-t_0|<\delta$, выполнено неравенство
\begin{gather*}
|\langle x^*,f(t)\rangle-\langle x^*,f(t_0)\rangle|\leqslant\varepsilon
\end{gather*}
Ввиду произвольности выбора $x^*\in X^*$ это означает, что функция (\ref{Cs([0,T],X).podposledovatelnost'!m8}) непрерывна на отрезке $[0,T]$ для всех $x^*\in X^*$, или, что то же самое,
$f\in C_s([0,T],X)$. Отсюда, из предельного соотношения (\ref{Cs([0,T],X).podposledovatelnost'!m7}), неравенства (\ref{Cs([0,T],X).podposledovatelnost'!m1}) и леммы
\ref{X*convergence.criteria[in.Cs([0,T],X)]}, следует предельное соотношение (\ref{Cs([0,T],X).podposledovatelnost'!m4}).

Докажем теперь неравенство (\ref{Cs([0,T],X).podposledovatelnost'!m5}). В самом деле,
\begin{gather*}
|\langle x^*,f_{k_m}(t)\rangle|\leqslant\|x^*\|_{X^*}\|f_{k_m}(t)\|_X\leqslant C\|x^*\|_{X^*}\,\,\,\forall\,x^*\in X^*,\,\,\,t\in[0,T].
\end{gather*}
Переходя здесь к пределу при $m\to\infty$, будем иметь
\begin{gather*}
|\langle x^*,f(t)\rangle|\leqslant C\|x^*\|_{X^*}\,\,\,\forall\,x^*\in X^*,\,\,\,t\in[0,T].
\end{gather*}
Отсюда, ввиду вложения $X\subset X^{**}$, получаем, что
\begin{gather*}
\|f(t)\|_{X^{**}}\leqslant C\,\,\,\forall\,t\in[0,T].
\end{gather*}
Ввиду изометричности вложения $X\subset X^{**}$ отсюда следует неравенство (\ref{Cs([0,T],X).podposledovatelnost'!m5}). Лемма полностью доказана.
\end{Proof}

\begin{Lemma}\label{Weierstrass:approx}\cite[стр.150, теорема 1.3]{Gaevskij:Greger:Zaharias}
Пусть $X$ --- банахово пространство. Тогда множество всевозможных многочленов с коэффициентами из $X$, т.е.
множество всевозможных функций вида $\xi(t)=\sum\limits_{k=0}^ma_kt^k$, $t\in \mathbb{R}$, $a_j\in X$, $j=\overline{0,m}$, $m=0,1,2,\dots$,
всюду плотно в $C([0,T],X)$.
\end{Lemma}

            \subsection{Функции нескольких вещественных переменных}
Пусть $Y$ --- банахово пространство с нормой $\|\cdot\|_Y$, $G\subset\mathbb{R}^m$ --- некоторое множество.

\begin{Definition}
Говорят, что функция $f\colon G\to Y$ имеет в норме пространства $Y$ предел, равный $y$, при $g\to g_0$, где $g_0$ --- предельная точка множества $G$, и пишут
$\lim\limits^{Y}_{g\to g_0}f(g)=y$, если
\begin{gather*}
\lim\limits_{g\to g_0}\|f(g)-y\|_Y=0,
\end{gather*}
иными словами,
\begin{gather*}
\forall\,\varepsilon>0\,\,\exists\,\delta=\delta(\varepsilon)>0\,\,\forall\,g\in G,\,\,|g-g_0|<\delta: \|f(g)-y\|_Y<\varepsilon.
\end{gather*}
\end{Definition}

\begin{Definition}
Говорят, что функция $f\colon G\to Y$ непрерывна в точке $g_0\in G$ в норме пространства $Y$, если $\lim\limits^{Y}_{g\to g_0}f(g)=f(g_0)$. Говорят, что функция $f\colon G\to Y$ непрерывна
на $G$ в норме пространства $Y$, если она непрерывна в норме пространства $Y$ в каждой точке множества $G$. Множество всех непрерывных на $G$ функций $f\colon G\to Y$ обозначают $C(G,Y)$.
\end{Definition}

\begin{Definition}
Говорят, что функция $f\colon G\to Y$ равномерно непрерывна на множестве $G$, если
\begin{gather*}
\forall\,\varepsilon>0\,\,\exists\,\delta=\delta(\varepsilon)>0\,\,\forall\,g',\,\,g''\in G,\,\,|g'-g''|<\delta: \|f(g')-f(g'')\|_Y<\varepsilon.
\end{gather*}
\end{Definition}

\begin{Theorem}\label{uniform_continuity}
Если функция $f\colon G\to Y$ непрерывна на $G$ в норме пространства $Y$ и $G$ --- компакт, то функция $f$ равномерно непрерывна на этом компакте.
\end{Theorem}
\begin{Proof}
Пусть это не так, и найдётся $\varepsilon_0>0$, такое, что для любого $\delta>0$ найдутся точки $g'_\delta$, $g''_\delta\in G$, $|g'_\delta-g''_\delta|<\delta$, для которых
$\|f(g'_\delta)-f(g''_\delta)\|_Y\geqslant\varepsilon_0$. Пусть $\delta_j>0$, $j=1,2,\dots$, $\delta_j\to0$, $j\to\infty$, --- некоторая последовательность чисел. Тогда
\begin{gather}\label{mg'deltajg''deltaj}
|g'_{\delta_j}-g''_{\delta_j}|<\delta_j,\,\,\,\|f(g'_{\delta_j})-f(g''_{\delta_j})\|_Y\geqslant\varepsilon_0,\,\,\,j=1,2,\dots
\end{gather}
Поскольку $G\subset\mathbb{R}^m$ --- компакт, то найдутся подпоследовательности $g'_{\delta_{j_i}}$, $g''_{\delta_{j_i}}$, $i=1,2\dots$, последовательностей $g'_{\delta_{j}}$ и
$g''_{\delta_{j}}$, $j=1,2\dots$, соответственно и элементы $g^*$, $g^{**}\in G$, такие, что
\begin{gather*}
g'_{\delta_{j_i}}\to g^*,\,\,\,g''_{\delta_{j_i}}\to g^{**},\,\,\,i\to\infty,
\end{gather*}
откуда, в силу первого из неравенств (\ref{mg'deltajg''deltaj}), извлекаем, что $g^*=g^{**}$. Полагая во втором из соотношений (\ref{mg'deltajg''deltaj}) $j=j_i$ и переходя затем к пределу
при $i\to\infty$, получим, что $0\geqslant\varepsilon_0>0$. Полученное противоречие доказывает теорему.
\end{Proof}

Пусть далее $G$ --- компакт, $X$ --- нормированное пространство с нормой $\|\cdot\|_X$.
\begin{Lemma}\label{boundness_Pi(g)_belonging_L(X,Y)}
Пусть $\Pi(g)\in\mathcal{L}(X,Y)$ при всех $g\in G$, и пусть при всех $x\in X$ функция $G\ni g\mapsto\Pi(g)x$ принадлежит пространству $C(G,Y)$. Тогда
$$
\sup\limits_{g\in G}\|\Pi(g)\|_{X\to Y}<+\infty.
$$
\end{Lemma}
\begin{Proof}
Поскольку при каждом фиксированном $x\in X$ функция $G\ni g\mapsto\Pi(g)x$ --- элемент пространства $C(G,Y)$, то найдётся зависящая от $x\in X$ постоянная $K=K(x)>0$, такая, что

$$
\sup\limits_{g\in G}\|\Pi(g)x\|_{Y}\leqslant K(x).
$$
Пользуясь теперь теоремой о резонансе, получаем утверждение настоящей леммы.
\end{Proof}

Пусть далее $X$ --- банахово пространство.


\begin{Lemma}\label{continuity_Pi(t,xi)z(xi)}
Пусть $\Pi(t,\xi)\in\mathcal{L}(X,Y)$ при всех $(t,\xi)\in\Gamma$, при всех $x\in X$ функция $\Gamma\ni(t,\xi)\mapsto \Pi(t,\xi)x$ принадлежит $C(\Gamma,Y)$. Если $z\in C([0,T],X)$, то
функция $\Gamma\ni(t,\xi)\mapsto\Pi(t,\xi)z(\xi)$ является элементом $C(\Gamma,Y)$.
\end{Lemma}
\begin{Proof}
В самом деле, пусть $(t,\xi)$, $(t+\Delta t,\xi+\Delta\xi)\in\Gamma$ --- произвольны, и пусть $z\in C([0,T],X)$. Тогда, на основании леммы \ref{boundness_Pi(g)_belonging_L(X,Y)},
\begin{gather*}
\|\Pi(t+\Delta t,\xi+\Delta\xi)z(\xi+\Delta\xi)-\Pi(t,\xi)z(\xi)\|_Y\leqslant\|\Pi(t+\Delta t,\xi+\Delta\xi)[z(\xi+\Delta\xi)-z(\xi)]\|_Y+\\
+\|[\Pi(t+\Delta t,\xi+\Delta\xi)-\Pi(t,\xi)]z(\xi)\|_Y\leqslant\sup\limits_{(\tau,\eta)\in\Gamma}\|\Pi(\tau,\eta)\|\|z(\xi+\Delta\xi)-z(\xi)\|_X+\\
+\|[\Pi(t+\Delta t,\xi+\Delta\xi)-\Pi(t,\xi)]z(\xi)\|_Y,
\end{gather*}
откуда, в силу условий настоящей леммы, и вытекает, что функция $\Gamma\ni(t,\xi)\mapsto\Pi(t,\xi)z(\xi)$ является элементом $C(\Gamma,Y)$.
\end{Proof}

\begin{Lemma}\label{differentiability_Pi(t,xi)z(xi)}
Пусть $\Pi(t,\xi)\in\mathcal{L}(X,Y)$ при всех $(t,\xi)\in\Gamma$, причём при всех $x\in X$ функция $\Gamma\ni(t,\xi)\mapsto\Pi(t,\xi)x$ принадлежит $C(\Gamma,Y)$ и при всех
$(t,\xi)\in\Gamma$ имеет непрерывную на $\Gamma$ в норме $Y$ производную $\Gamma\ni(t,\xi)\mapsto\Pi_t(t,\xi)x$. Если $z\in C([0,T],X)$, то функция $\Gamma\ni(t,\xi)\mapsto\Pi(t,\xi)z(\xi)$
является элементом $C(\Gamma,Y)$ и  при всех $(t,\xi)\in\Gamma$ имеет непрерывную в норме $Y$ на $\Gamma$ производную по переменной $t$. Кроме того,
\begin{gather*}
\frac{\partial}{\partial t}\Pi(t,\xi)z(\xi)=\Pi_t(t,\xi)z(\xi)\,\,\,\forall\,(t,\xi)\in\Gamma.
\end{gather*}
\end{Lemma}
\begin{Proof} Выберем произвольно функцию $z\in C([0,T],X)$ и зафиксируем. Введём обозначения
\begin{gather*}
\Theta_0(t,\xi)\equiv\Pi(t,\xi)z(\xi),\,\,\,\Theta_1(t,\xi)\equiv\Pi_t(t,\xi)z(\xi),\,\,\,(t,\xi)\in\Gamma.
\end{gather*}
Справедливость включений $\Theta_0$, $\Theta_1\in C(\Gamma,Y)$ вытекает из леммы \ref{continuity_Pi(t,xi)z(xi)}. Следовательно, нужно доказать лишь равенство
\begin{gather*}
\Theta_{0t}(t,\xi)=\Theta_1(t,\xi)\,\,\,\forall\,(t,\xi)\in\Gamma.
\end{gather*}
Действительно, пусть $(t,\xi)$, $(t+\Delta t,\xi)\in\Gamma$ --- произвольны. Тогда
\begin{gather*}
\left\|\frac{\Theta_0(t+\Delta t,\xi)-\Theta_0(t,\xi)}{\Delta t}-\Theta_1(t,\xi)\right\|_Y= \left\|\left[\frac{\Pi(t+\Delta t,\xi)-\Pi(t,\xi)}{\Delta t}-\Pi_t(t,\xi)\right]y(\xi)
\right\|_Y\to0,\,\,\,\Delta t\to0,
\end{gather*}
в силу условий леммы. Лемма доказана.
\end{Proof}


        \section{Интеграл Римана функции одной вещественной переменной}
В данном разделе приводятся определение и свойства интеграла Римана от функций одной переменной, принимающих значения в банаховом пространстве. Изложение этих сведений следует
\cite[\S23]{KudryavtsevT1}, \cite[\S25]{Trenogin} и \cite{ZorichTomI}. Всюду в настоящем разделе $X$ --- банахово пространство с нормой $\|\cdot\|_X$.

Прежде всего нам потребуется следующая

\begin{Lemma}\label{norm_x_j+y}
Пусть последовательность $x_j\in X$, $j=1,2,\dots$, такова, что $\|x_j\|_X\to\infty$, $j\to\infty$. Тогда для всех $y\in X$ справедливо соотношение $\|x_j+y\|_X\to\infty$, $j\to\infty$.
\end{Lemma}
\begin{Proof}
Заметим, что
\begin{gather*}
\|x_j+y\|_X=\|x_j-(-y)\|_X\geqslant|\,\|x_j\|_X-\|-y\|_X|=|\,\|x_j\|_X-\|y\|_X|\geqslant\|x_j\|_X-\|y\|_X,
\end{gather*}
то есть
\begin{gather*}
\|x_j+y\|_X\geqslant\|x_j\|_X-\|y\|_X,\,\,\,j=1,2,\dots
\end{gather*}
Переходя в последнем неравенстве к пределу при $j\to\infty$, получаем требуемое соотношение. Лемма доказана.
\end{Proof}

            \subsection{Определение интеграла и условия интегрируемости}
\textit{Разбиением} $\tau$ отрезка $[a,b]$ называется любая конечная система его точек $\{t_i\}^{i=i_\tau}_{i=0}$, такая, что
\begin{gather*}
a=t_0<t_1<\dots<t_{i_\tau-1}<t_{i_\tau}=b.
\end{gather*}
При этом пишут $\tau=\{t_i\}^{i=i_\tau}_{i=0}$. Каждый из отрезков $[t_{i-1},t_i]$ называется \textit{отрезком разбиения} $\tau$, длину этого отрезка обозначают через $\Delta t_i$,
$\Delta t_i\equiv t_i-t_{i-1}$, $i=\overline{1,i_\tau}$. Число $|\tau|=\max\limits_{i=\overline{1,i_\tau}}\Delta t_i$ называется \textit{мелкостью разбиения} $\tau$.

Разбиение $\tau'$ отрезка $[a,b]$ называется следующим за разбиением $\tau$ (или продолжающим разбиение $\tau$) того же отрезка, если каждая точка разбиения $\tau$ является и точкой
разбиения $\tau'$. Иначе говоря, если каждый отрезок разбиения $\tau'$ содержится в некотором отрезке разбиения $\tau$ (говорят ещё, что $\tau'$ --- измельчение разбиения $\tau$). В этом
случае пишут $\tau'\succ\tau$, или, что то же, $\tau\prec\tau'$.

Совокупность всех разбиений отрезка обладает следующими свойствами.
{\renewcommand{\labelenumi}{\theenumi$^\circ$}
\renewcommand{\theenumi}{\arabic{enumi}}
\begin{enumerate}
    \item Если $\tau_1\prec\tau_2$, а $\tau_2\prec\tau_3$, то $\tau_1\prec\tau_3$.
    \item Для любых $\tau_1$ и $\tau_2$ существует такое $\tau$, что $\tau\succ\tau_1$ и $\tau\succ\tau_2$.
\end{enumerate}}

Пусть теперь на отрезке $[a,b]$ определена функция $f$, принимающая значения в банаховом пространстве $X$, и пусть $\tau=\{t_i\}^{i=i_\tau}_{i=0}$ --- некоторое разбиение отрезка $[a,b]$,
$\Delta t_i\equiv t_i-t_{i-1}$, $i=\overline{1,i_\tau}$, а $|\tau|$ --- мелкость этого разбиения.

Зафиксируем произвольным образом точки $\xi_i\in[t_{i-1},t_i]$, $i=\overline{1,i_\tau}$, и составим сумму
\begin{gather*}
\sigma_\tau(f;\xi_1,\dots,\xi_{i_\tau})=\sum\limits_{i=1}^{i_\tau}f(\xi_i)\Delta t_i.
\end{gather*}
Суммы такого вида называются \textit{интегральными суммами Римана функции} $f$. Иногда будем обозначать их через $\sigma_\tau(f)$, или даже просто через $\sigma_\tau$. Точки $\xi_i$,
$i=\overline{1,i_\tau}$, будем называть \textit{отмеченными точками разбиения} $\tau$.


\begin{Definition}\label{Riemann1}
Функция $f$ называется интегрируемой (по Риману) на отрезке $[a,b]$, если существует такой элемент $A\in X$, что для любой последовательности разбиений отрезка $[a,b]$
\begin{gather*}
\tau_j=\{t_i^{(j)}\}^{i=i_{\tau_j}}_{i=0},\,\,\,j=1,2,\dots,
\end{gather*}
у которой $\lim\limits_{j\to\infty}|\tau_j|=0$, и для любого выбора точек $\xi^{(j)}_i\in[t^{(j)}_{i-1},t^{(j)}_i]$, $i=\overline{1,i_{\tau_j}}$, $j=1,2,\dots$, существует предел
последовательности интегральных сумм $\sigma_{\tau_j}(f;\xi^{(j)}_1,\dots,\xi^{(j)}_{i_{\tau_j}})$ и он равен $A$:
\begin{gather}\label{limintegralsums}
\lim\limits_{j\to\infty}\left\|\sum\limits_{i=1}^{i_{\tau_j}}f(\xi^{(j)}_i)\Delta t^{(j)}_i-A\right\|_X=0,
\end{gather}
где $\Delta t^{(j)}_i\equiv t^{(j)}_i-t^{(j)}_{i-1}$, $i=\overline{1,i_{\tau_j}}$, $j=1,2,\dots$

При выполнении этих условий элемент $A$ называется (римановым) определённым интегралом функции $f$ на отрезке $[a,b]$ и обозначается $(\textrm{Р})\int\limits_a^bf(t)dt$ или
$\int\limits_a^bf(t)dt$.
\end{Definition}

Таким образом,
\begin{gather*}
\int\limits_a^bf(t)dt=\lim\limits_{j\to\infty}\sigma_{\tau_j}(f;\xi^{(j)}_1,\dots,\xi^{(j)}_{i_{\tau_j}}),
\end{gather*}
где последовательность $\tau_j$, $j=1,2,\dots$, такова, что
\begin{gather*}
\lim\limits_{j\to\infty}|\tau_j|=0.
\end{gather*}
Для краткости в этом случае будем писать просто
\begin{gather*}
\int\limits_a^bf(t)dt=\lim\limits_{|\tau|\to0}\sigma_\tau(f).
\end{gather*}

Подобно тому, как определение предела функции можно сформулировать двумя эквивалентными способами --- с помощью пределов последовательностей и с помощью языка
\glqq$\varepsilon$--$\delta$\grqq, --- так и определение интеграла Римана можно сформулировать иначе.


\begin{Definition}\label{Riemann2}
Элемент $A\in X$ называется определённым интегралом функции $f$ на отрезке $[a,b]$, если для любого $\varepsilon>0$ найдётся такое $\delta=\delta(\varepsilon)>0$, что, каково бы ни было
разбиение $\tau=\{t_i\}^{i=i_\tau}_{i=0}$ (отрезка $[a,b]$) мелкости, меньшей $\delta$, и каковы бы ни были точки $\xi_i\in[t_{i-1},t_i]$, $i=\overline{1,i_\tau}$, выполняется неравенство
\begin{gather*}
\left\|\sum\limits_{i=1}^{i_{\tau}}f(\xi_i)\Delta t_i-A\right\|_X=0,
\end{gather*}
где $\Delta t_i\equiv t_i-t_{i-1}$, $i=\overline{1,i_{\tau_j}}$.
\end{Definition}

\begin{Theorem}
Два данных определения интеграла эквивалентны.
\end{Theorem}
\begin{Proof}
1) Покажем, что интеграл в смысле определения \ref{Riemann1} является интегралом в смысле определения \ref{Riemann2}. Предположим, что элемент $A\in X$ является интегралом в смысле
определения \ref{Riemann1}, но не является интегралом в смысле определения \ref{Riemann2}. Поскольку $A\in X$ не является интегралом в смысле определения \ref{Riemann2}, то найдутся
$\varepsilon_0>0$, последовательность чисел $\delta_j>0$, $j=1,2,\dots$, $\delta_j\to0$, $j\to\infty$, последовательность разбиений $\tau_j=\{t_i^{(j)}\}^{i=i_{\tau_j}}_{i=0}$,
$|\tau_j|<\delta_j$, $j=1,2,\dots$, и последовательность наборов $\xi^{(j)}\equiv\{\xi^{(j)}_1,\dots, \xi^{(j)}_{i_\tau}\}$, $\xi^{(j)}_i\in[t^{(j)}_{i-1},t^{(j)}_i]$,
$i=\overline{1,i_{\tau_j}}$, $j=1,2,\dots$, такие, что
\begin{gather*}
\left\|\sum\limits_{i=1}^{i_{\tau_j}}f(\xi^{(j)}_i)\Delta t^{(j)}_i-A\right\|_X\geqslant\varepsilon_0,
\end{gather*}
где $\Delta t^{(j)}_i\equiv t^{(j)}_i-t^{(j)}_{i-1}$, $i=\overline{1,i_{\tau_j}}$, $j=1,2,\dots$ А это противоречит тому, что $A\in X$ --- интеграл в смысле определения \ref{Riemann1}.

2) Покажем, что интеграл в смысле определения \ref{Riemann2} является интегралом в смысле определения \ref{Riemann1}. Предположим, что элемент $A\in X$ является интегралом в смысле
определения \ref{Riemann2}. Пусть $\tau_j=\{t_i^{(j)}\}^{i=i_{\tau_j}}_{i=0}$, $j=1,2,\dots$, --- последовательность разбиений, такая, что $\lim\limits_{j\to\infty}|\tau_j|=0$, и пусть
точки $\xi^{(j)}_i\in[t^{(j)}_{i-1},t^{(j)}_i]$, $i=\overline{1,i_{\tau_j}}$, $j=1,2,\dots$, --- произвольны.

Выберем произвольно $\varepsilon>0$ и зафиксируем, после чего подберём по нему число $\delta=\delta(\varepsilon)>0$ из определения \ref{Riemann2}. Так как
$\lim\limits_{j\to\infty}|\tau_j|=0$, то найдётся номер $j_0=j_0(\delta(\varepsilon))$, такой, что $|\tau_j|<\delta(\varepsilon)$ для всех $j\geqslant j_0(\delta(\varepsilon))$. Поэтому,
на основании определения \ref{Riemann2},
\begin{gather*}
\left\|\sum\limits_{i=1}^{i_{\tau_j}}f(\xi^{(j)}_i)\Delta t^{(j)}_i-A\right\|_X\leqslant\varepsilon
\end{gather*}
при всех $j\geqslant j_0(\delta(\varepsilon))$. Иными словами, для произвольной последовательности разбиений отрезка $[a,b]$
\begin{gather*}
\tau_j=\{t_i^{(j)}\}^{i=i_{\tau_j}}_{i=0},\,\,\,j=1,2,\dots,
\end{gather*}
у которой $\lim\limits_{j\to\infty}|\tau_j|=0$, и для любого выбора точек $\xi^{(j)}_i\in[t^{(j)}_{i-1},t^{(j)}_i]$, $i=\overline{1,i_{\tau_j}}$, $j=1,2,\dots$, существует предел
последовательности интегральных сумм $\sigma_{\tau_j}(f;\xi^{(j)}_1,\dots,\xi^{(j)}_{i_{\tau_j}})$ и он равен $A$:
\begin{gather*}
\lim\limits_{j\to\infty}\left\|\sum\limits_{i=1}^{i_{\tau_j}}f(\xi^{(j)}_i)\Delta t^{(j)}_i-A\right\|_X=0,
\end{gather*}
где $\Delta t^{(j)}_i\equiv t^{(j)}_i-t^{(j)}_{i-1}$, $i=\overline{1,i_{\tau_j}}$, $j=1,2,\dots$ Следовательно, элемент $A\in X$ --- интеграл в смысле определения \ref{Riemann1}.
\end{Proof}

Выше введено понятие определённого интеграла $\int\limits_a^bf(t)dt$ от функции $f$ по отрезку $[a,b]$, $a<b$.

Для любой функции $f$, определённой в точке $a$, по определению положим
\begin{gather*}
\int\limits_a^af(t)dt=0,
\end{gather*}
а для функции, интегрируемой на отрезке $[a,b]$,
\begin{gather*}
\int\limits_b^af(t)dt=-\int\limits_a^bf(t)dt.
\end{gather*}

Аналогично критерию Коши существования предела функции формулируется и доказывается аналогичный критерий существования предела интегральных сумм.
\begin{Theorem}\label{Cauchy_criterium_for_Riemann_integral::Theorem}
Для того, чтобы функция $f$ была интегрируема на отрезке $[a,b]$, необходимо и достаточно, чтобы для любого $\varepsilon>0$ существовало такое $\delta=\delta(\varepsilon)>0$, что, каковы бы
ни были разбиения $\tau'=\{t'_i\}_{i=0}^{i=i_{\tau'}}$ и $\tau''=\{t''_j\}_{j=0}^{j=j_{\tau''}}$, мелкости, меньшей $\delta$, и точки  $\xi'_i\in[t'_{i-1},t'_i]$, $i=
\overline{1,i_{\tau'}}$, $\xi''_j\in[t''_{j-1},t''_j]$, $j=\overline{1,j_{\tau''}}$, выполнено неравенство
\begin{gather}\label{Cauchy_criterium_for_Riemann_integral}
\|\sigma_{\tau'}(f;\xi'_1,\dots,\xi'_{i_{\tau'}})-\sigma_{\tau''}(f;\xi''_1,\dots,\xi''_{j_{\tau''}})\|_X<\varepsilon.
\end{gather}
\end{Theorem}
\begin{Proof}
1) Докажем необходимость условия (\ref{Cauchy_criterium_for_Riemann_integral}). Если функция $f$ интегрируема в смысле Римана, т.е. существует предел (\ref{limintegralsums}), то согласно
определению \ref{Riemann2}, для любого $\varepsilon>0$ найдётся такое $\delta=\delta(\varepsilon)>0$, что, каково бы ни было разбиение $\tau=\{t_i\}^{i=i_\tau}_{i=0}$ (отрезка $[a,b]$)
мелкости, меньшей $\delta$, и при любом выборе точек $\xi_i\in[t_{i-1},t_i]$, $i=\overline{1,i_\tau}$, для интегральных сумм $\sigma_\tau=\sigma_{\tau}(f;\xi_1,\dots,\xi_{i_{\tau}})$
выполняется неравенство
\begin{gather*}
\|\sigma_\tau-A\|_X<\frac\varepsilon2.
\end{gather*}

Если теперь $\sigma_{\tau'}\equiv\sigma_{\tau'}(f;\xi'_1,\dots,\xi'_{i_{\tau'}})$ и $\sigma_{\tau''}=\sigma_{\tau''}(f;\xi''_1,\dots,\xi''_{j_{\tau''}})$ --- две такие интегральные суммы,
что $|\tau'|<\delta$ и $|\tau''|<\delta$, то
\begin{gather*}
\|\sigma_{\tau'}-\sigma_{\tau''}\|_X=\|(\sigma_{\tau'}-A)+(A-\sigma_{\tau''})\|_X\leqslant\|\sigma_{\tau'}-A\|_X+\|A-\sigma_{\tau''}\|_X<\frac\varepsilon2+\frac\varepsilon2=\varepsilon.
\end{gather*}

2) Докажем достаточность условия (\ref{Cauchy_criterium_for_Riemann_integral}). Пусть для функции $f\colon[a,b]\to X$ выполнено условие (\ref{Cauchy_criterium_for_Riemann_integral}) и
$\sigma_{\tau_j}=\sigma_{\tau_j}(f;\xi^{(j)}_1,\dots,\xi^{(j)}_{i_{\tau_j}})$, $j=1,2,\dots$, --- такая последовательность интегральных сумм функции $f$, что
\begin{gather}\label{lim|tau_j|=0}
\lim\limits_{j\to\infty}|\tau_j|=0.
\end{gather}
Если  $\varepsilon>0$ произвольно фиксировано, а $\delta>0$ выбрано так, что выполняется условие (\ref{Cauchy_criterium_for_Riemann_integral}), то, в силу равенства (\ref{lim|tau_j|=0}),
существует такой номер $j_0$, что для всех $j>j_0$ выполняется условие $|\tau_j|<\delta$. Поэтому для всех $j'>j_0$ и $j''>j_0$ выполняются неравенства  $|\tau_{j'}|<\delta$,
$|\tau_{j''}|<\delta$, и, следовательно, согласно условию (\ref{Cauchy_criterium_for_Riemann_integral}), имеет место неравенство
\begin{gather*}
\|\sigma_{\tau_{j'}}-\sigma_{\tau_{j''}}\|_X<\varepsilon.
\end{gather*}
Это означает, что последовательность $\sigma_{\tau_j}$, $j=1,2,\dots$, фундаментальна в банаховом пространстве $X$. Ввиду полноты
пространства $X$ отсюда вытекает существование элемента $A\in X$, такого, что $\lim\limits_{j\to\infty}\|\sigma_{\tau_j}-A\|_X=0$.

Последовательность $\sigma_{\tau_j}$, $j=1,2,\dots$, являлась произвольной последовательностью интегральных сумм, для которой выполнялось условие (\ref{lim|tau_j|=0}). Поэтому все такие
последовательности сходятся, притом к одному и тому же элементу пространства $X$. В самом деле, пусть последовательности $\sigma_{\tau'_j}$, $j=1,2,\dots$, и  $\sigma_{\tau''_j}$,
$j=1,2,\dots$, таковы, что
\begin{gather*}
\lim\limits_{j\to\infty}|\tau'_j|=\lim\limits_{j\to\infty}|\tau''_j|=0,
\end{gather*}
и, следовательно, найдутся пределы $A'$, $A''\in X$:
\begin{gather*}
\lim\limits_{j\to\infty}\|\sigma_{\tau'_j}-A'\|_X=0,\,\,\,\lim\limits_{j\to\infty}\|\sigma_{\tau''_j}-A''\|_X=0.
\end{gather*}

Составим новую последовательность интегральных сумм, $\sigma_{\tau_k}$, $k=1,2,\dots$, так:
\begin{gather*}
\sigma_{\tau_k}=
\begin{cases}
\sigma_{\tau'_j},\text{ если $k=2j-1$;}\cr
\sigma_{\tau''_j},\text{ если $k=2j$;}\cr
\end{cases}
\,\,\,k=1,2,\dots
\end{gather*}

Тогда, очевидно, $\lim\limits_{k\to\infty}|\tau_k|=0$, и поэтому у последовательности  $\sigma_{\tau_k}$, $k=1,2,\dots$, существует предел (в смысле сходимости по норме пространства $X$).

Предел всякой подпоследовательности сходящейся последовательности равен пределу всей последовательности. Следовательно,
\begin{gather*}
A'=\lim\limits_{j\to\infty}\sigma_{\tau'_j}=\lim\limits_{k\to\infty}\sigma_{\tau_k}=\lim\limits_{j\to\infty}\sigma_{\tau''_j}=A''.
\end{gather*}
Теорема полностью доказана.
\end{Proof}

\begin{Theorem}
Если функция интегрируема (в смысле Римана) на некотором отрезке, то она ограничена на этом отрезке.
\end{Theorem}
\begin{Proof}
Пусть функция $f\colon[a,b]\to X$ неограничена на отрезке $[a,b]$, и пусть фиксировано некоторое разбиение $\tau=\{t_i\}_{i=0}^{i=i_{\tau}}$ этого отрезка. В силу неограниченности функции
$f$ на всём отрезке $[a,b]$, она неограничена по крайней мере на одном отрезке разбиения $\tau$. Пусть для определённости функция $f$ неограничена на отрезке $[t_0,t_1]]$. Тогда на этом
отрезке существует последовательность точек $\xi^{(j)}_1\in[t_0,t_1]$, $j=1,2,\dots$, такая, что
\begin{gather*}
\|f(\xi^{(j)}_1)\|_X\geqslant j,\,\,\,j=1,2,\dots
\end{gather*}
Отсюда следует, что
\begin{gather}\label{norm_of_f(xij1)_tends_to_+infty}
\lim\limits_{j\to\infty}\|f(\xi^{(j)}_1)\|_X=+\infty.
\end{gather}
Зафиксируем теперь каким--либо образом точки $\xi_i\in[t_{i-1},t_i]$, $i=\overline{2,i_\tau}$. Тогда сумма $\sum\limits_{i=2}^{i_\tau}f(\xi_i)\Delta t_i$ будет иметь определённое значение.
Поэтому, в силу (\ref{norm_of_f(xij1)_tends_to_+infty}) и леммы \ref{norm_x_j+y},
\begin{gather*}
\lim\limits_{j\to\infty}\|\sigma_{\tau}(f;\xi^{(j)}_1,\xi_2,\dots,\xi_{i_{\tau}})\|_X=
\lim\limits_{j\to\infty}\left\|f(\xi^{(j)}_1)\Delta t_1+\sum\limits_{i=2}^{i_\tau}f(\xi_i)\Delta t_i\right\|_X=+\infty,
\end{gather*}
и, следовательно, каково бы ни было число $K>0$, всегда можно подобрать такой номер $j_0$, что если на отрезке $[t_0,t_1]]$ взять точку
$\xi^{(j_0)}_1$, то
\begin{gather*}
\|\sigma_{\tau}(f;\xi^{(j_0)}_1,\xi_2,\dots,\xi_{i_{\tau}})\|_X>K
\end{gather*}
Отсюда следует, что суммы $\sigma_{\tau}$ не могут стремиться ни к какому конечному пределу при $|\tau|\to0$.

Действительно, если бы существовал элемент $A\in X$, такой, что
\begin{gather*}
\lim\limits_{|\tau|\to0}\sigma_\tau=A,
\end{gather*}
то для любого $\varepsilon>0$ нашлось бы такое $\delta=\delta(\varepsilon)>0$, что для всех разбиений $\tau=\{t_i\}_{i=0}^{i=i_{\tau}}$ отрезка $[a,b]$, у которых
$|\tau|<\delta(\varepsilon)$, при любом выборе точек $\xi_i\in[t_{i-1},t_i]$, $i=\overline{1,i_\tau}$, выполнялось бы неравенство $\|\sigma_\tau-A\|_X<\varepsilon$, и, следовательно,
\begin{gather*}
\|\sigma_\tau\|_X=\|(\sigma_\tau-A)+A\|_X\leqslant\|\sigma_\tau-A\|_X+\|A\|_X<\varepsilon+\|A\|_X.
\end{gather*}

Поскольку же функция $f$ неограничена, то для любого разбиения $\tau$ (в том числе и такого, что $|\tau|<\delta(\varepsilon)$, если существовало бы указанное $\delta(\varepsilon)$) при
любом фиксированном $\varepsilon>0$ можно так выбрать точки $\xi_i$, что будет выполняться неравенство
\begin{gather*}
\|\sigma_\tau\|_X>\varepsilon+\|A\|_X.
\end{gather*}
Полученное противоречие доказывает теорему.
\end{Proof}

Сформулируем и докажем следующее достаточное условие интегрируемости функции $f\colon[a,b]\to X$ на отрезке $[a,b]$.
\begin{Theorem}\label{Riemann_integrability_sufficient_conditions_in_terms_of_oscillation}
Для интегрируемости ограниченной на отрезке $[a,b]$ функции $f\colon[a,b]\to X$ достаточно, чтобы для любого числа $\varepsilon>0$ нашлось такое число $\delta=\delta(\varepsilon)>0$, что
при любом разбиении $\tau=\{t_i\}_{i=1}^{i=i_\tau}$ отрезка $[a,b]$, $|\tau|<\delta$, выполнено соотношение
\begin{gather*}
\sum\limits_{i=1}^{i_\tau}\osc(f;[t_{i-1},t_i])\Delta t_i<\varepsilon,
\end{gather*}
где $\Delta t_i=t_i-t_{i-1}$, $i=\overline{1,i_\tau}$.
\end{Theorem}
\begin{Proof}
Пусть $\tau=\{t_i\}_{i=0}^{i=i_\tau}$ --- разбиение отрезка $[a,b]$, а $\xi_i\in[t_{i-1},t_i]$, $i=\overline{1,i_\tau}$, --- произвольные точки. Пусть $\tilde{\tau}=
\{\tilde{t}_j\}_{j=0}^{j=i_\tau}$ --- измельчение разбиения $\tau$. Тогда некоторые (а может быть, и все) отрезки $[t_{i-1},t_i]$ разбиения $\tau$ сами подвергаются разбиению
$t_{i-1}=t_{i-1,0}<\dots<t_{i,k_i}=t_i$. В связи с этим нам нам будет удобно нумеровать точки разбиения $\tilde{\tau}$ двумя индексами. В записи $t_{i,j}$ первый индекс означает, что
$t_{i,j}\in[t_{i-1},t_i]$, а второй индекс есть порядковый номер точки на отрезке $[t_{i-1},t_i]$. Теперь естественно положить $\Delta t_{i,j}=t_{i,j}-t_{i,j-1}$.
Таким образом, $\Delta t_i=\Delta t_{i,1}+\dots+\Delta t_{i,k_i}$.

Выберем произвольно точки $\xi'_{i,j}\in[t_{i,j-1},t_{i,j}]$, $i=\overline{1,i_\tau}$, $j=\overline{1,k_i}$, и оценим норму разности
интегральных сумм $\sigma_\tau(f;\xi_1,\dots,\xi_{i_\tau})-\sigma_{\tilde{\tau}}(f;\xi'_{1,1},\dots,\xi'_{i_\tau,k_{i_\tau}})$:
\begin{gather*}
\|\sigma_\tau(f;\xi_1,\dots,\xi_{i_\tau})-\sigma_{\tilde{\tau}}(f;\xi'_{1,1},\dots,\xi'_{i_\tau,k_{i_\tau}})\|_X=
\left\|\sum\limits_{i=1}^{i_\tau}f(\xi_i)\Delta t_i-\sum\limits_{i=1}^{i_\tau}\sum\limits_{j=1}^{k_i}f(\xi'_{i,j})\Delta t_{i,j}\right\|_X=\\
=\left\|\sum\limits_{i=1}^{i_\tau}\sum\limits_{j=1}^{k_i}f(\xi_i)\Delta t_{i,j}-\sum\limits_{i=1}^{i_\tau}\sum\limits_{j=1}^{k_i}
f(\xi'_{i,j})\Delta t_{i,j}\right\|_X=\left\|\sum\limits_{i=1}^{i_\tau}\sum\limits_{j=1}^{k_i}[f(\xi_i)-f(\xi'_{i,j})]\Delta t_{i,j}\right\|_X\leqslant\\
\leqslant\sum\limits_{i=1}^{i_\tau}\sum\limits_{j=1}^{k_i}\|f(\xi_i)-f(\xi'_{i,j})\|_X\Delta t_{i,j}\leqslant\sum\limits_{i=1}^{i_\tau}
\sum\limits_{j=1}^{k_i}\osc(f;[t_{i-1},t_i])\Delta t_{i,j}=\sum\limits_{i=1}^{i_\tau}\osc(f;[t_{i-1},t_i])\Delta t_{i}.
\end{gather*}

Из полученной оценки нормы разности интегральных сумм следует, что если функция $f$ удовлетворяет достаточным условиям, сформулированным в данной теореме, то по любому $\varepsilon>0$ можно
найти такое $\delta>0$, что для любого разбиения $\tau$ отрезка $[a,b]$, $|\tau|<\delta$, и для измельчения $\tilde\tau$ разбиения $\tau$ будем иметь
\begin{gather*}
\|\sigma_\tau(f)-\sigma_{\tilde{\tau}}(f)\|_X<\frac\varepsilon2.
\end{gather*}

Если теперь $\tau'$ и $\tau''$ --- произвольные разбиения отрезка $[a,b]$, $|\tau'|<\delta$, $|\tau''|<\delta$, то, рассмотрев разбиение $\tilde\tau=\tau'\cup\tau''$, являющееся
измельчением обоих разбиений $\tau'$, $\tau''$, по доказанному будем иметь
\begin{gather*}
\|\sigma_{\tau'}(f)-\sigma_{\tilde{\tau}}(f)\|_X<\frac\varepsilon2,\,\,\,\|\sigma_{\tau''}(f)-\sigma_{\tilde{\tau}}(f)\|_X<\frac\varepsilon2.
\end{gather*}
Отсюда следует, что
\begin{gather*}
\|\sigma_{\tau'}-\sigma_{\tau''}\|_X=\|(\sigma_{\tau'}-\sigma_{\tilde{\tau}})+(\sigma_{\tilde{\tau}}-\sigma_{\tau''})\|_X\leqslant
\|\sigma_{\tau'}-\sigma_{\tilde{\tau}}\|_X+\|\sigma_{\tilde{\tau}}-\sigma_{\tau''}\|_X<\frac\varepsilon2+\frac\varepsilon2=\varepsilon,
\end{gather*}
как только $|\tau'|<\delta$, $|\tau''|<\delta$. Таким образом, в силу теоремы \ref{Cauchy_criterium_for_Riemann_integral::Theorem}, функция $f$ --- интегрируема в смысле Римана на отрезке
$[a,b]$.
\end{Proof}


\begin{Corrolary}
Если функция $f\colon[a,b]\to X$ сильно непрерывна на отрезке $[a,b]$, то $f$ интегрируема на отрезке $[a,b]$.
\end{Corrolary}
\begin{Proof}
Если $f$ непрерывна на отрезке, то она ограничена на нём. Поэтому на основании теоремы \ref{Riemann_integrability_sufficient_conditions_in_terms_of_oscillation} достаточно показать, что
\begin{gather}\label{Scosc}
\forall\,\varepsilon>0\,\,\exists\,\delta=\delta(\varepsilon)>0\,\,\forall\,\text{разбиения $\tau=\{t_i\}_{i=0}^{i_\tau}$, $|\tau|<\delta$}:
\sum\limits_{i=1}^{i_\tau}\osc(f;[t_{i-1},t_i])\Delta t_i<\varepsilon,
\end{gather}
где $\Delta t_i=t_i-t_{i-1}$, $i=\overline{1,i_\tau}$.

Так как $f$ непрерывна на отрезке $[a,b]$, то она равномерно непрерывна на этом отрезке, ввиду чего
\begin{gather}\label{ucon}
\forall\,\varepsilon>0\,\,\exists\,\eta=\eta(\varepsilon)>0\,\,\forall\,t',\,t''\in[a,b],\,\,\,|t'-t''|<\eta:
\|f(t')-f(t'')\|_X\leqslant\frac{\varepsilon}{2(b-a)}.
\end{gather}

Выберем произвольно $\varepsilon>0$ и подберём по нему $\eta=\eta(\varepsilon)>0$ согласно (\ref{ucon}). Выберем произвольно разбиение $\tau=\{t_i\}_{i=0}^{i_\tau}$, у которого
$|\tau|<\eta$. Выберем $i=\overline{1,i_\tau}$ и зафиксируем. Тогда для любых $\xi'$, $\xi''\in[t_{i-1},t_i]$, ввиду неравенства $\Delta t_i\leqslant|\tau|<\eta$ и условия (\ref{ucon})
имеет место неравенство
\begin{gather*}
\|f(\xi')-f(\xi'')\|_X\leqslant\frac{\varepsilon}{2(b-a)},
\end{gather*}
откуда вытекает, что
\begin{gather*}
\osc(f;[t_{i-1},t_i])\leqslant\frac{\varepsilon}{2(b-a)}.
\end{gather*}
Итак,
\begin{gather*}
\osc(f;[t_{i-1},t_i])\leqslant\frac{\varepsilon}{2(b-a)}\,\,\,i=\overline{1,i_\tau}.
\end{gather*}
Поэтому
\begin{gather*}
\sum\limits_{i=1}^{i_\tau}\osc(f;[t_{i-1},t_i])\Delta t_i\leqslant\frac{\varepsilon}{2(b-a)}\sum\limits_{i=1}^{i_\tau}\Delta t_i=\frac{\varepsilon}{2(b-a)}(b-a)<\varepsilon.
\end{gather*}
Таким образом, мы доказали (\ref{Scosc}) с $\delta(\varepsilon)=\eta(\varepsilon)$. Следствие доказано.
\end{Proof}

Далее нам потребуется несколько результатов о колебаниях функций. Во всех этих результатах $\mathcal{T}\subset\mathbb{R}$ --- некоторое множество.
\begin{Lemma}
Если функция $f\colon\mathcal{T}\to X$ ограничена на множестве $\mathcal{T}$, т.е. если найдётся постоянная $c>0$, такая, что
при всех $t\in\mathcal{T}$ выполнено неравенство $\|f(t)\|_X\leqslant c$, то для всех $t\in\mathcal{T}$ выполняется неравенство
\begin{gather}\label{oscftineq}
\osc(f;t)\leqslant2c.
\end{gather}
\end{Lemma}
\begin{Proof}
В самом деле, для любого $t_0\in\mathcal{T}$
\begin{gather*}
\osc(f;t_0)=\inf\limits_{r>0}\osc(f;\mathcal{T}\cap(t_0-r,t_0+r))=\inf\limits_{r>0}\sup\limits_{t',\,t''\in\mathcal{T}\cap(t_0-r,t_0+r)}\|f(t')-f(t'')\|_X\leqslant\\
\leqslant\inf\limits_{r>0}\sup\limits_{t',\,t''\in\mathcal{T}\cap(t_0-r,t_0+r)}[\|f(t')\|_X+\|f(t'')\|_X]\leqslant2c.
\end{gather*}
Лемма доказана.
\end{Proof}

Для дальнейшего полезно ввести множество
\begin{gather}\label{setoscftgeqeps}
\mathcal{T}_\varepsilon\equiv\{t\in\mathcal{T}:\osc(f;t)\geqslant\varepsilon\},
\end{gather}
где $\varepsilon>0$ --- произвольно.

Если $\eta<\varepsilon$, то ясно, что из неравенства $\osc(f;t)\geqslant\varepsilon$ следует неравенство $\osc(f;t)\geqslant\eta$, и поэтому
\begin{gather}\label{Teps_SUBSET_Teta}
\mathcal{T}_\varepsilon\subset\mathcal{T}_\eta.
\end{gather}

\begin{Lemma}\label{strong_continuity_criterium}
Функция $f\colon\mathcal{T}\to X$ сильно непрерывна в точке $t\in\mathcal{T}$ тогда и только тогда, когда
\begin{gather}\label{oscfteq0}
\osc(f;t)=0.
\end{gather}
\end{Lemma}
\begin{Proof}
1) Если функция $f$ сильно непрерывна в точке $t_0\in\mathcal{T}$, то для любого $\varepsilon>0$ существует такое $\delta>0$, что для всех точек $t\in(t_0-\delta,t-0+\delta)\cap
\mathcal{T}$ выполняется неравенство $\|f(t)-f(t_0)\|_X<\frac\varepsilon2$. Поэтому для любых точек $t'$, $t''\in(t_0-\delta,t-0+\delta)\cap\mathcal{T}$ имеем
\begin{gather*}
\|f(t')-f(t'')\|_X\leqslant\|f(t')-f(t_0)\|_X+\|f(t_0)-f(t'')\|_X<\frac\varepsilon2+\frac\varepsilon2=\varepsilon.,
\end{gather*}
и, следовательно,
\begin{gather*}
\osc(f;t_0)=\inf\limits_{r>0}\osc(f;\mathcal{T}\cap(t_0-r,t_0+r))\leqslant\osc(f;\mathcal{T}\cap(t_0-\delta,t_0+\delta))\leqslant\varepsilon.
\end{gather*}
А так как $\varepsilon>0$ --- произвольно, то это означает, что $\osc(f;t_0)=0$.

2) Наоборот, если $\osc(f;t_0)=0$, то для любого $\varepsilon>0$ существует такое $\delta>0$, что $\osc(f;\mathcal{T}\cap(t_0-\delta,t_0+\delta))<\varepsilon$. Тогда для любого $t\in
\mathcal{T}\cap(t_0-\delta,t_0+\delta)$ получаем, что
\begin{gather*}
\|f(t)-f(t_0)\|_X\leqslant\osc(f;\mathcal{T}\cap(t_0-\delta,t_0+\delta))<\varepsilon,
\end{gather*}
т.е. функция $f$ сильно непрерывна в точке $t_0$.
\end{Proof}

\begin{Corrolary}
Если $\mathcal{T}^*$  --- множество точек разрыва функции $f\colon\mathcal{T}\to X$, то
\begin{gather}\label{repres_of_set_of_discont_points}
\mathcal{T}^*=\bigcup\limits_{j=1}^\infty\mathcal{T}_{1/j}.
\end{gather}
\end{Corrolary}
\begin{Proof}
Если точка $t_0\in\mathcal{T}$ является точкой разрыва функции $f$, то, в силу леммы \ref{strong_continuity_criterium}, $\osc(f;t_0)>0$, а поэтому $t_0\in\mathcal{T}_\varepsilon$ при
$\varepsilon=\osc(f;t_0)$. Отсюда следует, что множество $\mathcal{T}^*$ точек разрыва функции $f$ представимо в виде
\begin{gather*}
\mathcal{T}^*=\bigcup\limits_{\varepsilon>0}\mathcal{T}_{\varepsilon}.
\end{gather*}

Ясно, что $\bigcup\limits_{j=1}^\infty\mathcal{T}_{1/j}\subset\bigcup\limits_{\varepsilon>0}\mathcal{T}_{\varepsilon}$, ибо каждое слагаемое левой части включения является слагаемым правой.
С другой стороны, если для данного $\varepsilon>0$ выбрать натуральное $j$ так, чтобы $\frac1{j}<\varepsilon$, то, в силу включения (\ref{Teps_SUBSET_Teta}), будем иметь
$\mathcal{T}_{\varepsilon}\subset\mathcal{T}_{1/j}$, и, следовательно, $\bigcup\limits_{\varepsilon>0}\mathcal{T}_{\varepsilon}\subset\bigcup\limits_{j=1}^\infty\mathcal{T}_{1/j}$. Таким
образом,
\begin{gather*}
\bigcup\limits_{j=1}^\infty\mathcal{T}_{1/j}=\bigcup\limits_{\varepsilon>0}\mathcal{T}_{\varepsilon}=\mathcal{T}^*.
\end{gather*}
Следствие доказано.
\end{Proof}


\begin{Lemma}\label{limit_points_of_Teps}
При любом $\varepsilon>0$ все точки прикосновения множества $\mathcal{T}_{\varepsilon}$, содержащиеся во множестве $\mathcal{T}$, содержатся и в $\mathcal{T}_{\varepsilon}$, т.е. если
$t\in(cl\mathcal{T}_{\varepsilon})\cap\mathcal{T}$, то $t\in\mathcal{T}_{\varepsilon}$.
\end{Lemma}
\begin{Proof}
Пусть $\varepsilon>0$ и $t_0\in(cl\mathcal{T}_{\varepsilon})\cap\mathcal{T}$. Зададим произвольно $\eta>0$. В силу определения колебания функции в точке существует такой интервал
$(t_0-\delta,t_0+\delta)$, что
\begin{gather*}
\osc(f;(t_0-\delta,t_0+\delta)\cap\mathcal{T})<\osc(f;t_0)+\eta.
\end{gather*}
Точка $t_0$ является точкой прикосновения множества $\mathcal{T}_{\varepsilon}$, ввиду чего существует такая последовательность $t_j\in\mathcal{T}_{\varepsilon}$, $j=1,2,\dots$, что
$\lim\limits_{j\to\infty}t_j=t_0$. Следовательно, найдётся такой номер $j_0$, что $t_{j_0}\in(t_0-\delta,t_0+\delta)\cap\mathcal{T}_\varepsilon$. Согласно определению колебания функции в
точке, отсюда вытекает, что
\begin{gather*}
\osc(f;(t_0-\delta,t_0+\delta)\cap\mathcal{T})\geqslant\osc(f;(t_{j_0}-\delta',t_{j_0}+\delta')\cap\mathcal{T})\geqslant\osc(f;t_{j_0}),
\end{gather*}
где $\delta'=\frac12\min\{t_{j_0}-(t_0-\delta),(t_0+\delta)-t_{j_0}\}$.

Таким образом,
\begin{gather*}
\osc(f;t_0)>\osc(f;(t_0-\delta,t_0+\delta)\cap\mathcal{T})-\eta\geqslant\osc(f;t_{j_0})-\eta\geqslant\varepsilon-\eta,
\end{gather*}
так как из $t_{j_0}\in\mathcal{T}_{\varepsilon}$ следует $\osc(f;t_{j_0})\geqslant\varepsilon$.

Поскольку $\osc(f;t_{0})\geqslant\varepsilon-\eta$ при любом $\eta>0$, то  $\osc(f;t_{0})\geqslant\varepsilon$, т.е. $t_0\in\mathcal{T}_{\varepsilon}$.
\end{Proof}

\begin{Corrolary}
Если множество $\mathcal{T}$, на котором задана функция $f\colon\mathcal{T}\to X$, --- замкнуто, то при любом $\varepsilon>0$ множество $\mathcal{T}_{\varepsilon}$ --- также замкнуто.
\end{Corrolary}

\begin{Corrolary}\label{([a,b])_eps_is_compact_set}
Пусть дана функция $f\colon[a,b]\to X$. Тогда при любом $\varepsilon>0$ множество $([a,b])_\varepsilon$ --- замкнуто и ограничено.
\end{Corrolary}


\begin{Lemma}\label{oscft_lt_eps::Lemma}
Пусть задана функция $f\colon[a,b]\to X$ и существует такое $\varepsilon>0$, что для всех точек $t$ отрезка $[a,b]$ выполняется неравенство
\begin{gather}\label{oscft_lt_eps}
\osc(f;t)<\varepsilon.
\end{gather}
Тогда существует такое разбиение $\tau=\{t_i\}_{i=0}^{i=i_\tau}$ отрезка $[a,b]$, что для всех $i=\overline{1,i_\tau}$ имеет место неравенство
\begin{gather}\label{oscf[t_i-1,t_i]_lt_varepsilon}
\osc(f;[t_{i-1},t_i])<\varepsilon.
\end{gather}
\end{Lemma}
\begin{Proof}
В силу выполнения условия (\ref{oscft_lt_eps}) для любой точки $\xi\in[a,b]$ существует такой интервал $(\xi-r_\xi,\xi+r_\xi)$, что
\begin{gather}\label{oscfxi-rxi,xi+rxicap[a,b]_lt_eps}
\osc(f;(\xi-r_\xi,\xi+r_\xi)\cap[a,b])<\varepsilon.
\end{gather}

Система интервалов
\begin{gather}\label{(xi-rxi,xi+rxi)system}
(\xi-r_\xi,\xi+r_\xi),\,\,\,\xi\in[a,b],
\end{gather}
образует покрытие отрезка $[a,b]$, и если
\begin{gather}\label{Delta_xi_segments}
\Delta_\xi\equiv[\xi-\frac12r_\xi,\xi+\frac12r_\xi]\cap[a,b],
\end{gather}
то
\begin{gather*}
\osc(f;\Delta_\xi)\leqslant\osc(f;(\xi-r_\xi,\xi+r_\xi)\cap[a,b])<\varepsilon.
\end{gather*}
Выделим, согласно лемме Гейне--Бореля, из покрытия (\ref{(xi-rxi,xi+rxi)system}) конечное подпокрытие
\begin{gather*}
(\xi_1-r_{\xi_1},\xi_1+r_{\xi_1}),\dots,(\xi_m-r_{\xi_m},\xi_m+r_{\xi_m})
\end{gather*}
и обозначим концы промежутков
\begin{gather*}
(\xi_j-r_{\xi_j},\xi_j+r_{\xi_j})\cap[a,b]
\end{gather*}
через $\alpha_j$ и $\beta_j$, $j=\overline{1,m}$.

Пусть $\tau=\{t_i\}_{i=0}^{i=i_\tau}$ --- разбиение отрезка $[a,b]$, состоящее из всех точек $\alpha_j$, $\beta_j$, $j=\overline{1,m}$. Каждый отрезок $[t_{i-1},t_i]$ этого разбиения имеет
одну из следующих форм: $[\alpha_j,\beta_j]$, $[\alpha_j,\alpha_k]$, $[\beta_j,\alpha_k]$, $[\beta_j,\beta_k]$, $j,\,k=\overline{1,m}$, и целиком содержится в одном из отрезков
$\Delta_{\xi_1}$, $\Delta_{\xi_m}$ (см. (\ref{Delta_xi_segments})). Иначе говоря, для каждого отрезка $[t_{i-1},t_i]$ существует такой отрезок $\Delta_{\xi_{j_i}}$, $1\leqslant j_i
\leqslant m$, что $[t_{i-1},t_i]\subset\Delta_{\xi_{j_i}}$. Поэтому
\begin{gather*}
\osc(f;[t_{i-1},t_i])\leqslant\osc(f;\Delta_{\xi_{j_i}})<\varepsilon.
\end{gather*}

Лемма доказана.
\end{Proof}


\begin{Corrolary}\label{sumoscifDelta_ti_lt_epsb-a::Corrolary}
В условиях леммы \ref{oscft_lt_eps::Lemma}
\begin{gather}\label{sumoscifDelta_ti_lt_epsb-a}
\sum\limits_{i=1}^{i_\tau}\osc(f;[t_{i-1},t_i])\Delta t_i<\varepsilon(b-a).
\end{gather}
\end{Corrolary}

\begin{Theorem}\label{DuBuaRaymond} (Дю Буа--Реймон)
Для интегрируемости ограниченной на отрезке $[a,b]$ функции $f\colon[a,b]\to X$ достаточно, чтобы для любого $\varepsilon>0$ и любого $\delta>0$ множество всех точек $t\in[a,b]$, в которых
$\osc(f;t)\geqslant\varepsilon$, можно было покрыть конечной системой интервалов с суммой длин, меньшей $\delta$.
\end{Theorem}
\begin{Proof}
Пусть для любых чисел $\varepsilon>0$ и $\delta>0$ множество $\mathcal{T}_\varepsilon=[a,b]_\varepsilon$ можно покрыть конечной системой интервалов, сумма длин которых меньше $\delta>0$.
Функция $f$ ограничена на отрезке $[a,b]$, поэтому существует такая постоянная $c>0$, что
\begin{gather}\label{norm_f(t)leq_c::DuBuaRaymond}
\|f(t)\|_X\leqslant c\,\,\,\forall\,t\in[a,b].
\end{gather}

Зададим произвольно $\varepsilon>0$ и возьмём $\delta=\frac{\varepsilon}{4c}$. Существует конечная система интервалов $(\alpha_i,\beta_i)$, $i=\overline{1,p}$, покрывающая множество
$\mathcal{T}_{\frac{\varepsilon}{2(b-a)}}$, с суммой длин, меньшей $\frac{\varepsilon}{4c}$:
\begin{gather}
\label{calTeps2b-asubset}
\mathcal{T}_{\frac{\varepsilon}{2(b-a)}}\subset\bigcup\limits_{i=1}^p(\alpha_i,\beta_i),\\
\label{total_length_alphabeta_lt}
\sum\limits_{i=1}^p(\beta_i-\alpha_i)<\frac{\varepsilon}{4c}.
\end{gather}

Объединение всех соответствующих отрезков $[\alpha_i,\beta_i]$, $i=\overline{1,p}$, можно представить в виде объединеня конечного множества отрезков $[\lambda_l,\mu_l]$, $l=\overline{1,m}$,
с непересекающимися попарно внутренностями и с концами $\lambda_l$, $\mu_l$, равными либо $\alpha_i$, либо $\beta_i$, либо $a$, либо $b$. Тогда в силу (\ref{total_length_alphabeta_lt})
\begin{gather*}
\sum\limits_{l=1}^m(\mu_l-\lambda_l)=\sum\limits_{i=1}^p(\beta_i-\alpha_i)<\frac{\varepsilon}{4c}.
\end{gather*}
Так как ввиду (\ref{norm_f(t)leq_c::DuBuaRaymond})
\begin{gather*}
\osc(f;[\lambda_l,\mu_l])<2c,
\end{gather*}
то
\begin{gather*}
\sum\limits_{l=1}^m\osc(f;[\lambda_l,\mu_l])(\mu_l-\lambda_l)<2c\sum\limits_{l=1}^m(\mu_l-\lambda_l)<\frac\varepsilon2.
\end{gather*}

Удалим из отрезка $[a,b]$ все точки, принадлежащие отрезкам $[\alpha_i,\beta_i]$, $i=\overline{1,p}$. Оставшееся множество представляет собой объединение конечного множества промежутков с
концами $\xi_j$ и $\eta_j$, $\xi_j<\eta_j$, $j=\overline{1,r}$, среди которых может быть не более двух полуинтервалов с концами $\xi_j=a$ или $\eta_j=b$, а все остальные являются
интервалами. При этом, согласно включению (\ref{calTeps2b-asubset}), пересечение каждого из отрезков $[\xi_j,\eta_j]$ с множеством $\mathcal{T}_{\frac{\varepsilon}{2(b-a)}}$ пусто.
Следовательно, в любой точке $t\in[\xi_j,\eta_j]$, $j=\overline{1,r}$, выполняется неравенство $\osc(f;t)<\frac{\varepsilon}{2(b-a)}$. Ясно также, что $\sum\limits_{j=1}^r(\eta_j-\xi_j)
\leqslant b-a$. Отсюда, в силу следствия \ref{sumoscifDelta_ti_lt_epsb-a::Corrolary}, вытекает, что для каждого отрезка $[\xi_j,\eta_j]$ существует такое его разбиение $\tau_j=\{
\zeta_{k_j}\}_{k_j=0}^{k_j=k_{\tau_j}}$, что
\begin{gather*}
\sum\limits_{k_j=1}^{k_{\tau_j}}\osc(f;[\zeta_{k_j-1},\zeta_{k_j}])(\zeta_{k_j}-\zeta_{k_j-1})<\frac{\varepsilon}{2(b-a)}(\eta_j-\xi_j).
\end{gather*}
Пусть теперь $\tau=\{\tau_\nu\}_{\nu=0}^{\nu=\nu_\tau}$ --- разбиение всего отрезка $[a,b]$, состоящее из всех точек $\lambda_l$, $\mu_l$,
$l=\overline{1,m}$, и точек $\zeta_{k_j}$, $k_j=\overline{1,k_{\tau_j}}$, $j=\overline{1,r}$. Тогда
\begin{gather*}
\sum\limits_{\nu=1}^{\nu_\tau}\osc(f;[t_{\nu-1},t_\nu])\Delta t_\nu=\sum\limits_{l=1}^m\osc(f;[\lambda_l,\mu_l])(\mu_l-\lambda_l)+
\sum\limits_{j=1}^r\sum\limits_{k_j=1}^{k_{\tau_j}}\osc(f;[\zeta_{k_j-1},\zeta_{k_j}])(\zeta_{k_j}-\zeta_{k_j-1})<\\
<\frac\varepsilon2+\frac{\varepsilon}{2(b-a)}\sum\limits_{j=1}^r(\eta_j-\xi_j)\leqslant\frac\varepsilon2+\frac\varepsilon2=\varepsilon.
\end{gather*}
Согласно теореме \ref{Riemann_integrability_sufficient_conditions_in_terms_of_oscillation} это означает, что функция $f$ интегрируема на отрезке $[a,b]$.
\end{Proof}

\begin{Theorem} (Лебег)
Для интегрируемости ограниченной на отрезке $[a,b]$ функции $f\colon[a,b]\to X$ достаточно, чтобы множество её точек разрыва было множеством лебеговой меры нуль.
\end{Theorem}
\begin{Proof}
Пусть множество $\mathcal{T}^*$ точек разрыва функции $f$, ограниченной на отрезке $\mathcal{T}=[a,b]$, является множеством лебеговой меры нуль. Зададим произвольно $\varepsilon>0$ и
$\delta>0$. Тогда существует не более чем счётная система интервалов, покрывающая множество $\mathcal{T}^*$, с суммой длин интервалов, меньшей $\delta$. Выберем натуральное число $j$ так,
чтобы $\frac1j<\varepsilon$. Указанная выше система интервалов, являясь покрытием множества $\mathcal{T}^*$, покрывает, в силу формулы $\mathcal{T}^*=\bigcup\limits_{k=1}^\infty
\mathcal{T}_{1/k}$ (см. (\ref{repres_of_set_of_discont_points})), множество $\mathcal{T}_{1/j}$, а следовательно, и множество $\mathcal{T}_{\varepsilon}$, ибо $\mathcal{T}_{\varepsilon}
\subset\mathcal{T}_{1/j}$ (см. (\ref{Teps_SUBSET_Teta})). Множество $\mathcal{T}_{\varepsilon}$ является ограниченным замкнутым множеством (см. следствие \ref{([a,b])_eps_is_compact_set}).
Поэтому, согласно лемме Гейне--Бореля, из рассматриваемой системы покрывающих его интервалов можно выделить конечную систему интервалов, по прежнему покрывающих множество
$\mathcal{T}_{\varepsilon}$, причём сумма длин входящих в неё интервалов (она не превосходит суммы длин всех интервалов исходной системы, покрывающей множество $\mathcal{T}^*$) меньше
$\delta$. В силу теоремы \ref{DuBuaRaymond} отсюда следует интегрируемость функции $f$.
\end{Proof}

            \subsection{Свойства интеграла}
\begin{Theorem}
Если $\varphi\colon[a,b]\to\mathbb{R}$ --- скалярная интегрируемая (в смысле Римана) на отрезке $[a,b]$ функция, а $x_0\in X$, то функция
$[0,T]\ni t\mapsto x_0\varphi(t)$ --- интегрируема на отрезке $[a,b]$, и
\begin{gather*}
\int\limits_a^bx_0\varphi(t)dt=x_0\int\limits_a^b\varphi(t)dt.
\end{gather*}
\end{Theorem}

\begin{Proof}
Положим $f(t)=x_0\varphi(t)$, $t\in[a,b]$. Далее, пусть $\tau=\{t_i\}_{i=0}^{i=i_\tau}$ --- некоторое разбиение отрезка $[a,b]$, и пусть
точки $\xi_i\in[t_{i-1},t_i]$, $i=\overline{1,i_\tau}$, --- произвольны. Тогда
\begin{gather*}
\sigma_\tau(f;\xi_1,\dots,\xi_{i_\tau})=\sum\limits_{i=1}^{i_\tau}f(\xi_i)\Delta t_i=\sum\limits_{i=1}^{i_\tau}x_0\varphi(\xi_i)\Delta t_i=
x_0\sum\limits_{i=1}^{i_\tau}\varphi(\xi_i)\Delta t_i=x_0\sigma_\tau(\varphi;\xi_1,\dots,\xi_{i_\tau}),
\end{gather*}
то есть
\begin{gather*}
\sigma_\tau(f;\xi_1,\dots,\xi_{i_\tau})=x_0\sigma_\tau(\varphi;\xi_1,\dots,\xi_{i_\tau}).
\end{gather*}
Поскольку функция $\varphi\colon[a,b]\to\mathbb{R}$ --- интегрируема (в смысле Римана) на отрезке $[a,b]$, то существует предел интегральных сумм $\sigma_\tau(\varphi;\xi_1,\dots,
\xi_{i_\tau})$ при $|\tau|\to0$, в силу чего существует и предел интегральных сумм $\sigma_\tau(f;\xi_1,\dots,\xi_{i_\tau})$ при $|\tau|\to0$. Переходя затем к пределу при $|\tau|\to0$,
получаем требуемое равенство.
\end{Proof}

\begin{Theorem}
Если функция $f\colon[a,b]\to X$ интегрируема на отрезке $[a,b]$, а $k\in\mathbb{R}$ --- константа, то функция $kf$ также интегрируема по отрезку $[a,b]$, причём
\begin{gather*}
\int\limits_a^bkf(t)dt=k\int\limits_a^bf(t)dt
\end{gather*}
\end{Theorem}
\begin{Proof}
Действительно, для любого разбиения $\tau$ справедливо равенство $\sigma_\tau(kf)=k\sigma_\tau(f)$, откуда и следует утверждение теоремы.
\end{Proof}

\begin{Theorem}
Если функции $f\colon[a,b]\to X$ и $g\colon[a,b]\to X$ интегрируемы на отрезке $[a,b]$, то функция $f+g$ также интегрируема по отрезку $[a,b]$, причём
\begin{gather*}
\int\limits_a^b[f(t)+g(t)]dt=\int\limits_a^bf(t)dt+\int\limits_a^bg(t)dt.
\end{gather*}
\end{Theorem}
\begin{Proof}
В самом деле, для любого разбиения $\tau$ имеет место равенство $\sigma_\tau(f+g)=\sigma_\tau(f)+\sigma_\tau(g)$, откуда и следует утверждение теоремы.
\end{Proof}


\begin{Theorem}
Если функция $f\colon[a,b]\to X$ п.в. на отрезке $[a,b]$ сильно непрерывна и ограничена на этом отрезке, то для всякого $c\in(a,b)$ она интегрируема по отрезкам $[a,c]$ и $[c,b]$, причём
\begin{gather*}
\int\limits_a^bf(t)dt=\int\limits_a^cf(t)dt+\int\limits_c^bf(t)dt.
\end{gather*}
\end{Theorem}
\begin{Proof}
Пусть $\tau_1$ --- разбиение отрезка $[a,c]$, $\tau_2$ --- разбиение отрезка $[c,b]$, а $\tau=\tau_1\cup\tau_2$ --- разбиение отрезка $[a,b]$. Тогда $\sigma_\tau(f)=\sigma_{\tau_1}(f)+
\sigma_{\tau_2}(f)$. Если $|\tau_1|\to0$ и $|\tau_2|\to0$, то и $|\tau|\to0$, и в пределе получаем нужное равенство.
\end{Proof}

\begin{Theorem}
Если функция $f\colon[a,b]\to X$ п.в. на отрезке $[a,b]$ сильно непрерывна и ограничена на этом отрезке, то функция $[a,b]\ni t\mapsto\|f(t)\|_X$ интегрируема в смысле Римана по отрезку
$[a,b]$, причём
\begin{gather*}
\left\|\int\limits_a^bf(t)dt\right\|_X\leqslant\int\limits_a^b\|f(t)\|_Xdt.
\end{gather*}
\end{Theorem}
\begin{Proof}
Поскольку функция $f\colon[a,b]\to X$ п.в. на отрезке $[a,b]$ сильно непрерывна  и ограничена на этом отрезке, то функция $g(t)=\|f(t)\|_X$, $t\in[a,b]$, непрерывна п.в. на отрезке $[a,b]$
и ограничена на этом отрезке, и, как следствие, в силу критерия Лебега интегрируемости числовых функций, интегрируема по Риману на отрезке $[a,b]$. Для завершения доказательства осталось
заметить, что для любого разбиения $\tau$ имеет место соотношение $\|\sigma_\tau(f)\|_X\leqslant\sigma_\tau(g)$.
\end{Proof}

\begin{Theorem}
Пусть для элементов пространства $X$ определено умножение справа на элементы банахова пространства $Y$. Тогда если функция $f\colon[a,b]\to X$ интегрируема на отрезке $[a,b]$, а $y\in Y$
--- константа, то функция $[a,b]\ni t\mapsto f(t)\bullet y$ также интегрируема по отрезку $[a,b]$, причём
\begin{gather*}
\int\limits_a^b[f(t)\bullet y]dt=\left[\int\limits_a^bf(t)dt\right]\bullet y.
\end{gather*}
\end{Theorem}
\begin{Proof}
Положим $g(t)=f(t)\bullet y$, $t\in[a,b]$. Далее, пусть $\tau=\{t_i\}_{i=0}^{i=i_\tau}$ --- некоторое разбиение отрезка $[a,b]$, и пусть
точки $\xi_i\in[t_{i-1},t_i]$, $i=\overline{1,i_\tau}$, --- произвольны. Тогда
\begin{gather*}
\sigma_\tau(g;\xi_1,\dots,\xi_{i_\tau})=\sum\limits_{i=1}^{i_\tau}[f(\xi_i)\bullet y]\Delta t_i=
\left[\sum\limits_{i=1}^{i_\tau}f(\xi_i)\Delta t_i\right]\bullet y=\sigma_\tau(f;\xi_1,\dots,\xi_{i_\tau})\bullet y,
\end{gather*}
то есть
\begin{gather*}
\sigma_\tau(g;\xi_1,\dots,\xi_{i_\tau})=\sigma_\tau(f;\xi_1,\dots,\xi_{i_\tau})\bullet y.
\end{gather*}
Поскольку функция $f\colon[a,b]\to X$ --- интегрируема (в смысле Римана) на отрезке $[a,b]$, то существует предел интегральных сумм $\sigma_\tau(f;\xi_1,\dots,\xi_{i_\tau})$ при
$|\tau|\to0$, в силу чего существует и предел интегральных сумм $\sigma_\tau(g;\xi_1,\dots,\xi_{i_\tau})$ при $|\tau|\to0$. Переходя затем к пределу при $|\tau|\to0$, получаем требуемое
равенство.
\end{Proof}

\begin{Theorem}
Пусть для элементов пространства $X$ определено умножение слева на элементы банахова пространства $Z$. Тогда если функция $f\colon[a,b]\to X$ интегрируема на отрезке $[a,b]$, а $z\in Z$
--- константа, то функция $[a,b]\ni t\mapsto z\bullet f(t)$ также интегрируема по отрезку $[a,b]$, причём
\begin{gather*}
\int\limits_a^b[z\bullet f(t)]dt=z\bullet\left[\int\limits_a^bf(t)dt\right].
\end{gather*}
\end{Theorem}
\begin{Proof}
Положим $g(t)=z\bullet f(t)$, $t\in[a,b]$. Далее, пусть $\tau=\{t_i\}_{i=0}^{i=i_\tau}$ --- некоторое разбиение отрезка $[a,b]$, и пусть точки $\xi_i\in[t_{i-1},t_i]$,
$i=\overline{1,i_\tau}$, --- произвольны. Тогда
\begin{gather*}
\sigma_\tau(g;\xi_1,\dots,\xi_{i_\tau})=\sum\limits_{i=1}^{i_\tau}[z\bullet f(\xi_i)]\Delta t_i=
z\bullet\left[\sum\limits_{i=1}^{i_\tau}f(\xi_i)\Delta t_i\right]=z\bullet\sigma_\tau(f;\xi_1,\dots,\xi_{i_\tau}),
\end{gather*}
то есть
\begin{gather*}
\sigma_\tau(g;\xi_1,\dots,\xi_{i_\tau})=z\bullet\sigma_\tau(f;\xi_1,\dots,\xi_{i_\tau}).
\end{gather*}
Поскольку функция $f\colon[a,b]\to X$ --- интегрируема (в смысле Римана) на отрезке $[a,b]$, то существует предел интегральных сумм $\sigma_\tau(f;\xi_1,\dots,\xi_{i_\tau})$ при
$|\tau|\to0$, в силу чего существует и предел интегральных сумм $\sigma_\tau(g;\xi_1,\dots,\xi_{i_\tau})$ при $|\tau|\to0$. Переходя затем к пределу при $|\tau|\to0$, получаем требуемое
равенство.
\end{Proof}


\begin{Theorem}
Пусть функция $f\colon[a,b]\to X$ ограничена на отрезке $[a,b]$ и п.в. на этом отрезке сильно непрерывна. Пусть $g(t)=\int\limits_a^tf(p)dp$, $t\in[a,b]$. Тогда функция $g$ сильно
непрерывна в каждой точке отрезка $[a,b]$.
\end{Theorem}
\begin{Proof}
Поскольку функция $f$ ограничена на отрезке $[a,b]$, то найдётся постоянная $c>0$, такая, что
\begin{gather*}
\|f(t)\|_X\leqslant c\,\,\,\,\forall\,t\in[a,b].
\end{gather*}
Поэтому для любого $t_0\in[a,b]$ и для всех $\Delta t$, $|\Delta t|\leqslant\min\{t_0-a,b-t_0\}$
\begin{gather*}
\|g(t_0+\Delta t)-g(t_0)\|_X=\left\|\int\limits_{t_0}^{t_0+\Delta t}f(p)dp\right\|_X\leqslant\left|\int\limits_{t_0}^{t_0+\Delta t}\|f(p)\|_Xdp\right|\leqslant c|\Delta t|,
\end{gather*}
ввиду чего
\begin{gather*}
\lim\limits_{\Delta t\to0}\|g(t_0+\Delta t)-g(t_0)\|_X=0.
\end{gather*}
Как следствие, функция $g$ сильно непрерывна в точке $t_0\in[a,b]$. Так как точка $t_0\in[a,b]$ выбрана произвольно, то функция $g$ сильно непрерывна в каждой точке отрезка $[a,b]$.
Теорема доказана.
\end{Proof}

\begin{Theorem}
Пусть функция $f\colon[a,b]\to X$ ограничена на отрезке $[a,b]$ и п.в. на этом отрезке сильно непрерывна. Пусть $g(t)=\int\limits_a^tf(p)dp$, $t\in[a,b]$. Тогда в точке $t_0\in[a,b]$
сильной непрерывности функции $f$ функция $g$ сильно дифференцируема, и $g'(t_0)=f(t_0)$.
\end{Theorem}
\begin{Proof}
Так как функция $f$ сильно непрывна в точке $t_0\in[a,b]$, то
\begin{gather}\label{fcontint_0::intatf(p)dp}
\forall\,\varepsilon>0\,\,\exists\,\delta=\delta(\varepsilon)>0\,\,\forall\,p\in[a,b],\,\,\,|p-t_0|<\delta:\|f(p)-f(t_0)\|_X<\varepsilon.
\end{gather}

Кроме того,
\begin{gather}\label{estnormdifffrac}
\left\|\frac{g(t_0+\Delta t)-g(t_0)}{\Delta t}-f(t_0)\right\|_X\leqslant
\frac1{|\Delta t|}\left|\int\limits_{t_0}^{t_0+\Delta t}\|f(p)-f(t_0)\|_Xdp\right|.
\end{gather}

Выберем произвольно $\varepsilon>0$ и подберём по нему $\delta=\delta(\varepsilon)>0$ согласно (\ref{fcontint_0::intatf(p)dp}), и пусть
$0<|\Delta t|<\delta$. Тогда, согласно (\ref{fcontint_0::intatf(p)dp}), для всех $p$, лежащих между $t_0$ и $t_0+\Delta_t$, имеем
\begin{gather*}
\|f(p)-f(t_0)\|_X\leqslant\varepsilon,
\end{gather*}
ввиду чего
\begin{gather*}
\left|\int\limits_{t_0}^{t_0+\Delta t}\|f(p)-f(t_0)\|_Xdp\right|\leqslant\varepsilon|\Delta t|.
\end{gather*}
Подставляя последнее неравенство в соотношение (\ref{estnormdifffrac}), заключаем, что
\begin{gather}\label{ineq::intatf(p)dp}
\left\|\frac{g(t_0+\Delta t)-g(t_0)}{\Delta t}-f(t_0)\right\|_X\leqslant\varepsilon.
\end{gather}

Иными словами, для любого $\varepsilon>0$ можно подобрать (согласно (\ref{fcontint_0::intatf(p)dp})) $\delta=\delta(\varepsilon)>0$ так, чтобы при всех $0<|\Delta t|<\delta$ выполнялось
неравенство (\ref{ineq::intatf(p)dp}). А это и означает, что функция $g$ сильно дифференцируема в точке $t_0$, и выполнено равенство $g'(t_0)=f(t_0)$.
\end{Proof}


\begin{Corrolary}
Если функция $f\colon[a,b]\to X$ сильно непрерывна в каждой точке отрезка $[a,b]$, то функция
\begin{gather*}
g(t)=\int\limits_a^tf(p)dp,\,\,\,t\in[a,b],
\end{gather*}
сильно непрерывно дифференцируема на отрезке $[a,b]$.
\end{Corrolary}


\begin{Theorem}
Пусть функция $f\colon[a,b]\to X$ сильно дифференцируема во всех точках отрезка $[a,b]$, а производная $f'\colon[a,b]\to X$ интегрируема в смысле Римана по отрезку $[a,b]$. Тогда
справедлива формула Ньютона--Лейбница
\begin{gather*}
\int\limits_a^bf'(t)dt=f(b)-f(a).
\end{gather*}
\end{Theorem}
\begin{Proof}
Для элементов пространства $X$ определено умножение справа на элементы сопряжённого пространства $X^*$. Поэтому для любого $x^*\in X^*$
\begin{gather*}
\left\langle\int\limits_a^bf'(t)dt,x^*\right\rangle=\int\limits_a^b\langle f'(t),x^*\rangle dt=
\int\limits_a^b\frac{d}{dt}\langle f(t),x^*\rangle dt=\langle f(t),x^*\rangle|_a^b=\langle f(b)-f(a),x^*\rangle,
\end{gather*}
поскольку для скалярных функций формула Ньютона--Лейбница имеет место. Следовательно, $\langle z,x^*\rangle=0$ для любого $x^*\in X^*$,
где $z=\int\limits_a^bf'(t)dt-f(b)-f(a)$. Это возможно только при $z=0$, что и требовалось доказать.
\end{Proof}


\begin{Theorem}
Пусть $\varphi\colon[\alpha,\beta]\to[a,b]$ --- непрерывно дифференцируемое строго монотонное отображение отрезка $\alpha\leqslant p\leqslant\beta$ в отрезок $a\leqslant t\leqslant b$, с
сооответствием концов $\varphi(\alpha)=a$, $\varphi(\beta)=b$ или $\varphi(\alpha)=b$, $\varphi(\beta)=a$. Тогда при любой функции $f\colon[a,b]\to X$, интегрируемой в смысле Римана на
отрезке $[a,b]$, функция $g(p)\equiv f(\varphi(p))\varphi'(p)$, $p\in[\alpha,\beta]$, интегрируема на отрезке $[\alpha,\beta]$ и справедливо равенство
\begin{gather}\label{zamena_peremennoj::Riemann}
\int\limits_{\varphi(\alpha)}^{\varphi(\beta)}f(t)dt=\int\limits_\alpha^\beta f(\varphi(p))\varphi'(p)dp.
\end{gather}
\end{Theorem}
\begin{Proof}
Поскольку $\varphi$ --- строго монотонное отображение отрезка $[\alpha,\beta]$ на отрезок $[a,b]$ с соответствием концов, то любое разбиение $\tau_p=\{p_i\}_{i=0}^m$ отрезка
$[\alpha,\beta]$ посредством образов $t_i=\varphi(p_i)$, $i=\overline{0,m}$, точек разбиения $\tau_p$ порождает разбиение $\tau_t$ отрезка $[a,b]$, которое можно условно обозначить
$\varphi(\tau_p)$. При этом $t_0=a$, если $\varphi(\alpha)=a$, и $t_0=b$, если $\varphi(\alpha)=b$. Из равномерной непрерывности функции $\varphi$ на отрезке $[\alpha,\beta]$ следует, что
если $|\tau_p|\to0$, то величина $|\tau_t|$ также стремится к нулю. Произвольно выберем отмеченные точки $\xi_i$, $i=\overline{1,m}$, разбиения $\tau_t$.

Используя теорему Лагранжа, преобразуем интегральную сумму $\sigma_{\tau_t}(f;\xi_1,\dots,\xi_m)$ следующим образом:
\begin{gather*}
\sum\limits_{i=1}^mf(\xi_i)\Delta t_i=\sum\limits_{i=1}^mf(\xi_i)(t_i-t_{i-1})=
\sum\limits_{i=1}^mf(\varphi(\eta_i))\varphi'(\bar\eta_i)(p_i-p_{i-1})=\sum\limits_{i=1}^mf(\varphi(\eta_i))\varphi'(\bar\eta_i)\Delta p_i.
\end{gather*}

Здесь $t_i=\varphi(p_i)$, $\xi_i=\varphi(\eta_i)$, $\xi_i$ лежит на отрезке с концами $t_{i-1}$ и $t_i$, а точки $\eta_i$ и $\bar\eta_i$ лежат на отрезке с концами $p_{i-1}$ и $p_i$.

Далее,
\begin{gather*}
\sum\limits_{i=1}^mf(\varphi(\eta_i))\varphi'(\bar\eta_i)\Delta p_i=\sum\limits_{i=1}^mf(\varphi(\eta_i))\varphi'(\eta_i)\Delta p_i+
\sum\limits_{i=1}^mf(\varphi(\eta_i))[\varphi'(\bar\eta_i)-\varphi'(\eta_i)]\Delta p_i.
\end{gather*}

Оценим последнюю сумму. Поскольку функция $f$ интегрируема в смысле Римана на отрезке $[a,b]$, то она ограничена на этом отрезке, т.е.
найдётся постоянная $c>0$, такая, что $\|f(t)\|_X\leqslant c$ для всех $t\in[a,b]$. Поэтому
\begin{gather*}
\left\|\sum\limits_{i=1}^mf(\varphi(\eta_i))[\varphi'(\bar\eta_i)-\varphi'(\eta_i)]\Delta p_i\right\|_X\leqslant
c\sum\limits_{i=1}^m\osc(\varphi';\Delta_i)\Delta p_i,
\end{gather*}
где $\Delta_i$ --- отрезок с концами $p_{i-1}$ и $p_i$.

Последняя сумма стремится к нулю при $|\tau_p|\to0$, поскольку $\varphi'$ --- непрерывная на отрезке $[\alpha,\beta]$ вещественнозначная функция.

Таким образом, мы показали, что
\begin{gather*}
\sum\limits_{i=1}^mf(\xi_i)\Delta t_i=\sum\limits_{i=1}^mf(\varphi(\eta_i))\varphi'(\eta_i)\Delta p_i+\gamma,
\end{gather*}
где $\gamma\to0$ при $|\tau_p|\to0$. Как уже отмечалось, если $|\tau_p|\to0$, то и $|\tau_t|\to0$. Так как функция $f$ интегрируема в смысле Римана на отрезке $[a,b]$, то при
$|\tau_t|\to0$ сумма в левой части последнего равенства стремится к интегралу $\int\limits_{\varphi(\alpha)}^{\varphi(\beta)}f(t)dt$. Значит, при $|\tau_p|\to0$ и сумма в правой части
этого равенства имеет (и притом тот же) предел.

Но сумму $\sum\limits_{i=1}^mf(\varphi(\eta_i))\varphi'(\eta_i)\Delta p_i$ можно считать совершенно произвольной интегральной суммой функции $g$, соответствующей разбиению $\tau_p$, с
отмеченными точками $\eta_1,\dots,\eta_m$, поскольку, ввиду строгой монотонности функции $\varphi$, любой набор точек $\eta_1,\dots,\eta_m$ можно получить из некоторого соответствующего
ему набора $\xi_1,\dots,\xi_m$ отмеченных точек разбиения $\tau_t=\varphi(\tau_p)$.

Таким образом, предел этой суммы есть, по определению, интеграл от функции $g$ по отрезку $[\alpha,\beta]$, и мы доказали одновременно как
интегрируемость функции $g$, так и формулу (\ref{zamena_peremennoj::Riemann}).
\end{Proof}


\begin{Theorem}
Пусть для элементов банахова пространства $X$ определено умножение справа на элементы банахова пространства $Y$. Пусть, кроме того, функции $f\colon[a,b]\to X$ и $g\colon[a,b]\to Y$ сильно
дифференцируемы всюду на отрезке $[a,b]$, а функции $f'$ и $g'$ ограничены на отрезке $[a,b]$ и почти всюду на этом отрезке сильно непрерывны. Тогда
\begin{gather*}
\int\limits_a^b[f(t)\bullet g'(t)]dt=[f(t)\bullet g(t)]|_a^b-\int\limits_a^b[f'(t)\bullet g(t)]dt.
\end{gather*}
\end{Theorem}
\begin{Proof}
Поскольку функции $f$ и $g$ сильно дифференцирумы всюду на отрезке $[a,b]$, то функция $h(t)\equiv f(t)\bullet g(t)$, $t\in[a,b]$, при всех $t\in[a,b]$ сильно дифференцируема, причём
\begin{gather*}
\frac{dh(t)}{dt}=f'(t)\bullet g(t)+f(t)\bullet g'(t),\,\,\,t\in[a,b].
\end{gather*}
В силу данной формулы и условий на функции $f$, $g$, $f'$, $g'$ получаем, что функция $h'$ ограничена на отрезке $[a,b]$ и почти всюду на этом отрезке сильно непрерывна, ввиду чего функция
$h'$ интегрируема в смысле Римана по отрезку $[a,b]$. Поэтому применима формула Ньютона--Лейбница, согласно которой
\begin{gather*}
\int\limits_{a}^bh'(t)dt=h(t)|_a^b.
\end{gather*}
Подставляя сюда выражение для функции $h$, будем иметь
\begin{gather*}
\int\limits_{a}^b[f'(t)\bullet g(t)+f(t)\bullet g'(t)]dt=[f(t)\bullet g(t)]|_a^b,\,\,\,
\int\limits_{a}^b[f'(t)\bullet g(t)]dt+\int\limits_{a}^b[f(t)\bullet g'(t)]dt=[f(t)\bullet g(t)]|_a^b,\\
\int\limits_a^b[f(t)\bullet g'(t)]dt=[f(t)\bullet g(t)]|_a^b-\int\limits_a^b[f'(t)\bullet g(t)]dt.
\end{gather*}
Теорема доказана.
\end{Proof}

        \section{Интеграл Бохнера}
Материал настоящего раздела взят из \cite[глава V, с.187--194]{iosida} и \cite{Edwards}. Всюду в данном разделе $X$ --- сепарабельное банахово пространство с нормой $\|\cdot\|_X$.
            \subsection{Определение интеграла Бохнера}
Дадим следующее определение.
\begin{Definition}
Пусть $(\mathfrak{S},\mathfrak{B},\mu)$ --- положительное пространство с мерой, и пусть $f:\mathfrak{S}\to X$ --- некоторое отображение.

Это отображение называется \textbf{слабо $\mathfrak{B}$--измеримым}, если для любого элемента $x^*\in X^*$ числовая функция
$\mathfrak{S}\ni\mathfrak{s}\mapsto\langle f(\mathfrak{s}),x^*\rangle$ является $\mathfrak{B}$--измеримой.

Отображение $f$ называется \textbf{простым}, если оно принимает постоянные отличные от нуля значения на каждом из множеств $B_j$, образующих конечную систему непересекающихся
$\mathfrak{B}$--измеримых множеств, причём $\mu(B_j)<+\infty$ и $f(\mathfrak{s})=0$ для $\mathfrak{s}\in\mathfrak{S}\setminus\bigcup\limits_jB_j$.

Отображение $f$ называется \textbf{сильно $\mathfrak{B}$--измеримым}, если существует последовательность простых отображений $f_k$, $k=1,2,\dots$, такая, что
$\lim\limits_{k\to\infty}\|f_k(\mathfrak{s})-f(\mathfrak{s})\|_X=0$ при $\mu$--п.в. $\mathfrak{s}\in\mathfrak{S}$.
\end{Definition}

\begin{Theorem}(Петтис)
Для того, чтобы функция $f:\mathfrak{S}\to X$ была сильно $\mathfrak{B}$--измеримой, необходимо и достаточно, чтобы она была слабо  $\mathfrak{B}$--измеримой.
\end{Theorem}

Пусть на пространстве с мерой $(\mathfrak{S},\mathfrak{B},\mu)$ задана простая функция $f$, принимающая значения в пространстве $X$. Пусть $f$ принимает значения $x_i\neq0$,
$i=\overline{1,r}$, на множествах $B_i\in\mathfrak{B}_i$, $i=\overline{1,r}$, соответственно, все множества $B_i$ попарно не пересекаются и $\mu(B_i)$, $i=\overline{1,r}$. Пусть, кроме
того, $f(\mathfrak{s})=0$ для $\mathfrak{s}\in\mathfrak{S}\setminus\bigcup\limits_{j=1}^rB_j$. Тогда \textbf{интегралом (Бохнера) функции $f$ по множеству $\mathfrak{S}$} называется
сумма $\sum\limits_{j=1}^rx_j\mu(B_j)$, и эта сумма обозначается через $(\textrm{Б})\int\limits_{\mathfrak{S}}f(\mathfrak{s})\mu(d\mathfrak{s})$ или
через $\int\limits_{\mathfrak{S}}f(\mathfrak{s})\mu(d\mathfrak{s})$.

\begin{Definition}
Функция $f:\mathfrak{S}\to X$, называется \textbf{$\mu$--интегрируемой по Бохнеру}, если существует последовательность простых функций
$f_k$, $k=1,2,\dots$, сильно сходящаяся к $f$ при $\mu$--п.в. $\mathfrak{s}\in\mathfrak{S}$, и такая, что
\begin{gather}\label{Bocher.integrability.def}
\lim\limits_{k\to\infty}\int\limits_{\mathfrak{S}}\|f_k(\mathfrak{s})-f(\mathfrak{s})\|_X\mu(d\mathfrak{s})=0.
\end{gather}
Тогда предел
\begin{gather}\label{Bocher.integral:limit}
\lim\limits_{k\to\infty}^X\left[(\textrm{Б})\int\limits_{\mathfrak{S}}f_k(\mathfrak{s})\mu(d\mathfrak{s})\right]
\end{gather}
называется \textbf{интегралом (Бохнера) функции $f$ по множеству $\mathfrak{S}$} и обозначается либо
как $\int\limits_{\mathfrak{S}}f(\mathfrak{s})\mu(d\mathfrak{s})$, либо как $(\textrm{Б})\int\limits_{\mathfrak{S}}f(\mathfrak{s})\mu(d\mathfrak{s})$.

При этом для любого $B\in\mathfrak{B}$ по определению положим
\begin{gather}\label{Bocher.integral:limit.B}
(\textrm{Б})\int\limits_{B}f(\mathfrak{s})\mu(d\mathfrak{s})=\lim\limits_{k\to\infty}^X\left[(\textrm{Б})\int\limits_{\mathfrak{S}}\chi_B(\mathfrak{s})
f_k(\mathfrak{s})\mu(d\mathfrak{s})\right],
\end{gather}
где $\chi_B$ --- характеристическая функция множества $B$.
\end{Definition}

\begin{Theorem}
Определение интеграла Бохнера корректно.
\end{Theorem}
\begin{Proof}
Так как функция $f$ сильно $\mathfrak{B}$--измерима, то условие (\ref{Bocher.integrability.def}) имеет смысл.

Покажем, что предел (\ref{Bocher.integral:limit.B}) существует. В самом деле, при всех $k$, $j\geqslant1$
\begin{gather*}
\left\|(\textrm{Б})\int\limits_{\mathfrak{S}}\chi_B(\mathfrak{s})f_k(\mathfrak{s})\mu(d\mathfrak{s})-(\textrm{Б})\int\limits_{\mathfrak{S}}\chi_B(\mathfrak{s})f_j(\mathfrak{s}
)\mu(d\mathfrak{s})\right\|_X=\left\|(\textrm{Б})\int\limits_{\mathfrak{S}}\chi_B(\mathfrak{s})[f_k(\mathfrak{s})-f_j(\mathfrak{s})]\mu(d\mathfrak{s})\right\|_X\leqslant\\
\leqslant\int\limits_{B}\|f_k(\mathfrak{s})-f_j(\mathfrak{s})\|_X\mu(d\mathfrak{s})\leqslant\int\limits_{B}\|f_k(\mathfrak{s})-f(\mathfrak{s})\|_X\mu(d\mathfrak{s})+
\int\limits_{B}\|f(\mathfrak{s})-f_j(\mathfrak{s})\|_X\mu(d\mathfrak{s}).
\end{gather*}
В силу условия (\ref{Bocher.integrability.def}) отсюда вытекает фундаментальность последовательности
\begin{gather*}
\int\limits_{\mathfrak{S}}\chi_B(\mathfrak{s})f_k(\mathfrak{s})\mu(d\mathfrak{s}),\,\,\,k=1,2,\dots,
\end{gather*}
в норме пространства $X$. Поскольку же $X$ полно, то у этой последовательности существует предел. Иными словами, предел (\ref{Bocher.integral:limit.B}) существует.

Докажем теперь, что этот предел не зависит от выбора последовательности $f_k$, $k=1,2,\dots$ В самом деле, пусть $f'_k$, $k=1,2,\dots$, и $f''_k$, $k=1,2,\dots$, --- две последовательности
из определения интеграла, и пусть
\begin{gather*}
\lim\limits_{k\to\infty}^X\left[(\textrm{Б})\int\limits_{\mathfrak{S}}\chi_B(\mathfrak{s})f'_k(\mathfrak{s})\mu(d\mathfrak{s})\right]=A,\,\,\,
\lim\limits_{k\to\infty}^X\left[(\textrm{Б})\int\limits_{\mathfrak{S}}\chi_B(\mathfrak{s})f''_k(\mathfrak{s})\mu(d\mathfrak{s})\right]=B.
\end{gather*}
Составив последовательность
\begin{gather*}
\tilde{f}_k=
\begin{cases}
f'_m,\text{ $k=2m-1$, $m=1,2,\dots$;}\cr
f''_m,\text{ $k=2m$, $m=1,2,\dots$;}
\end{cases}
\end{gather*}
получим, что для неё тоже существует предел (\ref{Bocher.integral:limit.B}), причём этот предел равен как $A$, так и $B$. Следовательно, $A=B$. Теорема доказана.
\end{Proof}

\begin{Theorem} (Бохнер) Для того, чтобы сильно $\mathfrak{B}$--измеримая функция была $\mu$--интегрируемой по Бохнеру, необходимо и достаточно, чтобы функция
$\mathfrak{S}\ni\mathfrak{s}\mapsto\|f(\mathfrak{s})\|_X$ была $\mu$--интегрируемой.
\end{Theorem}
\begin{Proof}
1) Докажем необходимость. Нетрудно видеть, что
\begin{gather*}
\|f(\mathfrak{s})\|_X\leqslant\|f_k(\mathfrak{s})\|_X+\|f(\mathfrak{s})-f_k(\mathfrak{s})\|_X.
\end{gather*}
Поэтому из условия (\ref{Bocher.integrability.def}) и $\mu$--интегрируемости функции
$$
\mathfrak{S}\ni\mathfrak{s}\mapsto\|f_k(\mathfrak{s})\|_X
$$
следует, что функция
$$
\mathfrak{S}\ni\mathfrak{s}\mapsto\|f(\mathfrak{s})\|_X
$$
тоже $\mu$--интегрируема, и
\begin{gather*}
\int\limits_B\|f(\mathfrak{s})\|_X\mu(d\mathfrak{s})\leqslant\int\limits_B\|f_k(\mathfrak{s})\|_X\mu(d\mathfrak{s})+\int\limits_B\|f(\mathfrak{s})-f_k(\mathfrak{s})\|_X\mu(d\mathfrak{s}).
\end{gather*}
Поскольку же при всех $k$, $j\geqslant1$
\begin{gather*}
\left|\int\limits_B\|f_k(\mathfrak{s})\|_X\mu(d\mathfrak{s})-\int\limits_B\|f_j(\mathfrak{s})\|_X\mu(d\mathfrak{s})\right|\leqslant
\int\limits_B|\,\|f_k(\mathfrak{s})\|_X-\|f_j(\mathfrak{s})\|_X|\mu(d\mathfrak{s})\leqslant\int\limits_{B}\|f_k(\mathfrak{s})-f_j(\mathfrak{s})\|_X\mu(d\mathfrak{s}),
\end{gather*}
то, ввиду условия (\ref{Bocher.integrability.def}), существует предел
\begin{gather*}
\lim\limits_{k\to\infty}\int\limits_{B}\|f_k(\mathfrak{s})\|_X\mu(d\mathfrak{s}),
\end{gather*}
причём
\begin{gather*}
\int\limits_{B}\|f(\mathfrak{s})\|_X\mu(d\mathfrak{s})\leqslant\varliminf\limits_{k\to\infty}\int\limits_{B}\|f_k(\mathfrak{s})\|_X\mu(d\mathfrak{s}).
\end{gather*}

2) Докажем достаточность. Пусть $f_k$, $k=1,2,\dots$,  --- произвольная последовательность простых функций, такая, что $\lim\limits_{k\to\infty}\|f_k(\mathfrak{s})-f(\mathfrak{s})\|_X=0$
при $\mu$--п.в. $\mathfrak{s}\in\mathfrak{S}$. Введём вспомогательные функции $g_k$, $k=1,2,\dots$, следующим образом:
\begin{gather*}
g_k(\mathfrak{s})=
\left\{\begin{array}{ll}
f_k(\mathfrak{s}), & \text{ если $\|f_k(\mathfrak{s})\|_X\leqslant\|f(\mathfrak{s})\|_X[1+\frac1k]$;}\\
0                , & \text{ если $\|f_k(\mathfrak{s})\|_X>\|f(\mathfrak{s})\|_X[1+\frac1k]$.}\\
\end{array}\right.
\end{gather*}
Тогда
\begin{gather*}
\|g_k(\mathfrak{s})\|_X\leqslant\|f(\mathfrak{s})\|_X[1+\frac1k]
\end{gather*}
и $\lim\limits_{k\to\infty}\|g_k(\mathfrak{s})-f(\mathfrak{s})\|_X=0$ при $\mu$--п.в. $\mathfrak{s}\in\mathfrak{S}$. Так как функция $\mathfrak{S}\ni\mathfrak{s}\mapsto
\|f(\mathfrak{s})\|_X$ --- $\mu$--интегрируема и $\|g_k(\mathfrak{s})-f(\mathfrak{s})\|_X\leqslant2\|f(\mathfrak{s})\|_X[1+\frac1k]$, то, в силу теоремы Лебега о предельном переходе под
знаком интеграла Лебега,
\begin{gather*}
\lim\limits_{k\to\infty}\int\limits_{\mathfrak{S}}\|g_k(\mathfrak{s})-f(\mathfrak{s})\|_X\mu(d\mathfrak{s})=0.
\end{gather*}
А это и означает, что функция $f$ интегрируема по Бохнеру. Теорема полностью доказана.
\end{Proof}

            \subsection{Свойства интеграла Бохнера}
\begin{Corrolary}
Если функция $f:\mathfrak{S}\to X$ $\mu$--интегрируема по Бохнеру, то для любого $B\in\mathfrak{B}$
\begin{gather*}
\left\|\int\limits_Bf(\mathfrak{s})\mu(d\mathfrak{s})\right\|_X\leqslant\int\limits_{B}\|f(\mathfrak{s})\|_X\mu(d\mathfrak{s}).
\end{gather*}
\end{Corrolary}

Из этого следствия и абсолютной непрерывности интеграла Лебега вытекает
\begin{Corrolary}
Интеграл Бохнера --- $\mu$--абсолютно--непрерывен.
\end{Corrolary}

Из линейности операции предельного перехода вытекает
\begin{Corrolary}
Пусть функции $f:\mathfrak{S}\to X$ и $g:\mathfrak{S}\to X$ --- $\mu$--интегрируемы по Бохнеру, а $\lambda\in\mathbb{R}$. Тогда
для всех $B\in\mathfrak{B}$
\begin{gather*}
\int\limits_B[f(\mathfrak{s})+g(\mathfrak{s})]\mu(d\mathfrak{s})=\int\limits_Bf(\mathfrak{s})\mu(d\mathfrak{s})+\int\limits_Bg(\mathfrak{s})\mu(d\mathfrak{s}),\,\,\,
\int\limits_B[\lambda f(\mathfrak{s})]\mu(d\mathfrak{s})=\lambda\int\limits_Bf(\mathfrak{s})\mu(d\mathfrak{s}).
\end{gather*}
\end{Corrolary}

\begin{Corrolary}
Пусть $Y$ и $Z$ --- банаховы пространства, функции $f:\mathfrak{S}\to X$ и $g:\mathfrak{S}\to Y$ --- $\mu$--интегрируемы по Бохнеру, $x_0\in X$ и $y_0\in Y$ --- константы, и пусть
определено умножение $\bullet$ элементов пространств $X$ и $Y$, принимающее значения в $Z$. Тогда для всех $B\in\mathfrak{B}$
\begin{gather*}
\int\limits_B[f(\mathfrak{s})\bullet y_0]\mu(d\mathfrak{s})=\left[\int\limits_Bf(\mathfrak{s})\mu(d\mathfrak{s})\right]\bullet y_0,\,\,\,
\int\limits_B[x_0\bullet g(\mathfrak{s})]\mu(d\mathfrak{s})=x_0\bullet\left[\int\limits_Bg(\mathfrak{s})\mu(d\mathfrak{s})\right].
\end{gather*}
\end{Corrolary}


Пусть $\mathfrak{S}\equiv[t_0,t_1]$ --- отрезок числовой оси, $\mathfrak{B}$ --- $\sigma$--алгебра измеримых по Лебегу подмножеств этого отрезка, $\mu$ --- мера Лебега на числовой оси.
Тогда если функция $f:[t_0,t_1]\to X$ --- $\mu$--интегрируема, то для всех $t$, $\tau\in[t_0,t_1]$ положим по определению
\begin{gather*}
\int\limits_t^\tau f(\xi)d\xi=
\begin{cases}
\int\limits_{[t,\tau]}f(\xi)d\xi,\,\,\,\text{если $t\leqslant\tau$;}\cr
-\int\limits_{[\tau,t]}f(\xi)d\xi,\,\,\,\text{если $t\geqslant\tau$.}\cr
\end{cases}
\end{gather*}



\begin{Theorem}
Пусть $\mathfrak{S}\equiv[t_0,t_1]$ --- отрезок числовой оси, $\mathfrak{B}$ --- $\sigma$--алгебра измеримых по Лебегу подмножеств этого отрезка, $\mu$ --- мера Лебега на числовой оси.
Тогда если функция $f:[t_0,t_1]\to X$ --- $\mu$--интегрируема по Бохнеру, то функция
\begin{gather*}
[t_0,t_1]\ni t\mapsto(\textrm{Б})\int\limits_{t_0}^tf(\xi)d\xi
\end{gather*}
сильно дифференцируема при п.в. $t\in[t_0,t_1]$ и справедливо равенство
\begin{gather*}
\frac{d}{dt}\left[(\textrm{Б})\int\limits_{t_0}^tf(\xi)d\xi\right]=f(t)\text{ при п.в. $t\in[t_0,t_1]$.}
\end{gather*}
\end{Theorem}
\begin{Proof}
Введём обозначение
\begin{gather*}
\Phi(t)\equiv(\textrm{Б})\int\limits_{t_0}^tf(\xi)d\xi,\,\,\,t\in[t_0,t_1].
\end{gather*}
Таким образом, нам нужно показать, что функция $\Phi$ при п.в. $t\in[t_0,t_1]$ имеет сильную производную $\Phi'(t)$ и $\Phi'(t)=f(t)$.

Пусть $t$, $t+\Delta t\in[t_0,t_1]$. Тогда
\begin{gather*}
\frac{\Phi(t+\Delta t)-\Phi(t)}{\Delta t}=\frac1{\Delta t}\left[\int\limits_{t_0}^{t+\Delta t}f(\xi)d\xi-\int\limits_{t_0}^{t}f(\xi)d\xi\right]=
\frac1{\Delta t}\int\limits_{t}^{t+\Delta t}f(\xi)d\xi.
\end{gather*}

Пусть $f_k$, $k=1,2,\dots$, --- последовательность простых функций, такая, что
\begin{gather*}
\|f_k(t')\|_X\leqslant\|f(t')\|_X\left[1+\frac1k\right]
\end{gather*}
и $\lim\limits_{k\to\infty}\|f_k(t')-f(t')\|_X=0$ при п.в. $t'\in[t_0,t_1]$. Нетрудно видеть, что
\begin{gather*}
\frac1{\Delta t}\int\limits_{t}^{t+\Delta t}f(\xi)d\xi-f(t)=\frac1{\Delta t}\int\limits_{t}^{t+\Delta t}[f(\xi)-f_k(\xi)]d\xi+
\left[\frac1{\Delta t}\int\limits_{t}^{t+\Delta t}f_k(\xi)d\xi-f_k(t)\right]+[f_k(t)-f(t)].
\end{gather*}
Поэтому
\begin{gather*}
\left\|\frac1{\Delta t}\int\limits_{t}^{t+\Delta t}f(\xi)d\xi-f(t)\right\|_X\leqslant\left|\frac1{\Delta t}\int\limits_{t}^{t+\Delta t}
\|f(\xi)-f_k(\xi)\|_Xd\xi\right|+\left\|\frac1{\Delta t}\int\limits_{t}^{t+\Delta t}f_k(\xi)d\xi-f_k(t)\right\|_X+\|f_k(t)-f(t)\|_X.
\end{gather*}
Поскольку $f_k$ --- простая функция, то второе слагаемое в правой части данного равенства почти всюду равно нулю. Таким образом,
\begin{gather*}
\left\|\frac1{\Delta t}\int\limits_{t}^{t+\Delta t}f(\xi)d\xi-f(t)\right\|_X\leqslant\left|\frac1{\Delta t}\int\limits_{t}^{t+\Delta t}\|f(\xi)-f_k(\xi)\|_Xd\xi\right|+\|f_k(t)-f(t)\|_X.
\end{gather*}
Следовательно,
\begin{gather*}
\varlimsup\limits_{\Delta t\to0}\left\|\frac1{\Delta t}\int\limits_{t}^{t+\Delta t}f(\xi)d\xi-f(t)\right\|_X\leqslant
\varlimsup\limits_{\Delta t\to0}\left|\frac1{\Delta t}\int\limits_{t}^{t+\Delta t}\|f(\xi)-f_k(\xi)\|_Xd\xi\right|+\|f_k(t)-f(t)\|_X.
\end{gather*}
Так как функция $[t_0,t_1]\ni\xi\mapsto\|f(\xi)-f_k(\xi)\|_X$ интегрируема в смысле Лебега, то предел в правой части данного неравенства почти всюду равен $\|f_k(t)-f(t)\|_X$.

Итак, при п.в. $t\in[t_0,t_1]$
\begin{gather*}
\varlimsup\limits_{\Delta t\to0}\left\|\frac1{\Delta t}\int\limits_{t}^{t+\Delta t}f(\xi)d\xi-f(t)\right\|_X\leqslant2\|f_k(t)-f(t)\|_X.
\end{gather*}
Устремляя затем $k$ к бесконечности, получаем, что при п.в. $t\in[t_0,t_1]$
\begin{gather*}
\lim\limits_{\Delta t\to0}\left\|\frac1{\Delta t}\int\limits_{t}^{t+\Delta t}f(\xi)d\xi-f(t)\right\|_X=0.
\end{gather*}
Теорема доказана.
\end{Proof}

\begin{Theorem}
Пусть функция $f:[0,T]\to X$ интегрируема в смысле Римана. Тогда она интегрируема и в смысле Бохнера, и её интегралы Римана и Бохнера совпадают.
\end{Theorem}
\begin{Proof}
В самом деле, поскольку функция $f$ интегрируема в смысле Римана, то при всех $x^*\in X^*$ интегрируема в смысле Римана же числовая функция
\begin{gather}\label{Rim.Leb!m1}
[0,T]\ni t\mapsto\langle f(t),x^*\rangle.
\end{gather}
Следовательно, при всех $x^*\in X^*$ функция (\ref{Rim.Leb!m1}) измерима в смысле Лебега и интегрируема в смысле Лебега. Значит функция $f$ сильно измерима и интегрируема в смысле Бохнера.
Поэтому
\begin{gather*}
\left\langle(\text{Р})\int\limits_0^Tf(t)dt,x^*\right\rangle=(\text{Л})\langle f(t),x^*\rangle dt=\left\langle(\text{Б})\int\limits_0^Tf(t)dt,x^*\right\rangle.
\end{gather*}
Таким образом, при всех $x^*\in X^*$
\begin{gather*}
\left\langle(\text{Р})\int\limits_0^Tf(t)dt-(\text{Б})\int\limits_0^Tf(t)dt,x^*\right\rangle=0.
\end{gather*}
А это и означает, что
\begin{gather*}
(\text{Р})\int\limits_0^Tf(t)dt=(\text{Б})\int\limits_0^Tf(t)dt.
\end{gather*}
Теорема доказана.
\end{Proof}

            \subsection{Пространства измеримых функций}
Пусть $1\leqslant p\leqslant\infty$, а  $(\mathfrak{S},\mathfrak{B},\mu)$ --- положительное пространство с мерой. Через $\mathfrak{L}_p((\mathfrak{S},\mathfrak{B},\mu),X)$ обозначаем
множество всех сильно $\mathfrak{B}$--измеримых функций $f:\mathfrak{S}\to X$, таких, что
\begin{gather*}
\int\limits_{\mathfrak{S}}\|f(\mathfrak{s})\|_X^p\mu(d\mathfrak{s})<+\infty\text{ при $1\leqslant p<\infty$;}\,\,\,
\mu\text{--}\vraisup\limits_{\mathfrak{s}\in\mathfrak{S}}\|f(\mathfrak{s})\|_X<+\infty\text{ при $p=\infty$}.
\end{gather*}
Нетрудно видеть, что $\mathfrak{L}_p((\mathfrak{S},\mathfrak{B},\mu),X)$ --- линейное пространство.

Далее, через $L_p((\mathfrak{S},\mathfrak{B},\mu),X)$ обозначим множество классов эквивалентных (в смысле равенства $\mu$--п.в.) функций из
$\mathfrak{L}_p((\mathfrak{S},\mathfrak{B},\mu),X)$. Положим
\begin{gather*}
\|f\|_{p,(\mathfrak{S},\mathfrak{B},\mu),X}\equiv\left[\int\limits_{\mathfrak{S}}\|f(\mathfrak{s})\|_X^p\mu(d\mathfrak{s})\right]^{1/p}\text{ при $1\leqslant p<\infty$;}\,\,\,
\|f\|_{p,(\mathfrak{S},\mathfrak{B},\mu),X}\equiv\mu\text{--}\vraisup\limits_{\mathfrak{s}\in\mathfrak{S}}\|f(\mathfrak{s})\|_X
\text{ при $p=\infty$}.
\end{gather*}
Несложно показать, что $\|\cdot\|_{p,(\mathfrak{S},\mathfrak{B},\mu),X}$ --- норма в $L_p((\mathfrak{S},\mathfrak{B},\mu),X)$, и что
пространство $L_p((\mathfrak{S},\mathfrak{B},\mu),X)$, наделённое этой нормой, --- полно.

В завершение настоящего раздела приведём следующий результат.
\begin{Lemma}\label{Cs[0,T]X:sup::vraisup}Если $f\in C_s([0,T],X)$, то $f\in L_\infty([0,T],X)$, причём
$$
\sup\limits_{t\in[0,T]}\|f(t)\|_X=\vraisup\limits_{t\in[0,T]}\|f(t)\|_X.
$$
\end{Lemma}
\begin{Proof}
Поскольку $f$ слабо непрерывна на отрезке $[0,T]$, то она слабо измерима на этом отрезке. Так как пространство $X$ --- сепарабельно, то из слабой измеримости следует сильная измеримость.
Из этого обстоятельства и леммы \ref{Cs[0,T]X::finiteness.of.sup.of.normX} вытекает, что
$$
\vraisup\limits_{t\in[0,T]}\|f(t)\|_X\leqslant\sup\limits_{t\in[0,T]}\|f(t)\|_X<+\infty,
$$
откуда и получаем включение $f\in L_\infty([0,T],X)$.

Итак, для завершения доказательства нам достаточно показать, что
$$
\sup\limits_{t\in[0,T]}\|f(t)\|_X\leqslant\vraisup\limits_{t\in[0,T]}\|f(t)\|_X.
$$
В самом деле, пусть $x^*\in X^*$ --- произвольный элемент. Ввиду слабой непрерывности функции $f$ на отрезке $[0,T]$, вещественнозначная функция
$$
[0,T]\ni t\mapsto\langle f(t), x^*\rangle
$$
непрерывна на отрезке $[0,T]$. Как следствие,
\begin{gather*}
\sup\limits_{t\in[0,T]}\langle f(t), x^*\rangle=\vraisup\limits_{\xi\in[0,T]}\langle f(\xi), x^*\rangle.
\end{gather*}
Поэтому при всех $t\in[0,T]$
\begin{gather*}
\langle f(t), x^*\rangle\leqslant\vraisup\limits_{\xi\in[0,T]}\langle f(\xi), x^*\rangle\leqslant\vraisup\limits_{\xi\in[0,T]}\|f(\xi)\|_X\|x^*\|_{X^*}.
\end{gather*}
Переходя здесь к точной верхней грани по $x^*\in X^*$, у которых $\|x^*\|_{X^*}\leqslant1$, получим, что при любом $t\in[0,T]$
\begin{gather*}
\|f(t)\|_{X^{**}}\leqslant\vraisup\limits_{\xi\in[0,T]}\|f(\xi)\|_X,
\end{gather*}
откуда, на основании  изометричности вложения $X\subset X^{**}$, вытекает, что
$$
\sup\limits_{t\in[0,T]}\|f(t)\|_X\leqslant\vraisup\limits_{t\in[0,T]}\|f(t)\|_X.
$$
Лемма полностью доказана.
\end{Proof}

        \section{Интеграл Стильтьеса}
Материал данного раздела, за исключением, может быть, раздела \ref{functional.on.C([0,T],X)!representation}, можно найти в \cite[раздел 25.3]{Trenogin}.
            \subsection{Определение интеграла и условия интегрируемости}
Прежде чем определить понятие интеграла Стильтьеса, введём понятие банаховозначной функции ограниченной вариации.

Пусть на отрезке $[a,b]$ числовой оси задана функция $y$, принимающая значения в банаховом пространстве $Y$ с нормой $\|\cdot\|_Y$. Пусть $\tau=\{t_i\}_{i=0}^N$ --- некоторое разбиение
отрезка $[a,b]$. Составим сумму
\begin{gather*}
\Var^{b}_{a}[y;\tau]\equiv\sum\limits_{i=1}^N\|y(t_i)-y(t_{i-1})\|_Y.
\end{gather*}

\begin{Definition}
Величина $\Var^b_a[y]\equiv\sup\limits_{\tau}\Var^{b}_{a}[y;\tau]$ называется \textbf{полной вариацией} функции $y$ на отрезке $[a,b]$. Если эта величина конечна, то будем называть
функцию $y$ \textbf{функцией ограниченной вариации}. Множество всех функций ограниченной вариации, принимающих значения в $Y$, будем обозначать $\mathbf{BV}([a,b],Y)$.
\end{Definition}

\begin{Lemma}
Если функция $y$ удовлетворяет на отрезке $[a,b]$ условию Липшица, то есть найдётся постоянная $L>0$, такая, что для всех $t'$, $t''\in[a,b]$ выполнено неравенство
$\|y(t'')-y(t')\|_Y\leqslant L|t'-t''|$, то функция $y$ является функцией ограниченной вариации, причём
\begin{gather*}
\Var^b_a[y]\leqslant L(b-a).
\end{gather*}
\end{Lemma}
\begin{Proof}
В самом деле, пусть $\tau=\{t_i\}_{i=0}^N$ --- некоторое разбиение отрезка $[a,b]$. Тогда, в силу липшицевости функции $y$,
\begin{gather*}
\Var^{b}_{a}[y;\tau]\equiv\sum\limits_{i=1}^N\|y(t_i)-y(t_{i-1})\|_Y\leqslant L\sum\limits_{i=1}^N|t_i-t_{i-1}|=L(b-a).
\end{gather*}
Взяв в получившемся неравенстве точную верхнюю грань по всевозможным разбиениям $\tau$, получим требуемое.
\end{Proof}

\begin{Lemma}
Если функция $y$ сильно дифференцируема всюду на отрезке $[a,b]$, и производная $y'$ ограничена на этом отрезке, то функция $y$ --- функция ограниченной вариации.
\end{Lemma}
\begin{Proof}
1) Прежде всего докажем нужное для доказательства вспомогательное неравенство. Выберем произвольно элемент $y^*\in Y^*$ и зафиксируем. Введём функцию $\Phi:[a,b]\to\mathbb{R}$ равенством
\begin{gather*}
\Phi(t)\equiv\langle y(t),y^*\rangle,\,\,\,t\in[a,b].
\end{gather*}
Выберем затем произвольно точки $t'$, $t''\in[a,b]$, $t'<t''$, и зафиксируем. Поскольку функция $y$ всюду на отрезке $[a,b]$ сильно дифференцируема, то всюду на этом отрезке дифференцируема
функция $\Phi$. Как следствие, функция $\Phi$ дифференцируема на отрезке $[t',t'']$. Значит, в силу теоремы Лагранжа о среднем, найдётся число $\theta\in(0,1)$, такое, что
\begin{gather*}
\Phi(t'')-\Phi(t')=\Phi'(t'+\theta(t''-t'))(t''-t'),
\end{gather*}
или, в силу определения функции $\Phi$,
\begin{gather}\label{bounded.derivative.implies.bounded.variation!m1}
\langle y(t'')-y(t'),y^*\rangle=\langle y'(t'+\theta(t''-t')),y^*\rangle(t''-t').
\end{gather}
Здесь возможны два случая: $y(t')=y(t'')$ и $y(t')\neq y(t'')$.

Предположим, что $y(t')=y(t'')$. Тогда, очевидно, справедливо неравенство
\begin{gather}\label{bounded.derivative.implies.bounded.variation!m2}
\|y(t'')-y(t')\|_Y\leqslant\sup\limits_{t\in[t',t'']}\|y'(t)\|_Y(t''-t').
\end{gather}

Пусть теперь $y(t')\neq y(t'')$. Тогда, в силу следствия из теоремы Хана--Банаха, найдётся линейный непрерывный функционал $y^*\in Y^*$, такой, что $\|y^*\|_{Y^*}=1$,
$\langle y(t'')-y(t'),y^*\rangle=\|y(t'')-y(t')\|_{Y}$. Подставляя такой функционал $y^*$ в \eqref{bounded.derivative.implies.bounded.variation!m1}, получаем, что
\begin{gather*}
\|y(t'')-y(t')\|_{Y}=\langle y'(t'+\theta(t''-t')),y^*\rangle(t''-t')\leqslant\\
\leqslant\|y'(t'+\theta(t''-t'))\|_Y\|y^*\|_{Y^*}(t''-t')\leqslant\sup\limits_{t\in[t',t'']}\|y'(t)\|_Y(t''-t').
\end{gather*}
Иными словами, и в этом случае справедливо неравенство \eqref{bounded.derivative.implies.bounded.variation!m2}.

Таким образом, для всех $t'$, $t''\in[a,b]$, $t'<t''$, справедливо неравенство \eqref{bounded.derivative.implies.bounded.variation!m2}.

2) Докажем теперь конечность полной вариации функции $y$.В самом деле, пусть $\tau=\{t_i\}_{i=0}^N$ --- некоторое разбиение отрезка $[a,b]$. Тогда, в силу доказанного неравенства
\eqref{bounded.derivative.implies.bounded.variation!m2},
\begin{gather*}
\Var^{b}_{a}[y;\tau]\equiv\sum\limits_{i=1}^N\|y(t_i)-y(t_{i-1})\|_Y\leqslant\sum\limits_{i=1}^N\sup\limits_{t\in[t_{i-1},t_i]}\|y'(t)\|_Y|t_i-t_{i-1}|\leqslant\\
\leqslant[\sup\limits_{t\in[a,b]}\|y'(t)\|_Y]\sum\limits_{i=1}^N|t_i-t_{i-1}|=(b-a)\sup\limits_{t\in[a,b]}\|y'(t)\|_Y,
\end{gather*}
то есть
\begin{gather*}
\Var^{b}_{a}[y;\tau]\leqslant(b-a)\sup\limits_{t\in[a,b]}\|y'(t)\|_Y.
\end{gather*}
Взяв в получившемся неравенстве точную верхнюю грань по всевозможным разбиениям $\tau$, получим требуемое утверждение.
\end{Proof}

Пусть $X$, $Y$, $Z$ --- банаховы пространства с нормами $\|\cdot\|_X$, $\|\cdot\|_Y$ и $\|\cdot\|_Z$ соответственно, причём для элементов $x\in X$ определено умножение справа на элементы
$y\in Y$ со значениями в $Z$ ($x\bullet y\in Z$). Пусть, далее, на отрезке $[a,b]$ заданы функции $x$ и $y$, принимающие значения в пространствах $X$ и $Y$ соответственно.

Пусть $\tau=\{t_i\}_{i=0}^N$ --- некоторое разбиение отрезка $[a,b]$, а $\theta_i\in[t_{i-1},t_i]$, $i=\overline{1,N}$, --- промежуточные точки. Составим интегральную сумму
\begin{gather*}
\sigma_\tau=\sigma_\tau(x;y;\theta_1,\dots,\theta_N)=\sum\limits_{i=1}^Nx(\theta_i)\bullet[y(t_i)-y(t_{i-1})].
\end{gather*}

\begin{Definition}
Функция $x$ называется \textbf{интегрируемой по Стильтьесу} на отрезке $[a,b]$ относительно функции $y$, если существует такой элемент $A\in Z$, что для любой последовательности разбиений
отрезка $[a,b]$
\begin{gather*}
\tau_j=\{t_i^{(j)}\}^{i=i_{\tau_j}}_{i=0},\,\,\,j=1,2,\dots,
\end{gather*}
у которой $\lim\limits_{j\to\infty}|\tau_j|=0$, и для любого выбора точек $\xi^{(j)}_i\in[t^{(j)}_{i-1},t^{(j)}_i]$, $i=\overline{1,i_{\tau_j}}$, $j=1,2,\dots$, существует предел
последовательности интегральных сумм $\sigma_{\tau_j}(x;y;\xi^{(j)}_1,\dots,\xi^{(j)}_{i_{\tau_j}})$ и он равен $A$:
\begin{gather}\label{limintegralsums::Stiltjes}
\lim\limits_{j\to\infty}\left\|\sum\limits_{i=1}^{i_{\tau_j}}x(\theta_i)\bullet[y(t_i)-y(t_{i-1})]-A\right\|_X=0.
\end{gather}

При выполнении этих условий элемент $A$ называется \textbf{\textbf{интегралом Стильтьеса}} функции $x$ на отрезке $[a,b]$ относительно функции $y$ и обозначается
$(\textrm{С})\int\limits_a^bx(t)\bullet dy(t)$ или $\int\limits_a^bf(t)\bullet dy(t)$.
\end{Definition}

Можно показать, что справедлива следующая
\begin{Theorem}
Если функция $x$ сильно непрерывна на $[a,b]$, а функция $y$ является функцией ограниченной вариации на $[a,b]$, то функция $x$ интегрируема в смысле Стильтьеса на $[a,b]$ относительно
функции $y$, причём
\begin{gather*}
\left\|\int\limits_a^bx(t)\bullet dy(t)\right\|_Z\leqslant K\pmb{|}x\pmb{|}^{(0)}_{[a,b],X}\Var^b_a[y],
\end{gather*}
где $K$ --- норма билинейной формы, задающей умножение.
\end{Theorem}

            \subsection{Свойства интеграла}
Приведём теперь некоторые свойства интеграла Стильтьеса.

\begin{Lemma}
Если $x_0\in X$, то
\begin{gather*}
\int\limits_a^bx_0\bullet dy(t)=x_0\bullet[y(b)-y(a)].
\end{gather*}
\end{Lemma}

\begin{Lemma}
Если $x(t)=x_0\varphi(t)$, $t\in[a,b]$, где $x_0\in X$, $\varphi\in C[a,b]$, то
\begin{gather*}
\int\limits_a^bx_0\varphi(t)\bullet dy(t)=x_0\bullet\int\limits_a^b\varphi(t)\bullet dy(t).
\end{gather*}
\end{Lemma}

\begin{Lemma}
Если $y(t)=y_0\psi(t)$, $t\in[a,b]$, где $y_0\in Y$, $\varphi\in \mathbf{BV}[a,b]$, то
\begin{gather*}
\int\limits_a^bx(t)\bullet d[y_0\psi(t)]=\left[\int\limits_a^bx(t)\bullet d\psi(t)\right]\bullet y_0.
\end{gather*}
\end{Lemma}

\begin{Lemma}Если $x$, $x_1$, $x_2\in C([a,b],X)$, а $y$, $y_1$, $y_2\in\mathbf{BV}([a,b],Y)$, то
\begin{gather*}
\int\limits_a^b[x_1(t)+x_2(t)]\bullet dy(t)=\int\limits_a^bx_1(t)\bullet dy(t)+\int\limits_a^bx_2(t)\bullet dy(t),\\
\int\limits_a^bx(t)\bullet d[y_1(t)+y_2(t)]=\int\limits_a^bx(t)\bullet dy_1(t)+\int\limits_a^bx(t)\bullet dy_2(t).
\end{gather*}
\end{Lemma}

\begin{Lemma}
Если $x$ и $y$ --- непрерывные функции ограниченной вариации, то справедлива формула интегрирования по частям,
\begin{gather*}
\int\limits_a^bx(t)\bullet dy(t)+\int\limits_a^bdx(t)\bullet y(t)=x(t)\bullet y(t)|_a^b.
\end{gather*}
\end{Lemma}

\begin{Lemma}
Если $c\in(a,b)$, то
\begin{gather*}
\int\limits_a^bx(t)\bullet dy(t)=\int\limits_a^cx(t)\bullet dy(t)+\int\limits_c^bx(t)\bullet dy(t).
\end{gather*}
\end{Lemma}

\begin{Lemma}
Пусть функция $x$ --- непрерывна на $[a,b]$, функция $y$ --- функция ограниченной вариации на $[a,b]$. Пусть, кроме того, $t=\omega(p)$, $p\in[\alpha,\beta]$, --- строго возрастающая,
непрерывная на $[\alpha,\beta]$ функция, причём $\omega(\alpha)=a$, $\omega(\beta)=b$. Тогда функция $y(\omega(p))$, $p\in[\alpha,\beta]$, --- функция ограниченной вариации на $[\alpha,
\beta]$, и справедливо равенство
\begin{gather}\label{zamena_peremennoj::Stiltjes}
\int\limits_a^bx(t)\bullet dy(t)=\int\limits_\alpha^\beta x(\omega(p))\bullet dy(\omega(p)).
\end{gather}
\end{Lemma}

            \subsection{Представление линейного непрерывного функционала на пространстве непрерывных банаховозначных функций}\label{functional.on.C([0,T],X)!representation}
В данном разделе мы докажем теорему о представлении линейного непрерывного функционала над банаховым пространством $C([a,b],X)$, где $X$ --- банахово пространство с нормой $\|\cdot\|_X$.

\begin{Theorem}\label{functional.on.C([0,T],X)!representation::Theorem}
Пусть банахово пространство $X$ --- рефлексивно, и пусть $\mathcal{F}:C([a,b],X)\to\mathbb{R}$ --- линейный непрерывный функционал. Тогда найдётся функция ограниченной вариации
$g:[a,b]\to X^*$, такая, что
\begin{gather}\label{functional.on.C([0,T],X)!representation::Theorem!representation}
\mathcal{F}[f]=\int\limits_a^b\langle x(t),dg(t)\rangle\,\,\,\forall\,f\in C([a,b],X),
\end{gather}
причём
\begin{gather}\label{functional.on.C([0,T],X)!representation::Theorem!norm}
\|\mathcal{F}\|_{(C([a,b],X))^*}=\Var^{b}_{a}[g].
\end{gather}
\end{Theorem}
\begin{Proof}
Поскольку пространство $C([a,b],X)$ является замкнутым подпространством пространства ограниченных функций, $BF([a,b],X)$, то, согласно теореме Хана--Банаха, функционал $\mathcal{F}$
можно с сохранением нормы продолжить до непрерывного функционала на $BF([a,b],X)$. Результат продолжения также обозначим $\mathcal{F}$.

Введём семейство функций $w_{\tau,x}:[a,b]\to X$, где $\tau\in[a,b]$, $x\in X$ --- параметры, формулами
\begin{gather*}
w_{a,x}(t)\equiv0;\,\,\,
w_{\tau,x}(t)\equiv
\begin{cases}
x,\text{ если $t\in[a,\tau]$},\cr
0,\text{ если $t\in(\tau,b]$},
\end{cases}
\,\,\,\text{при $\tau>a$.}
\end{gather*}

Нетрудно видеть, что функции $w_{\tau,x}$ удовлетворяют следующим условиям:
\begin{gather}\label{functional.on.C([0,T],X)!representation::Theorem!wfunc.properties}
w_{\tau,x}\in BF([a,b],X),\,\,\,\|w_{\tau,x}\|_{BF([a,b],X)}\leqslant\|x\|_X\,\,\,\forall\,x\in X,\,\,\,\tau\in[a,b];\\
\notag w_{\tau,x_1+x_2}=w_{\tau,x_1}+w_{\tau,x_2},\,\,\,w_{\tau,\lambda x}=\lambda w_{\tau,x}\,\,\,\forall\,x,\,\,x_1,\,\,x_2\in X,\,\,\,\lambda\in\mathbb{R};\\
\notag w_{\tau_2,x}(t)-w_{\tau_1,x}(t)=\chi_{(\tau_1,\tau_2]}(t)x,\,\,\,\forall\,\tau_1,\,\,\tau_2\in[a,b],\,\,\,\tau_1<\tau_2\,\,\,\forall\,x\in X.
\end{gather}

Далее, поскольку $\mathcal{F}$ --- линейный непрерывный функционал, то при всех $\tau\in[a,b]$ и $x\in X$
\begin{gather*}
|\mathcal{F}[w_{\tau,x}]|\leqslant\|\mathcal{F}\|\|w_{\tau,x}\|_{BF([a,b],X)}\leqslant\|\mathcal{F}\|\|x\|_X,
\end{gather*}
ввиду чего при всех $\tau\in[a,b]$ отображение
\begin{gather*}
X\ni x\mapsto\mathcal{F}[w_{\tau,x}]
\end{gather*}
является линейным непрерывным функционалом над $X$. Значит найдётся элемент $g(\tau)\in X^*$, такой, что
\begin{gather}\label{functional.on.C([0,T],X)!representation::Theorem!g(tau)}
\langle x,g(\tau)\rangle=\mathcal{F}[w_{\tau,x}]\,\,\,\forall\,x\in X,\,\,\,\tau\in[a,b].
\end{gather}
Таким образом, мы построили функцию $g:[a,b]\to X^*$. Докажем, что она является функцией ограниченной вариации.

В самом деле, пусть $\xi=\{t_i\}_{i=0}^N$ --- некоторое разбиение отрезка $[a,b]$. Рассмотрим величину
$$
\Var^{b}_{a}[g;\xi]\equiv\sum\limits_{i=1}^N\|g(t_i)-g(t_{i-1})\|_{X^*}.
$$
Для каждого $i=\overline{1,N}$ возможны два случая: либо $g(t_i)-g(t_{i-1})\neq0$, либо $g(t_i)-g(t_{i-1})=0$.

Пусть $g(t_i)-g(t_{i-1})\neq0$. Тогда, по следствию из теоремы Хана--Банаха, найдётся линейный непрерывный функционал $y_i\in X^{**}$, такой, что
\begin{gather*}
\|y_i\|_{X^{**}}=1,\,\,\,\langle y_i,g(t_i)-g(t_{i-1})\rangle=\|g(t_i)-g(t_{i-1})\|_{X^*}.
\end{gather*}
Поскольку $X$ --- рефлексивно, то последние соотношения означают, что найдётся элемент $x_i\in X$, такой, что
\begin{gather*}
\|x_i\|_{X}=1,\,\,\,\langle x_i,g(t_i)-g(t_{i-1})\rangle=\|g(t_i)-g(t_{i-1})\|_{X^*}.
\end{gather*}

Введём элементы $\alpha_i\in X$, $i=\overline{1,N}$, формулами
\begin{gather*}
\alpha_i=
\begin{cases}
x_i,\text{ при $g(t_i)-g(t_{i-1})\neq0$};\cr
0,\text{ при $g(t_i)-g(t_{i-1})=0$}.
\end{cases}
\end{gather*}
Тогда выводим, что
\begin{gather*}
\|\alpha_i\|_{X}\leqslant1,\,\,\,\langle\alpha_i,g(t_i)-g(t_{i-1})\rangle=\|g(t_i)-g(t_{i-1})\|_{X^*},\,\,\,i=\overline{1,N}.
\end{gather*}
Как следствие,
\begin{gather*}
\Var^{b}_{a}[g;\xi]\equiv\sum\limits_{i=1}^N\|g(t_i)-g(t_{i-1})\|_{X^*}=\sum\limits_{i=1}^N\langle\alpha_i,g(t_i)-g(t_{i-1})\rangle=
\sum\limits_{i=1}^N[\langle\alpha_i,g(t_i)\rangle-\langle\alpha_i,g(t_{i-1})\rangle]=\\
=\sum\limits_{i=1}^N[\mathcal{F}[w_{t_i,\alpha_i}]-\mathcal{F}[w_{t_{i-1},\alpha_i}]]=\sum\limits_{i=1}^N\mathcal{F}[w_{t_i,\alpha_i}-w_{t_{i-1},\alpha_i}]=
\sum\limits_{i=1}^N\mathcal{F}[\chi_{(t_{i-1},t_i]}\alpha_i]=\mathcal{F}\left[\sum\limits_{i=1}^N\chi_{(t_{i-1},t_i]}\alpha_i\right]\leqslant\\
\leqslant\|\mathcal{F}\|\sup\limits_{t\in[a,b]}\left\|\sum\limits_{i=1}^N\chi_{(t_{i-1},t_i]}(t)\alpha_i\right\|_X
\leqslant\|\mathcal{F}\|\sup\limits_{t\in[a,b]}\sum\limits_{i=1}^N\chi_{(t_{i-1},t_i]}(t)\|\alpha_i\|_X\leqslant\|\mathcal{F}\|.
\end{gather*}
Итак,
\begin{gather*}
\Var^{b}_{a}[g;\xi]\leqslant\|\mathcal{F}\|.
\end{gather*}
Поскольку же разбиение отрезка $[a,b]$ было выбрано произвольно, то
\begin{gather}\label{functional.on.C([0,T],X)!representation::Theorem!norm.estimated.from.below}
\Var^{b}_{a}[g]\leqslant\|\mathcal{F}\|.
\end{gather}

Итак, мы построили по функционалу $\mathcal{F}$ функцию ограниченной вариации $g:[a,b]\to X^*$. Покажем теперь, что с помощью этой функции функционал $\mathcal{F}$ можно записать в виде
\eqref{functional.on.C([0,T],X)!representation::Theorem!representation}.

Пусть $f:[a,b]\to X$ --- произвольная непрерывная функция. Поскольку она непрерывна на отрезке, то она и равномерно непрерывна на этом отрезке. Выберем теперь произвольно число
$\varepsilon>0$ и зафиксируем. Подберём по этому $\varepsilon>0$ число $\delta=\delta(\varepsilon)$ так, чтобы при всех $t'$, $t''\in [a,b]$, $|t'-t''|\leqslant\delta$ выполнялось
неравенство $\|f(t')-f(t'')\|_X<\varepsilon$. Выберем теперь разбиение $\xi$ так, чтобы его мелкость была меньше $\delta$, и рассмотрим кусочно--постоянную функцию $f_\varepsilon$,
\begin{gather*}
f_\varepsilon(t)=
\begin{cases}
f(t_i),\text{ при $t_{i-1}<t\leqslant t_i$, $i=\overline{1,N}$};\cr
f(t_1),\text{ при $t=a$}.
\end{cases}
\end{gather*}
Эту функцию можно записать в виде
\begin{gather*}
f_\varepsilon(t)=\sum\limits_{i=1}^N[w_{t_i,f(t_i)}(t)-w_{t_{i-1},f(t_i)}(t)].
\end{gather*}
Из определения функции $f_\varepsilon$ следует, что при всех $t\in[a,b]$
\begin{gather*}
\|f(t)-f_\varepsilon(t)\|_X<\varepsilon,
\end{gather*}
или, иначе говоря,
\begin{gather*}
\|f-f_\varepsilon\|_{BF([a,b],X)}\leqslant\varepsilon.
\end{gather*}
Найдём значение функционала $\mathcal{F}$ на элементе $f_\varepsilon$. Ввиду линейности этого функционала и определения функций $w_{\tau,x}$ оно равно
\begin{gather*}
\mathcal{F}[f_\varepsilon]=\mathcal{F}\left[\sum\limits_{i=1}^N[w_{t_i,f(t_i)}-w_{t_{i-1},f(t_i)}]\right]=
\sum\limits_{i=1}^N[\mathcal{F}[w_{t_i,f(t_i)}]-\mathcal{F}[w_{t_{i-1},f(t_i)}]]=\\
=\sum\limits_{i=1}^N[\langle f(t_i),g(t_i)\rangle-\langle f(t_i),g(t_{i-1})\rangle]=\sum\limits_{i=1}^N\langle f(t_i),g(t_i)-g(t_{i-1})\rangle,
\end{gather*}
т.е. представляет собой интегральную сумму для интеграла
\begin{gather*}
\int\limits_a^b\langle x(t),dg(t)\rangle.
\end{gather*}
Поэтому при достаточно мелком разбиении отрезка $[a,b]$
\begin{gather*}
\left|\mathcal{F}[f_\varepsilon]-\int\limits_a^b\langle x(t),dg(t)\rangle\right|<\varepsilon.
\end{gather*}
В то же время
\begin{gather*}
|\mathcal{F}[f]-\mathcal{F}[f_\varepsilon]|\leqslant\|\mathcal{F}\|\|f-f_\varepsilon\|_{BF([a,b],X)}\leqslant\|\mathcal{F}\|\varepsilon.
\end{gather*}
Следовательно,
\begin{gather*}
\left|\mathcal{F}[f]-\int\limits_a^b\langle x(t),dg(t)\rangle\right|<\varepsilon[1+\|\mathcal{F}\|].
\end{gather*}
Отсюда в силу произвольности $\varepsilon$ получаем равенство \eqref{functional.on.C([0,T],X)!representation::Theorem!representation}.

Соотношение же \eqref{functional.on.C([0,T],X)!representation::Theorem!norm} следует из оценки \eqref{functional.on.C([0,T],X)!representation::Theorem!norm.estimated.from.below} и
свойств интеграла Стильтьеса.

Теорема полностью доказана.
\end{Proof}

\begin{Remark}
Из свойств интеграла Стильтьеса следует, что формула \eqref{functional.on.C([0,T],X)!representation::Theorem!representation} для любой функции $g\in \mathbf{BV}([a,b],X^*)$
задаёт линейный непрерывный функционал над $C([a,b],X)$. При этом несложно показать, что если функции $g_1$ и $g_2$ задают один и тот же функционал, то $g_1-g_2\equiv const$ во всех
точках непрерывности функции $g_1-g_2$.

Таким образом, каждому линейному непрерывному функционалу над $C([a,b],X)$ соответствует целый класс функций ограниченной вариации. В каждом таком классе можно выбрать одну и только одну
функцию, непрерывную справа в каждой точке полуинтервала $(a,b]$ и равную нулю в точке $a$. Множество всех таких функций ограниченной вариации обозначим через $\mathbf{BV}^0([0,T],X)$.
Можно показать, что это множество, наделённое нормой $\|\varphi\|_{\mathbf{BV}^0,X}\equiv\Var^{T}_{0}[\varphi]$, является банаховым пространством. Заметим, что функция $g$, построенная
при доказательстве теоремы, является именно функцией из $\mathbf{BV}^0([0,T],X)$.
\end{Remark}

С учётом вышеприведённого замечания теорему \ref{functional.on.C([0,T],X)!representation::Theorem} можно переписать в следующем виде.
\begin{Theorem}\label{functional.on.C([0,T],X)!representation::Theorem!!final}
Существует изометричный изоморфизм пространств $(C([a,b],X))^*$ и $\mathbf{BV}^0([0,T],X)$, устанавливаемый равенством
\begin{gather*}
\mathcal{F}[f]=\int\limits_a^b\langle x(t),dg(t)\rangle\,\,\,\forall\,f\in C([a,b],X).
\end{gather*}
\end{Theorem}


        \subsection{Аппроксимация банаховозначных мер Радона, заданных на отрезке числовой оси}
\begin{Lemma}\label{measapprox!Banach.valued} Пусть $X$ --- рефлексивное банахово пространство, $[a,b]$ --- отрезок числовой оси. Для любой меры $\mu\in \mathbf{M}([a,b],X^*)$ найдётся
последовательность функций~$\omega^k\in C([a,b],X^*)$, $k=1,2,\dots$, такая, что
\begin{equation}\label{measapproxc!Banach.valued}
\lim\limits_{k\to\infty}\int\limits_{[a,b]}\langle\zeta(t),\mu^k(dt)\rangle=\int\limits_{[a,b]}\langle\zeta(t),\mu(dt)\rangle, \,\,\,\forall\,\zeta\in C([a,\,b],X),
\end{equation}
где $\displaystyle\mu^k(E)\equiv\int\limits_E\omega^k(t)dt$, $E\subseteq [a,\,b]$ --- борелевское подмножество отрезка $[a,\,b]$, $k=1,2,\dots$.%, причём если мера $\mu\in \mathbf{M}[0,\,T]$
%--- неотрицательна, то $\omega^k(t)\geqslant0$, $\forall\,t\in[0,\,T]$, $k=1,2,\dots$.
\end{Lemma}
\begin{Proof} Разобьём доказательство на несколько этапов.

1) Покажем вначале, что для любой меры~$\mu\in \mathbf{M}([a,\,b],X^*)$ найдётся последовательность мер
\begin{gather*}
\bar{\mu}^m\equiv\sum\limits_{i=1}^{i_m}\lambda_{i,m}\mydelta_{t_{i,m}},\,\,\,m=1,2,\dots,
\end{gather*}
где %$\lambda_{i,m}\in R$, $i= \overline{1,\,i_m}$, $m=1,2,\dots$, --- неотрицательны, если мера $\mu\in \mathbf{M}[0,\,T]$ --- неотрицательна,
$t_{i,m}\in[0,\,T]$, $i=\overline{1,\,i_m}$, $m=1,2,\dots$, такая, что
\begin{equation}\label{fawlc!Banach.valued}
\lim\limits_{m\to\infty}\int\limits_{[a,b]}\langle\zeta(t),\bar{\mu}^m(dt)\rangle= \int\limits_{[a,b]}\langle\zeta(t),{\mu}(dt)\rangle,\,\,\,\forall\,\zeta\in C([a,\,b],X).
\end{equation}
В самом деле, $(C([a,\,b],X))^*$ изометрично изоморфно~$\mathbf{M}([a,\,b],X^*)$. С другой стороны, $(C([a,\,b],X))^*$ изометрично изоморфно
$\mathbf{BV}^0([a,\,b],X^*)$. Следовательно, существует изоморфизм $${\cal F}\colon \mathbf{M}([a,\,b],X^*)\to\mathbf{BV}^0([a,\,b],X^*),$$ такой, что
\begin{equation}\label{isomfv!Banach.valued}
\|{\cal F}[\mu]\|_{\mathbf{BV}^0}=\|\mu\|,\,\,\,\forall\,\mu\in \mathbf{M}([a,\,b],X^*).
\end{equation}
Пусть функционал~${\cal A}\,:\,C([a,\,b],X)\to \mathbb{R}$ задаётся формулой
\begin{equation}\label{ffunca!Banach.valued}
{\cal A}[\zeta]=\int\limits_{[a,\,b]}\langle\zeta(t),\mu(dt)\rangle,\,\,\,\forall\,\zeta\in C([a,\,b],X).
\end{equation}
Тогда, очевидно, ${\cal A}\in(C([a,\,b],X))^*$, и, стало быть,
\begin{equation}\label{azis!Banach.valued}
{\cal A}[\zeta]=\int\limits_{[a,b]}\langle\zeta(t),d{\cal F}[\mu](t)\rangle,\,\,\,\forall\,\zeta \in C([a,\,b],X),
\end{equation}
где интеграл понимается в смысле интеграла Стильтьеса по отрезку~$[a,\,b]$. Пусть~$\bar{t}_{i,m}= \frac{Ti}{m}$, $i=\overline{0,\,m}$, $\bar{t}_{i,m+1}=t_{i,m}$, $\lambda_{i,m}\equiv
{\cal F}[\mu](\bar{t}_{i,m})-{\cal F}[\mu](\bar{t}_{i-1,m})$, $i= \overline{1,\,m}$, $\lambda_{i,m+1}={\cal F}[\mu](\bar{t}_{i,m})$, $i_m=m+1$, $t_{i,m}= \bar{t}_{i-1,m}$,
$i=\overline{1,\,i_m}$. Тогда, в силу определения интеграла Стильтьеса по отрезку~$[a,\,b]$,
\begin{gather*}
\displaystyle\lim\limits_{m\to\infty}\sum\limits_{i=1}^{i_m}\langle\zeta(t_{i,m}),\lambda_{i,m}\rangle= \int\limits_{[a,b]}\langle\zeta(t),d{\cal F}[\mu](t)\rangle,\,\,\,
\forall\,\zeta \in C([a,\,b],X).
\end{gather*}
Полагая~$\displaystyle\bar{\mu}^m\equiv\sum\limits_{i=1}^{i_m} \lambda_{i,m}\mydelta_{t_{i,m}}$, $m=1,2,\dots$, перепишем последнее равенство в виде
\begin{gather*}
\lim\limits_{m\to\infty}\int\limits_{[a,b]}\langle\zeta(t),\bar{\mu}^m(dt)\rangle= \int\limits_{[a,b]}\langle\zeta(t),{\mu}(dt)\rangle,\,\,\,\forall\,\zeta\in C([a,\,b],X),
\end{gather*}
что в совокупности с~(\ref{isomfv!Banach.valued})--(\ref{azis!Banach.valued}) и даёт~(\ref{fawlc!Banach.valued}).

%Ясно, что если функция~${\cal F}[\mu]$ монотонно не убывает, то построенные меры~$\bar\mu^m$, $m=1,2,\dots$, будут неотрицательными, и, следовательно, мера~$\mu$ также будет
%неотрицательной, как $*$--слабый предел таких мер. В силу же~(\ref{isomfv}), неотрицательная мера~$\mu$ может породить лишь монотонно неубывающую функция~${\cal F}[\mu]$. Итак, коэффициенты
%мер~$\bar\mu^m$, $m=1,2,\dots$, можно считать неотрицательными, если мера~$\mu$ --- неотрицательна.

2) Докажем существование непрерывных функций, упомянутых в формулировке леммы. Пусть $\varepsilon_s>0$, $s=1,2,\dots$, $\varepsilon_s\to0$, $s\to\infty$, --- некоторая последовательность
чисел, и пусть
\begin{gather*}
\omega^s_{i,m}(t)\equiv \frac{\chi_{(t_{i,m}-\varepsilon_s,t_{i,m}+\varepsilon_s)\cap(a,\,b)}(t)}{\meas\{(t_{i,m}-\varepsilon_s,t_{i,m}+\varepsilon_s)\cap(a,\,b)\}},\,\,\,
i=\overline{1,\,i_m},\\
\bar{\omega}^s_m(t)\equiv\sum\limits_{i=1}^{i_m}\lambda_{i,m}\omega^s_{i,m}(t),\,\,\,m,\,s=1,2,\dots,\,\,\,t\in[a,\,a],\\
\bar{\mu}_s^m(E)\equiv\int\limits_E\bar{\omega}^s_m(t)dt,\,\,\, E\subseteq[a,\,b],\,\,\,m,\,s=1,2,\dots.
\end{gather*}
Тогда, очевидно,
\begin{equation}\label{sawlc!Banach.valued}
\lim\limits_{s\to\infty}\int\limits_{[a,a]}\langle\zeta(t),\bar{\mu}^m_s(dt)\rangle= \int\limits_{[a,b]}\langle\zeta(t),\bar{\mu}^m(dt)\rangle,\,\,\,\forall\,\zeta\in C([a,\,b],X).
\end{equation}
Пусть теперь~$h_p>0$, $p=1,2,\dots$, $h_p\to0$, $p\to\infty$, --- некоторая последовательность чисел, и пусть $\displaystyle\bar{\mu}^m_{s,p}(E)\equiv \int\limits_E\bar{\omega}^{s,p}_m(t)
dt$, $E\subseteq[a,\,b]$, $m,\,s,\,p=1,2,\dots$, где~$\bar{\omega}^{s,p}_m$ --- усреднение с параметром~$h_p>0$ с ядром, независящим от $m,\,s,\,p=1,2,\dots$, функции $\bar{\omega}^{s}_m$.
Пользуясь свойствами средних функций и теоремой Радона--Никодима, заключаем, что
\begin{equation}\label{tawlc!Banach.valued}
\lim\limits_{p\to\infty}\int\limits_{[a,b]}\langle\zeta(t),\bar{\mu}^m_{s,p}(dt)\rangle= \int\limits_{[a,b]}\langle\zeta(t),\bar{\mu}^m_s(dt)\rangle,\,\,\,\forall\,\zeta\in C([a,\,b],X).
\end{equation}
Из~(\ref{fawlc!Banach.valued}) следует, что
\begin{gather}\label{weaknc:m!Banach.valued}
\lim\limits_{m\to\infty}|\bar{\mu}^m-\mu|_w=0,
\end{gather}
где~$|\cdot|_w$ --- слабая норма~\cite{warga} в~$\mathbf{M}([a,\,b],X^*)$.

Выберем произвольно $\alpha>0$ и зафиксируем.

Согласно (\ref{weaknc:m!Banach.valued}), найдётся номер $m_0(\alpha)\geqslant1$, такой, что
\begin{gather*}
|\bar{\mu}^{m_0(\alpha)}-\mu|_w\leqslant\frac\alpha3.
\end{gather*}

Далее, согласно (\ref{sawlc!Banach.valued}),
\begin{gather*}
\lim\limits_{s\to\infty}|\bar{\mu}^{m_0(\alpha)}_s-\bar\mu^{m_0(\alpha)}|_w=0.
\end{gather*}
Поэтому найдётся номер $s_0(\alpha)\geqslant1$, такой, что
\begin{gather*}
|\bar{\mu}^{m_0(\alpha)}_{s_0(\alpha)}-\bar\mu^{m_0(\alpha)}|_w\leqslant\frac\alpha3.
\end{gather*}

Наконец, согласно (\ref{tawlc!Banach.valued}),
\begin{gather*}
\lim\limits_{p\to\infty}|\bar{\mu}^{m_0(\alpha)}_{s_0(\alpha),p}-\bar\mu^{m_0(\alpha)}_{s_0(\alpha)}|_w=0,
\end{gather*}
в силу чего найдётся номер $p_0(\alpha)\geqslant1$, такой, что
\begin{gather*}
|\bar{\mu}^{m_0(\alpha)}_{s_0(\alpha),p_0(\alpha)}-\bar\mu^{m_0(\alpha)}_{s_0(\alpha)}|_w\leqslant\frac\alpha3.
\end{gather*}

Таким образом,
\begin{gather*}
|\bar{\mu}^{m_0(\alpha)}_{s_0(\alpha),p_0(\alpha)}-\mu|_w\leqslant |\bar{\mu}^{m_0(\alpha)}_{s_0(\alpha),p_0(\alpha)}-\bar\mu^{m_0(\alpha)}_{s_0(\alpha)}|_w+
|\bar{\mu}^{m_0(\alpha)}_{s_0(\alpha)}-\bar\mu^{m_0(\alpha)}|_w+|\bar{\mu}^{m_0(\alpha)}-\mu|_w\leqslant \frac\alpha3+\frac\alpha3+\frac\alpha3=\alpha,
\end{gather*}
то есть
\begin{gather}\label{iskomaya_posledovatel'nost'_mer!Banach.valued}
|\bar{\mu}^{m_0(\alpha)}_{s_0(\alpha),p_0(\alpha)}-\mu|_w\leqslant\alpha.
\end{gather}
Пусть $\alpha_k>0$, $k=1,2,\dots$, $\alpha_k\to0$, $k\to\infty$, --- некоторая последовательность. Тогда из (\ref{iskomaya_posledovatel'nost'_mer!Banach.valued}) следует, что
\begin{gather*}
\bar{\mu}^{m_0(\alpha_k)}_{s_0(\alpha_k),p_0(\alpha_k)}\to\mu,\,\,\,k\to\infty,\text{ $*$--слабо.}
\end{gather*}
Следовательно, в качестве искомой последовательности~$\omega^k$, $k=1,2,\dots$, можно взять последовательность~$\omega^k\equiv\bar\omega_{m_0(\alpha_k)}^{s_0(\alpha_k),p_0(\alpha_k)}$,
$k=1,2,\dots$. Лемма доказана.
\end{Proof}



        \section{Функциональные последовательности и ряды}
Пусть $X$ --- банахово пространство с нормой $\|\cdot\|_X$, $\mathcal{P}$ --- некоторое множество.
\begin{Definition}
Говорят, что последовательность функций $f_j\colon\mathcal{P}\to X$, $j=1,2,\dots$, равномерно по $p\in\mathcal{P}$ сходится в норме пространства $X$ к функции $f\colon\mathcal{P}\to X$
при $j\to\infty$ и пишем $f_j\mathop{\rightrightarrows}\limits^{X}_{p\in\mathcal{P}}f$, $j\to\infty$, если
\begin{gather*}
\forall\,\varepsilon>0\,\,\exists\,j_0=j_0(\varepsilon)\geqslant1\,\,\forall\,j\geqslant j_0(\varepsilon)\,\,\,\,\forall\,p\in\mathcal{P}:\|f_j(p)-f(p)\|_X\leqslant\varepsilon.
\end{gather*}
\end{Definition}

\begin{Theorem}\label{uniform_convergence_criterium} (Критерий равномерной сходимости) Для того, чтобы последовательность функций $f_j\colon\mathcal{P}\to X$, $j=1,2,\dots$, равномерно по
$p\in\mathcal{P}$ сходилась в норме пространства $X$ к некоторой функции $f\colon\mathcal{P}\to X$ при $j\to\infty$, необходимо и достаточно, чтобы
\begin{gather}\label{uniform_convergence_criterium:inequality}
\forall\,\varepsilon>0\,\,\exists\,j_0=j_0(\varepsilon)\geqslant1\,\,\forall\,j\geqslant j_0(\varepsilon)\,\,\forall\,k=1,2,\dots\,\,\forall\,p\in\mathcal{P}:\|f_{j+k}(p)-f_j(p)\|_X
\leqslant\varepsilon.
\end{gather}
\end{Theorem}
\begin{Proof}
1) Необходимость. Пусть $f_j\mathop{\rightrightarrows}\limits^{X}_{p\in\mathcal{P}}f$, $j\to\infty$, для некоторой функции $f\colon\mathcal{P}\to X$. Тогда для любого $\varepsilon>0$
найдётся номер $j_0=j_0(\varepsilon)\geqslant1$, такой, что
\begin{gather*}
\sup\limits_{p\in\mathcal{P}}\|f_j(p)-f(p)\|_X\leqslant\frac\varepsilon2\,\,\,\forall\,j\geqslant j_0(\varepsilon).
\end{gather*}
Поэтому
\begin{gather*}
\sup\limits_{p\in\mathcal{P}}\|f_{j+k}(p)-f(p)\|_X\leqslant\frac\varepsilon2\,\,\,\forall\,j\geqslant j_0(\varepsilon)\,\,\,\forall\,k=1,2,\dots
\end{gather*}
Как следствие,
\begin{gather*}
\sup\limits_{p\in\mathcal{P}}\|f_{j+k}(p)-f_j(p)\|_X\leqslant\sup\limits_{p\in\mathcal{P}}\|f_{j+k}(p)-f(p)\|_X+
\sup\limits_{p\in\mathcal{P}}\|f(p)-f_j(p)\|_X\leqslant\frac\varepsilon2+\frac\varepsilon2=\varepsilon\\
\forall\,j\geqslant j_0(\varepsilon)\,\,\,\forall\,k=1,2,\dots
\end{gather*}
Таким образом, необходимость доказана.

2) Достаточность. Пусть выполнено условие (\ref{uniform_convergence_criterium:inequality}). Тогда для каждого фиксированного $p\in\mathcal{P}$ получаем последовательность $f_j(p)\in X$,
$j=1,2,\dots$, фундаментальную в $X$. Поскольку $X$ --- полно, то при каждом фиксированном $p\in\mathcal{P}$ найдётся элемент $f(p)\in X$, такой, что
\begin{gather*}
\lim\limits_{j\to\infty}\|f_j(p)-f(p)\|_X=0.
\end{gather*}
Устремляя теперь в (\ref{uniform_convergence_criterium:inequality}) $k$ к бесконечности, выводим, что
\begin{gather*}
\forall\,\varepsilon>0\,\,\exists\,j_0=j_0(\varepsilon)\geqslant1\,\,\forall\,j\geqslant j_0(\varepsilon)\,\,\,\,\forall\,p\in\mathcal{P}:\|f_j(p)-f(p)\|_X\leqslant\varepsilon.
\end{gather*}
А это и означает, что $f_j\mathop{\rightrightarrows}\limits^{X}_{p\in\mathcal{P}}f$, $j\to\infty$. Теорема доказана.
\end{Proof}

\begin{Definition}
Пусть дана последовательность функций $f_j\colon\mathcal{P}\to X$, $j=1,2,\dots$ Говорят, что функциональный ряд
\begin{gather*}
\sum\limits_{j=1}^\infty f_j(p)
\end{gather*}
сходится к некоторой функции $F(p)$ равномерно на $\mathcal{P}$, если
\begin{gather*}
\lim\limits_{k\to\infty}\sup\limits_{p\in\mathcal{P}}\Biggl\|\sum\limits_{j=1}^kf_j(p)-F(p)\Biggr\|_X=0.
\end{gather*}
\end{Definition}

Пусть $\mathcal{H}$ --- сепарабельное гильбертово пространство со скалярным произведением $\langle\cdot,\cdot\rangle_\mathcal{H}$ и соответствующей нормой $\|\cdot\|_\mathcal{H}$. Пусть,
кроме того, $e_j$, $j=1,2,\dots$, --- ортогональный базис в $\mathcal{H}$.
\begin{Theorem}\label{priznak:uniform_convergence_of_functional_sequence}
Пусть даны функции $f_j\colon\mathcal{P}\to\mathbb{R}$, $j=1,2,\dots$ Тогда если найдётся числовая последовательность $\alpha_j$, $j=1,2,\dots$, такая, что
\begin{gather*}
\sup\limits_{p\in\mathcal{P}}|f_j(p)|\leqslant\alpha_j,\,\,\,j=1,2,\dots,
\end{gather*}
и сходится числовой ряд
\begin{gather*}
\sum\limits_{j=1}^\infty\alpha_j^2\|e_j\|^2_{\mathcal{H}},
\end{gather*}
то функциональный ряд
\begin{gather*}
\sum\limits_{j=1}^\infty f_j(p)e_j
\end{gather*}
сходится равномерно.
\end{Theorem}
\begin{Proof}
Для доказательства воспользуемся теоремой \ref{uniform_convergence_criterium}. Прежде всего заметим, что для всех $i,\,\,k=1,2,\dots$ и всех $p\in\mathcal{P}$ имеет место неравенство
\begin{gather*}
\left\|\sum\limits_{j=1}^{k+i}f_j(p)e_j-\sum\limits_{j=1}^{i}f_j(p)e_j\right\|^2=\left\|\sum\limits_{j=i+1}^{k+i}f_j(p)e_j\right\|^2=
\sum\limits_{j=i+1}^{k+i}|f_j(p)|^2\|e_j\|^2_{\mathcal{H}}\leqslant\sum\limits_{j=i+1}^{k+i}\alpha_j^2\|e_j\|^2_{\mathcal{H}}.
\end{gather*}
Таким образом,
\begin{gather*}
\left\|\sum\limits_{j=1}^{k+i}f_j(p)e_j-\sum\limits_{j=1}^{i}f_j(p)e_j\right\|^2\leqslant\sum\limits_{j=i+1}^{k+i}\alpha_j^2\|e_j\|^2_{\mathcal{H}}
\end{gather*}
для всех $i,\,\,k=1,2,\dots$ и всех $p\in\mathcal{P}$.

Поскольку числовой ряд
\begin{gather*}
\sum\limits_{j=1}^\infty\alpha_j^2\|e_j\|^2_{\mathcal{H}},
\end{gather*}
сходится, то для любого $\varepsilon>0$ найдётся номер $i_0=i_0(\varepsilon)\geqslant1$, такой, что для всех $i\geqslant i_0(\varepsilon)$ и всех $k\geqslant1$
\begin{gather*}
\sum\limits_{j=i+1}^{k+i}\alpha_j^2\|e_j\|^2_{\mathcal{H}}\leqslant\varepsilon^2.
\end{gather*}
Поэтому
\begin{gather*}
\sup\limits_{p\in\mathcal{P}}\left\|\sum\limits_{j=1}^{k+i}f_j(p)e_j-\sum\limits_{j=1}^{i}f_j(p)e_j\right\|\leqslant
\sqrt{\sum\limits_{j=i+1}^{k+i}\alpha_j^2\|e_j\|^2_{\mathcal{H}}}\leqslant\varepsilon
\end{gather*}
при всех $i\geqslant i_0(\varepsilon)$ и всех $k\geqslant1$. Согласно теореме \ref{uniform_convergence_criterium} это означает, что функциональная последовательность
\begin{gather*}
\mathcal{P}\ni p\mapsto\sum\limits_{j=1}^k f_j(p)e_j,\,\,\,k=1,2,\dots,
\end{gather*}
равномерно по $p\in\mathcal{P}$ сходится в норме пространства $\mathcal{H}$ к некоторой функции $F\colon\mathcal{P}\to\mathcal{H}$ при $k\to\infty$, что, в свой черёд, означает равномерную
сходимость функционального ряда
\begin{gather*}
\sum\limits_{j=1}^\infty f_j(p)e_j.
\end{gather*}
Теорема доказана.
\end{Proof}

Пусть теперь $\mathcal{P}$ --- компактное топологическое пространство.
\begin{Theorem}\label{consequence_of_uniform_convergence_of_functional_sequence}
Пусть последовательность функций $f_j\in C(\mathcal{P},X)$, $j=1,2,\dots$, равномерно по $p\in\mathcal{P}$ сходится в норме пространства
$X$ к некоторой функции $f\colon\mathcal{P}\to X$ при $j\to\infty$. Тогда $f\in C(\mathcal{P},X)$.
\end{Theorem}
\begin{Proof}
Выберем произвольно $\varepsilon>0$ и зафиксируем. В силу того, что $f_j\mathop{\rightrightarrows}\limits^{X}_{p\in\mathcal{P}}f$, $j\to\infty$, можно выбрать номер $j_0=j_0(\varepsilon)
\geqslant1$ так, чтобы
\begin{gather*}
\sup\limits_{p\in\mathcal{P}}\|f_j(p)-f(p)\|_X\leqslant\frac\varepsilon3
\end{gather*}
при всех $j\geqslant j_0(\varepsilon)$, и, в частности,
\begin{gather*}
\sup\limits_{p\in\mathcal{P}}\|f_{j_0(\varepsilon)}(p)-f(p)\|_X\leqslant\frac\varepsilon3.
\end{gather*}
Выберем теперь $p_0\in\mathcal{P}$ и зафиксируем. Поскольку $f_{j_0(\varepsilon)}\in C(\mathcal{P},X)$, то по заданному $\varepsilon>0$ можно так подобрать окрестность $\mathcal{V}=
\mathcal{V}(\varepsilon)$ точки $p_0$, чтобы для всех $p\in\mathcal{V}(\varepsilon)$
\begin{gather*}
\|f_{j_0(\varepsilon)}(p)-f_{j_0(\varepsilon)}(p_0)\|_X\leqslant\frac\varepsilon3.
\end{gather*}
Поэтому
\begin{gather*}
\|f(p)-f(p_0)\|_X\leqslant\|f(p)-f_{j_0(\varepsilon)}(p)\|_X+\|f_{j_0(\varepsilon)}(p)-f_{j_0(\varepsilon)}(p_0)\|_X+\|f_{j_0(\varepsilon)}(p_0)-f(p_0)\|_X\leqslant\\
\leqslant\frac\varepsilon3+\frac\varepsilon3+\frac\varepsilon3=\varepsilon
\end{gather*}
для всех $p\in\mathcal{V}(\varepsilon)$. Таким образом, для любого $\varepsilon>0$ найдётся окрестность $\mathcal{V}=\mathcal{V}(\varepsilon)$ точки $p_0$, такая, что при всех $p\in
\mathcal{V}(\varepsilon)$
\begin{gather*}
\|f(p)-f(p_0)\|_X\leqslant\varepsilon.
\end{gather*}
В силу произвольности $p_0\in\mathcal{P}$ это означает, что $f\in C(\mathcal{P},X)$.
\end{Proof}

\begin{Corrolary}\label{consequence_of_uniform_convergence_of_functional_series}
Пусть дана последовательность функций $f_j\in C(\mathcal{P},X)$, $j=1,2,\dots$ Если функциональный ряд
\begin{gather*}
\sum\limits_{j=1}^\infty f_j(p)
\end{gather*}
сходится равномерно на $\mathcal{P}$, то сумма ряда является элементом $f\in C(\mathcal{P},X)$.
\end{Corrolary}

\begin{Theorem}\label{differetiability_of_functional_sequence}
Пусть последовательность функций $f_j\in C^1([0,T],X)$, $j=1,2,\dots$, при каждом $t\in[0,T]$ сходится в норме $X$ к функции $f\colon[0,T]\to X$, а последовательность производных $f_j'$,
$j=1,2,\dots$, равномерно по $t\in[0,T]$ сходится в норме пространства $X$ к некоторой функции $g\colon[0,T]\to X$ при $j\to\infty$. Тогда $f\in C^1([0,T],X)$ и $f'=g$. Иными словами,
\begin{gather*}
\frac{d}{dt}\left[\lim\limits_{j\to\infty}f_j(t)\right]=\lim\limits_{j\to\infty}\frac{df_j(t)}{dt}\,\,\,\forall\,t\in[0,T].
\end{gather*}
\end{Theorem}
\begin{Proof} В силу теоремы \ref{consequence_of_uniform_convergence_of_functional_sequence} имеет место включение $g\in C([0,T],X)$. Далее, поскольку $f_j\in C^1([0,T],X)$, $j=1,2,\dots$,
то
\begin{gather}\label{m1}
f_j(t)-f_j(0)-\int\limits_0^tf_j'(\tau)d\tau=0\,\,\,\forall\,t\in[0,T].
\end{gather}
Выберем произвольно $t\in[0,T]$ и зафиксируем. Тогда, в силу условий теоремы,
\begin{gather*}
\left\|\left[f_j(t)-f_j(0)-\int\limits_0^tf_j'(\tau)d\tau\right]-\left[f(t)-f(0)-\int\limits_0^tg'(\tau)d\tau\right]\right\|_X\leqslant\|f_j(t)-f(t)\|_X+\|f_j(0)-f(0)\|_X+\\
+\left\|\int\limits_0^t[f_j'(\tau)-g(\tau)]d\tau\right\|_X\leqslant\|f_j(t)-f(t)\|_X+\|f_j(0)-f(0)\|_X+\int\limits_0^t\|f_j'(\tau)-g(\tau)\|_Xd\tau\leqslant\\
\leqslant\|f_j(t)-f(t)\|_X+\|f_j(0)-f(0)\|_X+t\max\limits_{\tau\in[0,T]}\|f_j'(\tau)-g(\tau)\|_X\to0,\,\,\,j\to\infty.
\end{gather*}
Переходя теперь с учётом данного соотношения к пределу при $j\to\infty$ в равенстве (\ref{m1}), получим, что
\begin{gather*}
f(t)-f(0)-\int\limits_0^tg(\tau)d\tau=0\,\,\,\forall\,t\in[0,T].
\end{gather*}
В силу свойств интеграла Римана и непрерывности функции $g$ в норме $X$ это означает, что $f\in C^1([0,T],X)$ и $f'=g$. Теорема доказана.
\end{Proof}

Следствием данной теоремы является
\begin{Theorem}\label{differetiability_of_functional_sequence:Delta}
Пусть дана последовательность функций $f_j\in C(\Gamma,X)$, $j=1,2,\dots$, таких, что  $f_{jt}\in C(\Gamma,X)$, $j=1,2,\dots$ Пусть, кроме того, последовательность $f_j$, $j=1,2,\dots$,
сходится при каждом $(t,\xi)\in\Gamma$ в норме $X$ к функции $f\colon\Gamma\to X$, и при каждом фиксированном $\xi\in[0,T]$ последовательность $f_{jt}(\cdot,\xi)$, $j=1,2,\dots$, равномерно
по $t\in[0,T]$ сходится в норме $X$ к функции $g(\cdot,\xi)$. Тогда при каждом фиксированном $\xi\in[0,T]$ функция $f(\cdot,\xi)$ является элементом $C^1([0,T],X)$ и $f_t=g$.
\end{Theorem}
\begin{Proof}
Выберем произвольно $\xi\in[0,T]$ и зафиксируем. Положим $\Phi_{j,\xi}(t)\equiv f_j(t,\xi)$, $\Psi_{j,\xi}(t)\equiv f_{jt}(t,\xi)$, $\Phi_{0,\xi}(t)\equiv f(t,\xi)$, $\Psi_{0,\xi}(t)\equiv
g(t,\xi)$, $j=1,2,\dots$, $t\in[0,T]$. Тогда получим, что при каждом $t\in[0,T]$ последовательность $\Phi_{j,\xi}$, $j=1,2,\dots$, $t\in[0,T]$, сходится в норме $X$ к $\Phi_{0,\xi}$, а
последовательность $\Psi_{j,\xi}$, $j=1,2,\dots$, равномерно по $t\in[0,T]$ сходится в норме $X$ к $\Psi_{0,\xi}$. Таким образом, для последовательности $\Phi_{j,\xi}(t)\equiv f_j(t,\xi)$,
$j=1,2,\dots$, $t\in[0,T]$, выполнены условия теоремы \ref{differetiability_of_functional_sequence}, откуда и следуют утверждения доказываемой теоремы. Теорема доказана.
\end{Proof}

\begin{Corrolary}\label{consequence::differetiability_of_functional_sequence}
Пусть дана последовательность функций $f_j\in C^1([0,T],X)$, $j=1,2,\dots$, и пусть функциональный ряд
\begin{gather*}
\sum\limits_{j=1}^\infty f_j(t)
\end{gather*}
сходится в норме $X$ при каждом фиксированном $t\in[0,T]$ к функции $F\colon[0,T]\to X$, а функциональный ряд
\begin{gather*}
\sum\limits_{j=1}^\infty f_j'(t)
\end{gather*}
равномерно по $t\in[0,T]$ сходится в норме $X$ к функции $G\colon[0,T]\to X$. Тогда $F\in C^1([0,T],X)$ и $F'=G$. Иными словами,
\begin{gather*}
\frac{d}{dt}\Biggl[\sum\limits_{j=1}^\infty f_j(t)\Biggr]=\sum\limits_{j=1}^\infty\frac{df_j(t)}{dt}\,\,\,\forall\,t\in[0,T].
\end{gather*}
\end{Corrolary}


\begin{Corrolary}\label{consequence::differetiability_of_functional_sequence:Delta}
Пусть дана последовательность функций $f_j\in C(\Gamma,X)$, $j=1,2,\dots$, таких, что  $f_{jt}\in C(\Gamma,X)$, $j=1,2,\dots$ Пусть, кроме того, функциональный ряд
\begin{gather*}
\sum\limits_{j=1}^\infty f_j(t,\xi)
\end{gather*}
сходится в норме $X$ при каждом фиксированном $(t,\xi)\in\Gamma$ к функции $F\colon\Gamma\to X$, и при каждом фиксированном $\xi\in[0,T]$
ряд
\begin{gather*}
\sum\limits_{j=1}^\infty f_{jt}(t,\xi)
\end{gather*}
равномерно по $t\in[0,T]$ сходится в норме $X$ к функции $G(\cdot,\xi)$. Тогда при каждом фиксированном $\xi\in[0,T]$ функция $F(\cdot,\xi)$ является элементом $C^1([0,T],X)$ и $F_t=G$.
\end{Corrolary}

\begin{Lemma}\label{Dini1} \cite{Il'inPoznyak} (Признак Дини равномерной сходимости)
Пусть $G\subset \mathbb{R}^m$ --- компакт, а функции $f_k\colon G\to \mathbb{R}$, $k=1,2,\dots$, и функция $f\colon G\to \mathbb{R}$ непрерывны на $G$, причём $f_k(x)\to f(x)$,
$k\to\infty$, при всех $x\in G$. Тогда если либо
\begin{gather*}
f_k(x)\geqslant f_{k+1}(x)\,\,\,\forall\,k=1,2,\dots,\,\,\,x\in G,
\end{gather*}
либо
\begin{gather*}
f_{k+1}(x)\geqslant f_{k}(x)\,\,\,\forall\,k=1,2,\dots,\,\,\,x\in G,
\end{gather*}
то
\begin{gather*}
\lim\limits_{k\to\infty}\max\limits_{x\in G}|f_k(x)-f(x)|=0.
\end{gather*}
\end{Lemma}


\begin{Lemma}\label{CkGH_LpGH:approximation}
Пусть $H$ --- сепарабельное гильбертово пространство со скалярным произведением $\langle\cdot,\cdot\rangle_H$ и соответствующей нормой $\|\cdot\|_H$. Пусть $h_k\in H$, $k=1,2,\dots$, ---
ортогональный базис в $H$, $p\in[1,+\infty)$, $s\geqslant0$ --- фиксированное целое число, $G_1\subset \mathbb{R}^{m_1}$, $G_2\subset \mathbb{R}^{m_2}$ --- компакты. Тогда для любых
функций $\vartheta_0\in L_p(G_1,H)$, $\vartheta_1\in C^s(G_1,H)$, $\vartheta_2\in C^s(G_1,L_p(G_2,H))$ справедливы равенства
\begin{gather*}
\lim\limits_{N\to\infty}\|\vartheta_0^N-\vartheta_0\|_{p,G_1,H}=0,\,\,\,\lim\limits_{N\to\infty}\pmb{|}\vartheta_1^N-\vartheta_1\pmb{|}^{(s)}_{G_1,H}=0,\,\,\,
\lim\limits_{N\to\infty}\pmb{|}\vartheta_2^N-\vartheta_2\pmb{|}^{(s)}_{G_1,L_p(G_2,H)}=0,
\end{gather*}
где введены обозначения
\begin{gather*}
\vartheta_0^N(x)\equiv\sum\limits_{k=1}^N\vartheta_{0k}(x)h_k,\,\,\, \vartheta_1^N(x)\equiv\sum\limits_{k=1}^N\vartheta_{1k}(x)h_k,\,\,\,
\vartheta_2^N(x,y)\equiv\sum\limits_{k=1}^N\vartheta_{2k}(x,y)h_k,\\
\vartheta_{0j}(x)\equiv\langle\vartheta_0(x),h_j\rangle_H,\,\,\, \vartheta_{1j}(x)\equiv\langle\vartheta_1(x),h_j\rangle_H,\,\,\,
\vartheta_{2j}(x,y)\equiv\langle\vartheta_2(x,y),h_j\rangle_H,\\
(x,y)\in G_1\times G_2,\,\,\,j,\,\,N=1,2,\dots
\end{gather*}
\end{Lemma}
\begin{Proof}
1) Докажем сначала утверждение о функции $\vartheta_0$. Действительно, поскольку  $h_k\in H$, $k=1,2,\dots$, --- ортогональный базис в $H$, то
\begin{gather*}
\|\vartheta_0^N(x)-\vartheta_0(x)\|_H^p= \left[\|\vartheta_0(x)\|_H^2-\sum\limits_{j=1}^N[\|h_j\|_H\vartheta_{0j}(x)]^2\right]^{p/2}
\leqslant\|\vartheta_0(x)\|_H^p\mbox{ при п.в. $x\in G_1$;}\\
\|\vartheta_0^N(x)-\vartheta_0(x)\|_H^p\to0,\,\,\,N\to\infty,\mbox{ при п.в. $x\in G_1$.}
\end{gather*}
Пользуясь затем теоремой Лебега о предельном переходе под знаком интеграла Лебега, заключаем, что
\begin{gather*}
\lim\limits_{N\to\infty}\|\vartheta_0^N-\vartheta_0\|_{p,G_1,H}=0.
\end{gather*}

2) Докажем утверждение о функции $\vartheta_1$. Пусть мультииндекс $i=(i_1,\dots,i_{m_1})$, $|i|=\overline{0,s}$, --- произволен. Положим $r_{N,i}(x)\equiv \left\|\frac{\partial^{|i|}
\vartheta_1^N(x)}{\partial x_1^{i_1}\dots\partial x_{m_1}^{i_{m_1}}}-\frac{\partial^{|i|}\vartheta_1(x)}{\partial x_1^{i_1}\dots\partial x_{m_1}^{i_{m_1}}}\right\|_H$, $x\in G_1$. Тогда
нетрудно видеть, что $r_{N,i}$ непрерывна на $G_1$, причём
\begin{gather*}
r_{N,i}(x)\geqslant r_{N+1,i}(x)\,\,\,\forall\,x\in G_1,\,\,\,N=1,2,\dots;\,\,\, r_{N,i}(x)\to0,\,\,\,N\to\infty,\,\,\,\forall\,x\in G_1.
\end{gather*}
Применяя лемму \ref{Dini1}, получаем, что
\begin{gather*}
\lim\limits_{N\to\infty}\pmb{|}\vartheta_1^N-\vartheta_1\pmb{|}^{(s)}_{G_1,H}=0.
\end{gather*}

3) Докажем утверждение о функции $\vartheta_2$. Пусть мультииндекс $i=(i_1,\dots,i_{m_1})$, $|i|=\overline{0,s}$, --- произволен. Положим
$\tau_{N,i}(x)\equiv \left\|\frac{\partial^{|i|}\vartheta_2^N(x,\cdot)}{\partial x_1^{i_1}\dots\partial x_{m_1}^{i_{m_1}}}-
\frac{\partial^{|i|}\vartheta_2(x,\cdot)}{\partial x_1^{i_1}\dots\partial x_{m_1}^{i_{m_1}}} \right\|_{p,G_2,H}$, $x\in G_1$. В силу доказанного в первом пункте справедливы соотношения
\begin{gather*}
\tau_{N,i}(x)\geqslant \tau_{N+1,i}(x)\,\,\,\forall\,x\in G_1,\,\,\,N=1,2,\dots;\,\,\, \tau_{N,i}(x)\to0,\,\,\,N\to\infty,\,\,\,
\forall\,x\in G_1.
\end{gather*}
Применяя лемму \ref{Dini1}, получаем, что
\begin{gather*}
\lim\limits_{N\to\infty}\pmb{|}\vartheta_2^N-\vartheta_2\pmb{|}^{(s)}_{G_1,L_p(G_2,H)}=0.
\end{gather*}
Лемма полностью доказана.
\end{Proof}

Пусть $V$ и $H$ --- сепарабельные гильбертовы пространства со скалярными произведениями $\langle\cdot,\cdot\rangle_V$ и $\langle\cdot,\cdot\rangle_H$ соответственно. Нормы, соответствующие
этим скалярным произведениям, обозначим через $\|\cdot\|_V$ и $\|\cdot\|_H$. Пусть, кроме того, $V\subset H$, причём это вложение плотно и компактно. Наконец, пусть $g_k\in V$,
$k=1,2,\dots$, --- ортогональная в $V$ и ортонормированная в $H$ система, такая, что для любых $\varphi\in V$ и $\psi\in H$ справедливо равенство
\begin{gather}\label{VHphipsi_approx}
\lim\limits_{N\to\infty}\|\varphi^N-\varphi\|_V=0,\,\,\,\lim\limits_{N\to\infty}\|\psi^N-\psi\|_H=0,
\end{gather}
где
\begin{gather*}
\varphi^N\equiv\sum\limits_{m=1}^N\varphi_mg_m,\,\,\,\psi^N\equiv\sum\limits_{m=1}^N\psi_mg_m,
\,\,\,\varphi_k\equiv\langle\varphi,g_k\rangle_H,\,\,\, \psi_k\equiv\langle\psi,g_k\rangle_H,\,\,\,k,\,\,N=1,2,\dots
\end{gather*}
Пусть $s\geqslant0$ --- фиксированное целое число, $G\subset \mathbb{R}^{m}$ --- замкнутая ограниченная область с кусочно--гладкой границей.

Покажем, что справедлива
\begin{Lemma}\label{CkVH_approx}
Для любых функций $\vartheta_0\in C^s(G,V)$, $\vartheta_1\in C^s(G,H)$, $\vartheta_2\in C^s([0,T],L_1(G,H))$, $\vartheta_3\in C^s(G\times[0,T],V)$ имеют место соотношения
\begin{gather}\label{vartheta01approx}
\lim\limits_{N\to\infty}\pmb{|}\vartheta_0^N-\vartheta_0\pmb{|}^{(s)}_{G,V}=0,\,\,\,\lim\limits_{N\to\infty}\pmb{|}\vartheta_1^N-\vartheta_1\pmb{|}^{(s)}_{G,H}=0,\\
\label{vartheta23approx} \lim\limits_{N\to\infty}\pmb{|}\vartheta_2^N-\vartheta_2\pmb{|}^{(s)}_{[0,T],L_1(G,H)}=0,\,\,\,
\lim\limits_{N\to\infty}\pmb{|}\vartheta_3^N-\vartheta_3\pmb{|}^{(s)}_{G\times[0,T],V}=0,
\end{gather}
где введены обозначения
\begin{gather*}
\vartheta_0^N(x)\equiv\sum\limits_{k=1}^N\vartheta_{0k}(x)g_k,\,\,\, \vartheta_1^N(x)\equiv\sum\limits_{k=1}^N\vartheta_{1k}(x)g_k,\,\,\,
\vartheta_2^N(x,t)\equiv\sum\limits_{k=1}^N\vartheta_{2k}(x,t)g_k,\\
\vartheta_3^N(x,t)\equiv\sum\limits_{k=1}^N\vartheta_{3k}(x,t)g_k,\,\,\, \vartheta_{0j}(x)\equiv\langle\vartheta_0(x),g_j\rangle_H,\,\,\,
\vartheta_{1j}(x)\equiv\langle\vartheta_1(x),g_j\rangle_H,\,\,\,\vartheta_{2j}(x,t)\equiv\langle\vartheta_2(x,t),g_j\rangle_H,\\
\vartheta_{3j}(x,t)\equiv\langle\vartheta_3(x,t),g_j\rangle_H,\,\,\,(x,t)\in G\times[0,T],\,\,\,j,\,\,N=1,2,\dots
\end{gather*}
При этом пространство $C^1(G,V)$ всюду плотно в  $C(G,H)$, а пространство $C^1(G\times[0,T],V)$ всюду плотно в $C([0,T],L_1(G,H))$.
\end{Lemma}
\begin{Proof}
1) Предельные соотношения (\ref{vartheta01approx}) и (\ref{vartheta23approx}) являются непосредственными следствиями условий на систему
$g_j\in V$, $j=1,2,\dots$, и леммы \ref{CkGH_LpGH:approximation}.

2) Утверждение о плотности  $C^1(G_1,V)$ в $C(G_1,H)$ является следствием классической теоремы Вейерштрасса об аппроксимации непрерывных на
отрезке вещественнозначных функций многочленами и предельных соотношений (\ref{vartheta01approx}) и (\ref{vartheta23approx}).

3) Докажем утверждение о плотности $C^1(G\times[0,T],V)$ в $C([0,T],L_1(G,H))$. Нетрудно видеть, что $\vartheta_{2k}\in C([0,T],L_1(G))$, $k=1,2,\dots$ Поэтому нам достаточно доказать, что
пространство $C^1(G\times[0,T])$ всюду плотно в пространстве $C([0,T],L_1(G))$. В самом деле, согласно лемме \ref{Weierstrass:approx}, множество многочленов с коэффициентами из $L_1(G)$
всюду плотно в $C([0,T],L_1(G))$. Поскольку же, как известно, множество $C^1(G)$ всюду плотно в  $L_1(G)$, то множество $C^1(G\times[0,T])$ всюду плотно в пространстве
$C([0,T],L_1(G))$. Лемма полностью доказана.
\end{Proof}

        \section{Интегралы, зависящие от параметра}
Пусть $X$ --- банахово пространство с нормой $\|\cdot\|_X$, $Y$ --- банахово пространство с нормой $\|\cdot\|_Y$.
\begin{Theorem}\label{parametric_integral_0}
Пусть задана функция $f\colon\Gamma\to X$. Если $f\in C(\Gamma,X)$, то функция
\begin{gather*}
[0,T]\ni t\mapsto(\textrm{Р})\int\limits_0^tf(t,\xi)d\xi
\end{gather*}
принадлежит классу $C([0,T],X)$.
\end{Theorem}
\begin{Proof}
Поскольку $f\in C(\Gamma,X)$, то согласно теореме \ref{uniform_continuity} она равномерно непрерывна на $\Gamma$, т.е.
\begin{gather}\label{fucontinuityDelta}
\forall\,\varepsilon>0\,\,\exists\,\delta=\delta(\varepsilon)>0\,\,\forall\,(t',\xi'),\,\,(t'',\xi'')\in\Gamma,\,\,|(t',\xi')-(t'',\xi'')|<\delta:\\
\notag\|f(t',\xi')-f(t'',\xi'')\|_X<\frac\varepsilon{T}.
\end{gather}
Пусть $t_0$, $t\in[0,T]$ --- произвольны. Тогда
\begin{gather*}
\int\limits_0^tf(t,\xi)d\xi-\int\limits_0^{t_0}f(t_0,\xi)d\xi=\int\limits_0^tf(t,\xi)d\xi-\int\limits_0^{t_0}f(t,\xi)d\xi+\int\limits_0^{t_0}[f(t,\xi)-f(t_0,\xi)]d\xi=\\
=\int\limits_{t_0}^tf(t,\xi)d\xi+\int\limits_0^{t_0}[f(t,\xi)-f(t_0,\xi)]d\xi,
\end{gather*}
откуда следует, что
\begin{gather*}
\left\|\int\limits_0^tf(t,\xi)d\xi-\int\limits_0^{t_0}f(t_0,\xi)d\xi\right\|_X\leqslant\left|\int\limits_{t_0}^t\|f(t,\xi)\|_Xd\xi\right|+\int\limits_0^{t_0}\|f(t,\xi)-f(t_0,\xi)\|_Xd\xi
\leqslant\\
\leqslant|t-t_0|\max\limits_{(\tau,\xi)\in\Gamma}\|f(\tau,\xi)\|_X+ \int\limits_0^{T}\|f(t,\xi)-f(t_0,\xi)\|_Xd\xi.
\end{gather*}
Таким образом, нам достаточно лишь доказать, что
\begin{gather}\label{m100}
\lim\limits_{t\to_0}\int\limits_0^{T}\|f(t,\xi)-f(t_0,\xi)\|_Xd\xi=0.
\end{gather}
В самом деле, выберем произвольно $\varepsilon>0$ и зафиксирум. Подберём $\delta=\delta(\varepsilon)$ согласно (\ref{fucontinuityDelta}) и
выберем $t\in[0,T]$ так, чтобы $|t-t_0|<\delta$. Тогда, в силу (\ref{fucontinuityDelta}),
\begin{gather*}
\|f(t,\xi)-f(t_0,\xi)\|_X<\frac\varepsilon T
\end{gather*}
при всех $\xi\in[0,T]$. Поэтому
\begin{gather*}
\int\limits_0^{T}\|f(t,\xi)-f(t_0,\xi)\|_Xd\xi\leqslant\varepsilon.
\end{gather*}
Таким образом, соотношение (\ref{m100}), а вместе с ним и настоящая теорема, доказаны.
\end{Proof}


\begin{Lemma}\label{parametric_integral_1}
Пусть $\Pi(t,\xi)\in\mathcal{L}(X,Y)$ при всех $(t,\xi)\in\Gamma$, при всех $x\in X$ функция $\Gamma\ni(t,\xi)\mapsto\Pi(t,\xi)x$ принадлежит $C(\Gamma,Y)$, и пусть $z\in L_1([0,T],X)$.
Тогда функция
\begin{gather*}
[0,T]\ni t\mapsto\int\limits_0^t\Pi(t,\xi)z(\xi)d\xi
\end{gather*}
непрерывна на $[0,T]$ в норме $Y$.
\end{Lemma}
\begin{Proof}
Выберем произвольно функцию $y\in L_1([0,T],X)$ и зафиксируем. Введём обозначение
\begin{gather*}
\Theta(t)=\int\limits_0^t\Pi(t,\xi)y(\xi)d\xi,\,\,\,t\in[0,T].
\end{gather*}
Согласно лемме \ref{boundness_Pi(g)_belonging_L(X,Y)}, найдётся постоянная $K>0$, такая, что
$$
\sup\limits_{(t,\xi)\in\Gamma}\|\Pi(t,\xi)\|_{X\to Y}\leqslant K.
$$
Поэтому при каждом фиксированном $t\in[0,T]$ функция $[0,T]]\ni\xi\mapsto\Pi(t,\xi)z(\xi)$ принадлежит пространству $L_1([0,T],Y)$.


Утверждение леммы эквивалентно включению $\Theta\in C([0,T],Y)$. Докажем его.

Пусть $t$, $t+\Delta t\in[0,T]$ --- произвольны. Тогда
\begin{gather*}
\Theta(t+\Delta t)-\Theta(t)=\int\limits_0^{t+\Delta t}\Pi(t+\Delta t,\xi)y(\xi)d\xi-\int\limits_0^t\Pi(t,\xi)y(\xi)d\xi=\\
=\int\limits_0^{t+\Delta t}[\Pi(t+\Delta t,\xi)-\Pi(t,\xi)]y(\xi)d\xi+\int\limits_t^{t+\Delta t}\Pi(t,\xi)y(\xi)d\xi.
\end{gather*}
Поэтому
\begin{gather*}
\|\Theta(t+\Delta t)-\Theta(t)\|_Y\leqslant\int\limits_0^T\|[\Pi(t+\Delta t,\xi)-\Pi(t,\xi)]y(\xi)\|_Yd\xi+\left\|\int\limits_t^{t+\Delta t}\Pi(t,\xi)y(\xi)d\xi\right\|_Y.
\end{gather*}
Первое слагаемое в правой части последнего неравенства стремится к нулю при $\Delta t\to0$ в силу теоремы Лебега о предельном переходе под знаком интеграла Лебега, а второе --- за счёт
абсолютной непрерывности интеграла Бохнера. Таким образом, включение $\Theta\in C([0,T],Y)$ доказано, а вместе с ним полностью доказана и данная лемма.
\end{Proof}

\begin{Theorem}
Пусть задана функция $f\colon\Gamma\to X$, $f=f(t,\xi)$. Если $f$, $f_t\in C(\Gamma,X)$, то функция
\begin{gather}\label{m300}
[0,T]\ni t\mapsto\int\limits_0^tf(t,\xi)d\xi
\end{gather}
непрерывно дифференцируема на $[0,T]$ в норме $X$, причём
\begin{gather}\label{m200}
\frac{d}{dt}\int\limits_0^tf(t,\xi)d\xi=f(t,t)+\int\limits_0^tf_t(t,\xi)d\xi\,\,\,\forall\,t\in[0,T].
\end{gather}
\end{Theorem}
\begin{Proof}
Введём обозначения
\begin{gather*}
\Theta_0(t)\equiv\int\limits_0^tf(t,\xi)d\xi,\,\,\,\Theta_1(t)\equiv f(t,t)+\int\limits_0^tf_t(t,\xi)d\xi,\,\,\,t\in[0,T].
\end{gather*}
Принадлежность функций $\Theta_0$ и $\Theta_1$ банахову пространству $C([0,T],X)$ следует из условий на функцию $f$  и теоремы
\ref{parametric_integral_0}. Таким образом, нам нужно лишь доказать, что $\Theta_0'=\Theta_1$.

В самом деле, пусть $t$, $t+\Delta t\in [0,T]$ --- произвольны. Тогда
\begin{gather*}
\Theta_0(t+\Delta t)-\Theta_0(t)=\int\limits_0^{t+\Delta t}f(t+\Delta t,\xi)d\xi-\int\limits_0^tf(t,\xi)d\xi=
\int\limits_0^{t+\Delta t}f(t+\Delta t,\xi)d\xi-\int\limits_0^{t}f(t+\Delta t,\xi)d\xi+\\
+\int\limits_0^{t}[f(t+\Delta t,\xi)-f(t,\xi)]d\xi=\int\limits_t^{t+\Delta t}f(t+\Delta t,\xi)d\xi+\int\limits_0^{t}[f(t+\Delta t,\xi)-f(t,\xi)]d\xi=\\
=\int\limits_t^{t+\Delta t}[f(t+\Delta t,\xi)-f(t,t)]d\xi+\int\limits_t^{t+\Delta t}f(t,t)d\xi+\int\limits_0^{t}[f(t+\Delta t,\xi)-f(t,\xi)]d\xi=\Delta tf(t,t)+\\
+\int\limits_0^{t}[f(t+\Delta t,\xi)-f(t,\xi)]d\xi+\int\limits_t^{t+\Delta t}[f(t+\Delta t,\xi)-f(t,t)]d\xi,
\end{gather*}
откуда вытекает, что

\begin{gather*}
\frac{\Theta_0(t+\Delta t)-\Theta_0(t)}{\Delta t}=f(t,t)+\int\limits_0^{t}\frac{f(t+\Delta t,\xi)-f(t,\xi)}{\Delta t}d\xi+\frac1{\Delta t}
\int\limits_t^{t+\Delta t}[f(t+\Delta t,\xi)-f(t,t)]d\xi=\\
=f(t,t)+\int\limits_0^{t}f_t(t,\xi)d\xi+\int\limits_0^{t}\left[\frac{f(t+\Delta t,\xi)-f(t,\xi)}{\Delta t}-f_t(t,\xi)\right]d\xi+
\frac1{\Delta t}\int\limits_t^{t+\Delta t}[f(t+\Delta t,\xi)-f(t,t)]d\xi=\\
=\Theta_1(t)+\int\limits_0^{t}\left[\frac{f(t+\Delta t,\xi)-f(t,\xi)}{\Delta t}-f_t(t,\xi)\right]d\xi+\frac1{\Delta t}\int\limits_t^{t+\Delta t}[f(t+\Delta t,\xi)-f(t,t)]d\xi=\Theta_1(t)+\\
+\int\limits_0^{t}\left[\frac1{\Delta t}\int\limits_t^{t+\Delta t}f_t(\eta,\xi)d\eta-f_t(t,\xi)\right]d\xi+
\frac1{\Delta t}\int\limits_t^{t+\Delta t}[f(t+\Delta t,\xi)-f(t,t)]d\xi=\Theta_1(t)+\\
+\int\limits_0^{t}\left[\frac1{\Delta t}\int\limits_t^{t+\Delta t}[f_t(\eta,\xi)-f_t(t,\xi)]d\eta\right]d\xi+\frac1{\Delta t}\int\limits_t^{t+\Delta t}[f(t+\Delta t,\xi)-f(t,t)]d\xi.
\end{gather*}
Итак,
\begin{gather*}
\frac{\Theta_0(t+\Delta t)-\Theta_0(t)}{\Delta t}-\Theta_1(t)= \int\limits_0^{t}\left[\frac1{\Delta t}\int\limits_t^{t+\Delta t}[f_t(\eta,\xi)-f_t(t,\xi)]d\eta\right]d\xi+
\frac1{\Delta t}\int\limits_t^{t+\Delta t}\!\![f(t+\Delta t,\xi)-f(t,t)]d\xi.
\end{gather*}
Поэтому
\begin{gather*}
\left\|\frac{\Theta_0(t+\Delta t)-\Theta_0(t)}{\Delta t}-\Theta_1(t)\right\|_X\leqslant
\left|\frac1{\Delta t}\int\limits_0^{T}\left[\int\limits_t^{t+\Delta t}\|f_t(\eta,\xi)-f_t(t,\xi)\|_Xd\eta\right]d\xi\right|+\\
+\left|\frac1{\Delta t}\int\limits_t^{t+\Delta t}\|f(t+\Delta t,\xi)-f(t,t)\|_Xd\xi\right|.
\end{gather*}
Поскольку $f$, $f_t\in C(\Gamma,X)$, то, согласно теореме \ref{uniform_continuity} они равномерно непрерывны на $\Gamma$, т.е.
\begin{gather}\label{fftucontinuityDelta}
\forall\,\varepsilon>0\,\,\exists\,\delta=\delta(\varepsilon)>0\,\,\forall\,(t',\xi'),\,\,(t'',\xi'')\in\Gamma,\,\,|(t',\xi')-(t'',\xi'')|<\delta:\\
\notag\|f(t',\xi')-f(t'',\xi'')\|_X<\frac\varepsilon{2T},\,\,\,\|f(t',\xi')-f(t'',\xi'')\|_X<\frac\varepsilon{2}.
\end{gather}
Выберем произвольно $\varepsilon>0$ и подберём по нему $\delta=\delta(\varepsilon)>0$ согласно (\ref{fftucontinuityDelta}). Пусть теперь $|\Delta t|<\frac\delta{\sqrt{2}}$. Тогда
$|(\eta,\xi)-(t,\xi)|<\delta$ при всех $\eta$, лежащих между $t$ и $t+\Delta t$ и всех $\xi\in[0,T]$, откуда, в силу (\ref{fftucontinuityDelta}), вытекает, что при всех таких $\eta$ и $\xi$
\begin{gather*}
\|f_t(\eta,\xi)-f_t(t,\xi)\|_X\leqslant\frac\varepsilon{2T},
\end{gather*}
и, как следствие
\begin{gather*}
\left|\frac1{\Delta t}\int\limits_0^{T}\left[\int\limits_t^{t+\Delta t}\|f_t(\eta,\xi)-f_t(t,\xi)\|_Xd\eta\right]d\xi\right|\leqslant\frac\varepsilon{2}.
\end{gather*}
Отсюда выводим, что при $|\Delta t|<\frac\delta{\sqrt{2}}$
\begin{gather*}
\left\|\frac{\Theta_0(t+\Delta t)-\Theta_0(t)}{\Delta t}-\Theta_1(t)\right\|_X\leqslant\frac\varepsilon{2}
+\left|\frac1{\Delta t}\int\limits_t^{t+\Delta t}\|f(t+\Delta t,\xi)-f(t,t)\|_Xd\xi\right|.
\end{gather*}
Поскольку $|\Delta t|<\frac\delta{\sqrt{2}}$, то $|(t+\Delta t,\xi)-(t,t)|=\sqrt{|\Delta t|^2+|\xi-t|^2}\leqslant|\Delta t|\sqrt2<\delta$
при всех $\xi$, лежащих между $t$ и $t+\Delta t$. Поэтому, на основании (\ref{fftucontinuityDelta}), при всех таких $\xi$
\begin{gather*}
\|f(t+\Delta t,\xi)-f(t,t)\|_X\leqslant\frac\varepsilon{2},
\end{gather*}
вследствие чего
\begin{gather*}
\left|\frac1{\Delta t}\int\limits_t^{t+\Delta t}\|f(t+\Delta t,\xi)-f(t,t)\|_Xd\xi\right|\leqslant\frac\varepsilon{2}.
\end{gather*}
Итак,
\begin{gather*}
\left\|\frac{\Theta_0(t+\Delta t)-\Theta_0(t)}{\Delta t}-\Theta_1(t)\right\|_X\leqslant\frac\varepsilon{2}+\frac\varepsilon{2}=\varepsilon
\end{gather*}
при $|\Delta t|<\frac\delta{\sqrt{2}}$, что и доказывает равенство $\Theta_0'=\Theta_1$. Теорема полностью доказана.
\end{Proof}

Из леммы \ref{differentiability_Pi(t,xi)z(xi)} и только что доказанной теоремы вытекает
\begin{Theorem}\label{parametric_integral_2}
Пусть $\Pi(t,\xi)\in\mathcal{L}(X,Y)$ при всех $(t,\xi)\in\Gamma$, причём при всех $x\in X$ функция $\Gamma\ni(t,\xi)\mapsto\Pi(t,\xi)x$ принадлежит $C(\Gamma,Y)$ и при всех
$(t,\xi)\in\Gamma$ имеет непрерывную на $\Gamma$ в норме $Y$ производную $\Gamma\ni(t,\xi)\mapsto\Pi_t(t,\xi)x$. Если $z\in C([0,T],X)$, то функция
\begin{gather*}
[0,T]\ni t\mapsto\int\limits_0^t\Pi(t,\xi)z(\xi)d\xi
\end{gather*}
непрерывно дифференцируема на $[0,T]$ в норме $X$ и
\begin{gather*}
\frac{d}{dt}\int\limits_0^t\Pi(t,\xi)z(\xi)d\xi=\Pi(t,t)z(t)+\int\limits_0^t\Pi_t(t,\xi)z(\xi)d\xi\,\,\,\forall\,t\in[0,T].
\end{gather*}
\end{Theorem}

        \section{Сведения из негладкого анализа}
Для дальнейшего нам потребуются нижеследующие определение и результаты \cite{mord1}, \cite{mordukh}. Пусть $\Xi\subset \mathbb{R}^m$ --- непустое замкнутое множество,
$\varepsilon\geqslant0$, $x\in\bar{\Xi}$. Непустое множество
$$
\hat{N}_\varepsilon(x;\Xi)\equiv\{x^*\in \mathbb{R}^m\,:\,\limsup\limits_{u\stackrel{\Xi}{\rightarrow}x}
\frac{\displaystyle \langle x^*,u-x\rangle}{\displaystyle |u-x|}\leqslant\varepsilon\}
$$
называется множеством $\varepsilon$-нормалей Фреше ко множеству $\Xi$ в точке $x$. Здесь $u\stackrel{\Xi}{\rightarrow}x$ означает, что $u\to x$ при $u\in \Xi$. В частности,
$\hat{N}_0(x;\Xi)$ называется конусом нормалей Фреше ко множеству $\Xi$ в точке $x$ и обозначается через $\hat{N}(x;\Xi)$. Определим (основной, предельный) нормальный конус в точке
$\bar{x}\in\Xi$ как $N(\hat{x};\Xi)\equiv\limsup\limits_{x\stackrel{\Xi}{\rightarrow}\hat{x},\,\varepsilon\downarrow0}\hat{N}_\varepsilon(x;\Xi)$.

Можно показать (подробности в \cite{mord1}, \cite{mordukh}), что $N(\hat{x};\Xi)\equiv\limsup\limits_{x\stackrel{\Xi}{\rightarrow}\hat{x}}\hat{N}(x;\Xi).$ Для полунепрерывной снизу функции
$f\colon \mathbb{R}^m\to \mathbb{R}\cup\{+\infty\}$ и $\bar{x}\in \dom\, f$ субдифференциал Фреше $\hat{\partial}f(\bar{x})$ функции $f$ в точке $\bar{x}\in\dom\,f$ определяется как
$$
\displaystyle \hat{\partial}f(\bar{x})\equiv \left\{x^*\in\mathbb{R}^m\,:\,\liminf\limits_{x\to\bar{x}} \frac{f(x)-f(\bar{x})-\langle x^*,\,x-\bar{x}\rangle}{|x-\bar{x}|}\geqslant0\right\},
$$
или, эквивалентно, как
$$
\displaystyle\hat{\partial}f(\bar{x})\equiv\left\{x^*\in \mathbb{R}^m\,:\,(x^*,-1)\in\hat{N}((\bar{x},\,f(\bar{x}));\epi\,f)\right\}.
$$
Для любого $\bar{x}\in dom\,f$ множества
\begin{gather*}
\partial f(\bar{x})\equiv\{x^*\in \mathbb{R}^m\,:\,(x^*,-1)\in N((\bar{x},\,f(\bar{x}));\epi\, f)\},\\
\partial^\infty f(\bar{x})\equiv\{x^*\in \mathbb{R}^m\,:\,(x^*,0)\in N((\bar{x},\,f(\bar{x}));\epi\, f)\},
\end{gather*}
называются соответственно субдифференциалом и сингулярным субдифференциалом функции $f$ в точке $\bar{x}$ в смысле \cite{mord1},
\cite{mordukh}. Если функция~$f$ полунепрерывна снизу, то справедливы следующие соотношения:
\begin{equation}\label{repsd}
\partial f(\bar{x})=\limsup\limits_{x\stackrel{f}{\rightarrow}\bar{x}}\hat{\partial}f(x),\,\,\,
\partial^\infty f(\bar{x})=\limsup\limits_{x\stackrel{f}{\rightarrow}\bar{x};\,\bar\varepsilon\downarrow0}\bar\varepsilon\hat{\partial}f(x),
\end{equation}
где $x\stackrel{f}{\rightarrow}\bar{x}$ означает, что $x\to\bar{x}$, $f(x)\to f(\bar{x})$. Мы имеем $\partial^\infty f(\bar{x})=\{0\}$, если $f$ липшицева в окрестности точки $x$.

Справедлив следующий важный результат (см. \cite{mord1}, \cite{mordukh}).

\begin{Lemma}\label{dens:Dirichlet2!suboptimal} Пусть $\Xi\subset \mathbb{R}^m$ --- непустое замкнутое множество. Тогда множество точек $\{x\in\Xi:\hat{N}(x;\Xi)\ne\{0\}\}$, т.е. множество
всех граничных точек множества $\Xi$, в которых существует ненулевая нормаль Фреше, всюду плотно во множестве всех граничных точек множества $\Xi$. Кроме того, для любой полунепрерывной
снизу функции~$f\colon \mathbb{R}^m\to \mathbb{R}\cup\{\pm\infty\}$ множество $\{x\in\dom\, f:\hat{\partial}f(x)\ne\emptyset\}$ всюду плотно в $\dom f$.
\end{Lemma}
Из определения субдифференциала Фреше функции $f$ в точке $x$ непосредственно вытекает
\begin{Lemma}\label{princ:Dirichlet2!suboptimal}
Пусть $f\colon \mathbb{R}^m\to \mathbb{R}\cup\{\pm\infty\}$ -- полунепрерывная снизу функция, $x\in\dom\,f$. Если $(x^*,-\mho)\in\hat{N}((x,f(x));\epi\,f)$, $\mho>0$, то для любого
$\varepsilon>0$ найдется окрестность $\textrm{Ш}^m_\varepsilon(x)$ точки~$x$, такая, что
$\mho f(x')-\mho f(x)-\langle x^*,x'-x\rangle+\varepsilon|x'-x|\geqslant0\,\,\forall\,x'\in \textrm{Ш}^m_\varepsilon(x).$
\end{Lemma}


    \chapter{Теоремы вложения}
        \section{Вещественные функции одного вещественного переменного}
Сформулируем прежде всего следующий классический результат.
\begin{Theorem}\label{W11[0,T]_charchterization}
Множество $W^1_1[0,T]$ совпадает с множеством всех абсолютно непрерывных на отрезке $[0,T]$ функций. При этом если функция $\xi\in W^1_1[0,T]$, то производная функции $\xi$, понимаемая в
классическом смысле, существует почти всюду на отрезке $[0,T]$ и почти всюду совпадает с обобщённой производной в смысле Соболева.
\end{Theorem}
Следствием данной теоремы является
\begin{Theorem}\label{W11:embedding}
Пусть $\xi\in W^1_p[0,T]$, $p\in[1,+\infty]$. Тогда найдётся константа $A_1=A_1(p,T)>0$, зависящая лишь от $p\in[1,+\infty]$ и от $T>0$, такая, что
$$
\max\limits_{t\in[0,T]}|\xi(t)|\leqslant A_1\|\xi\|^{(1)}_{p,[0,T]}.
$$
При этом если $p>1$, то вложение $W^1_p[0,T]\subset C[0,T]$ --- компактно.
\end{Theorem}
\begin{Proof}
1) Пусть $p=1$. Пусть $\xi\in W^1_1[0,T]$. Тогда, на основании теоремы \ref{W11[0,T]_charchterization}, $\xi$ абсолютно непрерывна на отрезке $[0,T]$, причём производная функции $\xi$,
понимаемая в классическом смысле, существует почти всюду на отрезке $[0,T]$ и почти всюду совпадает с обобщённой производной в смысле Соболева. Поэтому при всех $t$, $\tau\in[0,T]$
\begin{gather*}
\xi(t)=\xi(\tau)+\int\limits_\tau^t\xi'(\omega)d\omega.
\end{gather*}
Следовательно,
\begin{gather*}
|\xi(t)|\leqslant|\xi(\tau)|+\left|\int\limits_\tau^t|\xi'(\omega)|d\omega\right|.
\end{gather*}
Интегрируя по переменной $\tau\in[0,T]$, получаем, что
\begin{gather*}
T|\xi(t)|\leqslant\int\limits_0^T|\xi(\tau)|d\tau+\int\limits_0^T\left|\int\limits_\tau^t|\xi'(\omega)|d\omega\right|d\tau\leqslant
\int\limits_0^T|\xi(\tau)|d\tau+T\int\limits_0^T|\xi'(\tau)|d\tau\leqslant\max\{1,T\}\|\xi\|_{1,[0,T]}^{(1)},
\end{gather*}
то есть
\begin{gather*}
|\xi(t)|\leqslant\max\{1,T^{-1}\}\|\xi\|_{1,[0,T]}^{(1)}\,\,\,\forall\, t\in[0,T],
\end{gather*}
откуда следует требуемая оценка с $A_1=\max\{1,\frac1T\}$.

2) Пусть $p=\infty$, и пусть $\xi\in W^1_\infty[0,T]$. Нетрудно видеть, что $\xi\in W^1_1[0,T]$. В силу теоремы \ref{W11[0,T]_charchterization} функция $\xi$  абсолютно непрерывна на
отрезке $[0,T]$, производная функции $\xi$, понимаемая в классическом смысле, существует почти всюду на отрезке $[0,T]$ и почти всюду совпадает с обобщённой производной в смысле Соболева.
Поэтому при всех $t$, $\tau\in[0,T]$
\begin{gather*}
\xi(t)=\xi(\tau)+\int\limits_\tau^t\xi'(\omega)d\omega.
\end{gather*}
Следовательно,
\begin{gather*}
|\xi(t)|\leqslant|\xi(\tau)|+\left|\int\limits_\tau^t|\xi'(\omega)|d\omega\right|\leqslant\|\xi\|_{\infty,[0,T]}+T\|\xi'\|_{\infty,[0,T]}\leqslant\max\{1,T\}\|\xi\|_{\infty,[0,T]}^{(1)}.
\end{gather*}
Таким образом,
\begin{gather*}
\max\limits_{t\in[0,T]}|\xi(t)|\leqslant\max\{1,T\}\|\xi\|_{\infty,[0,T]}^{(1)},
\end{gather*}
и, следовательно, можно взять $A_1=\max\{1,T\}$.

Докажем теперь компактность вложения $W^1_\infty[0,T]\subset C[0,T]$. Пусть $\mathfrak{M}\subset W^1_\infty[0,T]$ --- ограничено в норме пространства $W^1_\infty[0,T]$, то есть найдётся
постоянная $C>0$, такая, что
\begin{gather*}
\|\psi\|_{\infty,[0,T]}^{(1)}\leqslant C\,\,\,\forall\,\psi\in\mathfrak{M}.
\end{gather*}
Тогда
\begin{gather*}
\pmb{|}\psi\pmb{|}^{(0)}_{[0,T]}\leqslant A_1C\,\,\,\forall\,\psi\in\mathfrak{M}.
\end{gather*}
Пусть $t'$, $t''\in[0,T]$, $\psi\in\mathfrak{M}$ --- произвольны. Тогда
\begin{gather*}
|\psi(t')-\psi(t'')|\leqslant\left|\int\limits_{t'}^{t''}|\psi'(\omega)|d\omega\right|\leqslant C|t'-t''|.
\end{gather*}
Следовательно, множество $\mathfrak{M}$ равномерно ограничено в норме пространства $C[0,T]$ и равностепенно непрерывно. Поэтому, в силу
теоремы Арцела--Асколи, множество $\mathfrak{M}$ предкомпактно в $C[0,T]$.


3) Пусть $1<p<\infty$, и пусть $\xi\in W^1_p[0,T]$. Нетрудно видеть, что $\xi\in W^1_1[0,T]$. В силу теоремы \ref{W11[0,T]_charchterization} функция $\xi$  абсолютно непрерывна на отрезке
$[0,T]$, производная функции $\xi$, понимаемая в классическом смысле, существует почти всюду на отрезке $[0,T]$ и почти всюду совпадает с обобщённой производной в смысле Соболева. Поэтому
при всех $t$, $\tau\in[0,T]$
\begin{gather*}
|\xi(t)|^p=|\xi(\tau)|^p+\int\limits_\tau^tp[|\xi(\omega)|^{p-1}\sgn\xi(\omega)]\xi'(\omega)d\omega\leqslant
|\xi(\tau)|^p+p\left|\int\limits_\tau^t|\xi(\omega)|^{p-1}|\xi'(\omega)|d\omega\right|.
\end{gather*}
Проинтегрировав последнее неравенство по переменной $\tau\in[0,T]$, получим, что
\begin{gather*}
T|\xi(t)|^p\leqslant\int\limits_0^T|\xi(\tau)|^pd\tau+p\int\limits_0^T\left|\int\limits_\tau^t|\xi(\omega)|^{p-1}|\xi'(\omega)|d\omega\right|d\tau\leqslant
\int\limits_0^T|\xi(\tau)|^pd\tau+pT\int\limits_0^T|\xi(\omega)|^{p-1}|\xi'(\omega)|d\omega\leqslant\\
\leqslant\int\limits_0^T|\xi(\tau)|^pd\tau+pT\left[\int\limits_0^T|\xi(\omega)|^pd\omega\right]^{\frac{p-1}{p}}
\left[\int\limits_0^T|\xi'(\omega)|^pd\omega\right]^{1/p}=\|\xi\|^p_{p,[0,T]}+pT\|\xi\|^{p-1}_{p,[0,T]}\|\xi'\|_{p,[0,T]}=\\
=\|\xi\|^{p-1}_{p,[0,T]}[\|\xi\|_{p,[0,T]}+pT\|\xi'\|_{p,[0,T]}]\leqslant[\max\limits_{\omega\in[0,T]}|\xi(\omega)|]^{p-1}T^{\frac{p-1}p}
[\|\xi\|_{p,[0,T]}+pT\|\xi'\|_{p,[0,T]}]\leqslant\\
\leqslant[\max\limits_{\omega\in[0,T]}|\xi(\omega)|]^{p-1}T^{\frac{p-1}p}\left[1+(pT)^{\frac{p}{p-1}}\right]^{\frac{p-1}{p}}\|\xi\|^{(1)}_{p,[0,T]}.
\end{gather*}
Как следствие,
\begin{gather*}
[\max\limits_{t\in[0,T]}|\xi(t)|]^{p}\leqslant[\max\limits_{\omega\in[0,T]}|\xi(\omega)|]^{p-1}T^{\frac{p-1}p}T^{-1}
\left[1+(pT)^{\frac{p}{p-1}}\right]^{\frac{p-1}{p}}\|\xi\|^{(1)}_{p,[0,T]},
\end{gather*}
откуда
\begin{gather*}
\max\limits_{t\in[0,T]}|\xi(t)|\leqslant T^{\frac{p-1}p}T^{-1}\left[1+(pT)^{\frac{p}{p-1}}\right]^{\frac{p-1}{p}}\|\xi\|^{(1)}_{p,[0,T]},
\end{gather*}
так что можно взять постоянную $A_1=T^{\frac{p-1}p}T^{-1}\left[1+(pT)^{\frac{p}{p-1}}\right]^{\frac{p-1}{p}}$.

Докажем теперь компактность вложения $W^1_p[0,T]\subset C[0,T]$. Пусть $\mathfrak{M}\subset W^1_p[0,T]$ --- ограничено в норме пространства
$W^1_p[0,T]$, то есть найдётся постоянная $C>0$, такая, что
\begin{gather*}
\|\psi\|_{p,[0,T]}^{(1)}\leqslant C\,\,\,\forall\,\psi\in\mathfrak{M}.
\end{gather*}
Тогда
\begin{gather*}
\pmb{|}\psi\pmb{|}^{(0)}_{[0,T]}\leqslant A_1C\,\,\,\forall\,\psi\in\mathfrak{M}.
\end{gather*}
Пусть $t'$, $t''\in[0,T]$, $\psi\in\mathfrak{M}$ --- произвольны. Тогда
\begin{gather*}
|\psi(t')-\psi(t'')|\leqslant\left|\int\limits_{t'}^{t''}|\psi'(\omega)|d\omega\right|\leqslant
|t'-t''|^{\frac{p-1}{p}}\|\psi'\|_{p,[0,T]}\leqslant C|t'-t''|^{\frac{p-1}{p}}.
\end{gather*}
Следовательно, множество $\mathfrak{M}$ равномерно ограничено в норме пространства $C[0,T]$ и равностепенно непрерывно. Поэтому, в силу
теоремы Арцела--Асколи, множество $\mathfrak{M}$ предкомпактно в $C[0,T]$.
\end{Proof}

        \section{Вещественные функции нескольких вещественных переменных}
\begin{Theorem}\label{embeddingWm2Omega} \cite[стр.84--85]{lad}
Если $n<2m$, то справедливо вложение $W^m_2(\Omega)\subset C(\bar\Omega)$, причём найдётся постоянная $A_2=A_2(m,n,\Omega)>0$, зависящая
лишь от $m$, размерности $n$ и области $\Omega$, такая, что
\begin{gather*}
\pmb{|}z\pmb{|}^{(0)}_{\bar\Omega}\leqslant A_2\|z\|^{(m)}_{2,\Omega}.
\end{gather*}
Кроме того, вложение $W^m_2(\Omega)\subset C(\bar\Omega)$ компактно.

Если $n=2m$, то при всех $p\in(1,\infty)$ справедливо вложение $W^m_2(\Omega)\subset L_p(\Omega)$, причём найдётся постоянная
$A_3=A_3(m,n,p,\Omega)>0$, зависящая лишь от $p\in(1,\infty)$, $m$, размерности $n$ и области $\Omega$, такая, что
\begin{gather*}
\|z\|_{p,\Omega}\leqslant A_3\|z\|^{(m)}_{2,\Omega}.
\end{gather*}
Кроме того, вложение $W^m_2(\Omega)\subset L_p(\Omega)$ компактно.

Если $n>2m$, то при всех $p\in(1,\frac{2n}{n-2m})$ справедливо вложение $W^m_2(\Omega)\subset L_p(\Omega)$, причём найдётся постоянная
$A_4=A_4(m,n,p,\Omega)>0$, зависящая лишь от $p\in(1,\frac{2n}{n-2m})$, $m$, размерности $n$ и области $\Omega$, такая, что
\begin{gather*}
\|z\|_{p,\Omega}\leqslant A_4\|z\|^{(m)}_{2,\Omega}.
\end{gather*}
Кроме того, вложение $W^m_2(\Omega)\subset L_p(\Omega)$ компактно.
\end{Theorem}

\begin{Theorem}\label{LqboundaryOmega:embedding} \cite[стр.84--85]{lad}
Если $n=2m$, то при всех $q\in(1,\infty)$ справедливо вложение $W^m_2(\Omega)\subset L_q(S)$, причём найдётся постоянная
$A_5=A_5(m,n,q,\Omega)>0$, зависящая лишь от $q\in(1,\infty)$, $m$, размерности $n$ и области $\Omega$, такая, что
\begin{gather*}
\|z\|_{q,S}\leqslant A_5\|z\|^{(m)}_{2,\Omega}.
\end{gather*}
Кроме того, вложение $W^m_2(\Omega)\subset L_q(S)$ компактно.

Если $n>2m$, то при всех $q\in(1,\frac{2(n-1)}{n-2m})$ справедливо вложение $W^m_2(\Omega)\subset L_q(S)$, причём найдётся постоянная
$A_6=A_6(m,n,q,\Omega)>0$, зависящая лишь от $q\in(1,\frac{2(n-1)}{n-2m})$, $m$, размерности $n$ и области $\Omega$, такая, что
\begin{gather*}
\|z\|_{q,S}\leqslant A_6\|z\|^{(m)}_{2,\Omega}.
\end{gather*}
Кроме того, вложение $W^m_2(\Omega)\subset L_q(S)$ компактно.
\end{Theorem}

\begin{Theorem}\label{ineq6.24lad} \cite[неравенство (6.24) главы 1]{lad}
Для всех $z\in W^1_2(\Omega)$ и всех $\varepsilon>0$ справедливо неравенство
\begin{gather*}
\int\limits_S z^2ds\leqslant\int\limits_\Omega[\varepsilon|\nabla_xz|^2+A_7(\varepsilon)z^2]dx,
\end{gather*}
где постоянная $A_7=A_7(\varepsilon)>0$ зависит лишь от области $\Omega$, размерности $n$ и числа $\varepsilon>0$.
\end{Theorem}


        \section{Функции одного переменного, принимающие значения в банаховом пространстве}
Пусть $X$ --- сепарабельное банахово пространство с нормой $\|\cdot\|_X$, и пусть $\mathfrak{D}(0,T)\equiv C^\infty_0(0,T)$.
\begin{Definition}
Говорят, что последовательность функций $\varphi_j\in\mathfrak{D}(0,T)$, $j=1,2,\dots$, сходится в $\mathfrak{D}(0,T)$ к функции $\varphi\in\mathfrak{D}(0,T)$, если найдётся отрезок
$[\tau_1,\tau_2]\subset[0,T]$, $0<\tau_1<\tau_2<T$, такой, что носители $\supp\varphi_j\equiv cl{\{t\in[0,T]:\varphi_j(t)\neq0\}}$ функций $\varphi_j$, $j=1,2,\dots$, содержатся в
$[\tau_1,\tau_2]$ и
\begin{gather*}
\lim\limits_{j\to\infty}\pmb{|}\varphi^{(m)}_j-\varphi^{(m)}\pmb{|}^{(0)}_{[0,T]}=0,\,\,\,m=0,1,2,\dots
\end{gather*}
При этом будем писать $\varphi_j\stackrel{\mathfrak{D}(0,T)}{\longrightarrow}\varphi$, $j\to\infty$.
\end{Definition}
Линейное пространство $\mathfrak{D}(0,T)$ с данным понятием сходимости называется основным пространством, а его элементы --- основными (или пробными) функциями.

\begin{Definition}
Обобщённой функцией или распределением со значениями в $X$ называется всякий линейный оператор $f\colon\mathfrak{D}(0,T)\to X$, слабо непрерывный на $\mathfrak{D}(0,T)$, т.е. такой, что
\begin{gather*}
f(\varphi_j)\to f(\varphi),\,\,\,j\to\infty,\text{ слабо в $X$},
\end{gather*}
для любых $\varphi$, $\varphi_j\in\mathfrak{D}$, $j=1,2,\dots$, для которых $\varphi_j\stackrel{\mathfrak{D}(0,T)}{\longrightarrow}\varphi$, $j\to\infty$.

Говорят, что последовательность обобщённых функций $f_j$, $j=1,2,\dots$, сходится к обобщённой функции $f$, если для всех $\varphi\in \mathfrak{D}$
\begin{gather*}
f_j(\varphi)\to f(\varphi),\,\,\,j\to\infty,\text{ слабо в $X$}.
\end{gather*}
Линейное пространство всех обобщённых функций с таким понятием сходимости принято обозначать $\mathfrak{D}'((0,T),X)$, а саму сходимость
--- через $f_j\stackrel{\mathfrak{D}'((0,T),X)}{\longrightarrow}f$, $j\to\infty$.
\end{Definition}


Покажем, что всякую функцию $f\in L_1([0,T],X)$ можно интерпретировать как обобщённую функцию и тем самым установить поэлементное вложение
\begin{gather}\label{L_1([0,T],X)_is_subset_mathfrakD'((0,T),X)}
L_1([0,T],X)\subset\mathfrak{D}'((0,T),X).
\end{gather}
В самом деле, пусть $f\in L_1([0,T],X)$ --- некоторая функция. Определим оператор $F_f\colon\mathfrak{D}(0,T)\to X$ по правилу
\begin{gather}\label{X_valued_regular_distribution}
F_f(\varphi)=(\textrm{Б})\int\limits_0^Tf(t)\varphi(t)dt\,\,\,\forall\,\varphi\in\mathfrak{D}(0,T).
\end{gather}
Это определение корректно, ибо функция $[0,T]\ni t\mapsto f(t)\varphi(t)\in X$ интегрируема по Бохнеру на $[0,T]$. Следовательно, оператор $F_f$ определён всюду в $\mathfrak{D}(0,T)$.
Линейность оператора $F_f$ следует из линейности интеграла Бохнера. Проверим слабую непрерывность оператора $F_f$. В самом деле, пусть $x^*\in X^*$, $\varphi\in\mathfrak{D}(0,T)$,
$\varphi_j\in\mathfrak{D}(0,T)$, $j=1,2,\dots$, $\varphi_j\stackrel{\mathfrak{D}(0,T)}{\longrightarrow}\varphi$, $j\to\infty$, --- произвольны. Тогда
\begin{gather*}
|\langle F_f(\varphi_j)-F_f(\varphi),x^*\rangle|=\left|\left\langle(\textrm{Б})\int\limits_0^Tf(t)[\varphi_j(t)-\varphi(t)]dt,x^*\right\rangle\right|
\leqslant\int\limits_0^T|\langle f(t)[\varphi_j(t)-\varphi(t)],x^*\rangle|dt\leqslant\\
\leqslant\pmb{|}\varphi_j-\varphi\pmb{|}^{(0)}_{[0,T]}\|x^*\|_{X^*}\int\limits_0^T\|f(t)\|_Xdt\to0,\,\,\,j\to\infty.
\end{gather*}
Таким образом,
\begin{gather*}
F_f(\varphi_j)\to F_f(\varphi),\,\,\,j\to\infty,\text{ слабо в $X$},
\end{gather*}
для любых $\varphi$, $\varphi_j\in\mathfrak{D}$, $j=1,2,\dots$, для которых $\varphi_j\stackrel{\mathfrak{D}(0,T)}{\longrightarrow}\varphi$, $j\to\infty$. Итак,
$F_f\in\mathfrak{D}'((0,T),X)$. Убедимся в том, что разные элементы $f$, $g\in L_1([0,T],X)$ порождают разные операторы $F_f$, $F_g\in\mathfrak{D}'((0,T),X)$. В самом деле, пусть для
некоторых $f$, $g\in L_1([0,T],X)$ оказалось, что $F_f=F_g$, то есть
\begin{gather*}
(\textrm{Б})\int\limits_0^T[f(t)-g(t)]\varphi(t)dt=0\,\,\,\forall\,\varphi\in\mathfrak{D}(0,T).
\end{gather*}
Тогда для любого $x^*\in X^*$
\begin{gather*}
0=\left\langle(\textrm{Б})\int\limits_0^T[f(t)-g(t)]\varphi(t)dt,x^*\right\rangle=\int\limits_0^T\langle[f(t)-g(t)]\varphi(t),x^*\rangle dt%=\\
=\int\limits_0^T\varphi(t)\langle f(t)-g(t),x^*\rangle dt\,\,\,\forall\,\varphi\in\mathfrak{D}(0,T).
\end{gather*}
Иными словами,
\begin{gather*}
\int\limits_0^T\varphi(t)\langle f(t)-g(t),x^*\rangle dt=0\,\,\,\forall\,\varphi\in\mathfrak{D}(0,T).
\end{gather*}
А это возможно, лишь если при п.в. $t\in[0,T]$
\begin{gather*}
\langle f(t)-g(t),x^*\rangle=0.
\end{gather*}
В силу произвольности $x^*\in X^*$ и следствия из теоремы Хана--Банаха заключаем, что при п.в. $t\in[0,T]$
\begin{gather*}
f(t)=g(t),
\end{gather*}
то есть $f=g$ как элементы пространства $L_1([0,T],X)$.

Обобщённую функцию, порождённую обычной функцией $f\in L_1([0,T],X)$ по правилу (\ref{X_valued_regular_distribution}), называют регулярной обобщённой функцией.
\begin{Definition}\label{X_valued_distribution_derivative:Definition}
Производной порядка $m$ обобщённой функции $f\in\mathfrak{D}'((0,T),X)$, называется обобщённая функция $f^{(m)}\in\mathfrak{D}'((0,T),X)$, действующая по правилу
\begin{gather}\label{X_valued_regular_distribution_derivative}
f^{(m)}(\varphi)\stackrel{\mathrm{def}}{=}(-1)^mf(\varphi^{(m)})\,\,\,\forall\,\varphi\in\mathfrak{D}(0,T).
\end{gather}
\end{Definition}

Убедимся в том, что равенство (\ref{X_valued_regular_distribution_derivative}) действительно определяет обобщённую функцию. В самом деле, если $\varphi\in\mathfrak{D}(0,T)$, то и
$\varphi^{(m)}\in\mathfrak{D}(0,T)$, так что правая часть (\ref{X_valued_regular_distribution_derivative}) действительно имеет смысл. Следовательно, оператор $f^{(m)}$ определён на всём
основном пространстве. Линейность данного оператора следует из определения обобщённой функции. Проверим непрерывность оператора $f^{(m)}$. Если $\varphi_j
\stackrel{\mathfrak{D}(0,T)}{\longrightarrow}\varphi$, $j\to\infty$, то, очевидно, $\varphi_j^{(m)}\stackrel{\mathfrak{D}(0,T)}{\longrightarrow}\varphi^{(m)}$, $j\to\infty$. В силу слабой
непрерывности $f$ и равенства (\ref{X_valued_regular_distribution_derivative}) для всех $x^*\in X^*$ при $j\to\infty$ имеем сходимость
\begin{gather*}
\langle f^{(m)}(\varphi_j),x^*\rangle=\langle (-1)^mf(\varphi^{(m)}_j),x^*\rangle\to \langle (-1)^mf(\varphi^{(m)}),x^*\rangle=\langle f^{(m)}(\varphi),x^*\rangle.
\end{gather*}
Это означает, что $f^{(m)}(\varphi_j)\to f^{(m)}(\varphi)$, $j\to\infty$, слабо в $X$, и, как следствие, $f^{(m)}\in\mathfrak{D}'((0,T),X)$.

Итак, установлено, что любая обобщённая функция имеет производную любого порядка $m=0,1,2,\dots$, также являющуюся обобщённой функцией (здесь считается, что $f^{(0)}=f$).

Из (\ref{X_valued_regular_distribution_derivative}) видно, что если $f_j\stackrel{\mathfrak{D}'((0,T),X)}{\longrightarrow}f$, $j\to\infty$, то и $f_j^{(m)}
\stackrel{\mathfrak{D}'((0,T),X)}{\longrightarrow}f^{(m)}$, $j\to\infty$, для всех  $m=0,1,2,\dots$

Если обобщённая функция $f$ и её производная $f'$ являются регулярными, т.е.
\begin{gather*}
f(\varphi)=(\textrm{Б})\int\limits_0^T f_0(t)\varphi(t)dt,\,\,\,f'(\varphi)=(\textrm{Б})\int\limits_0^T f_1(t)\varphi(t)dt\,\,\,\forall\,\varphi\in\mathfrak{D}(0,T),
\end{gather*}
для некоторых функций $f_0$, $f_1\in L_1([0,T],X)$, то, в соответствии с определением \ref{X_valued_distribution_derivative:Definition},
\begin{gather}\label{X_valued_regular_distribution_derivative:regular}
(\textrm{Б})\int\limits_0^T f_1(t)\varphi(t)dt=-(\textrm{Б})\int\limits_0^T f_0(t)\varphi'(t)dt\,\,\,\forall\,\varphi\in\mathfrak{D}(0,T).
\end{gather}
В частности, если $X=\mathbb{R}$, то регулярная производная регулярной обобщённой функции совпадает с обобщённой производной в смысле Соболева. Если же $f\in C^k([0,T],X)$, то производные
функции $f$ в смысле определения \ref{X_valued_distribution_derivative:Definition} до порядка $k$ включительно  являются регулярными обобщёнными функциями и совпадают с поточечными
производными $f^{(m)}(t)$, $t\in[0,T]$, $m=0,1,\dots,k$. Если же $f\in C^\infty([0,T],X)$, то сказанное относится к производным всех порядков.


Пусть $1\leqslant p\leqslant \infty$. Через $W^1_p([0,T],X)$ обозначим множество функций $f\in L_p([0,T],X)$, имеющих первую обобщённую производную $f'$, принадлежащую $L_p([0,T],T)$.
Норму в $W^1_p([0,T],X)$ зададим равенством
\begin{gather*}
\|f\|_{p,[0,T],X}^{(1)}\equiv\left[\int\limits_0^T[\|f(t)\|^p_X+\|f'(t)\|^p_X]dt\right]^{1/p}\text{ при $1\leqslant p<\infty$};\\
\|f\|_{p,[0,T],X}^{(1)}\equiv\|f\|_{p,[0,T],X}+\|f'\|_{p,[0,T],X}\text{ при $p=\infty$}.
\end{gather*}

\begin{Theorem}
Пространство $W^1_p([0,T],X)$ полно при всех $1\leqslant p\leqslant \infty$.
\end{Theorem}
\begin{Proof}
Пусть последовательность $f_j\in W^1_p([0,T],X)$, $j=1,2,\dots$, --- фундаментальна в норме пространства $W^1_p([0,T],X)$, т.е.
\begin{gather}\label{fundamentality_in_W1p([0,T],X)}
\forall\,\varepsilon>0\,\,\exists\,j_0=j_0(\varepsilon)\geqslant1\,\,\forall\,j\geqslant j_0(\varepsilon),\,\,k\geqslant1:\|f_{j+k}-f_j\|_{p,[0,T],X}^{(1)}\leqslant\varepsilon.
\end{gather}
Из данного неравенства, в силу определения нормы в пространстве $W^1_p([0,T],X)$, выводим, что
\begin{gather*}
\forall\,\varepsilon>0\,\,\exists\,j_0=j_0(\varepsilon)\geqslant1\,\,\forall\,j\geqslant j_0(\varepsilon),\,\,k\geqslant1:
\|f_{j+k}-f_j\|_{p,[0,T],X}\leqslant\varepsilon\,\,\,\|f_{j+k}'-f_j'\|_{p,[0,T],X}\leqslant\varepsilon.
\end{gather*}
Это означает, что последовательности $f_j$, $j=1,2,\dots$, и $f_j'$, $j=1,2,\dots$, фундаментальны в пространстве $L_p([0,T],X)$.
Поскольку же пространство $L_p([0,T],X)$ полно, то найдутся функции $g_0$, $g_1\in L_p([0,T],X)$, такие, что
\begin{gather}\label{convergence_of_fjf'j}
\|f_{j}-g_0\|_{p,[0,T],X}\to0,\,\,\,\|f_{j}'-g_1\|_{p,[0,T],X}\to0,\,\,\,j\to\infty.
\end{gather}
Поэтому
\begin{gather}\label{convergence_of_fjf'j:::1}
\|f_{j}-g_0\|_{1,[0,T],X}\to0,\,\,\,\|f_{j}'-g_1\|_{1,[0,T],X}\to0,\,\,\,j\to\infty.
\end{gather}

Так как $f_j\in W^1_p([0,T],X)$, $j=1,2,\dots$, то, ввиду определения класса $W^1_p([0,T],X)$, справедливо интегральное тождество

\begin{gather}\label{X_valued_regular_distribution_derivative:regular:::1}
(\textrm{Б})\int\limits_0^T f_j'(t)\varphi(t)dt=-(\textrm{Б})\int\limits_0^T f_j(t)\varphi'(t)dt\,\,\,\forall\,\varphi\in\mathfrak{D}(0,T).
\end{gather}
Перейдём в данном тождестве к пределу при $j\to\infty$. Прежде всего заметим, что
\begin{gather*}
\left\|(\textrm{Б})\int\limits_0^T f_j'(t)\varphi(t)dt-(\textrm{Б})\int\limits_0^T g_1(t)\varphi(t)dt\right\|_X=\left\|(\textrm{Б})\int\limits_0^T [f_j'(t)-g_1(t)]\varphi(t)dt\right\|_X
\leqslant\\
\leqslant\int\limits_0^T \|[f_j'(t)-g_1(t)]\varphi(t)\|_Xdt=\int\limits_0^T \|f_j'(t)-g_1(t)\|_X|\varphi(t)|dt\leqslant
\pmb{|}\varphi\pmb{|}^{(0)}_{[0,T]}\|f_{j}'-g_1\|_{1,[0,T],X}\to0,\,\,\,j\to\infty,
\end{gather*}
на основании (\ref{convergence_of_fjf'j:::1}). Итак,
\begin{gather}\label{convergence_f'j_integrals}
\lim\limits_{j\to\infty}\left\|(\textrm{Б})\int\limits_0^T f_j'(t)\varphi(t)dt-(\textrm{Б})\int\limits_0^T g_1(t)\varphi(t)dt\right\|_X=0.
\end{gather}
Во--вторых,
\begin{gather*}
\left\|\left[-(\textrm{Б})\int\limits_0^T f_j(t)\varphi(t)dt\right]-\left[-(\textrm{Б})\int\limits_0^T g_0(t)\varphi(t)dt\right]\right\|_X=
\left\|-(\textrm{Б})\int\limits_0^T [f_j(t)-g_0(t)]\varphi(t)dt\right\|_X\leqslant\\
\leqslant\int\limits_0^T \|[f_j(t)-g_0(t)]\varphi(t)\|_Xdt=\int\limits_0^T \|f_j(t)-g_0(t)\|_X|\varphi(t)|dt\leqslant\pmb{|}\varphi\pmb{|}^{(0)}_{[0,T]}\|f_{j}-g_0\|_{1,[0,T],X}\to0,\,\,\,
j\to\infty,
\end{gather*}
на основании (\ref{convergence_of_fjf'j:::1}). Как следствие,
\begin{gather}\label{convergence_fj_integrals}
\lim\limits_{j\to\infty}
\left\|\left[-(\textrm{Б})\int\limits_0^T f_j(t)\varphi(t)dt\right]-\left[-(\textrm{Б})\int\limits_0^T g_0(t)\varphi(t)dt\right]\right\|_X=0.
\end{gather}
Переходя теперь в интегральном тождестве (\ref{X_valued_regular_distribution_derivative:regular:::1}) с учётом предельных соотношений
(\ref{convergence_f'j_integrals}) и (\ref{convergence_fj_integrals}), получаем, что
\begin{gather*}
(\text{Б})\int\limits_0^T g_1(t)\varphi(t)dt=-(\text{Б})\int\limits_0^T g_0(t)\varphi'(t)dt\,\,\,\forall\,\varphi\in\mathfrak{D}(0,T).
\end{gather*}

Данное тождество означает, что функция $g_0$ является элементом пространства $W^1_p([0,T],X)$, причём  $g_0'$, регулярная первая обобщённая производная функции $g_0$, совпадает с функцией
 $g_1$. Осталось доказать, что последовательность $f_j$, $j=1,2,\dots$, сходится в норме пространства $W^1_p([0,T],X)$ к функции $g_0$.

Предположим сначала, что $p\neq\infty$. Тогда неравенство (\ref{fundamentality_in_W1p([0,T],X)}) можно переписать в виде
\begin{gather*}
\forall\,\varepsilon>0\,\,\exists\,j_0=j_0(\varepsilon)\geqslant1\,\,\forall\,j\geqslant j_0(\varepsilon),\,\,k\geqslant1:
[\|f_{j+k}-f_j\|_{p,[0,T],X}^p+\|f_{j+k}'-f_j'\|_{p,[0,T],X}^p]^{1/p}\leqslant\varepsilon.
\end{gather*}
Устремляя здесь $k$ к бесконечности и учтя соотношения (\ref{convergence_of_fjf'j}), будем иметь
\begin{gather*}
\forall\,\varepsilon>0\,\,\exists\,j_0=j_0(\varepsilon)\geqslant1\,\,\forall\,j\geqslant j_0(\varepsilon):[\|g_{0}-f_j\|_{p,[0,T],X}^p+\|g_{0}'-f_j'\|_{p,[0,T],X}^p]^{1/p}\leqslant
\varepsilon.
\end{gather*}
А это и означает, что $\|g_{0}-f_j\|_{p,[0,T],X}^{(1)}\to0$, $j\to\infty$. Иными словами, мы доказали полноту пространства $W^1_p([0,T],X)$ при $p\neq\infty$.

Пусть теперь $p=\infty$. Тогда неравенство (\ref{fundamentality_in_W1p([0,T],X)}) можно записать в виде
\begin{gather*}
\forall\,\varepsilon>0\,\,\exists\,j_0=j_0(\varepsilon)\geqslant1\,\,\forall\,j\geqslant j_0(\varepsilon),\,\,k\geqslant1:
\|f_{j+k}-f_j\|_{\infty,[0,T],X}+\|f_{j+k}'-f_j'\|_{\infty,[0,T],X}\leqslant\varepsilon.
\end{gather*}
Переходя в данном неравенстве к пределу при $k\to\infty$ и учтя соотношения (\ref{convergence_of_fjf'j}), выводим, что
\begin{gather*}
\forall\,\varepsilon>0\,\,\exists\,j_0=j_0(\varepsilon)\geqslant1\,\,\forall\,j\geqslant j_0(\varepsilon),\,\,k\geqslant1:
\|g_{0}-f_j\|_{\infty,[0,T],X}+\|g_{0}'-f_j'\|_{\infty,[0,T],X}\leqslant\varepsilon.
\end{gather*}
Последнее и даёт, что $\|g_{0}-f_j\|_{p,[0,T],X}^{(1)}\to0$, $j\to\infty$. Таким образом, мы доказали полноту пространства  $W^1_\infty([0,T],X)$. Теорема полностью доказана.
\end{Proof}


\begin{Theorem}\label{W1p([0,T],X)_weak}
Если при некотором $p$, $1\leqslant p\leqslant \infty$, функция $f\in W^1_p([0,T],X)$, то при всех $x^*\in X^*$ функция
\begin{gather}\label{W1p([0,T],X)_weak::f,x*}
[0,T]\ni t\mapsto\langle f(t),x^*\rangle
\end{gather}
является элементом $W^1_p[0,T]$, а её обобщённой производной в смысле Соболева является функция
\begin{gather}\label{W1p([0,T],X)_weak::f',x*}
[0,T]\ni t\mapsto\langle f'(t),x^*\rangle.
\end{gather}
\end{Theorem}
\begin{Proof}
Покажем сначала, что функции (\ref{W1p([0,T],X)_weak::f,x*}) и (\ref{W1p([0,T],X)_weak::f',x*}) при всех $x^*\in X^*$ принадлежат
пространству $L_p[0,T]$. В самом деле, нетрудно видеть, что при всех $t\in[0,T]$
\begin{gather*}
|\langle f(t),x^*\rangle|\leqslant\|f(t)\|_X\|x^*\|_{X^*},\,\,\,|\langle f'(t),x^*\rangle|\leqslant\|f'(t)\|_X\|x^*\|_{X^*}.
\end{gather*}
Поскольку же $f$, $f'\in L_p([0,T],X)$, то функции (\ref{W1p([0,T],X)_weak::f,x*}) и (\ref{W1p([0,T],X)_weak::f',x*}) при всех $x^*\in X^*$ принадлежат пространству $L_p[0,T]$.

Докажем теперь, что функция (\ref{W1p([0,T],X)_weak::f',x*}) является обобщённой производной в смысле Соболева функции
(\ref{W1p([0,T],X)_weak::f,x*}). Действительно, поскольку $f\in W^1_p([0,T],X)$, то имеет место интегральное тождество
\begin{gather*}%\label{X_valued_regular_distribution_derivative:regular:::2}
(\textrm{Б})\int\limits_0^T f'(t)\varphi(t)dt=-(\textrm{Б})\int\limits_0^T f(t)\varphi'(t)dt\,\,\,\forall\,\varphi\in\mathfrak{D}(0,T).
\end{gather*}
Поэтому при всех $\varphi\in\mathfrak{D}(0,T)$ и всех $x^*\in X^*$
\begin{gather*}
\left\langle(\textrm{Б})\int\limits_0^T f'(t)\varphi(t)dt,x^*\right\rangle=\left\langle-(\textrm{Б})\int\limits_0^T f(t)\varphi'(t)dt,x^*\right\rangle;\,\,\,
\int\limits_0^T\left\langle f'(t)\varphi(t),x^*\right\rangle dt=-\int\limits_0^T\left\langle f(t)\varphi'(t),x^*\right\rangle dt;\\
\int\limits_0^T\left\langle f'(t),x^*\right\rangle\varphi(t) dt=-\int\limits_0^T\left\langle f(t),x^*\right\rangle\varphi'(t) dt;
\end{gather*}
то есть
\begin{gather*}
\int\limits_0^T\left\langle f'(t),x^*\right\rangle\varphi(t) dt=-\int\limits_0^T\left\langle f(t),x^*\right\rangle\varphi'(t) dt\,\,\,\forall\,\varphi\in\mathfrak{D}(0,T).
\end{gather*}
Последнее тождество и даёт утверждение леммы.
\end{Proof}

\begin{Theorem}\label{W1p([0,T],X)_embedding}
Пусть $1\leqslant p\leqslant \infty$, $f\in W^1_p([0,T],X)$. Тогда $f\in C([0,T],X)$, и справедлива оценка
\begin{gather}\label{W1p([0,T],X)_embedding:estimate}
\max\limits_{t\in[0,T]}\|f(t)\|_X\leqslant A_1\|f\|_{p,[0,T],X}^{(1)},
\end{gather}
где постоянная $A_1>0$, зависящая лишь от $1\leqslant p\leqslant \infty$ и от $T>0$ та же, что и в теореме \ref{W11:embedding}.
\end{Theorem}
\begin{Proof}
Пусть $1\leqslant p\leqslant \infty$ и $f\in W^1_p([0,T],X)$ --- произвольны. Выберем произвольно $x^*\in X^*$ и зафиксируем. Положим $F(t,x^*)\equiv\langle f(t),x^*\rangle$, $t\in[0,T]$.
Согласно теореме \ref{W1p([0,T],X)_weak}, $F(\cdot,x^*)\in W^1_p[0,T]$. В силу теоремы \ref{W11:embedding} отсюда следует, что $F(\cdot,x^*)\in C[0,T]$, причём справедлива оценка
\begin{gather*}
\max\limits_{t\in[0,T]}|F(t,x^*)|\leqslant A_1\|F(\cdot,x^*)\|^{(1)}_{p,[0,T]}.
\end{gather*}
Из определения норм в пространствах $W^1_p([0,T],X)$ и $W^1_p[0,T]$ следует, что
\begin{gather*}
\|F(\cdot,x^*)\|^{(1)}_{p,[0,T]}\leqslant\|f\|_{p,[0,T],X}^{(1)}\|x^*\|_{X^*}.
\end{gather*}
Поэтому при всех $x^*\in X^*$ и всех $t\in[0,T]$
\begin{gather*}
|\langle f(t),x^*\rangle|\leqslant A_1\|f\|_{p,[0,T],X}^{(1)}\|x^*\|_{X^*}.
\end{gather*}
Переходя здесь к точной верхней грани по всем $x^*\in X^*$, для которых $\|x^*\|_{X^*}\leqslant 1$ и пользуясь изометричностью вложения $X\subset X^{**}$, заключаем, что
\begin{gather}\label{W1p([0,T],X)_embedding:estimate:aux}
\sup\limits_{t\in[0,T]}\|f(t)\|_X\leqslant A_1\|f\|_{p,[0,T],X}^{(1)}.
\end{gather}

Итак, мы доказали, что если $f\in W^1_p([0,T],X)$, то $f\in C_s([0,T],X)$ и имеет место оценка (\ref{W1p([0,T],X)_embedding:estimate:aux}). Для завершения доказательства теоремы достаточно
показать, что $f\in C([0,T],X)$. Поскольку, как нетрудно видеть, $W^1_p([0,T],X)\subset W^1_1([0,T],X)$ при $p>1$, то достаточно доказать, что $W^1_1([0,T],X)\subset C([0,T],X)$.

В самом деле, пусть $f\in W^1_1([0,T],X)$, $x^*\in X^*$, $t'$, $t''\in[0,T]$, --- произвольны. Тогда
\begin{gather*}
|\langle f(t')-f(t''),x^*\rangle|=\left|\int\limits_{t'}^{t''}\langle f'(\xi),x^*\rangle d\xi\right|\leqslant
\left|\int\limits_{t'}^{t''}\|f'(\xi)\|_Xd\xi\right|\|x^*\|_{X^*}.
\end{gather*}
Взяв здесь точную верхнюю грань по всем $x^*\in X^*$, у которых $\|x^*\|_{X^*}\leqslant 1$ и пользуясь изометричностью вложения $X\subset X^{**}$, заключаем, что
\begin{gather*}
\|f(t')-f(t'')\|_X\leqslant\left|\int\limits_{t'}^{t''}\|f'(\xi)\|_Xd\xi\right|.
\end{gather*}
Из данного неравенства и абсолютной непрерывности интеграла Бохнера и следует справедливость включения $f\in C([0,T],X)$. Теорема полностью доказана.
\end{Proof}

\begin{Definition}
Функция $f\colon[0,T]\to X$ называется абсолютно непрерывной, если найдутся функция $g\in L_1([0,T],X)$ и константа $C\in X$, такие, что при всех $t\in[0,T]$ справедливо представление
\begin{gather*}
f(t)=C+(\textrm{Б})\int\limits_0^tg(\xi)d\xi.
\end{gather*}
Множество всех абсолютно непрерывных функций со значениями в $X$ обозначим $AC([0,T],X)$.
\end{Definition}

\begin{Theorem}\label{W11([0,T],X)_charachterization}
Множество $AC([0,T],X)$ совпадает с множеством $W^1_1([0,T],X)$.
\end{Theorem}
\begin{Proof}
1) Докажем сначала вложение $W^1_1([0,T],X)\subset AC([0,T],X)$. Выберем произвольно $f\in W^1_1([0,T],X)$ и зафиксируем. Пусть $x^*\in X^*$ --- произвольно. Тогда, согласно теореме
\ref{W1p([0,T],X)_weak}, функция
\begin{gather*}
[0,T]\ni t\mapsto\langle f(t),x^*\rangle
\end{gather*}
является элементом $W^1_1[0,T]$, а её обобщённой производной в смысле Соболева является функция
\begin{gather*}
[0,T]\ni t\mapsto\langle f'(t),x^*\rangle.
\end{gather*}
Поэтому, в силу теоремы \ref{W11[0,T]_charchterization}, при всех $t\in[0,T]$ справедливо представление
\begin{gather*}
\langle f(t),x^*\rangle=\langle f(0),x^*\rangle+\int\limits_0^t\langle f'(\xi),x^*\rangle d\xi,
\end{gather*}
эквивалентное соотношению
\begin{gather*}
\langle f(t)-f(0),x^*\rangle-\int\limits_0^t\langle f'(\xi),x^*\rangle d\xi=0.
\end{gather*}
В силу свойств интеграла Бохнера отсюда извлекаем, что
\begin{gather*}
\langle f(t)-f(0),x^*\rangle-\left\langle(\textrm{Б})\int\limits_0^t f'(\xi)d\xi,x^*\right\rangle =0,
\end{gather*}
или, иначе,
\begin{gather*}
\left\langle f(t)-f(0)-(\textrm{Б})\int\limits_0^t f'(\xi)d\xi,x^*\right\rangle =0\,\,\,\forall\,x^*\in X^*.
\end{gather*}
Это означает, что при всех $t\in[0,T]$
\begin{gather*}
 f(t)-f(0)-(\textrm{Б})\int\limits_0^t f'(\xi)d\xi=0.
\end{gather*}
Таким образом, функция $f$ абсолютно непрерывна на отрезке $[0,T]$, причём в качестве $C$ можно взять $f(0)$, а в качестве функции $g\in L_1([0,T],X)$ --- обобщённую производную $f'$. Итак,
в силу произвольности $f\in W^1_1([0,T],X)$, мы доказали вложение $W^1_1([0,T],X)\subset AC([0,T],X)$.

2) Докажем теперь вложение $AC([0,T],X)\subset W^1_1([0,T],X)$. Выберем произвольно $f\in AC([0,T],X)$ и зафиксируем. Тогда
найдутся функция $g\in L_1([0,T],X)$ и константа $C\in X$, такие, что при всех $t\in[0,T]$ справедливо представление
\begin{gather*}
f(t)=C+(\textrm{Б})\int\limits_0^tg(\xi)d\xi.
\end{gather*}
Следовательно, для всех $x^*\in X^*$
\begin{gather*}
\langle f(t),x^*\rangle=\langle C,x^*\rangle+\left\langle(\textrm{Б})\int\limits_0^tg(\xi)d\xi,x^*\right\rangle,
\end{gather*}
откуда, в силу свойств интеграла Бохнера,
\begin{gather*}
\langle f(t),x^*\rangle=\langle C,x^*\rangle+\int\limits_0^t\langle g(\xi),x^*\rangle d\xi\,\,\,\forall\,t\in[0,T].
\end{gather*}
Это означает, что функция
\begin{gather*}
[0,T]\ni t\mapsto\langle f(t),x^*\rangle
\end{gather*}
абсолютно непрерывна на отрезке $[0,T]$. Поэтому, в силу теоремы \ref{W11[0,T]_charchterization}, функция
\begin{gather*}
[0,T]\ni t\mapsto\langle f(t),x^*\rangle
\end{gather*}
является элементом $W^1_1[0,T]$,, а её обобщённой производной в смысле Соболева является функция
\begin{gather*}
[0,T]\ni t\mapsto\langle g(t),x^*\rangle.
\end{gather*}
Иными словами, справедливо интегральное тождество
\begin{gather*}
\int\limits_0^T\langle f(t),x^*\rangle\varphi'(t)dt=-\int\limits_0^T\langle g(t),x^*\rangle\varphi(t)dt\,\,\,\forall\,\varphi\in
\mathfrak{D}(0,T).
\end{gather*}
Ясно, что, в силу свойств интеграла Бохнера, данное тождество можно переписать в виде
\begin{gather*}
\left\langle (\textrm{Б})\int\limits_0^Tf(t)\varphi'(t)dt,x^*\right\rangle=-
\left\langle (\textrm{Б})\int\limits_0^Tg(t)\varphi(t)dt,x^*\right\rangle\,\,\,\forall\,\varphi\in \mathfrak{D}(0,T),
\end{gather*}
или, что то же самое,
\begin{gather*}
\left\langle (\textrm{Б})\int\limits_0^Tf(t)\varphi'(t)dt+(\textrm{Б})\int\limits_0^Tg(t)\varphi(t)dt,x^*\right\rangle=0\,\,\,\forall\,\varphi\in \mathfrak{D}(0,T).
\end{gather*}
В силу произвольности $x^*\in X^*$ отсюда следует, что
\begin{gather*}
(\textrm{Б})\int\limits_0^Tf(t)\varphi'(t)dt=-(\textrm{Б})\int\limits_0^Tg(t)\varphi(t)dt\,\,\,\forall\,\varphi\in\mathfrak{D}(0,T).
\end{gather*}
Последнее и означает, что $f\in W^1_1([0,T],X)$, причём функция $g$ является регулярной обобщённой производной функции $f$.

Теорема полностью доказана.
\end{Proof}

\begin{Corrolary}
Если $f\in W^1_1([0,T],X)$, то при всех $t$, $\tau\in[0,T]$
\begin{gather*}
f(t)-f(\tau)=(\textrm{Б})\int\limits_\tau^tf'(\xi)d\xi.
\end{gather*}
\end{Corrolary}

Нам также потребуется
\begin{Lemma}\cite[лемма 8.1, стр.307]{LionsMajenes}\label{LionsMajenesXY}
Пусть $X$ и $Y$ --- два банаховых пространства, $X\subset Y$ с непрерывным вложением, $X$ рефлексивно. Тогда
\begin{gather*}
L_\infty([0,T],X)\cap C_s([0,T],Y)=C_s([0,T],X).
\end{gather*}
\end{Lemma}


\begin{Theorem}\label{C1[0T]XdenseinW1p[0T]X}
Множество многочленов с коэффициентами из $X$ всюду плотно в $W^1_1([0,T],X)$.
\end{Theorem}
\begin{Proof}
Пусть $f\in W^1_1([0,T],X)$. В силу теоремы \ref{W11([0,T],X)_charachterization} найдутся функция $g\in L_1([0,T],X)$ и константа $C\in X$, такие, что
\begin{gather}\label{fW11represent}
f(t)=C+\int\limits_0^tg(\xi)d\xi\,\,\,\forall\,t\in[0,T].
\end{gather}
Поскольку, как известно, $C([0,T],X)$ всюду плотно в $L_1([0,T],X)$, а, в силу леммы \ref{Weierstrass:approx}, множество многочленов с
коэффициентами из $X$ всюду плотно в $C([0,T],X)$, то найдётся такая последовательность $g_j$, $j=1,2,\dots$, многочленов с коэффициентами из $X$, что
\begin{gather}\label{gjconvginL1[0T]X}
\lim\limits_{j\to\infty}\|g_j-g\|_{1,[0,T],X}=0.
\end{gather}
Из данного предельного соотношения и равенства (\ref{fW11represent}) вытекает, что
\begin{gather}\label{fjconvergencetofinL1}
\lim\limits_{j\to\infty}\|f_j-f\|_{1,[0,T],X}=0,
\end{gather}
где
\begin{gather*}
f_j(t)=C+\int\limits_0^tg_j(\xi)d\xi\,\,\,\forall\,t\in[0,T].
\end{gather*}

Так как при п.в. $t\in[0,T]$ справедливы равенства $f'_j(t)=g_j(t)$, $f'(t)=g(t)$, $j=1,2,\dots$, то, ввиду (\ref{gjconvginL1[0T]X}),
\begin{gather*}
\lim\limits_{j\to\infty}\|f_j'-f'\|_{1,[0,T],X}=0.
\end{gather*}
Из данного равенства совместно с (\ref{fjconvergencetofinL1}) вытекает, что
\begin{gather*}
\lim\limits_{j\to\infty}\|f_j-f\|_{1,[0,T],X}^{(1)}=0.
\end{gather*}

Отсюда и из произвольности функции $f\in W^1_1([0,T],X)$ вытекает  утверждение теоремы.
\end{Proof}


Приведём теперь следующие конструкции (см. \cite[стр.70]{Lions1972}). Пусть  $B_0$, $B$ и $B_1$ --- банаховы пространства, причём $B_0
\subset B\subset B_1$, $B_0$ и $B_1$ рефлексивны, вложение $B_0$ в $B$ компактно, а вложение $B$ в $B_1$ --- непрерывно. Пусть $\mathfrak{W}
\equiv\{\mathfrak{z}:\mathfrak{z}\in L_{p_0}([0,T],B_0),\,\,\,\dot{\mathfrak{z}}\in L_{p_1}([0,T],B_1)$, где $1<p_0<\infty$, $1<p_1<\infty$. Снабдив $\mathfrak{W}$ нормой
\begin{gather*}
\|\mathfrak{z}\|_{\mathfrak{W}}\equiv\|\mathfrak{z}\|_{p_0,[0,T],B_0}+\|\dot{\mathfrak{z}}\|_{p_1,[0,T],B_1},
\end{gather*}
получим банахово пространство. Нетрудно видеть, что $\mathfrak{W}\subset L_{p_0}([0,T],B)$.

\begin{Theorem}\label{B0BB1Theorem}\cite[теорема 5.1 на стр.70]{Lions1972}
При сделанных предположениях вложение пространства $\mathfrak{W}$ в пространство $L_{p_0}([0,T],B)$ --- компактно.
\end{Theorem}

        \section{Функции одного переменного и со значениями в гильбертовом пространстве}
Пусть $V$ и $H$ --- сепарабельные гильбертовы пространства со скалярными произведениями $\langle\cdot,\cdot\rangle_V$ и $\langle\cdot,\cdot\rangle_H$ соответственно, с соответствующими
нормами $\|\cdot\|_V$ и $\|\cdot\|_H$, $V\subset H$, это вложение непрерывно и компактно. Иными словами, найдётся постоянная $\nu>0$, такая, что
\begin{gather*}
\|v\|_H\leqslant\nu\|v\|_V\,\,\,\forall\,v\in V,
\end{gather*}
причём любое ограниченное в норме $V$ множество предкомпактно в норме $H$.

Через $\mathcal{W}^1_2([0,T];V,H)$ обозначим множество функций $\mathfrak{z}\in L_2([0,T],V)$, имеющих регулярную обобщённую производную
$\dot{\mathfrak{z}}\in L_2([0,T],H)$. Наделим $\mathcal{W}^1_2([0,T];V,H)$ скалярным произведением
\begin{gather*}
\langle\mathfrak{z}_1,\mathfrak{z}_2\rangle_{\mathcal{W}^1_2([0,T];V,H)}\equiv
\int\limits_0^T[\langle\mathfrak{z}_1(t),\mathfrak{z}_2(t)\rangle_V+\langle\dot{\mathfrak{z}}_1(t),\dot{\mathfrak{z}}_2(t)\rangle_H]dt,
\end{gather*}
с соответствующей нормой
\begin{gather*}
\|\mathfrak{z}\|_{\mathcal{W}^1_2([0,T];V,H)}=\sqrt{\langle\mathfrak{z},\mathfrak{z}\rangle_{\mathcal{W}^1_2([0,T];V,H)}}.
\end{gather*}

Через $\mathcal{W}^1_\infty([0,T];V,H)$ обозначим множество функций $\mathfrak{z}\in L_\infty([0,T],V)$, имеющих регулярную обобщённую
производную $\dot{\mathfrak{z}}\in L_\infty([0,T],H)$. Наделим $\mathcal{W}^1_\infty([0,T];V,H)$ нормой
\begin{gather*}
\|\mathfrak{z}\|_{\mathcal{W}^1_\infty([0,T];V,H)}\equiv\|\mathfrak{z}\|_{\infty,[0,T],V}+\|\dot{\mathfrak{z}}\|_{\infty,[0,T],H}.
\end{gather*}

\begin{Theorem}
Пространство  $\mathcal{W}^1_2([0,T];V,H)$ --- гильбертово.
\end{Theorem}
\begin{Proof}
Предположим, что последовательность $f_j\in\mathcal{W}^1_2([0,T];V,H)$, $j=1,2,\dots$, --- фундаментальна в норме пространства $\mathcal{W}^1_2([0,T];V,H)$, т.е.
\begin{gather}\label{fundamentality_in_calW12([0,T];V,H)}
\forall\,\varepsilon>0\,\,\exists\,j_0=j_0(\varepsilon)\geqslant1\,\,\forall\,j\geqslant j_0(\varepsilon),\,\,k\geqslant1:
\|f_{j+k}-f_j\|_{\mathcal{W}^1_2([0,T];V,H)}\leqslant\varepsilon.
\end{gather}
Отсюда, ввиду определения нормы в пространстве $\mathcal{W}^1_2([0,T];V,H)$, выводим, что
\begin{gather*}
\forall\,\varepsilon>0\,\,\exists\,j_0=j_0(\varepsilon)\geqslant1\,\,\forall\,j\geqslant j_0(\varepsilon),\,\,k\geqslant1:
\|f_{j+k}-f_j\|_{2,[0,T],V}\leqslant\varepsilon,\,\,\,\|f_{j+k}'-f_j'\|_{2,[0,T],H}\leqslant\varepsilon.
\end{gather*}
Это означает, что последовательности $f_j$, $j=1,2,\dots$, и $f_j'$, $j=1,2,\dots$, фундаментальны в пространствах $L_2([0,T],V)$ и $L_2([0,T],H)$ соответственно. А так как данные
пространства полны, то существуют функции $g_0\in L_2([0,T],V)$, $g_1\in L_2([0,T],H)$, такие, что
\begin{gather}\label{convergence_of_fjf'j::::fundamentality_in_calW12([0,T];V,H)}
\|f_{j}-g_0\|_{2,[0,T],V}\to0,\,\,\,\|f_{j}'-g_1\|_{2,[0,T],H}\to0,\,\,\,j\to\infty.
\end{gather}
Поэтому
\begin{gather}\label{convergence_of_fjf'j:::1::::fundamentality_in_calW12([0,T];V,H)}
\|f_{j}-g_0\|_{1,[0,T],V}\to0,\,\,\,\|f_{j}'-g_1\|_{1,[0,T],H}\to0,\,\,\,j\to\infty.
\end{gather}

Поскольку $f_j\in\mathcal{W}^1_2([0,T];V,H)$, $j=1,2,\dots$, то, на основании определения класса $\mathcal{W}^1_2([0,T];V,H)$, справедливо интегральное тождество
\begin{gather}\label{X_valued_regular_distribution_derivative:regular:::1::::fundamentality_in_calW12([0,T];V,H)}
(\textrm{Б})\int\limits_0^T f_j'(t)\varphi(t)dt=-(\textrm{Б})\int\limits_0^T f_j(t)\varphi'(t)dt\,\,\,\forall\,\varphi\in\mathfrak{D}(0,T),
\end{gather}
интегралы Бохнера в котором понимаются как интегралы Бохнера от функций со значениями в $H$.
Совершим в данном тождестве переход к пределу при $j\to\infty$. Во--первых, легко видеть, что
\begin{gather*}
\left\|(\textrm{Б})\int\limits_0^T f_j'(t)\varphi(t)dt-(\textrm{Б})\int\limits_0^T g_1(t)\varphi(t)dt\right\|_H=
\left\|(\textrm{Б})\int\limits_0^T [f_j'(t)-g_1(t)]\varphi(t)dt\right\|_H\leqslant\\
\leqslant\int\limits_0^T \|[f_j'(t)-g_1(t)]\varphi(t)\|_Hdt=\int\limits_0^T \|f_j'(t)-g_1(t)\|_H|\varphi(t)|dt\leqslant
\pmb{|}\varphi\pmb{|}^{(0)}_{[0,T]}\|f_{j}'-g_1\|_{1,[0,T],H}\to0,\,\,\,j\to\infty,
\end{gather*}
в силу (\ref{convergence_of_fjf'j:::1::::fundamentality_in_calW12([0,T];V,H)}). Итак,
\begin{gather}\label{convergence_f'j_integrals::::fundamentality_in_calW12([0,T];V,H)}
\lim\limits_{j\to\infty}\left\|(\textrm{Б})\int\limits_0^T f_j'(t)\varphi(t)dt-(\textrm{Б})\int\limits_0^T g_1(t)\varphi(t)dt\right\|_H=0.
\end{gather}
Во--вторых,
\begin{gather*}
\left\|\left[-(\textrm{Б})\int\limits_0^T f_j(t)\varphi(t)dt\right]-\left[-(\textrm{Б})\int\limits_0^T g_0(t)\varphi(t)dt\right]\right\|_H=
\left\|-(\textrm{Б})\int\limits_0^T [f_j(t)-g_0(t)]\varphi(t)dt\right\|_H\leqslant\\
\leqslant\int\limits_0^T \|[f_j(t)-g_0(t)]\varphi(t)\|_Xdt=\int\limits_0^T \|f_j(t)-g_0(t)\|_H|\varphi(t)|dt\leqslant
\pmb{|}\varphi\pmb{|}^{(0)}_{[0,T]}\|f_{j}-g_0\|_{1,[0,T],H}\leqslant\\
\leqslant\pmb{|}\varphi\pmb{|}^{(0)}_{[0,T]}\nu\|f_{j}-g_0\|_{1,[0,T],V}\to0,\,\,\,j\to\infty,
\end{gather*}
на основании (\ref{convergence_of_fjf'j:::1::::fundamentality_in_calW12([0,T];V,H)}). Как следствие,
\begin{gather}\label{convergence_fj_integrals::::fundamentality_in_calW12([0,T];V,H)}
\lim\limits_{j\to\infty}\left\|\left[-(\textrm{Б})\int\limits_0^T f_j(t)\varphi(t)dt\right]-\left[-(\textrm{Б})\int\limits_0^T g_0(t)\varphi(t)dt\right]\right\|_H=0.
\end{gather}
Переходя теперь в интегральном тождестве (\ref{X_valued_regular_distribution_derivative:regular:::1::::fundamentality_in_calW12([0,T];V,H)}), с учётом предельных соотношений
(\ref{convergence_f'j_integrals::::fundamentality_in_calW12([0,T];V,H)}) и (\ref{convergence_fj_integrals::::fundamentality_in_calW12([0,T];V,H)}), к пределу при $j\to\infty$, получаем, что
\begin{gather*}
(\textrm{Б})\int\limits_0^T g_1(t)\varphi(t)dt=-(\textrm{Б})\int\limits_0^T g_0(t)\varphi'(t)dt\,\,\,\forall\,\varphi\in\mathfrak{D}(0,T).
\end{gather*}

Данное тождество означает, что функция $g_0$ является элементом пространства $\mathcal{W}^1_2([0,T];V,H)$, причём  $g_0'$, регулярная первая обобщённая производная функции $g_0$, совпадает
с функцией  $g_1$. Осталось доказать, что последовательность $f_j$, $j=1,2,\dots$, сходится в $\mathcal{W}^1_2([0,T];V,H)$ к функции $g_0$.

Несложно видеть, что неравенство (\ref{fundamentality_in_calW12([0,T];V,H)}) можно переписать в виде
\begin{gather*}
\forall\,\varepsilon>0\,\,\exists\,j_0=j_0(\varepsilon)\geqslant1\,\,\forall\,j\geqslant j_0(\varepsilon),\,\,k\geqslant1:
[\|f_{j+k}-f_j\|_{2,[0,T],V}^2+\|f_{j+k}'-f_j'\|_{2,[0,T],H}^2]^{1/2}\leqslant\varepsilon.
\end{gather*}
Устремляя здесь $k$ к бесконечности и учтя соотношения (\ref{convergence_of_fjf'j::::fundamentality_in_calW12([0,T];V,H)}), будем иметь
\begin{gather*}
\forall\,\varepsilon>0\,\,\exists\,j_0=j_0(\varepsilon)\geqslant1\,\,\forall\,j\geqslant j_0(\varepsilon):
[\|g_{0}-f_j\|_{2,[0,T],V}^2+\|g_{0}'-f_j'\|_{2,[0,T],H}^2]^{1/2}\leqslant\varepsilon.
\end{gather*}
А это и означает, что $\|g_{0}-f_j\|_{\mathcal{W}^1_2([0,T];V,H)}\to0$, $j\to\infty$. Лемма полностью доказана.
\end{Proof}


\begin{Theorem}
Пространство  $\mathcal{W}^1_\infty([0,T];V,H)$ --- полно.
\end{Theorem}
\begin{Proof}
Предположим, что последовательность $f_j\in\mathcal{W}^1_\infty([0,T];V,H)$, $j=1,2,\dots$, --- фундаментальна в норме пространства $\mathcal{W}^1_\infty([0,T];V,H)$, т.е.
\begin{gather}\label{fundamentality_in_calW1infty([0,T];V,H)}
\forall\,\varepsilon>0\,\,\exists\,j_0=j_0(\varepsilon)\geqslant1\,\,\forall\,j\geqslant j_0(\varepsilon),\,\,k\geqslant1:
\|f_{j+k}-f_j\|_{\mathcal{W}^1_\infty([0,T];V,H)}\leqslant\varepsilon.
\end{gather}
Отсюда, ввиду определения нормы в пространстве $\mathcal{W}^1_\infty([0,T];V,H)$, выводим, что
\begin{gather*}
\forall\,\varepsilon>0\,\,\exists\,j_0=j_0(\varepsilon)\geqslant1\,\,\forall\,j\geqslant j_0(\varepsilon),\,\,k\geqslant1:
\|f_{j+k}-f_j\|_{\infty,[0,T],V}\leqslant\varepsilon,\,\,\,\|f_{j+k}'-f_j'\|_{\infty,[0,T],H}\leqslant\varepsilon.
\end{gather*}
Это означает, что последовательности $f_j$, $j=1,2,\dots$, и $f_j'$, $j=1,2,\dots$, фундаментальны в пространствах $L_\infty([0,T],V)$ и $L_\infty([0,T],H)$ соответственно. А так как данные
пространства полны, то существуют функции $g_0\in L_\infty([0,T],V)$, $g_1\in L_\infty([0,T],H)$, такие, что
\begin{gather}\label{convergence_of_fjf'j::::fundamentality_in_calW1infty([0,T];V,H)}
\|f_{j}-g_0\|_{\infty,[0,T],V}\to0,\,\,\,\|f_{j}'-g_1\|_{\infty,[0,T],H}\to0,\,\,\,j\to\infty.
\end{gather}
Поэтому
\begin{gather}\label{convergence_of_fjf'j:::1::::fundamentality_in_calW1infty([0,T];V,H)}
\|f_{j}-g_0\|_{1,[0,T],V}\to0,\,\,\,\|f_{j}'-g_1\|_{1,[0,T],H}\to0,\,\,\,j\to\infty.
\end{gather}

Так как $f_j\in\mathcal{W}^1_\infty([0,T];V,H)$, $j=1,2,\dots$, то, согласно определению класса $\mathcal{W}^1_\infty([0,T];V,H)$, имеет место интегральное тождество
\begin{gather}\label{X_valued_regular_distribution_derivative:regular:::1::::fundamentality_in_calW1infty([0,T];V,H)}
(\textrm{Б})\int\limits_0^T f_j'(t)\varphi(t)dt=-(\textrm{Б})\int\limits_0^T f_j(t)\varphi'(t)dt\,\,\,\forall\,\varphi\in\mathfrak{D}(0,T),
\end{gather}
интегралы Бохнера в котором понимаются как интегралы Бохнера от функций со значениями в $H$. Совершим в данном тождестве переход к пределу при $j\to\infty$. Во--первых, легко видеть, что
\begin{gather*}
\left\|(\textrm{Б})\int\limits_0^T f_j'(t)\varphi(t)dt-(\textrm{Б})\int\limits_0^T g_1(t)\varphi(t)dt\right\|_H=
\left\|(\textrm{Б})\int\limits_0^T [f_j'(t)-g_1(t)]\varphi(t)dt\right\|_H\leqslant\\
\leqslant\int\limits_0^T \|[f_j'(t)-g_1(t)]\varphi(t)\|_Hdt=\int\limits_0^T \|f_j'(t)-g_1(t)\|_H|\varphi(t)|dt\leqslant
\pmb{|}\varphi\pmb{|}^{(0)}_{[0,T]}\|f_{j}'-g_1\|_{1,[0,T],H}\to0,\,\,\,j\to\infty,
\end{gather*}
в силу (\ref{convergence_of_fjf'j:::1::::fundamentality_in_calW1infty([0,T];V,H)}). Итак,
\begin{gather}\label{convergence_f'j_integrals::::fundamentality_in_calW1infty([0,T];V,H)}
\lim\limits_{j\to\infty}\left\|(\textrm{Б})\int\limits_0^T f_j'(t)\varphi(t)dt-(\textrm{Б})\int\limits_0^T g_1(t)\varphi(t)dt\right\|_H=0.
\end{gather}
Во--вторых,
\begin{gather*}
\left\|\left[-(\textrm{Б})\int\limits_0^T f_j(t)\varphi(t)dt\right]-\left[-(\textrm{Б})\int\limits_0^T g_0(t)\varphi(t)dt\right]\right\|_H=
\left\|-(\textrm{Б})\int\limits_0^T [f_j(t)-g_0(t)]\varphi(t)dt\right\|_H\leqslant\\
%\end{gather*}
%\begin{gather*}
\leqslant\int\limits_0^T \|[f_j(t)-g_0(t)]\varphi(t)\|_Xdt=\int\limits_0^T \|f_j(t)-g_0(t)\|_H|\varphi(t)|dt\leqslant
\pmb{|}\varphi\pmb{|}^{(0)}_{[0,T]}\|f_{j}-g_0\|_{1,[0,T],H}\leqslant\\
\leqslant\pmb{|}\varphi\pmb{|}^{(0)}_{[0,T]}\nu\|f_{j}-g_0\|_{1,[0,T],V}\to0,\,\,\,j\to\infty,
\end{gather*}
вследствие (\ref{convergence_of_fjf'j:::1::::fundamentality_in_calW1infty([0,T];V,H)}). Как следствие,
\begin{gather}\label{convergence_fj_integrals::::fundamentality_in_calW1infty([0,T];V,H)}
\lim\limits_{j\to\infty}\left\|\left[-(\textrm{Б})\int\limits_0^T f_j(t)\varphi(t)dt\right]-\left[-(\textrm{Б})\int\limits_0^T g_0(t)\varphi(t)dt\right]\right\|_H=0.
\end{gather}
Переходя теперь в интегральном тождестве (\ref{X_valued_regular_distribution_derivative:regular:::1::::fundamentality_in_calW1infty([0,T];V,H)}), с учётом предельных соотношений
(\ref{convergence_f'j_integrals::::fundamentality_in_calW1infty([0,T];V,H)}) и (\ref{convergence_fj_integrals::::fundamentality_in_calW1infty([0,T];V,H)}), к пределу при $j\to\infty$,
получаем, что
\begin{gather*}
(\textrm{Б})\int\limits_0^T g_1(t)\varphi(t)dt=-(\textrm{Б})\int\limits_0^T g_0(t)\varphi'(t)dt\,\,\,\forall\,\varphi\in\mathfrak{D}(0,T).
\end{gather*}

Данное тождество означает, что функция $g_0$ является элементом пространства $\mathcal{W}^1_\infty([0,T];V,H)$, причём  $g_0'$, регулярная первая обобщённая производная функции $g_0$,
совпадает с функцией  $g_1$. Осталось доказать, что последовательность $f_j$, $j=1,2,\dots$, сходится в $\mathcal{W}^1_\infty([0,T];V,H)$ к функции $g_0$.

Несложно видеть, что неравенство (\ref{fundamentality_in_calW12([0,T];V,H)}) можно переписать в виде
\begin{gather*}
\forall\,\varepsilon>0\,\,\exists\,j_0=j_0(\varepsilon)\geqslant1\,\,\forall\,j\geqslant j_0(\varepsilon),\,\,k\geqslant1:
\|f_{j+k}-f_j\|_{\infty,[0,T],V}+\|f_{j+k}'-f_j'\|_{\infty,[0,T],H}\leqslant\varepsilon.
\end{gather*}
Устремляя здесь $k$ к бесконечности и учтя соотношения (\ref{convergence_of_fjf'j::::fundamentality_in_calW1infty([0,T];V,H)}), будем иметь
\begin{gather*}
\forall\,\varepsilon>0\,\,\exists\,j_0=j_0(\varepsilon)\geqslant1\,\,\forall\,j\geqslant j_0(\varepsilon):
[\|g_{0}-f_j\|_{\infty,[0,T],V}+\|g_{0}'-f_j'\|_{\infty,[0,T],H}\leqslant\varepsilon.
\end{gather*}
А это и означает, что $\|g_{0}-f_j\|_{\mathcal{W}^1_\infty([0,T];V,H)}\to0$, $j\to\infty$. Лемма полностью доказана.
\end{Proof}

Из определения классов $\mathcal{W}^1_2([0,T];V,H)$ и $\mathcal{W}^1_\infty([0,T];V,H)$ вытекают теоретико--множественные вложения $\mathcal{W}^1_2([0,T];V,H)\subset W^1_2([0,T],H)$,
$\mathcal{W}^1_\infty([0,T];V,H)\subset W^1_\infty([0,T],H)$. Поэтому, на основании теоремы \ref{W1p([0,T],X)_embedding}, справедлива
\begin{Theorem} Пусть $f\in \mathcal{W}^1_2([0,T];V,H)$, $g\in \mathcal{W}^1_\infty([0,T];V,H)$. Тогда $f$, $g\in C([0,T],H)$, причём
\begin{gather*}
\max\limits_{t\in[0,T]}\|f(t)\|_H\leqslant A_1(2,T)\max\{1,\nu\}\|f\|_{\mathcal{W}^1_2([0,T];V,H)},\\
\max\limits_{t\in[0,T]}\|g(t)\|_H\leqslant A_1(\infty,T)\max\{1,\nu\}\|g\|_{\mathcal{W}^1_\infty([0,T];V,H)},
\end{gather*}
где $A_1$ --- та же постоянная, что и в теореме \ref{W11:embedding}.
\end{Theorem}

Через $\textrm{Э}([0,T];V,H)$ обозначим множество функций из $\mathfrak{z}\in C_s([0,T],V)$, принадлежащих пространству $W^1_\infty([0,T],H)$. Зададим в $\textrm{Э}([0,T];V,H)$ норму
равенством
\begin{gather*}
\|\mathfrak{z}\|_{\textrm{Э}([0,T];V,H)}\equiv\|\mathfrak{z}\|_{C_s([0,T],V)}+\|\dot{\mathfrak{z}}\|_{\infty,[0,T],H}.
\end{gather*}

\begin{Theorem}
Пространство $\textrm{Э}([0,T];V,H)$ --- банахово.
\end{Theorem}
\begin{Proof} Пусть последовательность $\mathfrak{z}_j$, $j=1,2,\dots$, --- фундаментальна в норме пространства $\textrm{Э}([0,T];V,H)$. Это означает, что
\begin{gather}\label{fundamentality_in_energetic_class}
\forall\,\varepsilon>0\,\,\exists\,j_0=j_0(\varepsilon)\geqslant1\,\,\forall\,j\geqslant j_0(\varepsilon),\,\,k\geqslant1:\\
\notag\|\mathfrak{z}_{j+k}-\mathfrak{z}_{j}\|_{\textrm{Э}([0,T];V,H)}\equiv\|\mathfrak{z}_{j+k}-\mathfrak{z}_{j}\|_{C_s([0,T],V)}+
\|\dot{\mathfrak{z}}_{j+k}-\dot{\mathfrak{z}}_{j}\|_{\infty,[0,T],H}\leqslant\varepsilon.
\end{gather}
Таким образом, последовательность $\mathfrak{z}_j$, $j=1,2,\dots$,  фундаментальна в пространстве $C_s([0,T],V)$, наделённом нормой $\|\cdot\|_{C_s([0,T],V)}$, а последовательность
$\dot{\mathfrak{z}}_j$, $j=1,2,\dots$, фундаментальна в норме пространства $L_\infty([0,T],H)$. На основании леммы \ref{completness_of_Cs([0,T],X)_with_strong_norm} и полноты пространства
$L_\infty([0,T],H)$ найдутся функции $\mathfrak{g}_0\in C_s([0,T],V)$ и $\mathfrak{g}_1\in L_\infty([0,T],H)$, такие, что
\begin{gather}\label{convergence_zjz'j_in_energetic_class}
\lim\limits_{j\to\infty}\|\mathfrak{z}_{j}-\mathfrak{g}_{0}\|_{C_s([0,T],V)}=0,\,\,\,\lim\limits_{j\to\infty}\|\dot{\mathfrak{z}}_{j}-\mathfrak{g}_{1}\|_{\infty,[0,T],H}=0.
\end{gather}
Так как $\mathfrak{z}_j\in \textrm{Э}([0,T];V,H)$, $j=1,2,\dots$, то $\mathfrak{z}_j\in W^1_\infty([0,T],H)$, $j=1,2,\dots$, вследствие чего выполнено интегральное тождество
\begin{gather}\label{integral_identity_energetic_class}
(\textrm{Б})\int\limits_0^T\dot{\mathfrak{z}}_{j}(t)\varphi(t)dt=-(\textrm{Б})\int\limits_0^T\mathfrak{z}_{j}(t)\varphi'(t)dt\,\,\,\forall\,\varphi\in\mathfrak{D}(0,T).
\end{gather}
Заметим, что

\begin{gather*}
\left\|(\textrm{Б})\int\limits_0^T\dot{\mathfrak{z}}_{j}(t)\varphi(t)dt-(\textrm{Б})\int\limits_0^T\mathfrak{g}_{1}(t)\varphi(t)dt\right\|_H=
\left\|(\textrm{Б})\int\limits_0^T[\dot{\mathfrak{z}}_{j}(t)-\mathfrak{g}_{1}(t)]\varphi(t)dt\right\|_H\leqslant\\
\leqslant\int\limits_0^T\|[\dot{\mathfrak{z}}_{j}(t)-\mathfrak{g}_{1}(t)]\varphi(t)\|_Hdt=\int\limits_0^T\|\dot{\mathfrak{z}}_{j}(t)-\mathfrak{g}_{1}(t)\|_H|\varphi(t)|dt\leqslant
T\pmb{|}\varphi\pmb{|}^{(0)}_{[0,T]}\|\dot{\mathfrak{z}}_{j}-\mathfrak{g}_{1}\|_{\infty,[0,T],H}\to0,\,\,\,j\to\infty,
\end{gather*}
ввиду соотношений (\ref{convergence_zjz'j_in_energetic_class}). Таким образом,
\begin{gather}\label{convergence_integrals_mathfrak_zj'}
\lim\limits_{j\to\infty}\left\|(\textrm{Б})\int\limits_0^T\dot{\mathfrak{z}}_{j}(t)\varphi(t)dt-(\textrm{Б})\int\limits_0^T\mathfrak{g}_{1}(t)\varphi(t)dt\right\|_H=0.
\end{gather}
Кроме того,
\begin{gather*}
\left\|\left[-(\textrm{Б})\int\limits_0^T\mathfrak{z}_{j}(t)\varphi(t)dt\right]-\left[-(\textrm{Б})\int\limits_0^T\mathfrak{g}_{0}(t)\varphi(t)dt\right]
\right\|_H=\left\|-(\textrm{Б})\int\limits_0^T[\mathfrak{z}_{j}(t)-\mathfrak{g}_{0}(t)]\varphi(t)dt\right\|_H\leqslant\\
\leqslant\int\limits_0^T\|[\mathfrak{z}_{j}(t)-\mathfrak{g}_{0}(t)]\varphi(t)\|_Hdt=\int\limits_0^T\|\mathfrak{z}_{j}(t)-\mathfrak{g}_{0}(t)\|_H|\varphi(t)|dt\leqslant\nu T
\pmb{|}\varphi\pmb{|}^{(0)}_{[0,T]}\|\mathfrak{z}_{j}-\mathfrak{g}_{0}\|_{C_s([0,T],V)}\to0,\,\,\,
j\to\infty,
\end{gather*}
ввиду соотношений (\ref{convergence_zjz'j_in_energetic_class}). Следовательно,
\begin{gather}\label{convergence_integrals_mathfrak_zj}
\lim\limits_{j\to\infty}\left\|\left[-(\textrm{Б})\int\limits_0^T\mathfrak{z}_{j}(t)\varphi(t)dt\right]-\left[-(\textrm{Б})\int\limits_0^T\mathfrak{g}_{0}(t)\varphi(t)dt\right]\right\|_H=0.
\end{gather}
Перейдя теперь в (\ref{integral_identity_energetic_class}) с учётом (\ref{convergence_integrals_mathfrak_zj'}) и (\ref{convergence_integrals_mathfrak_zj}), получаем, что
\begin{gather*}
(\textrm{Б})\int\limits_0^T\mathfrak{g}_{1}(t)\varphi(t)dt=-(\textrm{Б})\int\limits_0^T\mathfrak{g}_{0}(t)\varphi'(t)dt\,\,\,\forall\,\varphi\in\mathfrak{D}(0,T).
\end{gather*}
Поэтому $g_0\in W^1_\infty([0,T],H)$, а её регулярная первая обобщённая производная $g_0'$ совпадает с $g_1$. Поскольку же ранее было доказано включение $g_0\in C_s([0,T],V)$, то
$g_0\in \textrm{Э}([0,T];V,H)$. Переходя затем к пределу при $k\to\infty$ в соотношении (\ref{fundamentality_in_energetic_class}) и учтя предельные соотношения
(\ref{convergence_zjz'j_in_energetic_class}), получим, что
\begin{gather*}
\forall\,\varepsilon>0\,\,\exists\,j_0=j_0(\varepsilon)\geqslant1\,\,\forall\,j\geqslant j_0(\varepsilon):\|\mathfrak{g}_{0}-\mathfrak{z}_{j}\|_{\textrm{Э}([0,T];V,H)}\equiv
\|\mathfrak{g}_{0}-\mathfrak{z}_{j}\|_{C_s([0,T],V)}+\|\dot{\mathfrak{g}}_{0}-\dot{\mathfrak{z}}_{j}\|_{\infty,[0,T],H}\leqslant\varepsilon.
\end{gather*}
Последнее же означает, что $\|\mathfrak{g}_{0}-\mathfrak{z}_{j}\|_{\textrm{Э}([0,T];V,H)}\to0$, $j\to\infty$. Теорема доказана.
\end{Proof}

\begin{Definition}\cite{warga}
Пусть $\mathcal{P}$ --- компактное метрическое пространство с метрикой $d(\cdot,\cdot)$, $X$ --- банахово пространство с нормой
$\|\cdot\|_X$. Множество $\mathfrak{M}\subset C(\mathcal{P},X)$ называется \textbf{равностепенно непрерывным}, если
\begin{gather*}
\forall\,\varepsilon>0\,\,\exists\,\delta=\delta(\varepsilon)>0\,\,\forall\,p',\,\,p''\in\mathcal{P},\,\,d(p',p'')\leqslant\delta\,\,\,
\forall\,f\in\mathfrak{M}:\|f(p')-f(p'')\|_X\leqslant\varepsilon.
\end{gather*}
\end{Definition}

\begin{Theorem}\cite{warga} (Теорема Арцела--Асколи)\label{Arcela--Ascoli}
Пусть $\mathcal{P}$ --- компактное метрическое пространство с метрикой $d(\cdot,\cdot)$, $X$ --- банахово пространство с нормой
$\|\cdot\|_X$. Множество $\mathfrak{M}\subset C(\mathcal{P},X)$ предкомпактно в $C(\mathcal{P},X)$ тогда и только тогда, когда
\begin{enumerate}
    \item
множество $\mathfrak{M}$ равностепенно непрерывно;
    \item
множество $\{f(p):p\in\mathcal{P},\,\,\,f\in\mathfrak{M}\}\subset X$ --- предкомпактно в $X$.
\end{enumerate}
\end{Theorem}

\begin{Theorem}\label{abstact_energetic_class_embedding:Theorem}
Если $\mathfrak{z}\in \textrm{Э}([0,T];V,H)$, то $\mathfrak{z}\in C([0,T],H)$, причём
\begin{gather*}
\max\limits_{t\in[0,T]}\|\mathfrak{z}(t)\|_H\leqslant\nu\|\mathfrak{z}\|_{\textrm{Э}([0,T];V,H)}.
\end{gather*}
Кроме того, вложение $\textrm{Э}([0,T];V,H)\subset C([0,T],H)$ --- компактно.
\end{Theorem}
\begin{Proof} Утверждение о том, что имеет место вложение $\textrm{Э}([0,T];V,H)\subset C([0,T],H)$ и это вложение непрерывно, следует из
предыдущей теоремы, очевидного вложения $$\textrm{Э}([0,T];V,H)\subset\mathcal{W}([0,T];V,H)$$ и определения нормы в пространстве $\textrm{Э}([0,T];V,H)$.
Таким образом, достаточно лишь доказать компактность вложения $\textrm{Э}([0,T];V,H)\subset C([0,T],H)$.

В самом деле, пусть $\mathfrak{M}\subset \textrm{Э}([0,T];V,H)$ --- ограниченное в норме $\textrm{Э}([0,T];V,H)$ множество. Тогда найдётся постоянная $K>0$, такая, что
\begin{gather*}
\forall\,\mathfrak{z}\in\mathfrak{M}:\sup\limits_{t\in[0,T]}\|\mathfrak{z}(t)\|_V+\vraisup\limits_{t\in[0,T]}\|\dot{\mathfrak{z}}(t)\|_H
\leqslant K.
\end{gather*}
Иными словами, множество
\begin{gather*}
\{\mathfrak{z}(t):t\in[0,T],\,\,\,\mathfrak{z}\in\mathfrak{M}\}\subset V,
\end{gather*}
ограничено в норме $V$, и, в силу компактности вложения $V\subset H$, предкомпактно в норме пространства $H$. Далее, при всех $h\in H$ и при всех $t'$, $t''\in[0,T]$
\begin{gather*}
|\langle\mathfrak{z}(t')-\mathfrak{z}(t''),h\rangle_H|=\left|\int\limits_{t''}^{t'}\langle\dot{\mathfrak{z}}(t),h\rangle_Hdt\right|
\leqslant\left|\int\limits_{t''}^{t'}\|\dot{\mathfrak{z}}(t)\|_Hdt\right|\|h\|_H\leqslant C|t'-t''|\|h\|_H.
\end{gather*}
Взяв точную верхнюю грань по всем $h\in H$, $\|h\|_H\leqslant1$, получим, что
\begin{gather*}
\|\mathfrak{z}(t')-\mathfrak{z}(t'')\|_H\leqslant C|t'-t''|.
\end{gather*}
Следовательно, для любого $\varepsilon>0$ найдётся $\delta=\delta(\varepsilon)=\frac\varepsilon C$, такое, что для всех $t'$, $t''\in[0,T]$, $|t'-t''|<\delta$ и для всех $\mathfrak{z}\in
\mathfrak{M}$ выполнено  условие
\begin{gather*}
\|\mathfrak{z}(t')-\mathfrak{z}(t'')\|_H<\varepsilon.
\end{gather*}
Следовательно, в силу теоремы \ref{Arcela--Ascoli}, множество $\mathfrak{M}\subset \textrm{Э}([0,T];V,H)$ предкомпактно в норме пространства $C([0,T],H)$. Таким образом, теорема полностью
доказана.
\end{Proof}

Через $\mathbb{W}([0,T];V,H)$ обозначим множество $W^1_\infty([0,T],H)\cap L_\infty([0,T],V)$. Согласно \cite[теорема 8.18.3, стр.809]{Edwards},
\begin{gather}\label{L1*Linfty_isometric_isomorphic}
L_\infty([0,T],V)\cong(L_1([0,T],V^*))^*,\,\,\,L_\infty([0,T],H)\cong(L_1([0,T],H))^*,
\end{gather}
где знаком $\cong$ обозначен изометрический изоморфизм банаховых пространств.


Будем говорить, что последовательность $\mathfrak{z}_j\in\mathbb{W}([0,T];V,H)$, $j=1,2,\dots$, сходится к функции $\mathfrak{z}\in \mathbb{W}([0,T];V,H)$, если
$$
\mathfrak{z}_j\to\mathfrak{z},\,\,\,j\to\infty,\text{ $*$--слабо в $L_\infty([0,T],V)$};\,\,\, \dot{\mathfrak{z}}_j\to\dot{\mathfrak{z}},
\,\,\,j\to\infty,\text{ $*$--слабо в $L_\infty([0,T],H)$;}
$$
то есть
\begin{gather*}
\lim\limits_{j\to\infty}\int\limits_0^T\langle\mathfrak{z}_j(t),\varphi(t)\rangle dt=\int\limits_0^T\langle\mathfrak{z}(t),\varphi(t)\rangle dt\,\,\,\forall\,\varphi\in L_1([0,T],V^*);\\
\lim\limits_{j\to\infty}\int\limits_0^T\langle\dot{\mathfrak{z}}_j(t),\varphi(t)\rangle_H dt=\int\limits_0^T\langle\dot{\mathfrak{z}}(t),\varphi(t)\rangle_H dt\,\,\,
\forall\,\varphi\in L_1([0,T],H).
\end{gather*}

\begin{Theorem}\label{closeness_mathbbW}
Пусть последовательность $\mathfrak{z}_j\in\mathbb{W}([0,T];V,H)$, $j=1,2,\dots$, такова, что для некоторых функций $\mathfrak{g}_0\in
L_\infty([0,T],V)$ и $\mathfrak{g}_1\in L_\infty([0,T],H)$
\begin{gather}\label{zjzj'coverg}
\mathfrak{z}_j\to\mathfrak{g}_0,\,\,\,j\to\infty,\text{ $*$--слабо в $L_\infty([0,T],V)$};\,\,\, \dot{\mathfrak{z}}_j\to\mathfrak{g}_1,
\,\,\,j\to\infty,\text{ $*$--слабо в $L_\infty([0,T],H)$.}
\end{gather}
Тогда $\mathfrak{g}_0\in\mathbb{W}([0,T];V,H)$, а её регулярная первая обобщённая производная $\mathfrak{g}_0'$ совпадает с $\mathfrak{g}_1$.
\end{Theorem}

\begin{Proof}
Пусть найдутся функции $\mathfrak{g}_0\in L_\infty([0,T],V)$ и $\mathfrak{g}_1\in L_\infty([0,T],H)$, такие, что выполнены соотношения (\ref{zjzj'coverg}). Поскольку
$\mathfrak{z}_j\in\mathbb{W}([0,T];V,H)$, $j=1,2,\dots$, то имеет место интегральное тождество
\begin{gather*}%\label{integral_identity_mathbbW_class}
(\textrm{Б})\int\limits_0^T\dot{\mathfrak{z}}_{j}(t)\varphi(t)dt=-(\textrm{Б})\int\limits_0^T\mathfrak{z}_{j}(t)\varphi'(t)dt\,\,\,\forall\,
\varphi\in\mathfrak{D}(0,T).
\end{gather*}
Отсюда следует, что при всех $h\in H$
\begin{gather*}
\left\langle(\textrm{Б})\int\limits_0^T\dot{\mathfrak{z}}_{j}(t)\varphi(t)dt,h\right\rangle_H=
\left\langle-(\textrm{Б})\int\limits_0^T\mathfrak{z}_{j}(t)\varphi'(t)dt,h\right\rangle_H\,\,\,\forall\,\varphi\in\mathfrak{D}(0,T);\\
\int\limits_0^T\left\langle\dot{\mathfrak{z}}_{j}(t)\varphi(t),h\right\rangle_Hdt=
-\int\limits_0^T\left\langle\mathfrak{z}_{j}(t)\varphi'(t),h\right\rangle_Hdt\,\,\,\forall\,\varphi\in\mathfrak{D}(0,T);\\
\int\limits_0^T\left\langle\dot{\mathfrak{z}}_{j}(t),\varphi(t)h\right\rangle_Hdt=
-\int\limits_0^T\left\langle\mathfrak{z}_{j}(t),\varphi'(t)h\right\rangle_Hdt\,\,\,\forall\,\varphi\in\mathfrak{D}(0,T).
\end{gather*}
Переходя здесь к пределу при $j\to\infty$ с учётом соотношений (\ref{zjzj'coverg}), будем иметь
\begin{gather*}
\int\limits_0^T\left\langle\mathfrak{g}_{1}(t),\varphi(t)h\right\rangle_Hdt=
-\int\limits_0^T\left\langle\mathfrak{g}_{0}(t),\varphi'(t)h\right\rangle_Hdt\,\,\,\forall\,\varphi\in\mathfrak{D}(0,T).
\end{gather*}
Поэтому при всех $h\in H$
\begin{gather*}
\left\langle(\textrm{Б})\int\limits_0^T\mathfrak{g}_{1}(t)\varphi(t)dt,h\right\rangle_H=
\left\langle-(\textrm{Б})\int\limits_0^T\mathfrak{g}_{0}(t)\varphi'(t)dt,h\right\rangle_H\,\,\,\forall\,\varphi\in\mathfrak{D}(0,T);\\
\left\langle(\textrm{Б})\int\limits_0^T\mathfrak{g}_{1}(t)\varphi(t)dt+
(\textrm{Б})\int\limits_0^T\mathfrak{g}_{0}(t)\varphi'(t)dt,h\right\rangle_H=0\,\,\,\forall\,\varphi\in\mathfrak{D}(0,T);
\end{gather*}
ввиду чего
\begin{gather*}
(\textrm{Б})\int\limits_0^T\mathfrak{g}_{1}(t)\varphi(t)dt+(\textrm{Б})\int\limits_0^T\mathfrak{g}_{0}(t)\varphi'(t)dt=0\,\,\,\forall\,\varphi\in\mathfrak{D}(0,T);
\end{gather*}
или, иначе,
\begin{gather*}
(\textrm{Б})\int\limits_0^T\mathfrak{g}_{1}(t)\varphi(t)dt=-(\textrm{Б})\int\limits_0^T\mathfrak{g}_{0}(t)\varphi'(t)dt\,\,\,\forall\,\varphi\in\mathfrak{D}(0,T);
\end{gather*}
Последнее и означает, что $\mathfrak{g}_0\in\mathbb{W}([0,T];V,H)$, а её регулярная первая обобщённая производная $\mathfrak{g}_0'$ совпадает с $\mathfrak{g}_1$. Теорема доказана.
\end{Proof}


\begin{Theorem}\label{sequential.compactness.in.mathbbW}
Любая функция $f\in \textrm{Э}([0,T];V,H)$ принадлежит пространству $\mathbb{W}([0,T];V,H)$, причём любое множество, ограниченное в норме пространства $\textrm{Э}([0,T];V,H)$, секвенциально
компактно в $\mathbb{W}([0,T];V,H)$.
\end{Theorem}
\begin{Proof}
Теоретико--множественное вложение $\textrm{Э}([0,T];V,H)\subset \mathbb{W}([0,T];V,H)$ вытекает из определения этих пространств.

Докажем, что любое множество, ограниченное в норме пространства $\textrm{Э}([0,T];V,H)$, секвенциально компактно в $\mathbb{W}([0,T];V,H)$. В самом деле, пусть последовательность
$\mathfrak{z}_j\in \textrm{Э}([0,T];V,H)$, $j=1,2,\dots$, ограничена в норме пространства $\textrm{Э}([0,T];V,H)$, т.е. найдётся константа $K>0$, такая, что для всех $j=1,2,\dots$
\begin{gather}\label{mathfrakzm:boundness}
\sup\limits_{t\in[0,T]}\|\mathfrak{z}_j(t)\|_V+\vraisup\limits_{t\in[0,T]}\|\dot{\mathfrak{z}}_j(t)\|_H\leqslant K.
\end{gather}
В силу леммы \ref{Cs[0,T]X:sup::vraisup} отсюда следует, что
\begin{gather}\label{mathfrakzm:boundness!1}
\vraisup\limits_{t\in[0,T]}\|\mathfrak{z}_j(t)\|_V+\vraisup\limits_{t\in[0,T]}\|\dot{\mathfrak{z}}_j(t)\|_H\leqslant K.
\end{gather}
Положим $Y\equiv L_1([0,T],V^*)\oplus L_1([0,T],H)$ и введём в этом пространстве норму равенством
\begin{gather*}
\|y\|_Y\equiv\max\{\|\mathfrak{m}\|_{1,[0,T],V^*},\,\|\mathfrak{n}\|_{1,[0,T],H}\}\,\,\,\forall\,y\equiv(\mathfrak{m},\mathfrak{n})\in Y.
\end{gather*}
В силу изометричных изоморфизмов (\ref{L1*Linfty_isometric_isomorphic}) пространство $Y^*$ изометрично изоморфно пространству
$Z\equiv L_\infty([0,T],V)\oplus L_\infty([0,T],H)$, наделённому нормой
\begin{gather*}
\|z\|_Z\equiv\|\mathfrak{p}\|_{\infty,[0,T],V}+\|\mathfrak{q}\|_{\infty,[0,T],H}\,\,\,\forall\,z\equiv(\mathfrak{p},\mathfrak{q})\in Z.
\end{gather*}
Ясно, что теперь (\ref{mathfrakzm:boundness!1}) можно переписать в виде
\begin{gather}\label{mathfrakzm:boundness!2}
\|(\mathfrak{z}_j,\dot{\mathfrak{z}}_j)\|_Z\leqslant K\,\,\,\forall\,j\geqslant1.
\end{gather}
Следовательно, на основании теорем Бишопа и Алаоглу, найдутся пара $(\tilde{\mathfrak{z}}_0,\tilde{\mathfrak{z}}_1)\in Z$ и
подпоследовательность $j_i$, $i=1,2,\dots$, последовательности $j=1,2,\dots$, такие, что
\begin{gather*}
(\mathfrak{z}_{j_i},\dot{\mathfrak{z}}_{j_i})\to(\tilde{\mathfrak{z}}_0,\tilde{\mathfrak{z}}_1),\,\,\,i\to\infty,\text{ $*$--слабо в $Z$};
\,\,\,\|(\tilde{\mathfrak{z}}_0,\tilde{\mathfrak{z}}_1)\|_Z\leqslant K.
\end{gather*}
Отсюда следует, что
\begin{gather*}
\mathfrak{z}_{j_i}\to\tilde{\mathfrak{z}}_0,\,\,\,i\to\infty,\text{ $*$--слабо в $L_\infty([0,T],V)$};\,\,\,
\dot{\mathfrak{z}}_{j_i}\to\tilde{\mathfrak{z}}_1,\,\,\,i\to\infty,\text{ $*$--слабо в $L_\infty([0,T],H)$;}\\
\|\tilde{\mathfrak{z}}_0\|_{\infty,[0,T],V}+\|\tilde{\mathfrak{z}}_1\|_{\infty,[0,T],H}\leqslant K.
\end{gather*}
Поэтому
\begin{gather*}
\lim\limits_{i\to\infty}\int\limits_0^T\langle\mathfrak{z}_{j_i}(t),\varphi(t)\rangle dt=
\int\limits_0^T\langle\tilde{\mathfrak{z}}_0(t),\varphi(t)\rangle dt\,\,\,\forall\,\varphi\in L_1([0,T],V^*);\\
\notag\lim\limits_{i\to\infty}\int\limits_0^T\langle\dot{\mathfrak{z}}_{j_i}(t),\varphi(t)\rangle_H dt=
\int\limits_0^T\langle\tilde{\mathfrak{z}}_1(t),\varphi(t)\rangle_H dt\,\,\,\forall\,\varphi\in L_1([0,T],H);
\end{gather*}
и, в частности,
\begin{gather}\label{*weak_convergence_of_mathfrak_zji_dot_mathfrak_z_ji}
\lim\limits_{i\to\infty}\int\limits_0^T\langle\mathfrak{z}_{j_i}(t),\varphi(t)\rangle_H dt=
\int\limits_0^T\langle\tilde{\mathfrak{z}}_0(t),\varphi(t)\rangle_H dt\,\,\,\forall\,\varphi\in L_1([0,T],H);\\
\notag\lim\limits_{i\to\infty}\int\limits_0^T\langle\dot{\mathfrak{z}}_{j_i}(t),\varphi(t)\rangle_H dt=
\int\limits_0^T\langle\tilde{\mathfrak{z}}_1(t),\varphi(t)\rangle_H dt\,\,\,\forall\,\varphi\in L_1([0,T],H);
\end{gather}
Поскольку $\mathfrak{z}_{j_i}\in \textrm{Э}([0,T];V,H)$, $i=1,2,\dots$, то $\mathfrak{z}_{j_i}\in W^1_\infty([0,T],H)$, $i=1,2,\dots$ Поэтому
справедливо интегральное тождество
\begin{gather*}
(\textrm{Б})\int\limits_0^T\mathfrak{z}_{j_i}(t)\psi'(t)dt=-(\textrm{Б})\int\limits_0^T\dot{\mathfrak{z}}_{j_i}(t)\psi(t)dt\,\,\,\forall\,\psi\in\mathfrak{D}(0,T),
\end{gather*}
из которого следует, что для всех $h\in H$
\begin{gather*}
\int\limits_0^T\langle\mathfrak{z}_{j_i}(t),\psi'(t)h\rangle_H dt=-\int\limits_0^T\langle\dot{\mathfrak{z}}_{j_i}(t),\psi(t)h\rangle_H dt\,\,\,\forall\,\psi\in\mathfrak{D}(0,T).
\end{gather*}
Переходя здесь на основании (\ref{*weak_convergence_of_mathfrak_zji_dot_mathfrak_z_ji}) к пределу при $i\to\infty$, получим, что
\begin{gather*}
\int\limits_0^T\langle\tilde{\mathfrak{z}}_0(t),\psi'(t)h\rangle_H dt=-\int\limits_0^T\langle\tilde{\mathfrak{z}}_1(t),\psi(t)h\rangle_H dt\,\,\,\forall\,\psi\in\mathfrak{D}(0,T).
\end{gather*}
Следовательно, для всех $h\in H$
\begin{gather*}
\left\langle(\textrm{Б})\int\limits_0^T\tilde{\mathfrak{z}}_0(t)\psi'(t)dt+(\textrm{Б})\int\limits_0^T\tilde{\mathfrak{z}}_1(t)\psi(t)dt,h\right\rangle_H=0\,\,\,\forall\,\psi\in
\mathfrak{D}(0,T).
\end{gather*}
Это означает, что
\begin{gather*}
(\textrm{Б})\int\limits_0^T\tilde{\mathfrak{z}}_0(t)\psi'(t)dt=-(\textrm{Б})\int\limits_0^T\tilde{\mathfrak{z}}_1(t)\psi(t)dt\,\,\,\forall\,\psi\in\mathfrak{D}(0,T),
\end{gather*}
откуда вытекает, что $\tilde{\mathfrak{z}}_0\in L_\infty([0,T],V)$, $\tilde{\mathfrak{z}}_1\in L_\infty([0,T],H)$, и $\tilde{\mathfrak{z}}_1$ является регулярной обобщённой производной
функции $\tilde{\mathfrak{z}}_0$. Отсюда и из непрерывности вложения $V\subset H$ следует, что $\tilde{\mathfrak{z}}_0\in W^1_\infty([0,T],H)$. Поэтому, согласно теореме
\ref{W1p([0,T],X)_embedding}, $\tilde{\mathfrak{z}}_0\in C([0,T],H)$. Таким образом, $\tilde{\mathfrak{z}}_0\in C_s([0,T],H)\cap L_\infty([0,T],V)$. Последнее на
основании леммы \ref{LionsMajenesXY} даёт включение $\tilde{\mathfrak{z}}_0\in C_s([0,T],V)$.

Итак, мы доказали, что $\tilde{\mathfrak{z}}_0\in \textrm{Э}([0,T];V,H)$, причём $\tilde{\mathfrak{z}}_0'=\tilde{\mathfrak{z}}_1$. Теорема полностью доказана.
\end{Proof}

%Следствием доказанных в данном разделе результатов является
Докажем теперь справедливость следующего результата.
\begin{Theorem}\label{V1120QTcompactness:abstract}
Пусть последовательность функций $\mathfrak{z}_j\in \textrm{Э}([0,T];V,H)$, $j=1,2,\dots$, --- ограничена в норме пространства $\textrm{Э}([0,T];V,H)$, т.е. найдётся постоянная $K>0$,
такая, что
\begin{gather}\label{V1120QTcompactness:abstract!m1}
\|\mathfrak{z}_j\|_{\textrm{Э}([0,T];V,H)}\leqslant K,\,\,\,j=1,2,\dots
\end{gather}
Тогда найдутся подпоследовательность $\mathfrak{z}_{j_i}$, $i=1,2,\dots$, последовательности $\mathfrak{z}_j$, $j=1,2,\dots$, и функция $\mathfrak{z}\in \textrm{Э}([0,T];V,H)$, такие, что
\begin{gather}\label{V1120QTcompactness:abstract!m2}
\mathfrak{z}_{j_i}\to\mathfrak{z},\,\,\,i\to\infty,\,\,\,\mbox{слабо в $\mathcal{W}^1_2([0,T];V,H)$},\,\,\,
\lim\limits_{i\to\infty}\max\limits_{t\in[0,T]}\|\mathfrak{z}_{j_i}(t)-\mathfrak{z}(t)\|_{H}=0;\\
\notag \mathfrak{z}_{j_i}\to\mathfrak{z},\,\,\,i\to\infty,\,\,\,\mbox{$*$--слабо в $L_\infty([0,T],V)$};\,\,\,
\dot{\mathfrak{z}}_{j_i}\to\dot{\mathfrak{z}},\,\,\,i\to\infty,\,\,\,\mbox{$*$--слабо в $L_\infty([0,T],H)$};\\
\notag\mathfrak{z}_{j_i}\to\mathfrak{z},\,\,\,i\to\infty,\,\,\,\mbox{в $V^*$--топологии пространства $C_s([0,T],V)$};
\end{gather}
причём
\begin{gather}\label{V1120QTcompactness:abstract!m3}
\|\mathfrak{z}\|_{\textrm{Э}([0,T];V,H)}\leqslant K.
\end{gather}
Кроме того, если $\mathbf{Z}$ --- банахово пространство, а $A\in\mathcal{L}(V,\mathbf{Z})$ --- некоторый компактный оператор, то
\begin{gather}\label{V1120QTcompactness:abstract!m2.1}
\lim\limits_{i\to\infty}\max\limits_{t\in[0,T]}\|A[\mathfrak{z}_{j_i}(t)]-A[\mathfrak{z}(t)]\|_{\mathbf{Z}}=0.
\end{gather}
\end{Theorem}
\begin{Proof} Разобьём доказательство на шесть шагов.

1) В силу теоремы \ref{sequential.compactness.in.mathbbW} и неравенства (\ref{V1120QTcompactness:abstract!m1}) найдутся подпоследовательность $\mathfrak{z}_{j_{i,1}}$, $i=1,2,\dots$,
последовательности $\mathfrak{z}_j$, $j=1,2,\dots$, и функция $\mathfrak{z}\in \textrm{Э}([0,T];V,H)$, такие, что
\begin{gather}\label{V1120QTcompactness:abstract!m4}
\mathfrak{z}_{j_{i,1}}\to\mathfrak{z},\,\,\,i\to\infty,\,\,\,\mbox{$*$--слабо в $L_\infty([0,T],V)$};\,\,\,
\dot{\mathfrak{z}}_{j_{i,1}}\to\dot{\mathfrak{z}},\,\,\,i\to\infty,\,\,\,\mbox{$*$--слабо в $L_\infty([0,T],H)$};
\end{gather}
причём выполнено неравенство (\ref{V1120QTcompactness:abstract!m3}).

%Далее, согласно лемме
2) Далее, для любого $h\in H$, $h\neq0$, и для любых $t'$, $t''\in[0,T]$
\begin{gather*}
|\langle h,\mathfrak{z}_{j_{i,1}}(t')-\mathfrak{z}_{j_{i,1}}(t'')\rangle_H|\leqslant\|h\|_H\|\mathfrak{z}_{j_{i,1}}(t')-\mathfrak{z}_{j_{i,1}}(t'')\|_H=
\|h\|_H\Biggl\|(\textrm{Б})\int\limits_{t''}^{t'}\dot{\mathfrak{z}}_{j_{i,1}}(t)dt\Biggr\|_H\leqslant\\
\leqslant\|h\|_H\Biggl|\int\limits_{t''}^{t'}\|\dot{\mathfrak{z}}_{j_{i,1}}(t)\|_Hdt\Biggr|\leqslant K|t'-t''|\|h\|_H.
\end{gather*}
Это значит, что при любом $h\in H$ семейство функций
\begin{gather*}
[0,T]\ni t\mapsto\langle h,\mathfrak{z}_{j_{i,1}}(t)\rangle_H,\,\,\,i=1,2,\dots,
\end{gather*}
равностепенно непрерывно. Поскольку же элемент $h\in H$ можно отождествить с линейным непрерывным на $V$ функционалом, действующим по правилу
\begin{gather*}
\langle h,v\rangle\equiv\langle h,v\rangle_H,\,\,\,v\in V,
\end{gather*}
то мы доказали, что семейство функций
\begin{gather}\label{V1120QTcompactness:abstract!m5}
[0,T]\ni t\mapsto\langle v^*,\mathfrak{z}_{j_{i,1}}(t)\rangle,\,\,\,i=1,2,\dots,
\end{gather}
равностепенно непрерывно при каждом фиксированном $v^*\in H$. Докажем теперь равностепенную непрерывность этого семейства для всех $v^*\in V^*$. Выберем произвольно $v^*\in V$ и
зафиксируем. Пусть $\varepsilon>0$ --- произвольное число. В силу плотности $H$ в $V^*$ найдётся элемент $h_\varepsilon\in H$, такой, что
$$
\|h_\varepsilon-v^*\|_{V^*}\leqslant\frac\varepsilon{4K}.
$$
Тогда для всех  $t'$, $t''\in[0,T]$
\begin{gather*}
|\langle v^*,\mathfrak{z}_{j_{i,1}}(t')-\mathfrak{z}_{j_{i,1}}(t'')\rangle|\leqslant|\langle v^*-h_\varepsilon,\mathfrak{z}_{j_{i,1}}(t')-\mathfrak{z}_{j_{i,1}}(t'')\rangle|+
|\langle h_\varepsilon,\mathfrak{z}_{j_{i,1}}(t')-\mathfrak{z}_{j_{i,1}}(t'')\rangle|\leqslant\frac\varepsilon{4K}2K+\\
+|\langle h_\varepsilon,\mathfrak{z}_{j_{i,1}}(t')-\mathfrak{z}_{j_{i,1}}(t'')\rangle|.
\end{gather*}
Таким образом,
\begin{gather*}
|\langle v^*,\mathfrak{z}_{j_{i,1}}(t')-\mathfrak{z}_{j_{i,1}}(t'')\rangle|\leqslant\frac\varepsilon2+|\langle h_\varepsilon,\mathfrak{z}_{j_{i,1}}(t')-\mathfrak{z}_{j_{i,1}}(t'')\rangle|.
\end{gather*}

В силу доказанной выше равностепенной непрерывности семейства (\ref{V1120QTcompactness:abstract!m5}) для всех фиксированных $v^*\in H$, найдётся $\delta=\delta(\varepsilon)>0$, не зависящее
от выбора номера $i\geqslant1$, такое, что для всех  $t'$, $t''\in[0,T]$, для которых $|t'-t''|\leqslant\delta$, выполнено неравенство
\begin{gather*}
|\langle h_\varepsilon,\mathfrak{z}_{j_{i,1}}(t')-\mathfrak{z}_{j_{i,1}}(t'')\rangle|\leqslant\frac\varepsilon2.
\end{gather*}
Следовательно, для любого $\varepsilon>0$ найдётся $\delta=\delta(\varepsilon)>0$, не зависящее от выбора номера $i\geqslant1$, такое, что
для всех  $t'$, $t''\in[0,T]$, для которых $|t'-t''|\leqslant\delta$, справедливо соотношение
\begin{gather*}
|\langle v^*,\mathfrak{z}_{j_{i,1}}(t')-\mathfrak{z}_{j_{i,1}}(t'')\rangle|\leqslant\varepsilon.
\end{gather*}
Иными словами, при любом фиксированном $v^*\in V^*$ семейство (\ref{V1120QTcompactness:abstract!m5}) --- равностепенно непрерывно.

В силу равностепенной непрерывности семейства (\ref{V1120QTcompactness:abstract!m5}) при всех фиксированных $v^*\in V^*$ и леммы \ref{Cs([0,T],X).podposledovatelnost'} найдутся
подпоследовательность $\mathfrak{z}_{j_{i,2}}$, $i=1,2,\dots$, последовательности $\mathfrak{z}_{j_{i,1}}$, $i=1,2,\dots$, и функция $\hat{\mathfrak{z}}\in C_s([0,T],V)$, такие, что
\begin{gather}\label{V1120QTcompactness:abstract!m6}
\mathfrak{z}_{j_{i,2}}\to\hat{\mathfrak{z}},\,\,\,i\to\infty,\,\,\,\mbox{в $V^*$--топологии пространства $C_s([0,T],V)$}.
\end{gather}
Покажем, что $\hat{\mathfrak{z}}=\mathfrak{z}$. В самом деле, из (\ref{V1120QTcompactness:abstract!m6}) следует, что при любых фиксированных $t\in[0,T]$ и $v^*\in V^*$ справедливо
предельное соотношение
\begin{gather*}
\lim\limits_{i\to\infty}\langle v^*,\mathfrak{z}_{j_{i,2}}(t)\rangle=\langle v^*,\hat{\mathfrak{z}}(t)\rangle.
\end{gather*}
Отсюда вытекает, что для всех фиксированных $g\in L_1([0,T],V^*)$ и при п.в. $t\in[0,T]$
\begin{gather*}
\lim\limits_{i\to\infty}\langle g(t),\mathfrak{z}_{j_{i,2}}(t)\rangle=\langle g(t),\hat{\mathfrak{z}}(t)\rangle.
\end{gather*}
Поскольку, кроме того, в силу неравенства (\ref{V1120QTcompactness:abstract!m3}), при п.в. $t\in[0,T]$
\begin{gather*}
|\langle g(t),\mathfrak{z}_{j_{i,2}}(t)\rangle|\leqslant\|g(t)\|_{V^*}\|\mathfrak{z}_{j_{i,2}}(t)\|_V\leqslant K\|g(t)\|_{V^*}
\end{gather*}
Поэтому, в силу теоремы Лебега о предельном переходе под знаком интеграла Лебега,
\begin{gather*}
\lim\limits_{i\to\infty}\int\limits_0^T\langle g(t),\mathfrak{z}_{j_{i,2}}(t)\rangle dt=\int\limits_0^T\langle g(t),\hat{\mathfrak{z}}(t)
\rangle dt.
\end{gather*}
Иными словами, последовательность $\mathfrak{z}_{j_{i,2}}$, $i=1,2,\dots$, $*$--слабо в $L_1([0,T],V)$ сходится к функции $\hat{\mathfrak{z}}\in C_s([0,T],V)$. Ввиду
(\ref{V1120QTcompactness:abstract!m4}) это означает, что $\hat{\mathfrak{z}}=\mathfrak{z}$.

3) Далее, поскольку последовательность  функций $\mathfrak{z}_j\in \textrm{Э}([0,T];V,H)$, $j=1,2,\dots$, --- ограничена в норме пространства $\textrm{Э}([0,T];V,H)$, то она ограничена и в
норме пространства $\mathcal{W}^1_2([0,T];V,H)$. Поскольку же последнее пространство гильбертово, то найдутся подпоследовательность $\mathfrak{z}_{j_{i,3}}$, $i=1,2,\dots$,
последовательности $\mathfrak{z}_{j_{i,2}}$, $i=1,2,\dots$, и функция $\tilde{\mathfrak{z}}\in\mathcal{W}^1_2([0,T];V,H)$, такие, что
\begin{gather}\label{V1120QTcompactness:abstract!m7}
\mathfrak{z}_{j_{i,3}}\to\tilde{\mathfrak{z}},\,\,\,i\to\infty,\,\,\,\mbox{слабо в $\mathcal{W}^1_2([0,T];V,H)$}.
\end{gather}
Из данного соотношения вытекает, что
\begin{gather*}
\lim\limits_{i\to\infty}\int\limits_0^T\langle g(t),\mathfrak{z}_{j_{i,3}}(t)\rangle_Vdt=\int\limits_0^T\langle g(t),
\tilde{\mathfrak{z}}(t)\rangle_Vdt\,\,\,\forall\,g\in L_2([0,T],V).
\end{gather*}
С другой стороны, в силу (\ref{V1120QTcompactness:abstract!m4}),
\begin{gather*}
\lim\limits_{i\to\infty}\int\limits_0^T\langle g(t),\mathfrak{z}_{j_{i,3}}(t)\rangle_Vdt=\int\limits_0^T\langle g(t),\mathfrak{z}(t)\rangle_Vdt\,\,\,\forall\,g\in L_2([0,T],V).
\end{gather*}
Отсюда следует, что $\tilde{\mathfrak{z}}=\mathfrak{z}$.

4) В силу теоремы \ref{abstact_energetic_class_embedding:Theorem} и неравенства (\ref{V1120QTcompactness:abstract!m1}) найдутся подпоследовательность $\mathfrak{z}_{j_{i,4}}$,
$i=1,2,\dots$, последовательности $\mathfrak{z}_{j_{i,3}}$, $i=1,2,\dots$, и функция $\check{\mathfrak{z}}\in C([0,T],H)$, такие, что
\begin{gather}\label{V1120QTcompactness:abstract!m8}
\lim\limits_{i\to\infty}\max\limits_{t\in[0,T]}\|\mathfrak{z}_{j_{i,4}}(t)-\check{\mathfrak{z}}(t)\|_{H}=0.
\end{gather}
Покажем, что $\check{\mathfrak{z}}=\mathfrak{z}$. В самом деле, из (\ref{V1120QTcompactness:abstract!m4}) следует, что
\begin{gather}\label{V1120QTcompactness:abstract!m9}
\lim\limits_{i\to\infty}\int\limits_0^T\langle g(t),\mathfrak{z}_{j_{i,4}}(t)\rangle_Hdt=\int\limits_0^T\langle g(t),\mathfrak{z}(t)\rangle_Hdt\,\,\,\forall\,g\in L_1([0,T],H).
\end{gather}
С другой стороны, из (\ref{V1120QTcompactness:abstract!m8}) следует, что
\begin{gather*}
\left|\int\limits_0^T\langle g(t),\mathfrak{z}_{j_{i,4}}(t)\rangle_Hdt-\int\limits_0^T\langle g(t),\check{\mathfrak{z}}(t)\rangle_Hdt\right|=\left|\int\limits_0^T\langle g(t),
\mathfrak{z}_{j_{i,4}}(t)-\check{\mathfrak{z}}(t)\rangle_Hdt\right|\leqslant\int\limits_0^T|\langle g(t),\mathfrak{z}_{j_{i,4}}(t)-\check{\mathfrak{z}}(t)\rangle_H|dt\leqslant\\
\leqslant\int\limits_0^T\|g(t)\|_H\|\mathfrak{z}_{j_{i,4}}(t)-\check{\mathfrak{z}}(t)\|_Hdt\leqslant
\max\limits_{t\in[0,T]}\|\mathfrak{z}_{j_{i,4}}(t)-\check{\mathfrak{z}}(t)\|_{H}\int\limits_0^T\|g(t)\|_Hdt\to0,\,\,\,i\to\infty.
\end{gather*}
Отсюда и из (\ref{V1120QTcompactness:abstract!m9}) вытекает, что $\check{\mathfrak{z}}=\mathfrak{z}$.

5) Собирая вместе соотношения (\ref{V1120QTcompactness:abstract!m4}) (\ref{V1120QTcompactness:abstract!m6}), (\ref{V1120QTcompactness:abstract!m7}), (\ref{V1120QTcompactness:abstract!m8}),
и полагая $\mathfrak{z}_{j_i}\equiv\mathfrak{z}_{j_{i,4}}$, $i=1,2,\dots$, получаем соотношения (\ref{V1120QTcompactness:abstract!m2}) и (\ref{V1120QTcompactness:abstract!m3}).

6) Докажем теперь соотношение (\ref{V1120QTcompactness:abstract!m2.1}). В самом деле, поскольку
\begin{gather*}
\mathfrak{z}_{j_i}\to\mathfrak{z},\,\,\,i\to\infty,\,\,\,\mbox{в $V^*$--топологии пространства $C_s([0,T],V)$},
\end{gather*}
то
\begin{gather}\label{V1120QTcompactness:abstract!m10}
\forall\,v^*\in V^*:\lim\limits_{i\to\infty}\max\limits_{t\in[0,T]}|\langle v^*,\mathfrak{z}_{j_i}(t)-\mathfrak{z}(t)\rangle|=0.
\end{gather}
Отсюда вытекает, что при каждом фиксированном $t\in[0,T]$
\begin{gather}\label{V1120QTcompactness:abstract!m11}
\mathfrak{z}_{j_i}(t)\to\mathfrak{z}(t),\,\,\,i\to\infty,\,\,\,\mbox{слабо в $V$},
\end{gather}
что, в силу компактности оператора $A$, даёт соотношение
\begin{gather}\label{V1120QTcompactness:abstract!m12}
\lim\limits_{i\to\infty}\|A[\mathfrak{z}_{j_i}(t)]-A[\mathfrak{z}(t)]\|_{\mathbf{Z}}=0\,\,\,\forall\,t\in[0,T].
\end{gather}
Предположим, однако, что соотношение (\ref{V1120QTcompactness:abstract!m2.1}) не выполнено. Тогда найдутся положительное число $\varepsilon_0$, подпоследовательность $i_k$, $k=1,2,\dots$,
последовательности $i=1,2,\dots$, и последовательность чисел $t_k\in[0,T]$, $k=1,2,\dots$, такие, что
\begin{gather}\label{V1120QTcompactness:abstract!m13}
\|A[\mathfrak{z}_{j_{i_k}}(t_k)]-A[\mathfrak{z}(t_k)]\|_{\mathbf{Z}}\geqslant\varepsilon_0,\,\,\,k=1,2,\dots
\end{gather}
В силу компактности отрезка $[0,T]$ найдутся подпоследовательность последовательности $t_k\in[0,T]$, $k=1,2,\dots$, которую мы обозначим так же, как и саму последовательность $t_k\in[0,T]$,
$k=1,2,\dots$, и точка $t^*\in[0,T]$, такие, что $t_k\to t^*$, $k\to\infty$. Выберем теперь произвольно $v^*\in V^*$ и зафиксируем. Тогда
\begin{gather}\label{V1120QTcompactness:abstract!m13?}
|\langle v^*,\mathfrak{z}_{j_{i_k}}(t_k)-\mathfrak{z}(t_k)\rangle|\leqslant|\langle v^*,\mathfrak{z}_{j_{i_k}}(t_k)-\mathfrak{z}_{j_{i_k}}(t^*)\rangle|+
|\langle v^*,\mathfrak{z}_{j_{i_k}}(t^*)-\mathfrak{z}(t^*)\rangle|+|\langle v^*,\mathfrak{z}(t^*)-\mathfrak{z}(t_k)\rangle|.
\end{gather}
Из включения $\mathfrak{z}\in C_s([0,T],V)$ следует, что функция $\mathfrak{z}$ слабо непрерывна в точке $t^*$. Поэтому
\begin{gather}\label{V1120QTcompactness:abstract!m14}
\forall\,\varepsilon>0\,\,\exists\,\delta_1=\delta_1(\varepsilon)>0\,\,\forall\,t\in[0,T],\,\,|t-t^*|\leqslant\delta_1:|\langle v^*,\mathfrak{z}(t)-\mathfrak{z}(t^*)\rangle|\leqslant
\frac\varepsilon3.
\end{gather}
Далее, в силу доказанной выше равностепенной непрерывности семейства функций
\begin{gather*}
[0,T]\ni t\mapsto\langle v^*,\mathfrak{z}_{j_{i_k}}(t_k)\rangle,\,\,\,k=1,2,\dots,
\end{gather*}
справедливо соотношение
\begin{gather}\label{V1120QTcompactness:abstract!m15}
\forall\,\varepsilon>0\,\,\exists\,\,\delta_2=\delta_2(\varepsilon)>0\,\,\forall\,t',\,\,\,t''\in[0,T],\,\,\,|t'-t''|\leqslant\delta_2:
|\langle v^*,\mathfrak{z}_{j_{i_k}}(t')-\mathfrak{z}_{j_{i_k}}(t'')\rangle|\leqslant\frac\varepsilon3.
\end{gather}
Кроме того, в силу сходимости $t_k\to t^*$, $k\to\infty$,
\begin{gather}\label{V1120QTcompactness:abstract!m16}
\forall\,\sigma>0\,\,\exists\,k_0=k_0(\sigma)\geqslant1\,\,\forall\,k\geqslant k_0(\sigma):|t_k-t^*|\leqslant\sigma.
\end{gather}
Произвольно зафиксируем $\varepsilon>0$ и подберём по нему $\delta_1=\delta_1(\varepsilon)>0$ согласно (\ref{V1120QTcompactness:abstract!m14}) и $\delta_2=\delta_2(\varepsilon)>0$ согласно
(\ref{V1120QTcompactness:abstract!m15}). Положив $\delta^*(\varepsilon)\equiv\min\{\delta_1(\varepsilon),\delta_2(\varepsilon)\}$, будем иметь
\begin{gather}\label{V1120QTcompactness:abstract!m17}
\forall\,t\in[0,T],\,\,|t-t^*|\leqslant\delta^*(\varepsilon):|\langle v^*,\mathfrak{z}(t)-\mathfrak{z}(t^*)\rangle|\leqslant\frac\varepsilon3;\\
\notag\forall\,t',\,\,\,t''\in[0,T],\,\,\,|t'-t''|\leqslant\delta^*(\varepsilon):|\langle v^*,\mathfrak{z}_{j_{i_k}}(t')-\mathfrak{z}_{j_{i_k}}(t'')\rangle|\leqslant\frac\varepsilon3.
\end{gather}
Подберём теперь по $\delta^*(\varepsilon)$ номер $k_0=k_0(\delta^*(\varepsilon))$ согласно (\ref{V1120QTcompactness:abstract!m16}). Тогда, в силу (\ref{V1120QTcompactness:abstract!m16}),
при всех $k\geqslant k_0$
\begin{gather*}
|t_k-t^*|\leqslant\delta^*(\varepsilon).
\end{gather*}
Следовательно, ввиду (\ref{V1120QTcompactness:abstract!m17}),
\begin{gather*}
|\langle v^*,\mathfrak{z}(t^*)-\mathfrak{z}(t_k)\rangle|\leqslant\frac\varepsilon3,\,\,\,|\langle v^*,\mathfrak{z}_{j_{i_k}}(t_k)-
\mathfrak{z}_{j_{i_k}}(t^*)\rangle|\leqslant\frac\varepsilon3\,\,\,\forall\,k\geqslant\tilde{k}_0(\varepsilon),
\end{gather*}
где $\tilde{k}_0(\varepsilon)\equiv k_0(\delta^*(\varepsilon))$. Поэтому из (\ref{V1120QTcompactness:abstract!m13?}) вытекает, что
\begin{gather*}
|\langle v^*,\mathfrak{z}_{j_{i_k}}(t_k)-\mathfrak{z}(t_k)\rangle|\leqslant\frac\varepsilon3+
|\langle v^*,\mathfrak{z}_{j_{i_k}}(t^*)-\mathfrak{z}(t^*)\rangle|+\frac\varepsilon3\,\,\,\forall\,k\geqslant\tilde{k}_0(\varepsilon).
\end{gather*}
Наконец, из (\ref{V1120QTcompactness:abstract!m11}) вытекает, что найдётся номер $k_1=k_1(\varepsilon)\geqslant1$, такой, что
\begin{gather*}
|\langle v^*,\mathfrak{z}_{j_{i_k}}(t^*)-\mathfrak{z}(t^*)\rangle|\leqslant\frac\varepsilon3\,\,\,\forall\,k\geqslant k_1(\varepsilon).
\end{gather*}
Взяв $k^*(\varepsilon)\equiv\max\{\tilde{k}_0(\varepsilon),\,k_1(\varepsilon)\}$, получим, что
\begin{gather*}
|\langle v^*,\mathfrak{z}_{j_{i_k}}(t_k)-\mathfrak{z}(t_k)\rangle|\leqslant\frac\varepsilon3+\frac\varepsilon3+\frac\varepsilon3=\varepsilon\,\,\,\forall\,k\geqslant {k}^*(\varepsilon).
\end{gather*}
Итак, для любого $\varepsilon>0$ найдётся номер $k^*(\varepsilon)$, такой, что для всех $k\geqslant {k}^*(\varepsilon)$ выполнено неравенство
\begin{gather*}
|\langle v^*,\mathfrak{z}_{j_{i_k}}(t_k)-\mathfrak{z}(t_k)\rangle|\leqslant\varepsilon.
\end{gather*}
В силу произвольности выбора $v^*\in V^*$ это означает, что
\begin{gather*}
\mathfrak{z}_{j_{i_k}}(t_k)-\mathfrak{z}(t_k)\to0,\,\,\,k\to\infty,\,\,\,\mbox{слабо в $V$},
\end{gather*}
что, ввиду компактности оператора $A$, даёт предельное соотношение
\begin{gather*}
\lim\limits_{k\to\infty}\|A[\mathfrak{z}_{j_{i_k}}(t_k)]-A[\mathfrak{z}(t_k)]\|_{\mathbf{Z}}=0,
\end{gather*}
противоречащее неравенству (\ref{V1120QTcompactness:abstract!m13}). Таким образом, соотношение (\ref{V1120QTcompactness:abstract!m2.1}) доказано.

Теорема полностью доказана.
\end{Proof}

Пусть $e_k\in V$, $k=1,2,\dots$, --- ортогональная в $V$ и ортонормированная в $H$ система, такая, что для любых $\varphi\in V$ и $\psi\in H$ справедливо равенство
\begin{gather*}
\lim\limits_{N\to\infty}\|\varphi^N-\varphi\|_V=0,\,\,\,\lim\limits_{N\to\infty}\|\psi^N-\psi\|_H=0,
\end{gather*}
где
\begin{gather*}
\varphi^N\equiv\sum\limits_{m=1}^N\varphi_me_m,\,\,\,\psi^N\equiv\sum\limits_{m=1}^N\psi_me_m, \,\,\,
\varphi_k\equiv\langle\varphi,e_k\rangle_H,\,\,\, \psi_k\equiv\langle\psi,e_k\rangle_H,\,\,\,k,\,\,N=1,2,\dots
\end{gather*}

\begin{Lemma}\label{approx:abstract} Пусть
${\mathfrak{M}}^N\equiv\{\sum\limits_{j=1}^N\zeta_je_j\,:\,\zeta_j\in W^1_2[0,T],\,\,\,\zeta_j(T)=0,\,\,\,j=\overline{1,\,N}\}$, ${\mathfrak{M}}\equiv \bigcup\limits_{N=1}^\infty
{\mathfrak{M}}^N$. Тогда ${\mathfrak{M}}$ плотно в $\hat{\textbf{Э}}_1([0,T];V,H)\equiv \{{\sf z}\in{\textbf{Э}}_1([0,T];V,H)\,:\,{\sf z}(T)=0\}$.
\end{Lemma}

\begin{Proof} Покажем, что ${\mathfrak{M}}$ плотно в $\hat{\textbf{Э}}_1([0,T];V,H)$.

В самом деле, пусть ${\sf z}\in \hat{\textbf{Э}}_1([0,T];V,H)$ --- произвольна. Тогда, в силу свойств последовательности $e_j\in V$, $j=1,2,\dots$,
$$
\lim\limits_{N\to\infty}\left\|\sum_{j=1}^N{\sf z}_j(t)e_j-{\sf z}(t)\right\|^2_V=0,\,\,\,
\lim\limits_{N\to\infty}\left\|\sum_{j=1}^N\dot{\sf z}_j(t)e_j-\dot{\sf z}(t)\right\|^2_H=0,\,\,\, \forall\,t\in[0,T],
$$
где ${\sf z}_j(t)\equiv\langle{\sf z}(t),\,e_j\rangle_H$, $j=1,2,\dots$.

Полагая ${\sf z}^N(t)\equiv\sum\limits_{j=1}^N{\sf z}_j(t)e_j$, $r_{N,\,0}(t)\equiv \|{\sf z}^N(t)-{\sf z}(t)\|^2_V$,
$r_{N,\,1}(t)\equiv \|\dot{\sf z}^N(t)-\dot{\sf z}(t)\|^2_H$, $t\in[0,T]$, запишем последние соотношения в виде
\begin{equation}\label{restconv}
\lim\limits_{N\to\infty}r_{N,\,0}(t)=0,\,\,\, \lim\limits_{N\to\infty}r_{N,\,1}(t)=0,\,\,\, \forall\,t\in[t_0,\,t_1].
\end{equation}
Кроме того, как нетрудно видеть,
\begin{equation}\label{monrest}
r_{N,\,0}(t)\geqslant r_{N+1,\,0}(t),\,\,\, r_{N,\,1}(t)\geqslant r_{N+1,\,1}(t),\,\,\, \forall\,t\in[0,T],\,\,\,N=1,2,\dots.
\end{equation}

В силу непрерывности $r_{N,\,0}$, $r_{N,\,1}$, на отрезке $[0,T]$, соотношений~(\ref{restconv}), (\ref{monrest}), и леммы
\ref{Dini1}, получаем, что
 $$
\lim\limits_{N\to\infty}\max\limits_{t\in[0,T]}r_{N,\,0}(t)=0,\,\,\, \lim\limits_{N\to\infty}\max\limits_{t\in[0,T]}r_{N,\,1}(t)=0,
 $$
 откуда вытекает, что
 $$
\lim\limits_{N\to\infty}\|{\sf z}^N-{\sf z}\|_{{\textbf{Э}}_1([0,T];V,H)}=0.
 $$

Итак, ${\mathfrak{M}}$ плотно в $\hat{\textbf{Э}}_1([0,T];V,H)$. Лемма доказана.
\end{Proof}


\begin{Lemma}\label{approx:w.abstract} Множество ${\mathfrak{M}}\equiv \bigcup\limits_{N=1}^\infty\mathfrak{M}^N$  плотно в 
$\hat{\mathcal{W}}{}^1_2([0,T];V,H)\equiv\{{\sf z}\in{\mathcal{W}}{}^1_2([0,T];V,H):{\sf z}(T)=0\}$.
\end{Lemma}
\begin{Proof}
Достаточно заметить, что последовательность функций 
$$
{\sf z}^N(t)\equiv\sum\limits_{k=1}^N{\sf z}_k(t)e_k,\,\,\,N=1,2,\dots,
$$ 
где ${\sf z}_k(t)\equiv\langle{\sf z}(t),e_k\rangle_H$, $k=\overline{1,N}$, такова, что 
\begin{gather*}
{\sf z}^N\in\mathfrak{M}^N,\,\,\,N=1,2,\dots,\,\,\,\|{\sf z}^N-{\sf z}\|_{\mathcal{W}{}^1_2([0,T];V,H)}\to0,\,\,\,N\to\infty.
\end{gather*}
\end{Proof}

        \section{Следствия}
Пусть $X$ --- компактное топологическое пространство, $\mu$ --- положительная мера Радона на нём. Через $L_\infty(X,\mu)$ будем обозначать банахово пространство $\mu$--существенно
ограниченных на $X$ функций $\varphi\colon X\to\mathbb{R}$, с нормой
\begin{gather*}
\|\varphi\|_{L_\infty(X,\mu)}\equiv\mu\text{--}\vraisup_{x\in X}|\varphi(x)|.
\end{gather*}
Через $L_{\infty,1}(X\times[0,T],\mu)$ обозначим множество функций $\varphi\colon X\times[0,T]\to\mathbb{R}$, измеримых относительно произведения мер $\mu\otimes\lambda$, где
$\lambda$ --- мера Лебега на отрезке $[0,T]$, и таких, что конечна норма
\begin{gather*}
\|\varphi\|_{L_{\infty,1}(X\times[0,T],\mu)}\equiv\int\limits_0^T\mu\text{--}\vraisup_{x\in X}|\varphi(x,t)|dt.
\end{gather*}
Наконец, через $W^{0,1}_{\infty,1}(X\times[0,T],\mu)$ обозначим множество функций $\varphi\in L_{\infty,1}(X\times[0,T],\mu)$, для которых
$\varphi_t\in L_{\infty,1}(X\times[0,T],\mu)$. Норму в классе $W^{0,1}_{\infty,1}(X\times[0,T],\mu)$ определим равенством
\begin{gather*}
\|\varphi\|_{W^{0,1}_{\infty,1}(X\times[0,T],\mu)}\equiv \|\varphi\|_{L_{\infty,1}(X\times[0,T],\mu)}+\|\varphi_t\|_{L_{\infty,1}(X\times[0,T],\mu)},
\end{gather*}
превратив тем самым класс $W^{0,1}_{\infty,1}(X\times[0,T],\mu)$ в банахово пространство.

Покажем, что справедлива
\begin{Theorem}\label{W:01:infty1::abstr::embedding}
У каждой функции $f\in W^{0,1}_{\infty,1}(X\times[0,T],\mu)$ при всех $t\in[0,T]$ существует след $f(\cdot,t)\in L_\infty(X,\mu)$, непрерывно зависящий от $t\in[0,T]$ в норме
$L_\infty(X,\mu)$, причём
\begin{gather}\label{W:01:infty1::abstr::embedding:::estimate}
    \max\limits_{t\in[0,T]}\|f(\cdot,t)\|_{L_\infty(X,\mu)}\leqslant A_1\|f\|_{W^{0,1}_{\infty,1}(X\times[0,T],\mu)}.
\end{gather}
\end{Theorem}
\begin{Proof} Выберем произвольно $f\in W^{0,1}_{\infty,1}(X\times[0,T],\mu)$ и зафиксируем. Для любой функции $\varphi\in C(X\times[0,T])$, равной нулю вблизи $X\times\{0\}$ и
$X\times\{T\}$ и имеющей непрерывную на $X\times[0,T]$ производную $\varphi_t$, справедливо тождество
\begin{gather*}
\int\limits_{X\times[0,1]}f(x,t)\varphi_t(x,t)(\mu\otimes\lambda)(dxdt)=-\int\limits_{X\times[0,1]}f_t(x,t)\varphi(x,t)(\mu\otimes\lambda)(dxdt).
\end{gather*}
Полагая $\varphi(x,t)\equiv p(x)q(t)$, где $p\in C(X)$, а $q\in C^\infty[0,T]$ --- финитная на отрезке $[0,T]$ функция, получим, что для любых таких $p$ и $q$
\begin{gather*}
\int\limits_X\left[\int\limits_0^Tf(x,t)q'(t)dt+\int\limits_0^Tf_t(x,t)q(t)dt\right]p(x)\mu(dx)=0.
\end{gather*}
Как следствие, какова бы ни была функция $q$ из указанного класса, при $\mu$--п.в. $x\in X$ имеет место соотношение
\begin{gather*}
\int\limits_0^Tf(x,t)q'(t)dt=-\int\limits_0^Tf_t(x,t)q(t)dt.
\end{gather*}
В силу этого при $\mu$--почти всех $x\in X$ функция $f(x,\cdot)$  --- элемент $W^1_1[0,T]$, и, в частности, имеет смысл говорить о следе $f(\cdot,t)$.

Покажем, что $f\in C([0,T],L_\infty(X,\mu))$. Действительно,
\begin{gather*}
|f(x,t+\Delta t)-f(x,t)|\leqslant\left|\int\limits_t^{t+\Delta t}|f_t(x,\tau)|d\tau\right|\leqslant\left|\int\limits_t^{t+\Delta t}\|f_t(\cdot,\tau)\|_{L_\infty(X,\mu)}d\tau\right|,
\end{gather*}
откуда
\begin{gather*}
\|f(\cdot,t+\Delta t)-f(\cdot,t)\|_{L_\infty(X,\mu)}\leqslant\left|\int\limits_t^{t+\Delta t}\|f_t(\cdot,\tau)\|_{L_\infty(X,\mu)}d\tau\right|.
\end{gather*}

Ввиду этого при всех $t\in[0,T]$ существует след $f(\cdot,t)\in L_\infty(X,\mu)$, непрерывно зависящий от $t\in[0,T]$ в норме $L_\infty(X,\mu)$.

Докажем теперь оценку (\ref{W:01:infty1::abstr::embedding:::estimate}). Так как функция $f(x,\cdot)$ при $\mu$--п.в. $x\in X$ принадлежит пространству $W^1_1[0,T]$, то, в соответствии с
теоремой \ref{W11:embedding}, для каждого  $t\in[0,T]$ при $\mu$--почти всех $x\in X$
\begin{gather*}
|f(x,t)|\leqslant A_1\int\limits_0^T[|f(x,\tau)|+|f_t(x,\tau)|]d\tau.
\end{gather*}
Поэтому $|f(x,t)|\leqslant A_1\|f\|_{W^{0,1}_{\infty,1}(X\times[0,T],\mu)}$.

Таким образом, $\|f(\cdot,t)\|_{L_\infty(X,\mu)}\leqslant A_1\|f\|_{W^{0,1}_{\infty,1}(X\times[0,T],\mu)}$, что совместно с доказанным ранее включением $W^{0,1}_{\infty,1}(X\times[0,T],\mu)
\subset C([0,T],L_\infty(X,\mu))$ даёт оценку (\ref{W:01:infty1::abstr::embedding:::estimate}). Теорема~\ref{W:01:infty1::abstr::embedding} полностью доказана.
\end{Proof}

Из теоремы~\ref{W:01:infty1::abstr::embedding} вытекают следующие три теоремы.

\begin{Theorem}\label{W:01:infty1:QT::embedding}
У каждой функции $f\in W^{0,1}_{\infty,1}(Q_T)$ при всех $t\in[0,T]$ существует след $f(\cdot,t)\in L_\infty(\Omega)$, непрерывно зависящий от $t\in[0,T]$ в норме $L_\infty(\Omega)$, причём
\begin{gather}\label{W:01:infty1:QT::embedding:::estimate}
    \max\limits_{t\in[0,T]}\|f(\cdot,t)\|_{\infty,\Omega}\leqslant A_1\|f\|^{(0,1)}_{\infty,1,Q_T}.
\end{gather}
\end{Theorem}

\begin{Theorem}\label{W:01:infty1:ST::embedding}
У любой функции $f\in W^{0,1}_{\infty,1}(S_T)$ при каждом $t\in[0,T]$ существует след $f(\cdot,t)\in L_\infty(S)$, непрерывно зависящий от $t\in[0,T]$ в норме $L_\infty(S)$, причём
\begin{gather}\label{W:01:infty1:ST::embedding:::estimate}
    \max\limits_{t\in[0,T]}\|f(\cdot,t)\|_{\infty,S}\leqslant A_1\|f\|^{(0,1)}_{\infty,1,S_T}.
\end{gather}
\end{Theorem}

\begin{Theorem}\label{W:01:infty1:S'T::embedding}
У любой функции $f\in W^{0,1}_{\infty,1}(S'_T)$ при каждом $t\in[0,T]$ существует след $f(\cdot,t)\in L_\infty(S')$, непрерывно зависящий от $t\in[0,T]$ в норме $L_\infty(S')$, причём
\begin{gather}\label{W:01:infty1:S'T::embedding:::estimate}
    \max\limits_{t\in[0,T]}\|f(\cdot,t)\|_{\infty,S'}\leqslant A_1\|f\|^{(0,1)}_{\infty,1,S'_T}.
\end{gather}
\end{Theorem}

Из теоремы \ref{W1p([0,T],X)_embedding} вытекают две следующие теоремы.
\begin{Theorem}\label{W:01:21:ST::embedding}
Пусть $1\leqslant p<\infty$. Тогда у любой функции $f\in W^{0,1}_{p,1}(S_T)$ при каждом $t\in[0,T]$ существует след $f(\cdot,t)\in L_p(S)$, непрерывно зависящий от $t\in[0,T]$ в норме
$L_p(S)$, причём
\begin{gather}\label{W:01:21:ST::embedding:::estimate}
    \max\limits_{t\in[0,T]}\|f(\cdot,t)\|_{p,S}\leqslant A_1\|f\|^{(0,1)}_{p,1,S_T}.
\end{gather}
\end{Theorem}

\begin{Theorem}\label{W:01:21:S'T::embedding}
Пусть $1\leqslant p<\infty$. Тогда у любой функции $f\in W^{0,1}_{p,1}(S'_T)$ при каждом $t\in[0,T]$ существует след $f(\cdot,t)\in L_p(S')$, непрерывно зависящий от $t\in[0,T]$ в норме
$L_p(S')$, причём
\begin{gather}\label{W:01:21:S'T::embedding:::estimate}
    \max\limits_{t\in[0,T]}\|f(\cdot,t)\|_{p,S'}\leqslant A_1\|f\|^{(0,1)}_{p,1,S'_T}.
\end{gather}
\end{Theorem}


Из теорем \ref{abstact_energetic_class_embedding:Theorem} и \ref{V1120QTcompactness:abstract} вытекает
\begin{Lemma}\label{En1_2Q_T::embedding}
\renewcommand{\labelenumi}{\arabic{enumi})}
\renewcommand{\labelenumii}{\asbuk{enumii})}
\begin{enumerate}
    \item
Если $z\in \textrm{Э}^{1}_{2}(Q_T)$, то $z\in W^{1}_{2}(Q_T)$, причём найдётся постоянная $c_0^*>0$, зависящая лишь от $T>0$ (например, можно взять $c_0^*\equiv\sqrt{2T}$), такая, что
\begin{gather*}
\|z\|^{(1)}_{2,Q_T}\leqslant c_0^*\|z\|_{\textrm{Э}^{1}_{2}(Q_T)}\,\,\,\forall\,z\in \textrm{Э}^{1}_{2}(Q_T).
\end{gather*}
    \item
Вложения $\textrm{Э}^{1}_{2}(Q_T)\subset L_2(S_T)$ и $\textrm{Э}^{1}_{2}(Q_T)\subset C([0,T],L_2(\Omega))$ непрерывны и компактны.
    \item
Справедливы следующие утверждения:
\begin{enumerate}
    \item если $n=1$, то любая функция $z\in \textrm{Э}^{1}_{2}(Q_T)$ является элементом пространства $C(\bar Q_T)$ и найдётся константа $c^*_1>0$, зависящая лишь от области $\Omega$,
такая, что
\begin{gather*}
\pmb{|}z\pmb{|}^{(0)}_{\bar Q_T}\leqslant c^*_1\|z\|_{\textrm{Э}^{1}_{2}(Q_T)}\,\,\,\forall\,z\in \textrm{Э}^{1}_{2}(Q_T);
\end{gather*}

    \item
 если $n=2$, а $p\in(1,\infty)$, то любая функция $z\in \textrm{Э}^{1}_{2}(Q_T)$ является элементом пространства $C([0,T],L_p(\Omega))$ и найдётся
константа $c^*_2=c_2^*(p)>0$, зависящая лишь от области $\Omega$ и от $p$, такая, что
\begin{gather*}
\max\limits_{t\in[0,T]}\|z(\cdot,t)\|_{p,\Omega}\leqslant c^*_2(p)\|z\|_{\textrm{Э}^{1}_{2}(Q_T)}\,\,\,\forall\,z\in \textrm{Э}^{1}_{2}(Q_T);
\end{gather*}

    \item
если $n>2$, а $p\in(1,\frac{2n}{n-2})$, то любая функция $z\in \textrm{Э}^{1}_{2}(Q_T)$ является элементом пространства $C([0,T],L_p(\Omega))$ и
найдётся константа $c^*_3=c_3^*(p)>0$, зависящая лишь от $p$, размерности $n$ и от области $\Omega$, такая, что
\begin{gather*}
\max\limits_{t\in[0,T]}\|z(\cdot,t)\|_{p,\Omega}\leqslant c^*_3(p)\|z\|_{\textrm{Э}^{1}_{2}(Q_T)}\,\,\,\forall\,z\in \textrm{Э}^{1}_{2}(Q_T).
\end{gather*}
\end{enumerate}

    \item
Пусть последовательность функций ${z}^m\in \textrm{Э}^{1}_{2}(Q_T)$, $m=1,2,\dots$, --- ограничена в норме пространства $\textrm{Э}^{1}_{2}(Q_T)$, т.е. найдётся постоянная $K>0$, такая, что
\begin{gather*}
\|{z}^m\|_{\textrm{Э}^{1}_{2}(Q_T)}\leqslant K,\,\,\,m=1,2,\dots
\end{gather*}
Тогда найдутся подпоследовательность $z^{m_l}$, $l=1,2,\dots$, последовательности $z^{m}$, $m=1,2,\dots$, и функция $z\in \textrm{Э}^{1}_{2}(Q_T)$,
такие, что
\begin{gather*}
z^{m_l}\to z,\,\,\,l\to\infty,\,\,\,\mbox{слабо в $W^1_{2}(Q_T)$},\,\,\,\lim\limits_{l\to\infty}\max\limits_{t\in[0,T]}\|z^{m_l}(\cdot,t)-z(\cdot,t)\|_{2,\Omega}=0;\\
\notag z^{m_l}\to z,\,\,\,l\to\infty,\,\,\,\mbox{$*$--слабо в $L_\infty([0,T],W^1_{2}(\Omega))$};\,\,\,z^{m_l}_t\to z_t,\,\,\,l\to\infty,\,\,\,
\mbox{$*$--слабо в $L_\infty([0,T],L_{2}(\Omega))$};\\
\lim\limits_{l\to\infty}\pmb{|}z^{m_l}-z\pmb{|}^{(0)}_{\bar Q_T}=0\,\,\,\text{ при $n=1$; }
\max\limits_{t\in[0,T]}\|z^{m_l}(\cdot,t)-z(\cdot,t)\|_{p,S}=0\,\,\,\text{ при $n>1$;}
\end{gather*}
где $p$ --- число из интервала $(1,+\infty)$ при $n=2$ и из интервала $(1,\frac{2(n-1)}{n-2})$ при $n>2$; причём
\begin{gather*}
\|z\|_{\textrm{Э}^{1}_{2}(Q_T)}\leqslant K.
\end{gather*}
\end{enumerate}
\end{Lemma}


Из теорем \ref{B0BB1Theorem}, \ref{abstact_energetic_class_embedding:Theorem} и \ref{V1120QTcompactness:abstract},
цепочки компактных вложений $W^2_2(\Omega)\subset W^1_2(\Omega)\subset L_2(\Omega)$ и рефлексивности пространств $W^2_2(\Omega)$ и $L_2(\Omega)$ вытекает
\begin{Lemma}\label{En2_2Q_T::embedding}
\renewcommand{\labelenumi}{\arabic{enumi})}
\renewcommand{\labelenumii}{\asbuk{enumii})}
\begin{enumerate}
    \item
Если $z\in \textrm{Э}^{2}_{2}(Q_T)$, то $z\in W^{2;1}_{2}(Q_T)$, причём
\begin{gather*}
\|z\|^{(2;1)}_{2,Q_T}\leqslant c_0^*\|z\|_{\textrm{Э}^{2}_{2}(Q_T)}\,\,\,\forall\,z\in \textrm{Э}^{2}_{2}(Q_T),
\end{gather*}
где постоянная $c_0^*>0$ --- та же, что и в лемме \ref{En1_2Q_T::embedding}.
    \item
Если $z\in \textrm{Э}^{2}_{2}(Q_T)$, то $z\in C([0,T],W^1_2(\Omega))$, причём
\begin{gather*}
\max\limits_{t\in[0,T]}\|z(\cdot,t)\|^{(1)}_{2,\Omega}\leqslant\|z\|_{\textrm{\textrm{Э}}^{2}_{2}(Q_T)}\,\,\,\forall\,z\in \textrm{Э}^{2}_{2}(Q_T).
\end{gather*}
   \item
Вложения $\textrm{Э}^{2}_{2}(Q_T)\subset L_2(S_T)$, $\textrm{Э}^{2}_{2}(Q_T)\subset C([0,T],L_2(\Omega))$ и $\textrm{Э}^{2}_{2}(Q_T)\subset L_p([0,T],W^1_2(\Omega))$
непрерывны и компактны при всех $p\in(1,\infty)$.
   \item
Если $z\in \textrm{Э}^{2}_{2}(Q_T)$, то $\nabla_xz\in C([0,T],(W^1_2(\Omega))^n)$, а если ещё $n\geqslant2$, то $\nabla_xz\in C([0,T],L_2^n(S))$.
   \item
Справедливы следующие утверждения:
\begin{enumerate}
     \item
если $n<4$, то любая функция $z\in \textrm{Э}^{2}_{2}(Q_T)$ является элементом пространства $C(\bar Q_T)$ и найдётся константа $c^*_4>0$,
зависящая лишь от области $\Omega$, такая, что
\begin{gather*}
\pmb{|}z\pmb{|}^{(0)}_{\bar Q_T}\leqslant c^*_4\|z\|_{\textrm{Э}^{2}_{2}(Q_T)}\,\,\,\forall\,z\in \textrm{Э}^{2}_{2}(Q_T);
\end{gather*}
    \item
если $n=4$, а $p\in(1,\infty)$, то любая функция $z\in \textrm{Э}^{2}_{2}(Q_T)$ является элементом пространства $C([0,T],L_p(\Omega))$ и найдётся
константа $c^*_5=c_5^*(p)>0$, зависящая лишь от области $\Omega$ и от $p$, такая, что
\begin{gather*}
\max\limits_{t\in[0,T]}\|z(\cdot,t)\|_{p,\Omega}\leqslant c^*_5(p)\|z\|_{\textrm{Э}^{2}_{2}(Q_T)}\,\,\,\forall\,z\in \textrm{Э}^{2}_{2}(Q_T);
\end{gather*}
    \item
если $n>4$, а $p\in(1,\frac{2n}{n-4})$, то любая функция $z\in \textrm{Э}^{2}_{2}(Q_T)$ является элементом пространства $C([0,T],L_p(\Omega))$ и
найдётся константа $c^*_6=c_6^*(p)>0$, зависящая лишь от $p$, размерности $n$ и от области $\Omega$, такая, что
\begin{gather*}
\max\limits_{t\in[0,T]}\|z(\cdot,t)\|_{p,\Omega}\leqslant c^*_6(p)\|z\|_{\textrm{Э}^{2}_{2}(Q_T)}\,\,\,\forall\,z\in \textrm{Э}^{2}_{2}(Q_T).
\end{gather*}
\end{enumerate}

    \item
Пусть последовательность функций ${z}^m\in \textrm{Э}^{2}_{2}(Q_T)$, $m=1,2,\dots$, --- ограничена в норме пространства $\textrm{Э}^{2}_{2}(Q_T)$, т.е. найдётся постоянная $K>0$, такая, что
\begin{gather*}
\|{z}^m\|_{\textrm{Э}^{2}_{2}(Q_T)}\leqslant K,\,\,\,m=1,2,\dots
\end{gather*}
Тогда найдутся подпоследовательность $z^{m_l}$, $l=1,2,\dots$, последовательности $z^{m}$, $m=1,2,\dots$, и функция $z\in \textrm{Э}^{2}_{2}(Q_T)$,
такие, что
\begin{gather*}
z^{m_l}\to z,\,\,\,l\to\infty,\,\,\,\mbox{слабо в $W^{2;1}_{2}(Q_T)$},\,\,\,\lim\limits_{l\to\infty}\max\limits_{t\in[0,T]}\|z^{m_l}(\cdot,t)-z(\cdot,t)\|_{2,\Omega}=0;\\
z^{m_l}\to z,\,\,\,l\to\infty,\,\,\,\mbox{$*$--слабо в $L_\infty([0,T],W^2_{2}(\Omega))$};\,\,\,
z^{m_l}_t\to z_t,\,\,\,l\to\infty,\,\,\,\mbox{$*$--слабо в $L_\infty([0,T],L_{2}(\Omega))$};\\
{z}^{m_l}\to{z},\,\,\,l\to\infty,\,\,\,\mbox{сильно в $L_2([0,T],{W}^1_{2}(\Omega))$;}\\
\lim\limits_{l\to\infty}\pmb{|}z^{m_l}-z\pmb{|}^{(0)}_{\bar Q_T}=0\,\,\,\text{при $n<4$;}\,\,\,
\lim\limits_{l\to\infty}\pmb{|}\nabla_xz^{m_l}-\nabla_xz\pmb{|}^{(0)}_{\bar Q_T}=0\,\,\,\text{при $n=1$;}\\
\lim\limits_{l\to\infty}\max\limits_{t\in[0,T]}\|z^{m_l}(\cdot,t)-z(\cdot,t)\|_{p,S}=0,\,\,\,\text{при $n>4$;}\\
\lim\limits_{l\to\infty}\max\limits_{t\in[0,T]}\|\nabla_xz^{m_l}(\cdot,t)-\nabla_xz(\cdot,t)\|_{q,S}=0,\,\,\,\text{при $n>1$;}
\end{gather*}
где $p$ --- число из интервала $(1,+\infty)$ при $n=2$ и из интервала $(1,\frac{2(n-1)}{n-2})$ при $n>2$, а
$q$ --- число из интервала $(1,+\infty)$ при $n=4$ и из интервала $(1,\frac{2(n-1)}{n-4})$ при $n>4$; причём
\begin{gather*}
\|z\|_{\textrm{Э}^{2}_{2}(Q_T)}\leqslant K.
\end{gather*}
\end{enumerate}
\end{Lemma}


\begin{Lemma}\label{EnergeticZ:function_theta01:2}
Пусть $\gamma\in(1,\infty)$ при $n=1$ или $2$ и $\gamma\in(1,\frac{n}{n-2})$ при $n>2$. Тогда найдутся неубывающие неотрицательные непрерывные функции $\vartheta_0[\gamma]:[0,+\infty)\to
\mathbb{R}$, $\vartheta_1[\gamma]:[0,\meas\Omega]\to\mathbb{R}$, $\vartheta_0[\gamma](0)=\vartheta_1[\gamma](0)=0$, такие, что для любой функции $z\in \textrm{Э}^1_2(Q_T)$ и любого
измеримого по Лебегу множества $E\subseteq \Omega$ справедливо неравенство
\begin{gather*}
\max\limits_{t\in[0,T]}\|z(\cdot,t)\|_{\gamma,E}\leqslant\vartheta_0[\gamma](\|z\|_{\textrm{Э}^1_2(Q_T)})\vartheta_1[\gamma](\meas_nE).
\end{gather*}
При этом когда понятно, о каком именно $\gamma$ идёт речь, вместо $\vartheta_0[\gamma]$ и $\vartheta_1[\gamma]$ пишем просто $\vartheta_0$ и $\vartheta_1$ соответственно.
\end{Lemma}
\begin{Proof}
Выберем произвольно функцию $z\in \textrm{Э}^{1}_{2}(Q_T)$ и измеримое по Лебегу множество $E\subseteq\Omega$ и зафиксируем. Рассмотрим отдельно случаи $n=1$, $n=2$ и $n>2$.

1) Пусть $n=1$. Тогда, согласно лемме \ref{En1_2Q_T::embedding}, $z\in C(\bar Q_T)$, откуда следует, что
\begin{gather*}
\int\limits_E|z(x,t)|^\gamma dx\leqslant[\pmb{|}z\pmb{|}^{(0)}_{\bar Q_T}]^\gamma\meas E.
\end{gather*}
Согласно пункту 3.а) леммы \ref{En1_2Q_T::embedding},
\begin{gather*}
\pmb{|}z\pmb{|}^{(0)}_{\bar Q_T}\leqslant c^*_1\|z\|_{\textrm{Э}^{1}_{2}(Q_T)}.
\end{gather*}
Поэтому
\begin{gather*}
\int\limits_E|z(x,t)|^\gamma dx\leqslant[\pmb{|}z\pmb{|}^{(0)}_{\bar Q_T}]^\gamma \meas E\leqslant[c^*_1\|z\|_{\textrm{Э}^1_2(Q_T)}]^\gamma\meas E.
\end{gather*}
Положив $\vartheta_0(r)\equiv c_1^*r$, $\vartheta_1(M)\equiv M^{1/\gamma}$ при всех $r\in[0,+\infty)$, $M\in[0,\meas\Omega]$, получаем требуемое утверждение.

2) Пусть $n=2$. Тогда, согласно лемме \ref{En1_2Q_T::embedding}, $z\in C([0,T],L_q(\Omega))$ при всех $q\in(1,+\infty)$, в силу чего
$|z|^\gamma\in C([0,T],L_p(\Omega))$ при всех $p\in(1,+\infty)$. Далее, по неравенству Гёльдера для интегралов,
\begin{gather*}
\int\limits_E|z(x,t)|^\gamma dx\leqslant\left[\int\limits_{E}|z(x,t)|^{\gamma p}dx\right]^{1/p}[\meas E]^{1-\frac1p}=\|z(\cdot,t)\|^\gamma_{\gamma p,E}[\meas E]^{1-\frac1p}\leqslant\\
\leqslant[\max\limits_{\tau\in[0,T]}\|z(\cdot,\tau)\|_{\gamma p,\Omega}]^\gamma[\meas E]^{1-\frac1p}.
\end{gather*}
Таким образом,
\begin{gather*}
\int\limits_E|z(x,t)|^\gamma dx\leqslant[\max\limits_{\tau\in[0,T]}\|z(\cdot,\tau)\|_{\gamma p,\Omega}]^\gamma[\meas E]^{1-\frac1p}.
\end{gather*}
Согласно пункту 3.б) леммы \ref{En1_2Q_T::embedding},
\begin{gather*}
\max\limits_{t\in[0,T]}\|z(\cdot,t)\|_{\gamma p,\Omega}\leqslant c^*_2(p\gamma)\|z\|_{\textrm{Э}^{1}_{2}(Q_T)}.
\end{gather*}
Как следствие,
\begin{gather*}
\int\limits_E|z(x,t)|^\gamma dx\leqslant[\max\limits_{\tau\in[0,T]}\|z(\cdot,\tau)\|_{\gamma p,\Omega}]^\gamma[\meas E]^{1-\frac1p}
\leqslant[c^*_2(\gamma p)\|z\|_{\textrm{Э}^{1}_{2}(Q_T)}]^\gamma[\meas E]^{1-\frac1p}.
\end{gather*}
Положив $\vartheta_0(r)\equiv c_2^*(\gamma p)r$, $\vartheta_1(M)\equiv M^{\frac{p-1}{p\gamma}}$ при всех $r\in[0,+\infty)$, $M\in[0,\meas\Omega]$, получаем требуемое утверждение.

3)  Пусть $n>2$. Тогда, согласно лемме \ref{En1_2Q_T::embedding}, $z\in C([0,T],L_q(\Omega))$ при всех $q\in(1,\frac{2n}{n-2})$, в силу чего $|z|^\gamma\in C([0,T],L_p(\Omega))$ при всех
$p\in(1,\frac{n}{(n-2)\gamma})$. Далее, по неравенству Гёльдера для интегралов,
\begin{gather*}
\int\limits_E|z(x,t)|^\gamma dx\leqslant\left[\int\limits_{E}|z(x,t)|^{\gamma p}dx\right]^{1/p}[\meas E]^{1-\frac1p}=\|z(\cdot,t)\|^\gamma_{\gamma p,E}[\meas E]^{1-\frac1p}\leqslant\\
\leqslant[\max\limits_{\tau\in[0,T]}\|z(\cdot,\tau)\|_{\gamma p,\Omega}]^\gamma[\meas E]^{1-\frac1p},
\end{gather*}
где $p\in(1,\frac{2n}{(n-2)\gamma})$ --- некоторое число. Таким образом,
\begin{gather*}
\int\limits_E|z(x,t)|^\gamma dx\leqslant[\max\limits_{\tau\in[0,T]}\|z(\cdot,\tau)\|_{\gamma p,\Omega}]^\gamma[\meas E]^{1-\frac1p}.
\end{gather*}
Согласно пункту 3.в) леммы \ref{En1_2Q_T::embedding},
\begin{gather*}
\max\limits_{t\in[0,T]}\|z(\cdot,t)\|_{\gamma p,\Omega}\leqslant c^*_3(p\gamma)\|z\|_{\textrm{Э}^{1}_{2}(Q_T)}.
\end{gather*}
Как следствие,
\begin{gather*}
\int\limits_E|z(x,t)|^\gamma dx\leqslant[\max\limits_{\tau\in[0,T]}\|z(\cdot,\tau)\|_{\gamma p,\Omega}]^\gamma[\meas E]^{1-\frac1p}
\leqslant[c^*_3(\gamma p)\|z\|_{\textrm{Э}^{1}_{2}(Q_T)}]^\gamma[\meas E]^{1-\frac1p}.
\end{gather*}

Положив $\vartheta_0(r)\equiv c_3^*(\gamma p)r$, $\vartheta_1(M)\equiv M^{\frac{p-1}{p\gamma}}$ при всех $r\in[0,+\infty)$, $M\in[0,\meas\Omega]$, получаем требуемое утверждение.

Лемма полностью доказана.
\end{Proof}


\begin{Lemma}\label{KL21QT_Ei(t)}
Пусть при каждом $t\in[0,T]$ заданы измеримые по Лебегу множества $E_i(t)\subseteq\Omega$, $i=1,2,\dots$, такие, что $\meas_nE_i(t)\to0$, $i\to\infty$, по мере Лебега на $[0,T]$, и пусть
$K\in L_{2,1}(Q_T)$. Тогда
\begin{gather}\label{KL21QT_Ei(t)::lim}
\lim\limits_{i\to\infty}\int\limits_0^T\|K(\cdot,t)\|_{2,E_i(t)}dt=0.
\end{gather}
\end{Lemma}
\begin{Proof}
Предположим, что утверждение леммы неверно. Тогда найдутся положительное число $\varepsilon_0$ и подпоследовательность $i_j$, $j=1,2,\dots$, последовательности $i=1,2,\dots$, такие, что
\begin{gather}\label{KL21QT_Ei(t)::lim::isnt_true}
\int\limits_0^T\|K(\cdot,t)\|_{2,E_{i_j}(t)}dt\geqslant\varepsilon_0,\,\,\,j=1,2,\dots
\end{gather}
Поскольку $\meas_nE_i(t)\to0$, $i\to\infty$, по мере Лебега на $[0,T]$, то $\meas_nE_{i_j}(t)\to0$, $j\to\infty$, по мере Лебега на $[0,T]$, в силу чего найдётся подпоследовательность
$j_m$, $m=1,2,\dots$, последовательности $j=1,2,\dots$, такая, что $\meas_nE_{i_{j_m}}(t)\to0$, $m\to\infty$, почти всюду на $[0,T]$. Пользуясь теперь абсолютной непрерывностью интеграла
Лебега, заключаем, что
\begin{gather*}
\|K(\cdot,t)\|_{2,E_{i_{j_m}}(t)}\to0,\,\,\,m\to\infty,\,\,\,\text{п.в. на $[0,T]$.}
\end{gather*}
Так как, кроме того,
\begin{gather*}
\|K(\cdot,t)\|_{2,E_{i_{j_m}}(t)}\leqslant\|K(\cdot,t)\|_{2,\Omega}\,\,\,\text{при п.в. на $[0,T]$,}
\end{gather*}
а (в силу включения $K\in L_{2,1}(Q_T)$) функция $[0,T]\ni t\mapsto\|K(\cdot,t)\|_{2,\Omega}$ принадлежит $L_1[0,T]$, то, в силу теоремы Лебега о предельном переходе под знаком интеграла
Лебега,
\begin{gather*}
\lim\limits_{m\to\infty}\int\limits_0^T\|K(\cdot,t)\|_{2,E_{i_{j_m}}(t)}dt=0,
\end{gather*}
что противоречит соотношению (\ref{KL21QT_Ei(t)::lim::isnt_true}). Полученное противоречие доказывает утверждение леммы.
\end{Proof}

\begin{Lemma}\label{KL1[0,T]_theta(measE_i(t))}
Пусть при каждом $t\in[0,T]$ заданы измеримые по Лебегу множества $E_i(t)\subseteq\Omega$, $i=1,2,\dots$, такие, что $\meas_nE_i(t)\to0$, $i\to\infty$, по мере Лебега на $[0,T]$; функция
$K(t)$, $t\in[0,T]$, --- неотрицательна и суммируема по отрезку $[0,T]$; а неотрицательная функция $\vartheta(q)$, $q\in[0,\meas_n\Omega]$, --- непрерывна и не убывает на
$[0,\meas_n\Omega]$, и такова, что $\vartheta(0)=0$. Тогда
\begin{gather}\label{KL1[0,T]_theta(measE_i(t))::lim}
\lim\limits_{i\to\infty}\int\limits_0^TK(t)\vartheta(\meas_nE_i(t))dt=0.
\end{gather}
\end{Lemma}
\begin{Proof}
Пусть это не так, и найдутся $\varepsilon_0>0$ и подпоследовательность $i_j$, $j=1,2,\dots$, последовательности $i=1,2,\dots$, такие, что
\begin{gather}\label{KL1[0,T]_theta(measE_i(t))::lim::isnt_true}
\int\limits_0^TK(t)\vartheta(\meas_nE_{i_j}(t))dt\geqslant\varepsilon_0,\,\,\,j=1,2,\dots
\end{gather}
Поскольку $\meas_nE_i(t)\to0$, $i\to\infty$, по мере Лебега на $[0,T]$, то $\meas_nE_{i_j}(t)\to0$, $j\to\infty$, по мере Лебега на $[0,T]$, в силу чего найдётся подпоследовательность
$j_m$, $m=1,2,\dots$, последовательности $j=1,2,\dots$, такая, что $\meas_nE_{i_{j_m}}(t)\to0$, $m\to\infty$, почти всюду на $[0,T]$. Поэтому, ввиду непрерывности функции $\vartheta$ на
$[0,\meas_n\Omega]$,
\begin{gather*}
K(t)\vartheta(\meas_nE_{i_{j_m}}(t))\to0,\,\,\,m\to\infty,\,\,\,\text{п.в. на $[0,T]$.}
\end{gather*}
Так как
\begin{gather*}
K(t)\vartheta(\meas_nE_{i_{j_m}}(t))\leqslant K(t)\vartheta(\meas_n\Omega),\,\,\,m=1,2,\dots,\,\,\,\text{п.в. на $[0,T]$,}
\end{gather*}
а функция $K$ является элементом $L_1[0,T]$, то после применения теоремы Лебега о мажорированной сходимости будем иметь
\begin{gather*}
\lim\limits_{m\to\infty}\int\limits_0^TK(t)\vartheta(\meas_nE_{i_{j_m}}(t))dt=0,
\end{gather*}
что противоречит неравенству (\ref{KL1[0,T]_theta(measE_i(t))::lim::isnt_true}). Следовательно, лемма доказана.
\end{Proof}

\begin{Lemma}\label{Fourierseriesconvergence}Пусть $\sigma\in C^\infty(\bar{Q}_T)$, и пусть $e_j\in L_2(\Omega)$, $j=1,2,\dots,$ --- некоторый ортонормированный базис в $L_2(\Omega)$.
Тогда ряд Фурье
\begin{equation}\label{sigmaFourierseries}
\sum_{j=1}^\infty\sigma_j(t)e_j,
\end{equation}
где
\begin{equation}\label{sigmakdef}
\sigma_j(t)\equiv\int\limits_\Omega\sigma(x,t)e_j(x)\,dx,\,\,\,j=1,2,\dots,
\end{equation}
сходится к функции $\sigma$ в норме пространства $C^m([0,T],L_2(\Omega))$ при любом $m\geqslant0$.
\end{Lemma}
\begin{Proof} Поскольку $\sigma\in C^\infty(\bar{Q}_T)$, то, очевидно,
\begin{equation}\label{sigmakincandder}
\sigma_j\in C^\infty[0,T],\,\,\,\frac{d^m\sigma_j(t)}{dt^m}= \int\limits_\Omega\frac{\partial^m\sigma(x,t)}{\partial t^m}e_k(x)\,dx,\,\,\,
m,\,\,j=1,2,\dots
\end{equation}
Ясно, что при любом $t\in[0,T]$ ряд
\begin{equation}\label{dssigma}
\sum_{j=1}^\infty\frac{d^m\sigma_j(t)}{dt^m}e_j
\end{equation}
сходится в норме $L_2(\Omega)$ к $\frac{\partial^m\sigma(x,t)}{\partial t^m}$, $m=1,2,\dots$ Полагая $r_{N,m}(t)\equiv\int\limits_\Omega\left[ \frac{\partial^m\sigma(x,t)}{\partial t^m}-
\sum\limits_{j=1}^N\frac{d^m\sigma_j(t)}{dt^m}e_j(x)\right]^2dx$, $N=1,2,\dots$, $m=0,1,2,\dots$, заключаем отсюда, что
\begin{gather*}
\lim_{N\to\infty}r_{N,m}(t)=0,\,\,\,\forall\,t\in[t_0,\,t_1];\,\,\, r_{N,m}\in C[0,T],\,\,\,N=1,2,\dots,\,\,m=0,1,2,\dots;\\
r_{N+1,m}(t)\leqslant r_{N,m}(t),\,\,\,\forall\,t\in[0,T],\,\,\,N=1,2,\dots,\,\,m=0,1,2,\dots
\end{gather*}
Поэтому, в силу леммы \ref{Dini1}, $\lim\limits_{N\to\infty} \max\limits_{t\in[0,T]}r_{N,m}(t)=0$, $m=0,1,2,\dots$, что означает сходимость
ряда~(\ref{sigmaFourierseries}) к функции $\sigma$ в норме пространства $C^m([0,T],\,L_2(\Omega))$ при любом $m\geqslant0$.
\end{Proof}

Положим $x'\equiv(x_1,\dots,x_{n-1})$.

Ниже нам потребуется следующая (см., например, \cite[лемма 6]{variation_2008})
\begin{Lemma}\label{a1a2zetaEn1qt!!abstract} Пусть множество конечной положительной меры Лебега ${\cal A}_1\subset \mathbb{R}^{n-1}$ и отрезок ${\cal A}_2\equiv[a_1,\,a_2]\subset
\mathbb{R}$  таковы, что ${\cal A}_1\times{\cal A}_2\subset\Omega$. Тогда найдётся, зависящая только от отрезка ${\cal A}_2$, постоянная $C>0$, такая, что для любой
функции $\zeta\in \textrm{Є}^1_{2}(Q_T)$
$$
\displaystyle\sup\limits_{t\in[0,T]}\left[\int\limits_{{\cal A}_1}\|\zeta(x',\cdot,t)\|_{\infty,{\cal A}_2}^2\,dx'\right]^{1/2}\leqslant
C\|\zeta\|_{\textrm{Є}^1_{2}(Q_T)}.
$$
\end{Lemma}
\begin{Proof}На основании классического свойства обобщённо дифференцируемых функций (см., например, \cite[с.344]{smirnov}) можно без ограничения общности считать при п.в. $x'\in{\cal A}_1$
и всех $t\in[0,T]$ функцию $\zeta_{x_n}(x',\cdot,t)\colon{\cal A}_2\to\mathbb{R}^1 $ обобщённой производной функции $ \zeta(x',\cdot,t)\colon{\cal A}_2\to \mathbb{R}^1$. Поэтому по
элементарной теореме вложения (см., например, \cite[с.359]{smirnov}, $p=2,\,l=1,\,n=1$) имеем при п.в. $x'\in{\cal A}_1$ и всех $t\in[0,T]$
$$
\|\zeta(x',\cdot,t)\|_{\infty,{\cal A}_2}\leqslant C\sqrt{\|\zeta(x',\cdot,t)\|^2_{2,{\cal A}_2}+\|\zeta_{x_n}(x',\cdot,t)\|^2_{2,{\cal A}_2}},
$$
где $C>0$ не зависит ни от $x'\in{\cal A}_1$, ни от $t\in[0,T]$. Возводя в квадрат обе части этого неравенства и интегрируя затем по множеству ${\cal A}_1$, получаем, что
\begin{gather*}
\left[\int\limits_{{\cal A}_1}\|\zeta(x',\cdot,t)\|_{\infty,{\cal A}_2}^2\,dx'\right]^{1/2}\leqslant C\left[\int\limits_{{\cal A}_1\times{\cal A}_2}[|\zeta(x,t)|^2+|\zeta_{x_n}(x,t)|^2]dx
\right]^{1/2}\leqslant C\|\zeta(\cdot,t)\|^{(1)}_{2,\Omega}\leqslant C\|\zeta\|_{\textrm{Є}^1_{2}(Q_T)}.
\end{gather*}
Лемма доказана.
\end{Proof}


\begin{Lemma}\label{a1a2zetaEn1qt_1!!abstract} Пусть множество конечной положительной меры Лебега ${\cal A}_1\subset \mathbb{R}^{n-1}$ и отрезок ${\cal A}_2\equiv[a_1,\,a_2]\subset
\mathbb{R}$  таковы, что ${\cal A}_1\times{\cal A}_2\subset\Omega$. Тогда для любой функции $\zeta\in \textrm{Є}^1_{2}(Q_T)$
$$
\displaystyle\left[\int\limits_0^Tdt\int\limits_{{\cal A}_1}\|\zeta(x',\cdot,t)\|_{\infty,{\cal A}_2}^2\,dx'\right]^{1/2}\leqslant
C[\|\zeta\|^2_{{\cal A}_1\times{\cal A}_2\times[0,T]}+\|\nabla_x\zeta\|^2_{{\cal A}_1\times{\cal A}_2\times[0,T]}]^{1/2},
$$
где положительную постоянную $C$ можно взять такой же, как в предыдущей лемме
\end{Lemma}
\begin{Proof}На основании классического свойства обобщённо дифференцируемых функций (см., например, \cite[с.344]{smirnov}) можно без ограничения общности считать при п.в. $x'\in{\cal A}_1$
и всех $t\in[0,T]$ функцию $\zeta_{x_n}(x',\cdot,t)\colon{\cal A}_2\to\mathbb{R}^1 $ обобщённой производной функции $ \zeta(x',\cdot,t)\colon{\cal A}_2\to \mathbb{R}^1$. Поэтому по
элементарной теореме вложения (см., например, \cite[с.359]{smirnov}, $p=2,\,l=1,\,n=1$) имеем при п.в. $x'\in{\cal A}_1$ и всех $t\in[0,T]$
$$
\|\zeta(x',\cdot,t)\|_{\infty,{\cal A}_2}\leqslant C\sqrt{\|\zeta(x',\cdot,t)\|^2_{2,{\cal A}_2}+\|\zeta_{x_n}(x',\cdot,t)\|^2_{2,{\cal A}_2}},
$$
где $C>0$ не зависит ни от $x'\in{\cal A}_1$, ни от $t\in[0,T]$. Возводя в квадрат обе части этого неравенства и интегрируя затем по множеству ${\cal A}_1\times[0,T]$, получаем требуемое
неравенство. Лемма доказана.
\end{Proof}

\begin{Lemma}\label{W11[0,T]is_dense_in_L1[0,T]}
Множество $W_1^1[0,T]$ всюду плотно в пространстве $L_1[0,T]$.
\end{Lemma}
\begin{Proof} В самом деле, как известно \cite{smirnov}, множество $C[0,T]$ всюду плотно в $L_1[0,T]$. Поскольку же в $C[0,T]$ всюду плотно множество всех многочленов с вещественными
коэффициентами, а любой многочлен является элементом пространства $W_1^1[0,T]$, то $W_1^1[0,T]$ всюду плотно в пространстве $L_1[0,T]$. Лемма доказана.
\end{Proof}

\begin{Lemma}\label{W01_21(S_T)_is_dense_in_L_21(S_T)}
Множество $W^{0,1}_{2,1}(S_T)$ всюду плотно в пространстве $L_{2,1}(S_T)$.
\end{Lemma}
\begin{Proof}
Нетрудно видеть, что пространство $W^{0,1}_{2,1}(S_T)$ изометрично изоморфно пространству $W^1_1([0,T],L_2(S))$, а пространство $L_{2,1}(S_T)$ изометрично изоморфно пространству
$L_1([0,T],L_2(S))$. Таким образом, нам нужно доказать, что $W^1_1([0,T],L_2(S))$ всюду плотно в $L_1([0,T],L_2(S))$. Как известно (см., например, \cite{warga}), множество
$C([0,T],L_2(S))$ всюду плотно в пространстве $L_1([0,T],L_2(S))$. Поскольку же, в силу леммы \ref{Weierstrass:approx}, множество всевозможных многочленов с коэффициентами из $L_2(S)$
всюду плотно в $C([0,T],L_2(S))$, а любой такой многочлен, как несложно заметить, является функцией из $W^1_1([0,T],L_2(S))$, то $W^1_1([0,T],L_2(S))$ всюду плотно в $L_1([0,T],L_2(S))$.
Лемма доказана.
\end{Proof}

\begin{Lemma}\label{W01_21(S'_T)_is_dense_in_L_21(S'_T)}
Множество $W^{0,1}_{2,1}(S'_T)$ всюду плотно в пространстве $L_{2,1}(S'_T)$.
\end{Lemma}
\begin{Proof}
Нетрудно видеть, что пространство $W^{0,1}_{2,1}(S'_T)$ изометрично изоморфно пространству $W^1_1([0,T],L_2(S'))$, а пространство $L_{2,1}(S'_T)$ изометрично изоморфно пространству
$L_1([0,T],L_2(S'))$. Таким образом, нам нужно доказать, что $W^1_1([0,T],L_2(S'))$ всюду плотно в $L_1([0,T],L_2(S'))$. Как известно (см., например, \cite{warga}), множество
$C([0,T],L_2(S'))$ всюду плотно в пространстве $L_1([0,T],L_2(S'))$. Поскольку же, в силу леммы \ref{Weierstrass:approx}, множество всевозможных многочленов с коэффициентами из $L_2(S')$
всюду плотно в $C([0,T],L_2(S'))$, а любой такой многочлен, как несложно заметить, является функцией из $W^1_1([0,T],L_2(S'))$, то $W^1_1([0,T],L_2(S'))$ всюду плотно в
$L_1([0,T],L_2(S'))$. Лемма доказана.
\end{Proof}

    \chapter{О представлении некоторых линейных непрерывных операторов}
        \section{Абстрактные теоремы}
В данном разделе мы доказываем некоторые результаты о представлении некоторых линейных непрерывных операторов. Всюду в данном разделе $X$, $Y$ --- банаховы пространства с нормами $\|\cdot\|_X$ и $\|\cdot\|_Y$ соответственно, $\mathcal{P}$ --- компактное метрическое пространство, $[a,b]$ --- отрезок числовой оси.

\begin{Theorem}\label{L.X->C(P,Y)!theorem}
Если оператор $A:X\to C(\mathcal{P},Y)$ --- линеен и ограничен, то найдётся функция $B:\mathcal{P}\to\mathcal{L}(X,Y)$, непрерывная в смысле сильной операторной топологии пространства $\mathcal{L}(X,Y)$, такая, что
\begin{gather}\label{L.X->C(P,Y)!representation}
A[f](p)=B(p)f\,\,\,\forall\,f\in X\,\,\forall\,p\in\mathcal{P}.
\end{gather}
При этом
\begin{gather}\label{L.X->C(P,Y)!norm}
\|A\|_{X\to C(\mathcal{P},Y)}=\sup\limits_{p\in\mathcal{P}}\|B(p)\|_{X\to Y}.
\end{gather}
Обратно, если функция $B:\mathcal{P}\to\mathcal{L}(X,Y)$ --- непрерывна в смысле сильной операторной топологии пространства $\mathcal{L}(X,Y)$, то формула~\eqref{L.X->C(P,Y)!representation} задаёт линейный непрерывный оператор, действующий из $X$ в $C(\mathcal{P},Y)$.
\end{Theorem}
\begin{Proof}
Доказательство разобьём на три части.

1) Покажем, что если функция $B:\mathcal{P}\to\mathcal{L}(X,Y)$ --- непрерывна в смысле сильной операторной топологии пространства $\mathcal{L}(X,Y)$, то формула~\eqref{L.X->C(P,Y)!representation} задаёт линейный непрерывный оператор, действующий из $X$ в $C(\mathcal{P},Y)$.

Действительно, пусть функция $B:\mathcal{P}\to\mathcal{L}(X,Y)$ --- непрерывна в смысле сильной операторной топологии пространства $\mathcal{L}(X,Y)$. Тогда при всех $f\in X$ справедливо включение $A[f]\in C(\mathcal{P},Y)$, а линейность оператора $A$ очевидна. Таким образом, $A$ --- линейный оператор, действующий из $X$ в $C(\mathcal{P},Y)$.

Докажем ограниченность оператора $A$. Поскольку функция $B:\mathcal{P}\to\mathcal{L}(X,Y)$ --- непрерывна в смысле сильной операторной топологии пространства $\mathcal{L}(X,Y)$, то при всех $f\in X$ функция
$$
\mathcal{P}\ni p\mapsto (B(p)f) \in Y
$$
непрерывна в смысле сильной топологии пространства $Y$. Следовательно, в силу теоремы о резонансе, при каждом фиксированном $f\in X$ конечна величина
$\sup\limits_{p\in\mathcal{P}}\|B(p)f\|_Y$. Поэтому, снова в силу теоремы о резонансе, конечна величина $\sup\limits_{p\in\mathcal{P}}\|B(p)\|_{X\to Y}$. Как следствие,
\begin{gather*}
\|A[f](p)\|_Y\leqslant\|B(p)\|_{X\to Y}\|f\|_X\leqslant\sup\limits_{q\in\mathcal{P}}\|B(q)\|_{X\to Y}\|f\|_{X}.
\end{gather*}
Переходя здесь к точной верхней грани по $p\in\mathcal{P}$, выводим, что
\begin{gather*}
\|A[f]\|_{C(\mathcal{P},Y)}\leqslant\sup\limits_{q\in\mathcal{P}}\|B(q)\|_{X\to Y}\|f\|_{X}.
\end{gather*}
А это и означает, что оператор $A$ --- ограничен.

2) Докажем, что для любого линейного непрерывного оператора $A:X\to C(\mathcal{P},Y)$ найдётся функция $B:\mathcal{P}\to\mathcal{L}(X,Y)$, непрерывная в смысле сильной операторной топологии пространства $\mathcal{L}(X,Y)$, такая, что справедливо представление \eqref{L.X->C(P,Y)!representation}. В самом деле, пусть оператор $A:X\to C(\mathcal{P},Y)$ --- линеен и ограничен. Тогда при всех $f\in X$
\begin{gather*}
\sup\limits_{p\in\mathcal{P}}\|A[f](p)\|_Y\leqslant\|A\|_{X\to C(\mathcal{P},Y)}\|f\|_X.
\end{gather*}
Следовательно, при каждом фиксированном $p\in\mathcal{P}$
\begin{gather*}
\|A[f](p)\|_Y\leqslant\|A\|_{X\to C(\mathcal{P},Y)}\|f\|_X\,\,\,\forall\,f\in X.
\end{gather*}
Иными словами, при каждом фиксированном $p$ отображение
\begin{gather*}
X\ni f\mapsto A[f](p)\in Y
\end{gather*}
является линейным ограниченным оператором, действующим из $X$ в $Y$. Обозначив этот оператор через $B(p)$, получаем представление~\eqref{L.X->C(P,Y)!representation}. Непрерывность функции $B:\mathcal{P}\to\mathcal{L}(X,Y)$ в смысле сильной операторной топологии пространства $\mathcal{L}(X,Y)$ следует из того, что при всех $f\in X$ справедливо включение $A[f]\in C(\mathcal{P},Y)$.

3) Докажем равенство~\eqref{L.X->C(P,Y)!norm}:
\begin{gather*}
\|A\|_{X\to C(\mathcal{P},Y)}=\sup\limits_{\|f\|_X\leqslant1}\sup\limits_{p\in\mathcal{P}}\|A[f](p)\|_Y=\sup\limits_{p\in\mathcal{P}}\sup\limits_{\|f\|_X\leqslant1}\|B(p)f\|_Y=
\sup\limits_{p\in\mathcal{P}}\|B(p)\|_{X\to Y},
\end{gather*}
что и требовалось доказать.
\end{Proof}

\begin{Theorem}\label{L.X->Cs(P,Y)!theorem}
Пусть пространство $Y$ --- рефлексивно. Если оператор $A:X\to C_s(\mathcal{P},Y)$ --- линеен и ограничен (считаем, что пространство $C_s(\mathcal{P},Y)$ наделено сильной нормой), то
найдётся функция $B:\mathcal{P}\to\mathcal{L}(X,Y)$, непрерывная в смысле слабой операторной топологии пространства $\mathcal{L}(X,Y)$, такая, что
\begin{gather}\label{L.X->Cs(P,Y)!representation}
A[f](p)=B(p)f\,\,\,\forall\,f\in X\,\,\forall\,p\in\mathcal{P}.
\end{gather}
При этом
\begin{gather}\label{L.X->Cs(P,Y)!norm}
\|A\|_{X\to C_s(\mathcal{P},Y)}=\sup\limits_{p\in\mathcal{P}}\|B(p)\|_{X\to Y}.
\end{gather}
Обратно, если функция $B:\mathcal{P}\to\mathcal{L}(X,Y)$ --- непрерывна в смысле слабой операторной топологии пространства $\mathcal{L}(X,Y)$, то формула
\eqref{L.X->Cs(P,Y)!representation} задаёт линейный непрерывный оператор, действующий из $X$ в $C_s(\mathcal{P},Y)$.
\end{Theorem}
\begin{Proof}
Доказательство разобьём на три части.

1) Покажем, что если функция $B:\mathcal{P}\to\mathcal{L}(X,Y)$ --- непрерывна в смысле слабой операторной топологии пространства $\mathcal{L}(X,Y)$, то формула
\eqref{L.X->Cs(P,Y)!representation} задаёт линейный непрерывный оператор, действующий из $X$ в $C_s(\mathcal{P},Y)$.

Действительно, пусть функция $B:\mathcal{P}\to\mathcal{L}(X,Y)$ --- непрерывна в смысле слабой операторной топологии пространства $\mathcal{L}(X,Y)$. Тогда при всех $f\in X$ справедливо
включение $A[f]\in C_s(\mathcal{P},Y)$, а линейность оператора $A$ очевидна. Таким образом, $A$ --- линейный оператор, действующий из $X$ в $C_s(\mathcal{P},Y)$.

Докажем ограниченность оператора $A$. Поскольку функция $B:\mathcal{P}\to\mathcal{L}(X,Y)$ --- непрерывна в смысле слабой операторной топологии пространства $\mathcal{L}(X,Y)$, то
при всех $f\in X$ функция
$$
\mathcal{P}\ni p\mapsto (B(p)f) \in Y
$$
непрерывна в смысле слабой топологии пространства $Y$. Следовательно, в силу теоремы о резонансе, при каждом фиксированном $f\in X$ конечна величина
$\sup\limits_{p\in\mathcal{P}}\|B(p)f\|_Y$. Поэтому, снова в силу теоремы о резонансе, конечна величина $\sup\limits_{p\in\mathcal{P}}\|B(p)\|_{X\to Y}$. Как следствие,
\begin{gather*}
\|A[f](p)\|_Y\leqslant\|B(p)\|_{X\to Y}\|f\|_X\leqslant\sup\limits_{q\in\mathcal{P}}\|B(q)\|_{X\to Y}\|f\|_{X}.
\end{gather*}
Переходя здесь к точной верхней грани по $p\in\mathcal{P}$, выводим, что
\begin{gather*}
\|A[f]\|_{C_s(\mathcal{P},Y)}\leqslant\sup\limits_{q\in\mathcal{P}}\|B(q)\|_{X\to Y}\|f\|_{X}.
\end{gather*}
А это и означает, что оператор $A$ --- ограничен.


2) Докажем, что для любого линейного непрерывного оператора $A:X\to C_s(\mathcal{P},Y)$ найдётся функция $B:\mathcal{P}\to\mathcal{L}(X,Y)$, непрерывная в смысле слабой
операторной топологии пространства $\mathcal{L}(X,Y)$, такая, что справедливо представление \eqref{L.X->Cs(P,Y)!representation}. В самом деле, пусть оператор $A:X\to C_s(\mathcal{P},Y)$
--- линеен и ограничен. Тогда при всех $f\in X$
\begin{gather*}
\sup\limits_{p\in\mathcal{P}}\|A[f](p)\|_Y\leqslant\|A\|_{X\to C_s(\mathcal{P},Y)}\|f\|_X.
\end{gather*}
Следовательно, при каждом фиксированном $p\in\mathcal{P}$
\begin{gather*}
\|A[f](p)\|_Y\leqslant\|A\|_{X\to C_s(\mathcal{P},Y)}\|f\|_X\,\,\,\forall\,f\in X.
\end{gather*}
Иными словами, при каждом фиксированном $p$ отображение
\begin{gather*}
X\ni f\mapsto A[f](p)\in Y
\end{gather*}
является линейным ограниченным оператором, действующим из $X$ в $Y$. Обозначив этот оператор через $B(p)$, получаем представление \eqref{L.X->Cs(P,Y)!representation}. Непрерывность функции
$B:\mathcal{P}\to\mathcal{L}(X,Y)$ в смысле слабой операторной топологии пространства $\mathcal{L}(X,Y)$ следует из того, что при всех $f\in X$ справедливо включение
$A[f]\in C_s(\mathcal{P},Y)$.

3) Докажем равенство \eqref{L.X->Cs(P,Y)!norm}:
\begin{gather*}
\|A\|_{X\to C_s(\mathcal{P},Y)}=\sup\limits_{\|f\|_X\leqslant1}\sup\limits_{p\in\mathcal{P}}\|A[f](p)\|_Y=\sup\limits_{p\in\mathcal{P}}\sup\limits_{\|f\|_X\leqslant1}\|B(p)f\|_Y=
\sup\limits_{p\in\mathcal{P}}\|B(p)\|_{X\to Y},
\end{gather*}
что и требовалось доказать.
\end{Proof}

\begin{Corrolary}\label{L.L1([a,b],X)->Cs(P,Y)!theorem.auxiliary}
Пусть пространство $X$ --- сепарабельно. Если оператор $A:L_1([a,b],X)\to C_s(\mathcal{P},Y)$ --- линеен и ограничен (считаем, что пространство $C_s(\mathcal{P},Y)$ наделено сильной
нормой), то найдётся функция $B:\mathcal{P}\to\mathcal{L}(L_1([a,b],X),Y)$, непрерывная в смысле слабой операторной топологии пространства $\mathcal{L}(L_1([a,b],X),Y)$, такая, что
\begin{gather}\label{L.L1([a,b],X)->Cs(P,Y)!representation.auxiliary}
A[f](p)=B(p)f\,\,\,\forall\,f\in L_1([a,b],X)\,\,\forall\,p\in\mathcal{P}.
\end{gather}
При этом
\begin{gather}\label{L.L1([a,b],X)->Cs(P,Y)!norm.auxiliary}
\|A\|_{L_1([a,b],X)\to C_s(\mathcal{P},Y)}=\sup\limits_{p\in\mathcal{P}}\|B(p)\|_{L_1([a,b],X)\to Y}.
\end{gather}
Обратно, если функция $B:\mathcal{P}\to\mathcal{L}(L_1([a,b],X),Y)$ --- непрерывна в смысле слабой операторной топологии пространства $\mathcal{L}(L_1([a,b],X),Y)$, то формула
\eqref{L.L1([a,b],X)->Cs(P,Y)!representation.auxiliary} задаёт линейный непрерывный оператор, действующий из $L_1([a,b],X)$ в $C_s(\mathcal{P},Y)$.
\end{Corrolary}

\begin{Theorem}\label{L.L1([a,b],X)->Y!theorem}
Пусть пространства $X$ и $Y$ --- рефлексивны и сепарабельны.

Если оператор $A:L_1([a,b],X)\to Y$ --- линеен и ограничен, то найдётся функция $B:[a,b]\to\mathcal{L}(X,Y)$, такая, что при всех $x\in X$ отображение
\begin{gather}\label{L.L1([a,b],X)->Y![a,b]nit->B(t)x}
[a,b]\ni t\mapsto (B(t)x)\in Y
\end{gather}
слабо измеримо, конечна величина
\begin{gather}\label{L.L1([a,b],X)->Y!vraisup.norm(B(t))}
\vraisup\limits_{t\in[a,b]}\|B(t)\|_{X\to Y},
\end{gather}
и справедливо представление
\begin{gather}\label{L.L1([a,b],X)->Y!representation}
A[f]=(\textrm{Б})\int\limits_a^bB(t)f(t)dt\,\,\,\forall\,f\in L_1([a,b],X).
\end{gather}
При этом
\begin{gather}\label{L.L1([a,b],X)->Y!norm}
\|A\|_{L_1([a,b],X)\to Y}=\vraisup\limits_{t\in[a,b]}\|B(t)\|_{X\to Y}.
\end{gather}
Обратно, если функция $B:[a,b]\to\mathcal{L}(X,Y)$ обладает указанными выше свойствами, то формула \eqref{L.L1([a,b],X)->Y!representation} задаёт линейный непрерывный оператор, действующий
из $L_1([a,b],X)$ в $Y$.
\end{Theorem}
\begin{Proof}
Разобьём доказательство на две части.

1) Докажем, что если функция $B:[a,b]\to\mathcal{L}(X,Y)$ обладает указанными выше свойствами, то формула \eqref{L.L1([a,b],X)->Y!representation} задаёт линейный непрерывный оператор,
действующий из $L_1([a,b],X)$ в $Y$. В самом деле, поскольку при любом $x\in X$ слабо измеримо отображение \eqref{L.L1([a,b],X)->Y![a,b]nit->B(t)x}, а пространства $X$ и $Y$ сепарабельны, то
сильно измеримо отображение
\begin{gather}\label{L.L1([a,b],X)->Y![a,b]nit->B(t)f(t)}
[a,b]\ni t\mapsto (B(t)f(t))\in Y.
\end{gather}
Далее, в силу непрерывности оператора $B(t)$ и конечности величины \eqref{L.L1([a,b],X)->Y!vraisup.norm(B(t))} заключаем, что при всех $f\in L_1([a,b],X)$ и п.в. $t\in[a,b]$
\begin{gather}\label{L.L1([a,b],X)->Y!vraisup.norm(B(t)ft))}
\|B(t)f(t)\|_Y\leqslant[\vraisup\limits_{\tau\in[a,b]}\|B(\tau)\|_Y]\|f(t)\|_X.
\end{gather}
Поэтому, в силу суммируемости функции $f$, суммируема и функция \eqref{L.L1([a,b],X)->Y![a,b]nit->B(t)f(t)}, так что формула \eqref{L.L1([a,b],X)->Y!representation} действительно задаёт
оператор, действующий из $L_1([a,b],X)$ в $Y$. Линейность этого оператора очевидна. Докажем его ограниченность. Из оценки \eqref{L.L1([a,b],X)->Y!vraisup.norm(B(t)ft))}, равенства
\eqref{L.L1([a,b],X)->Y!representation} и свойств интеграла Бохнера следует, что
\begin{gather*}
\|A[f]\|_Y\leqslant[\vraisup\limits_{\tau\in[a,b]}\|B(\tau)\|_Y]\|f\|_{1,[a,b],X}\,\,\,\forall\,f\in L_1([a,b],X),
\end{gather*}
что и означает ограниченность оператора $A$.

2) Докажем, что если оператор $A:L_1([a,b],X)\to Y$ --- линеен и ограничен, то найдётся функция $B:[a,b]\to\mathcal{L}(X,Y)$, такая, что при всех $x\in X$ отображение
\eqref{L.L1([a,b],X)->Y![a,b]nit->B(t)x} --- слабо измеримо, конечна величина \eqref{L.L1([a,b],X)->Y!vraisup.norm(B(t))}, справедливо представление
\eqref{L.L1([a,b],X)->Y!representation}, и имеет место равенство \eqref{L.L1([a,b],X)->Y!norm}.

Выберем произвольно $y^*\in Y^*$ и зафиксируем. Тогда отображение
\begin{gather}\label{L.L1([a,b],X)->Y!fniL1([a,b],X)->A[f].y*map}
L_1([a,b],X)\ni f\mapsto \langle A[f],y^*\rangle,
\end{gather}
является линейным непрерывным функционалом над $L_1([a,b],X)$, вследствие чего найдётся функция $g_{y^*}\in L_\infty([a,b],X^*)$, такая, что
\begin{gather}\label{L.L1([a,b],X)->Y!fniL1([a,b],X)->A[f].y*map-representation}
\langle A[f],y^*\rangle=\int\limits_a^b\langle f(t),g_{y^*}(t)\rangle dt\,\,\,\forall\,y^*\in Y^*\,\,\forall\,f\in L_1([a,b],X),
\end{gather}
причём
\begin{gather}\label{L.L1([a,b],X)->Y!fniL1([a,b],X)->A[f].y*map-norm}
\sup\limits_{\|f\|_{1,[a,b],X}\leqslant1}\langle A[f],y^*\rangle=\vraisup\limits_{t\in[a,b]}\|g_{y^*}(t)\|_{X^*}\,\,\,\forall\,y^*\in Y^*.
\end{gather}

Покажем, что при п.в. $t\in[a,b]$ отображение
\begin{gather}\label{L.L1([a,b],X)->Y!g[cdot](t)map}
Y^*\ni y^*\mapsto g_{y^*}(t)\in X^*
\end{gather}
линейно.

Действительно, если $y_1^*$, $y_2^*\in Y^*$, то, с одной стороны, в силу \eqref{L.L1([a,b],X)->Y!fniL1([a,b],X)->A[f].y*map-representation},
\begin{gather}\label{L.L1([a,b],X)->Y!g[y1*+y2*]1}
\langle A[f],y^*_1+y^*_2\rangle=\int\limits_a^b\langle f(t),g_{y^*_1+y^*_2}(t)\rangle dt.
\end{gather}
С другой стороны, в силу \eqref{L.L1([a,b],X)->Y!fniL1([a,b],X)->A[f].y*map-representation},
\begin{gather}\label{L.L1([a,b],X)->Y!g[y1*+y2*]2}
\langle A[f],y^*_1+y^*_2\rangle=\langle A[f],y^*_1\rangle+\langle A[f],y^*_2\rangle=\int\limits_a^b\langle f(t),g_{y^*_1}(t)\rangle dt+\int\limits_a^b\langle f(t),g_{y^*_2}(t)\rangle dt.
\end{gather}
Вычитая \eqref{L.L1([a,b],X)->Y!g[y1*+y2*]2} из \eqref{L.L1([a,b],X)->Y!g[y1*+y2*]1}, будем иметь
\begin{gather*}
0=\int\limits_a^b\langle f(t),g_{y^*_1+y^*_2}(t)-g_{y^*_1}(t)-g_{y^*_2}(t)\rangle dt\,\,\,\forall\,f\in L_1([a,b],X),
\end{gather*}
ввиду чего
\begin{gather}\label{L.L1([a,b],X)->Y!g[y1*+y2*]=g[y1*]+g[y2*]}
g_{y^*_1+y^*_2}(t)=g_{y^*_1}(t)+g_{y^*_2}(t)\,\,\,\text{при п.в. $t\in[a,b]$, }\forall\,y^*_1,\,y_2^*\in Y^*.
\end{gather}

Кроме того, если $y^*\in Y^*$, $\lambda\in\mathbb{R}$, то, с одной стороны, согласно \eqref{L.L1([a,b],X)->Y!fniL1([a,b],X)->A[f].y*map-representation},
\begin{gather}\label{L.L1([a,b],X)->Y!g[lambda.y*]1}
\langle A[f],\lambda y^*\rangle=\int\limits_a^b\langle f(t),g_{\lambda y^*}(t)\rangle dt.
\end{gather}
С другой стороны, в силу \eqref{L.L1([a,b],X)->Y!fniL1([a,b],X)->A[f].y*map-representation},
\begin{gather}\label{L.L1([a,b],X)->Y!g[lambda.y*]2}
\langle A[f],\lambda y^*\rangle=\lambda\langle A[f],y^*\rangle=\lambda\int\limits_a^b\langle f(t),g_{y^*}(t)\rangle dt=\int\limits_a^b\langle f(t),\lambda g_{y^*}(t)\rangle dt.
\end{gather}
Вычитая \eqref{L.L1([a,b],X)->Y!g[lambda.y*]2} из \eqref{L.L1([a,b],X)->Y!g[lambda.y*]1}, выводим, что
\begin{gather*}
0=\int\limits_a^b\langle f(t),g_{\lambda y^*}(t)-\lambda g_{y^*}(t)\rangle dt\,\,\,\forall\,f\in L_1([a,b],X).
\end{gather*}
Поэтому
\begin{gather}\label{L.L1([a,b],X)->Y!g[lambda.y*]=lambda.g[y*]}
g_{\lambda y^*}(t)=\lambda g_{y^*}(t)\,\,\,\text{при п.в. $t\in[a,b]$, }\forall\,y^*\in Y^*\,\,\forall\,\lambda\in\mathbb{R}.
\end{gather}

Из \eqref{L.L1([a,b],X)->Y!g[y1*+y2*]=g[y1*]+g[y2*]} и \eqref{L.L1([a,b],X)->Y!g[lambda.y*]=lambda.g[y*]} следует, что при почти всех $t\in[a,b]$ отображение
\eqref{L.L1([a,b],X)->Y!g[cdot](t)map} --- линейно. Обозначим это отображение через $\Lambda(t)$.

Докажем, что при почти всех $t\in[a,b]$ оператор $\Lambda(t):Y^*\to X^*$ --- ограничен.

Так как оператор $A$ --- линеен и ограничен, то конечна величина $\|A\|_{L_1([a,b],X)\to Y}\equiv\sup\limits_{\|f\|_{1,[a,b],X}\leqslant1}\|A[f]\|_Y$, и, значит,
\begin{gather*}
\|A\|_{L_1([a,b],X)\to Y}\equiv\sup\limits_{\|f\|_{1,[a,b],X}\leqslant1}\|A[f]\|_Y=\sup\limits_{\|f\|_{1,[a,b],X}\leqslant1}\sup\limits_{\|y^*\|_{Y^*}\leqslant1}|\langle A[f],y^*\rangle|=
\sup\limits_{\|y^*\|_{Y^*}\leqslant1}\sup\limits_{\|f\|_{1,[a,b],X}\leqslant1}|\langle A[f],y^*\rangle|,
\end{gather*}
откуда, пользуясь \eqref{L.L1([a,b],X)->Y!fniL1([a,b],X)->A[f].y*map-norm}, выводим, что
\begin{gather*}
\|A\|_{L_1([a,b],X)\to Y}=\sup\limits_{\|y^*\|_{Y^*}\leqslant1}\vraisup\limits_{t\in[a,b]}\|\Lambda(t)y^*\|_{X^*}=
\vraisup\limits_{t\in[a,b]}\sup\limits_{\|y^*\|_{Y^*}\leqslant1}\|\Lambda(t)y^*\|_{X^*}.
\end{gather*}
Отсюда следует, что, во--первых, $\Lambda(t)\in\mathcal{L}(Y^*,X^*)$ при п.в. $t\in[a,b]$; во--вторых, конечна величина $$\vraisup\limits_{t\in[a,b]}\|\Lambda(t)\|_{Y^*\to X^*};$$ и,
в--третьих, что
\begin{gather}\label{L.L1([a,b],X)->Y!norm2}
\|A\|_{L_1([a,b],X)\to Y}=\vraisup\limits_{t\in[a,b]}\|\Lambda(t)\|_{Y^*\to X^*}.
\end{gather}

Итак, мы доказали, что найдётся функция $\Lambda:[a,b]\to\mathcal{L}(Y^*,X^*)$, такая, что справедливо представление
\begin{gather}\label{L.L1([a,b],X)->Y!fniL1([a,b],X)->A[f].y*map-representation2}
\langle A[f],y^*\rangle=\int\limits_a^b\langle f(t),\Lambda(t)y^*\rangle dt\,\,\,\forall\,y^*\in Y^*\,\,\forall\,f\in L_1([a,b],X),
\end{gather}

Поскольку $\Lambda(t)$ --- линейный ограниченный оператор, то для него существует сопряжённый оператор $B(t)\in\mathcal{L}(X^{**},Y^{**})$, причём
$$
\|\Lambda(t)\|_{Y^*\to X^*}=\|B(t)\|_{X^{**}\to Y^{**}}.
$$
Поскольку пространства $X$ и $Y$ --- рефлексивны, то можно считать, что $B(t)\in\mathcal{L}(X,Y)$.

Поэтому равенства \eqref{L.L1([a,b],X)->Y!norm2} и \eqref{L.L1([a,b],X)->Y!fniL1([a,b],X)->A[f].y*map-representation2} можно переписать в виде следующих соотношений:
\begin{gather}
\label{L.L1([a,b],X)->Y!norm3}
\|A\|_{L_1([a,b],X)\to Y}=\vraisup\limits_{t\in[a,b]}\|B(t)\|_{X\to Y},\\
\label{L.L1([a,b],X)->Y!fniL1([a,b],X)->A[f].y*map-representation3}
\langle A[f],y^*\rangle=\int\limits_a^b\langle B(t)f(t),y^*\rangle dt\,\,\,\forall\,y^*\in Y^*\,\,\forall\,f\in L_1([a,b],X).
\end{gather}

Ясно, что равенство \eqref{L.L1([a,b],X)->Y!norm3} представляет собою соотношение \eqref{L.L1([a,b],X)->Y!norm}.

Далее, из \eqref{L.L1([a,b],X)->Y!fniL1([a,b],X)->A[f].y*map-representation3} следует, что при всех $y^*\in Y^*$
\begin{gather*}
\left\langle A[f]-\int\limits_a^bB(t)f(t)dt, y^*\right\rangle=0,
\end{gather*}
откуда, пользуясь теоремой Хана--Банаха, получаем представление \eqref{L.L1([a,b],X)->Y!representation}.
\end{Proof}

\begin{Theorem}\label{L.L1([a,b],X)->Cs(P,Y)!theorem}
Пусть пространства $X$ и $Y$ --- рефлексивны и сепарабельны.

Если оператор $A:L_1([a,b],X)\to C_s(\mathcal{P},Y)$ --- линеен и непрерывен (считаем, что пространство
$C_s(\mathcal{P},Y)$ наделено сильной нормой), то найдётся функция $F:\mathcal{P}\times[a,b]\to\mathcal{L}(X,Y)$, такая, что

1) при всех $p\in\mathcal{P}$ отображение
    \begin{gather}\label{L.L1([a,b],X)->Cs(P,Y)!weak.measurability.wrt.t}
    [a,b]\ni t\mapsto [F(p,t)x]\in Y
    \end{gather}
слабо измеримо при каждом фиксированном $x\in X$;

2) конечна величина
\begin{gather}\label{L.L1([a,b],X)->Cs(P,Y)!vraisup.wrt.t}
\sup\limits_{p\in\mathcal{P}}\vraisup\limits_{t\in[a,b]}\|F(p,t)\|_{X\to Y};
\end{gather}

3) при любом $p_0\in\mathcal{P}$
\begin{gather}\label{L.L1([a,b],X)->Cs(P,Y)!limit.wrt.p.of.int.a.bF(p,t)f(t)dt}
\int\limits_a^bF(p,t)f(t)dt\to\int\limits_a^bF(p_0,t)f(t)dt\text{ слабо в $Y$ при $p\to p_0$ }\forall\,f\in L_1([a,b],X);
\end{gather}

4) справедливо представление
    \begin{gather}\label{L.L1([a,b],X)->Cs(P,Y)!representation}
A[f](p)=(\textrm{Б})\int\limits_a^bF(p,t)f(t)dt\,\,\,\forall\,f\in L_1([a,b],X)\,\,\forall\,p\in \mathcal{P}.
    \end{gather}

При этом
\begin{gather}\label{L.L1([a,b],X)->Cs(P,Y)!norm}
\|A\|_{L_1([a,b],X)\to C_s(\mathcal{P},Y)}=\sup\limits_{p\in\mathcal{P}}\vraisup\limits_{t\in[a,b]}\|F(p,t)\|_{X\to Y}.
\end{gather}

Обратно, если функция $F:\mathcal{P}\times[a,b]\to\mathcal{L}(X,Y)$ обладает свойствами 1)--3), то формула \eqref{L.L1([a,b],X)->Cs(P,Y)!representation} задаёт линейный непрерывный
оператор, действующий из $L_1([a,b],X)$ в $C_s(\mathcal{P},Y)$.
\end{Theorem}
\begin{Proof}
1) Покажем, что если функция $F:\mathcal{P}\times[a,b]\to\mathcal{L}(X,Y)$ обладает свойствами 1)--3), то формула \eqref{L.L1([a,b],X)->Cs(P,Y)!representation} задаёт линейный непрерывный
оператор, действующий из $L_1([a,b],X)$ в $C_s(\mathcal{P},Y)$.

В самом деле, из первого свойства и сепарабельности пространства $Y$ следует, что отображение
\begin{gather*}
[a,b]\ni t\mapsto [F(p,t)f(t)]\in Y
\end{gather*}
сильно измеримо при всех $p\in \mathcal{P}$, $f\in L_1([a,b],X)$. Кроме того, при почти всех $t\in[a,b]$
\begin{gather*}
\|F(p,t)f(t)\|_Y\leqslant[\sup\limits_{q\in\mathcal{P}}\vraisup\limits_{t\in[a,b]}\|F(q,t)\|_{X\to Y}]\|f(t)\|_X.
\end{gather*}
Поэтому интеграл в правой части равенства \eqref{L.L1([a,b],X)->Cs(P,Y)!representation} существует при всех $p\in\mathcal{P}$. То, что при всех $f\in L_1([a,b],X)$ выполнено включение
$A[f]\in C_s(\mathcal{P},Y)$, следует из третьего свойства. Итак, оператор $A$ действует из $L_1([a,b],X)$ в $C_s(\mathcal{P},Y)$. Линейность этого оператора очевидна. Докажем
ограниченность:
\begin{gather*}
\|A[f](p)\|_Y\leqslant\int\limits_a^b\|F(p,t)f(t)\|_Ydt\leqslant[\sup\limits_{q\in\mathcal{P}}\vraisup\limits_{t\in[a,b]}\|F(q,t)\|_{X\to Y}]\|f\|_{1,[a,b],X}.
\end{gather*}
Переходя здесь к точной верхней грани по $p\in \mathcal{P}$, будем иметь
\begin{gather*}
\|A[f]\|_{C_s(\mathcal{P},Y)}\leqslant[\sup\limits_{q\in\mathcal{P}}\vraisup\limits_{t\in[a,b]}\|F(q,t)\|_{X\to Y}]\|f\|_{1,[a,b],X}\,\,\,\forall\,f\in L_1([a,b],X),
\end{gather*}
что и означает непрерывность оператора $A$.

2) Докажем теперь, что если оператор $A:L_1([a,b],X)\to C_s(\mathcal{P},Y)$ --- линеен и непрерывен, то найдётся функция $F:\mathcal{P}\times[a,b]\to\mathcal{L}(X,Y)$, обладающая
свойствами 1)--3), такая, что справедливы представление \eqref{L.L1([a,b],X)->Cs(P,Y)!representation} и равенство \eqref{L.L1([a,b],X)->Cs(P,Y)!norm}.

Действительно, поскольку оператор $A$ --- линеен и ограничен, то, согласно следствию \ref{L.L1([a,b],X)->Cs(P,Y)!theorem.auxiliary}, найдётся функция
$E:\mathcal{P}\to\mathcal{L}(L_1([a,b],X),Y)$, непрерывная в смысле слабой операторной топологии пространства $\mathcal{L}(L_1([a,b],X),Y)$, такая, что
\begin{gather}
\label{L.L1([a,b],X)->Cs(P,Y)!representation2}
A[f](p)=E(p)f\,\,\,\forall\,f\in L_1([a,b],X)\,\,\forall\,p\in \mathcal{P};\\
\label{L.L1([a,b],X)->Cs(P,Y)!norm2}
\|A\|_{L_1([a,b],X)\to C_s(\mathcal{P},Y)}=\sup\limits_{p\in\mathcal{P}}\|E(p)\|_{L_1([a,b],X)\to Y}.
\end{gather}

Выберем произвольно $p\in\mathcal{P}$ и зафиксируем. Поскольку $E(p)\in\mathcal{L}(L_1([a,b],X),Y)$, то, согласно теореме \ref{L.L1([a,b],X)->Y!theorem}, найдётся функция
$B_p:[a,b]\to\mathcal{L}(X,Y)$, такая, что

а) при всех $x\in X$ слабо измеримо отображение
\begin{gather*}
[a,b]\ni t\mapsto (B_p(t)x)\in Y;
\end{gather*}

б) конечна величина
\begin{gather*}
\vraisup\limits_{t\in[a,b]}\|B_p(t)\|_{X\to Y};
\end{gather*}

в) справедливо представление
\begin{gather*}
E(p)f=(\textrm{Б})\int\limits_a^bB_p(t)f(t)dt\,\,\,\forall\,p\in\mathcal{P}\,\,\forall\,f\in L_1([a,b],X);
\end{gather*}

г) имеет место равенство
\begin{gather*}
\|E(p)\|_{L_1([a,b],X)\to Y}=\vraisup\limits_{t\in[a,b]}\|B_p(t)\|_{X\to Y}.
\end{gather*}

Положив $F(p,t)\equiv B_p(t)$, $p\in\mathcal{P}$, $t\in[a,b]$, получим требуемое утверждение. Теорема полностью доказана.
\end{Proof}

\begin{Theorem}\label{L.X->Linfty([a,b],Y)!theorem}
Пусть пространство $Y$ --- сепарабельно. Если оператор $A:X\to L_\infty([a,b],Y)$ --- линеен и непрерывен, то найдётся функция $B:[a,b]\to\mathcal{L}(X,Y)$, такая, что

1) при всех $x\in X$ слабо измеримо отображение
\begin{gather}\label{L.X->Linfty([a,b],Y)![a,b]nit->B(t)x-map}
[a,b]\ni t\mapsto[B(t)x]\in Y;
\end{gather}

2) конечна величина
\begin{gather}\label{L.X->Linfty([a,b],Y)!vraisup.tin[a,b].norm(B(t))}
\vraisup\limits_{t\in[a,b]}\|B(t)\|_{X\to Y};
\end{gather}

3) справедливо представление
\begin{gather}\label{L.X->Linfty([a,b],Y)!representation}
A[f](t)=B(t)f\,\,\,\text{при п.в. $t\in[a,b]$ }\forall\,f\in X;
\end{gather}

4) имеет место равенство
\begin{gather}\label{L.X->Linfty([a,b],Y)!norm}
\|A\|_{X\to L_\infty([a,b],Y)}=\vraisup\limits_{t\in[a,b]}\|B(t)\|_{X\to Y}.
\end{gather}

Обратно, если функция $B:[a,b]\to\mathcal{L}(X,Y)$ обладает свойствами 1) и 2), то формула \eqref{L.X->Linfty([a,b],Y)!representation} задаёт линейный ограниченный оператор, действующий из
$X$ в $L_\infty([a,b],Y)$.
\end{Theorem}
\begin{Proof}
Доказательство разобьём на две части.

1) Докажем, что если $B:[a,b]\to\mathcal{L}(X,Y)$ обладает свойствами 1) и 2), то формула \eqref{L.X->Linfty([a,b],Y)!representation} задаёт линейный ограниченный оператор, действующий из
$X$ в $L_\infty([a,b],Y)$. Отметим, что из свойств 1)--2) следует корректность определения оператора $A$. Линейность оператора $A$ --- очевидна. Докажем его ограниченность.
Действительно, при почти всех $t\in[a,b]$, в силу свойства 2),
\begin{gather*}
\|A[f](t)\|_Y=\|B(t)f\|_Y\leqslant[\vraisup\limits_{\tau\in[a,b]}\|B(\tau)\|_{X\to Y}]\|f\|_X;
\end{gather*}
откуда вытекает, что
\begin{gather*}
\vraisup\limits_{t\in[a,b]}\|A[f](t)\|_Y=\|B(t)f\|_Y\leqslant[\vraisup\limits_{\tau\in[a,b]}\|B(\tau)\|_{X\to Y}]\|f\|_X.
\end{gather*}
А это и означает ограниченность оператора $A$.

2) Покажем, что если оператор $A:X\to L_\infty([a,b],Y)$ --- линеен и непрерывен, то найдётся функция $B:[a,b]\to\mathcal{L}(X,Y)$, обладающая свойствами 1) и 2), и такая, что справедливы
представление \eqref{L.X->Linfty([a,b],Y)!representation} и равенство \eqref{L.X->Linfty([a,b],Y)!norm}.

В самом деле, если оператор $A:X\to L_\infty([a,b],Y)$ --- линеен и непрерывен, то при всех $f\in X$
\begin{gather*}
\vraisup\limits_{t\in[a,b]}\|A[f](t)\|_Y\leqslant\|A\|_{X\to L_\infty([a,b],Y)}\|f\|_X.
\end{gather*}
Следовательно, при почти всех $t\in[a,b]$
\begin{gather*}
\|A[f](t)\|_Y\leqslant\|A\|_{X\to L_\infty([a,b],Y)}\|f\|_X.
\end{gather*}
Иными словами, при почти всех $t\in[a,b]$ отображение
\begin{gather*}
X\ni f\mapsto[A[f](t)]\in Y
\end{gather*}
является линейным ограниченным оператором, действующим из $X$ в $Y$. Обозначив этот оператор через $B(t)$, получаем представление \eqref{L.X->Linfty([a,b],Y)!representation}.

Далее, поскольку оператор $A$ --- линеен и ограничен, то конечна величина $$\|A\|_{X\to L_\infty([a,b],Y)}\equiv\sup\limits_{\|f\|_X\leqslant1}\|A[f]\|_{\infty,[a,b],Y}.$$ Поэтому
\begin{align*}
\|A\|_{X\to L_\infty([a,b],Y)}&\equiv\sup\limits_{f\in X}\|A[f]\|_{\infty,[a,b],Y}=\sup\limits_{\|f\|_X\leqslant1}\vraisup\limits_{t\in[a,b]}\|A[f](t)\|_{Y}=\\
&=\sup\limits_{\|f\|_X\leqslant1}\vraisup\limits_{t\in[a,b]}\|B(t)f\|_{Y}=\vraisup\limits_{t\in[a,b]}\sup\limits_{\|f\|_X\leqslant1}\|B(t)f\|_{Y}=
\vraisup\limits_{t\in[a,b]}\|B(t)\|_{X\to Y}.
\end{align*}
Отсюда следует, что, во--первых, конечна величина \eqref{L.X->Linfty([a,b],Y)!vraisup.tin[a,b].norm(B(t))}; и, во--вторых, что справедливо равенство \eqref{L.X->Linfty([a,b],Y)!norm}.
\end{Proof}

\begin{Theorem}\label{L.L1([a,b],X)->Linfty([c,d],Y)!theorem}
Пусть пространства $X$ и $Y$ --- рефлексивны и сепарабельны. Если оператор $A:L_1([a,b],X)\to L_\infty([c,d],Y)$ --- линеен и непрерывен, то найдётся функция
$F:[c,d]\times[a,b]\to\mathcal{L}(X,Y)$, такая, что

1) при всех $x\in X$ слабо измеримо отображение
\begin{gather}\label{L.L1([a,b],X)->Linfty([c,d],Y)!(t,tau)->F(t,tau)x-map}
[c,d]\times[a,b]\ni(t,\tau)\mapsto [F(t,\tau)x]\in Y;
\end{gather}

2) конечна величина
\begin{gather}\label{L.L1([a,b],X)->Linfty([c,d],Y)!vraisup.wrt.t-and-tau}
\vraisup\limits_{(t,\tau)\in[c,d]\times[a,b]}\|F(t,\tau)\|_{X\to Y};
\end{gather}

3) справедливо представление
\begin{gather}\label{L.L1([a,b],X)->Linfty([c,d],Y)!representation}
A[f](t)=(\textrm{Б})\int\limits_a^bF(t,\tau)f(\tau)d\tau\,\,\,\text{при п.в. $t\in[c,d]$, }\forall\,f\in L_1([a,b],X);
\end{gather}

4) имеет место равенство
\begin{gather}\label{L.L1([a,b],X)->Linfty([c,d],Y)!norm}
\|A\|_{L_1([a,b],X)\to L_\infty([c,d],Y)}=\vraisup\limits_{(t,\tau)\in[c,d]\times[a,b]}\|F(t,\tau)\|_{X\to Y}.
\end{gather}

Обратно, если функция $F:[c,d]\times[a,b]\to\mathcal{L}(X,Y)$ обладает свойствами 1) и 2), то формула \eqref{L.L1([a,b],X)->Linfty([c,d],Y)!representation} задаёт линейный непрерывный
оператор, действующий из $L_1([a,b],X)$ в $L_\infty([c,d],Y)$.
\end{Theorem}
\begin{Proof}
1) Докажем, что если функция $F:[c,d]\times[a,b]\to\mathcal{L}(X,Y)$ обладает свойствами 1) и 2), то формула \eqref{L.L1([a,b],X)->Linfty([c,d],Y)!representation} задаёт линейный
непрерывный оператор, действующий из $L_1([a,b],X)$ в $L_\infty([c,d],Y)$.

В самом деле, пусть это так. Тогда при почти всех $t\in[c,d]$ измеримо отображение
\begin{gather*}
[a,b]\ni\tau\mapsto [F(t,\tau)f(\tau)]\in Y.
\end{gather*}
Далее, при почти всех $(t,\tau)\in[c,d]\times[a,b]$, в силу свойства 2),
\begin{gather}\label{L.L1([a,b],X)->Linfty([c,d],Y)!estimate.of.norm(F(t,tau)f(tau))}
\|F(t,\tau)f(\tau)\|_Y\leqslant[\vraisup\limits_{(t',\tau')\in[c,d]\times[a,b]}\|F(t',\tau')\|_{X\to Y}]\|f(\tau)\|_X.
\end{gather}
Поэтому, ввиду суммируемости функции $f$, интеграл в правой части формулы \eqref{L.L1([a,b],X)->Linfty([c,d],Y)!representation} имеет смысл.

Из неравенства \eqref{L.L1([a,b],X)->Linfty([c,d],Y)!estimate.of.norm(F(t,tau)f(tau))} следует, что при почти всех $t\in[c,d]$
\begin{gather*}
\|A[f](t)\|_Y\leqslant[\vraisup\limits_{(t',\tau')\in[c,d]\times[a,b]}\|F(t',\tau')\|_{X\to Y}]\|f\|_{1,[a,b],X},
\end{gather*}
откуда вытекает, во--первых, что оператор $A$ корректно определён; и, во--вторых, что этот оператор ограничен. Линейность же оператора $A$ очевидна.

2) Докажем, что если оператор $A:L_1([a,b],X)\to L_\infty([c,d],Y)$ --- линеен и непрерывен, то найдётся функция $F:[c,d]\times[a,b]\to\mathcal{L}(X,Y)$, обладающая свойствами 1)--2),
и такая, что справедливы представление \eqref{L.L1([a,b],X)->Linfty([c,d],Y)!representation} и равенство \eqref{L.L1([a,b],X)->Linfty([c,d],Y)!norm}.

В самом деле, на основании теоремы \ref{L.X->Linfty([a,b],Y)!theorem}, найдётся функция $E:[c,d]\to\mathcal{L}(L_1([a,b],X),Y)$, такая, что

а) при всех $f\in L_1([a,b],X)$ слабо измеримо отображение
\begin{gather*}
[c,d]\ni t\mapsto[E(t)f]\in Y;
\end{gather*}

б) конечна величина
\begin{gather*}
\vraisup\limits_{t\in[a,b]}\|E(t)\|_{L_1([a,b],X)\to Y};
\end{gather*}

в) справедливо представление
\begin{gather*}
A[f](t)=E(t)f\,\,\,\text{при п.в. $t\in[c,d]$ }\forall\,f\in L_1([a,b],X);
\end{gather*}

г) имеет место равенство
\begin{gather*}
\|A\|_{L_1([a,b],X)\to L_\infty([c,d],Y)}=\vraisup\limits_{t\in[a,b]}\|E(t)\|_{L_1([a,b],X)\to Y}.
\end{gather*}

Выберем теперь произвольно $t\in[c,d]$ и зафиксируем. Поскольку $E(t)\in\mathcal{L}(L_1([a,b],X),Y)$, то, согласно теореме \ref{L.L1([a,b],X)->Y!theorem}, найдётся функция
$B_t:[a,b]\to\mathcal{L}(X,Y)$, такая, что

д) при каждом фиксированном $x\in X$ слабо измеримо отображение
\begin{gather*}
[a,b]\ni\tau\mapsto [B_t(\tau)x]\in Y;
\end{gather*}

е) конечна величина
\begin{gather*}
\vraisup\limits_{\tau\in[a,b]}\|B_t(\tau)\|_{X\to Y};
\end{gather*}

ё) справедливо представление
\begin{gather*}
E(t)f=(\textrm{Б})\int\limits_a^bB_t(\tau)f(\tau)d\tau\,\,\,\text{при п.в. $t\in[c,d]$ }\forall\,f\in L_1([a,b],X);
\end{gather*}

ж) выполняется равенство
\begin{gather*}
\|E(t)\|_{L_1([a,b],X)\to Y}=\vraisup\limits_{\tau\in[a,b]}\|B_t(\tau)\|_{X\to Y}.
\end{gather*}

Положив затем $F(t,\tau)\equiv B_t(\tau)$, $(t,\tau)\in[c,d]\times[a,b]$, получаем требуемое утверждение.
\end{Proof}

        \section{Применение абстрактных теорем к энергетическим классам}
В данном разделе мы выводим представления линейных непрерывных операторов, определённых на пространстве суммируемых по Бохнеру функций, и принимающих значения в энергетических классах.

Пусть $V$ и $H$ --- сепарабельные гильбертовы пространства со скалярными произведениями $\langle\cdot,\cdot\rangle_V$ и $\langle\cdot,\cdot\rangle_H$ соответственно, с соответствующими
нормами $\|\cdot\|_V$ и $\|\cdot\|_H$, $V\subset H$, и это вложение непрерывно. Иными словами, найдётся постоянная $\nu>0$, такая, что
\begin{gather*}
\|v\|_H\leqslant\nu\|v\|_V\,\,\,\forall\,v\in V.
\end{gather*}
%причём любое ограниченное в норме $V$ множество предкомпактно в норме $H$.

\begin{Theorem}\label{L.L1([0,T],H)->En([0,T];V,H)!theorem}
Пусть оператор $A:L_1([0,T],H)\to\textrm{Э}([0,T];V,H)$ --- линеен и непрерывен. Тогда найдётся функция $\Psi:\Gamma\to\mathcal{L}(V,H)$, такая, что

1) при каждом фиксированном $h\in H$ и всех $t\in[0,T]$ измеримо отображение
\begin{gather}\label{L.L1([0,T],H)->En([0,T];V,H)!tau->Psi(t,tau)h}
[0,T]\ni\tau\mapsto[\Psi(t,\tau)h]\in V;
\end{gather}

2) конечна величина
\begin{gather}\label{L.L1([0,T],H)->En([0,T];V,H)!vraisup.wrt.tau.of.norm(Psi(t,tau))}
\sup\limits_{t\in[0,T]}\vraisup\limits_{\tau\in[0,T]}\|\Psi(t,\tau)\|_{H\to V};
\end{gather}

3) при любом $t_0\in[0,T]$
\begin{gather}\label{L.L1([0,T],H)->En([0,T];V,H)!weak.continuity}
\int\limits_0^T\Psi(t,\tau)f(\tau)d\tau\to\int\limits_0^T\Psi(t_0,\tau)f(\tau)d\tau\,\,\,\text{слабо в $V$ при $t\to t_0$ }\forall\,f\in L_1([0,T],H);
\end{gather}

4) при почти всех $\tau\in[0,T]$ и при каждом фиксированном $h\in H$ функция
\begin{gather}\label{L.L1([0,T],H)->En([0,T];V,H)!t->Psi(t,tau)h}
[0,T]\ni t\mapsto[\Psi(t,\tau)h]\in V
\end{gather}
является элементом пространства $W^1_\infty([0,T],H)$ и $\Psi_t(t,\tau)\in\mathcal{L}(H,H)$ при всех $(t,\tau)\in\Gamma$;

5) при всех фиксированных $h\in H$ измеримо отображение
\begin{gather}\label{L.L1([0,T],H)->En([0,T];V,H)!(t,tau)->Psi.t(t,tau)h}
\Gamma\ni(t,\tau)\mapsto[\Psi_t(t,\tau)h]\in H;
\end{gather}

6) конечна величина
\begin{gather}\label{L.L1([0,T],H)->En([0,T];V,H)!vraisup.wrt.t.tau.of.norm(Psi.t(t,tau))}
\vraisup\limits_{(t,\tau)\in\Gamma}\|\Psi_t(t,\tau)\|_{H\to H};
\end{gather}

7) справедливы представления
\begin{gather}\label{L.L1([0,T],H)->En([0,T];V,H)!representations}
A[f](t)=\int\limits_a^b\Psi(t,\tau)f(\tau)d\tau,\,\,\,\frac{dA[f](t)}{dt}=\int\limits_a^b\Psi_t(t,\tau)f(\tau)d\tau\,\,\,\text{при п.в. $t\in[0,T]$ }\forall\,f\in L_1([0,T],H);
\end{gather}

8) имеет место неравенство
\begin{gather}\label{L.L1([0,T],H)->En([0,T];V,H)!norm}
\|A\|_{L_1([0,T],H)\to\textrm{Э}([0,T];V,H)}\leqslant\sup\limits_{t\in[0,T]}\vraisup\limits_{\tau\in[0,T]}\|\Psi(t,\tau)\|_{H\to V}+
\vraisup\limits_{(t,\tau)\in\Gamma}\|\Psi_t(t,\tau)\|_{H\to H}.
\end{gather}

Обратно, если функция $\Psi:\Gamma\to\mathcal{L}(V,H)$ обладает свойствами 1)--6), то оператор $A$, задаваемый соотношениями \eqref{L.L1([0,T],H)->En([0,T];V,H)!representations},
является линейным ограниченным оператором, действующим из $L_1([0,T],H)$ в $\textrm{Э}([0,T];V,H)$.
\end{Theorem}
\begin{Proof}
1) Пусть оператор $A:L_1([0,T],H)\to\textrm{Э}([0,T];V,H)$ --- линеен и непрерывен.

Введём операторы $B:L_1([0,T],H)\to C_s([0,T],V)$ и $E:L_1([0,T],H)\to L_\infty([0,T],H)$ равенствами
\begin{gather*}
B[f](t)=A[f](t),\,\,\,E[f](t)=\frac{dA[f](t)}{dt}\,\,\,\text{при п.в. $t\in[0,T]$ }\forall\,f\in L_1([0,T],H).
\end{gather*}

Нетрудно видеть, что операторы $B$ и $E$ --- линейны и непрерывны. Поэтому, согласно теоремам \ref{L.L1([a,b],X)->Cs(P,Y)!theorem} и \ref{L.L1([a,b],X)->Linfty([c,d],Y)!theorem}, найдутся
функции $F:\Gamma\to\mathcal{L}(H,V)$ и $G:\Gamma\to\mathcal{L}(H,H)$, такие, что

а) при всех фиксированных $t\in[0,T]$ и $h\in H$ измеримо отображение
\begin{gather*}
[0,T]\ni\tau\mapsto[F(t,\tau)h]\in V;
\end{gather*}

б) при всех фиксированных $h\in H$ измеримо отображение
\begin{gather*}
\Gamma\ni(t,\tau)\mapsto[G(t,\tau)h]\in H;
\end{gather*}

в) конечны величины
$$
\sup\limits_{t\in[0,T]}\vraisup\limits_{\tau\in[0,T]}\|F(t,\tau)\|_{H\to V}
$$
и
$$
\vraisup\limits_{(t,\tau)\in\Gamma}\|\Psi_t(t,\tau)\|_{H\to H};
$$

г) при любом $t_0\in[0,T]$
$$
\int\limits_0^TF(t,\tau)f(\tau)d\tau\to\int\limits_0^TF(t_0,\tau)f(\tau)d\tau\,\,\,\text{слабо в $V$ при $t\to t_0$ }\forall\,f\in L_1([0,T],H);
$$

д) имеют место представления
\begin{gather*}
B[f](t)=\int\limits_a^bF(t,\tau)f(\tau)d\tau,\,\,\,E[f](t)=\int\limits_a^bG(t,\tau)f(\tau)d\tau\,\,\,\text{при п.в. $t\in[0,T]$ }\forall\,f\in L_1([0,T],H);
\end{gather*}

е) справедливы равенства
\begin{gather*}
\|B\|_{L_1([0,T],H)\to C_s([0,T],V)}=\sup\limits_{t\in[0,T]}\vraisup\limits_{\tau\in[0,T]}\|F(t,\tau)\|_{H\to V},\,\,\,
\|E\|_{L_1([0,T],H)\to L_\infty([0,T],H)}=\vraisup\limits_{(t,\tau)\in\Gamma}\|G(t,\tau)\|_{H\to H}.
\end{gather*}

Далее, в силу определения класса $\textrm{Э}([0,T];V,H)$, операторов $B$ и $E$, и того, что оператор $A$ принимает значения в $\textrm{Э}([0,T];V,H)$, вытекает, что
\begin{gather*}
\int\limits_0^TE[f](t)\varphi(t)dt=-\int\limits_0^TB[f](t)\varphi'(t)dt\,\,\,\forall\,\varphi\in\mathfrak{D}(0,T)\,\,\forall\,f\in L_1([0,T],H).
\end{gather*}
Подставив сюда представления операторов $B$ и $E$, выводим, что
\begin{gather*}
\int\limits_\Gamma G(t,\tau)f(\tau)\varphi(t)dtd\tau=-\int\limits_\Gamma F(t,\tau)f(\tau)\varphi'(t)dtd\tau\,\,\,\forall\,\varphi\in\mathfrak{D}(0,T)\,\,\forall\,f\in L_1([0,T],H),
\end{gather*}
откуда следует, что
\begin{gather*}
\int\limits_0^T\left[\int\limits_0^TG(t,\tau)f(\tau)\varphi(t)dt\right]d\tau=-\int\limits_0^T\left[\int\limits_0^TF(t,\tau)f(\tau)\varphi'(t)dt\right]d\tau\,\,\,
\forall\,\varphi\in\mathfrak{D}(0,T)\,\,\forall\,f\in L_1([0,T],H).
\end{gather*}
Ограничившись в данном соотношении функциями $f$ вида $f(\tau)\equiv h\psi(\tau)$, $\tau\in[0,T]$, где $h\in H$, $\psi\in\mathfrak{D}(0,T)$, заключаем, что
\begin{gather*}
\int\limits_0^T\left[\int\limits_0^TG(t,\tau)h\varphi(t)dt+\int\limits_0^TF(t,\tau)h\varphi'(t)dt\right]\psi(\tau)d\tau=0\,\,\,\forall\,\varphi,\,\,\psi\in\mathfrak{D}(0,T).
\end{gather*}
Это означает, что при почти всех $t\in[0,T]$
\begin{gather*}
\int\limits_0^TG(t,\tau)h\varphi(t)dt=-\int\limits_0^TF(t,\tau)h\varphi'(t)dt\,\,\,\forall\,\varphi\in\mathfrak{D}(0,T).
\end{gather*}
Таким образом, в качестве функции $\Psi$ можно взять функцию $F$.

Докажем теперь неравенство \eqref{L.L1([0,T],H)->En([0,T];V,H)!norm}. В самом деле, при всех $f\in L_1([0,T],H)$
\begin{align*}
\|A[f]\|_{\textrm{Э}([0,T];V,H)}&=\|B[f]\|_{C_s([0,T],V)}+\|E[f]\|_{\infty,[0,T],H}\leqslant\\
                                &\leqslant[\sup\limits_{t\in[0,T]}\vraisup\limits_{\tau\in[0,T]}\|\Psi(t,\tau)\|_{H\to V}+
                                           \vraisup\limits_{(t,\tau)\in\Gamma}\|\Psi_t(t,\tau)\|_{H\to H}]\|f\|_{1,[0,T],H}.
\end{align*}
Перейдя здесь к точной верхней грани по $f\in L_1([0,T],H)$, получаем требуемое неравенство.

2) Утверждение о том, что если функция $\Psi:\Gamma\to\mathcal{L}(V,H)$ обладает свойствами 1)--6), то оператор $A$, задаваемый соотношениями
\eqref{L.L1([0,T],H)->En([0,T];V,H)!representations}, является линейным ограниченным оператором, действующим из $L_1([0,T],H)$ в $\textrm{Э}([0,T];V,H)$, вытекает непосредственно из
этих свойств.
\end{Proof}

Введём теперь пространство $\textrm{Є}([0,T];V,H)$ как множество функций $f\in \textrm{Э}([0,T];V,H)$, у которых $\dot{f}\in C_s([0,T],H)$. Норму в классе $\textrm{Є}([0,T];V,H)$
определим равенством
$$
\|f\|_{\textrm{Є}([0,T];V,H)}=\sup\limits_{t\in[0,T]}\sqrt{\|f(t)\|^2_V+\|\dot{f}(t)\|^2_H}.
$$
Нетрудно показать, что введённое пространство является банаховым.


\begin{Theorem}\label{L.L1([0,T],H)->En2([0,T];V,H)!theorem}
Пусть оператор $A:L_1([0,T],H)\to\textrm{Є}([0,T];V,H)$ --- линеен и непрерывен. Тогда найдётся функция $\Psi:\Gamma\to\mathcal{L}(H,V)$, такая, что

1) при каждом фиксированном $h\in H$ и всех $t\in[0,T]$ измеримо отображение
\begin{gather}\label{L.L1([0,T],H)->En2([0,T];V,H)!tau->Psi(t,tau)h}
[0,T]\ni\tau\mapsto[\Psi(t,\tau)h]\in V;
\end{gather}

2) конечна величина
\begin{gather}\label{L.L1([0,T],H)->En2([0,T];V,H)!vraisup.wrt.tau.of.norm(Psi(t,tau))}
\sup\limits_{t\in[0,T]}\vraisup\limits_{\tau\in[0,T]}\|\Psi(t,\tau)\|_{H\to V};
\end{gather}

3) при любом $t_0\in[0,T]$
\begin{gather}\label{L.L1([0,T],H)->En2([0,T];V,H)!weak.continuity.Psi}
\int\limits_0^T\Psi(t,\tau)f(\tau)d\tau\to\int\limits_0^T\Psi(t_0,\tau)f(\tau)d\tau\,\,\,\text{слабо в $V$ при $t\to t_0$ }\forall\,f\in L_1([0,T],H);
\end{gather}

4) при почти всех $\tau\in[0,T]$ и при каждом фиксированном $h\in H$ функция
\begin{gather}\label{L.L1([0,T],H)->En2([0,T];V,H)!t->Psi(t,tau)h}
[0,T]\ni t\mapsto[\Psi(t,\tau)h]\in V
\end{gather}
является элементом пространства $W^1_\infty([0,T],H)$ и $\Psi_t(t,\tau)\in\mathcal{L}(H,H)$ при всех $(t,\tau)\in\Gamma$;

5) при каждом фиксированном $h\in H$ и всех $t\in[0,T]$ измеримо отображение
\begin{gather}\label{L.L1([0,T],H)->En2([0,T];V,H)!tau->Psi.t(t,tau)h}
[0,T]\ni\tau\mapsto[\Psi_t(t,\tau)h]\in H;
\end{gather}

6) конечна величина
\begin{gather}\label{L.L1([0,T],H)->En2([0,T];V,H)!vraisup.wrt.tau.of.norm(Psi.t(t,tau))}
\sup\limits_{t\in[0,T]}\vraisup\limits_{\tau\in[0,T]}\|\Psi_t(t,\tau)\|_{H\to H};
\end{gather}

7) при любом $t_0\in[0,T]$
\begin{gather}\label{L.L1([0,T],H)->En2([0,T];V,H)!weak.continuity}
\int\limits_0^T\Psi_t(t,\tau)f(\tau)d\tau\to\int\limits_0^T\Psi_t(t_0,\tau)f(\tau)d\tau\,\,\,\text{слабо в $H$ при $t\to t_0$ }\forall\,f\in L_1([0,T],H);
\end{gather}

8) справедливы представления
\begin{gather}\label{L.L1([0,T],H)->En2([0,T];V,H)!representations}
A[f](t)=\int\limits_a^b\Psi(t,\tau)f(\tau)d\tau,\,\,\,\frac{dA[f](t)}{dt}=\int\limits_a^b\Psi_t(t,\tau)f(\tau)d\tau\,\,\,\text{при п.в. $t\in[0,T]$ }\forall\,f\in L_1([0,T],H);
\end{gather}

9) имеет место неравенство
\begin{gather}\label{L.L1([0,T],H)->En2([0,T];V,H)!norm}
\|A\|_{L_1([0,T],H)\to\textrm{Є}([0,T];V,H)}\leqslant[\sup\limits_{t\in[0,T]}\vraisup\limits_{\tau\in[0,T]}\|\Psi(t,\tau)\|^2_{H\to V}+
\sup\limits_{t\in[0,T]}\vraisup\limits_{\tau\in[0,T]}\|\Psi_t(t,\tau)\|^2_{H\to H}]^{1/2}.
\end{gather}

Обратно, если функция $\Psi:\Gamma\to\mathcal{L}(V,H)$ обладает свойствами 1)--7), то оператор $A$, задаваемый соотношениями \eqref{L.L1([0,T],H)->En2([0,T];V,H)!representations},
является линейным ограниченным оператором, действующим из $L_1([0,T],H)$ в $\textrm{Є}([0,T];V,H)$.
\end{Theorem}
\begin{Proof}
1) Пусть оператор $A:L_1([0,T],H)\to\textrm{Є}([0,T];V,H)$ --- линеен и непрерывен.

Введём операторы $B:L_1([0,T],H)\to C_s([0,T],V)$ и $E:L_1([0,T],H)\to C_s([0,T],H)$ равенствами
\begin{gather*}
B[f](t)=A[f](t),\,\,\,E[f](t)=\frac{dA[f](t)}{dt}\,\,\,\text{при п.в. $t\in[0,T]$ }\forall\,f\in L_1([0,T],H).
\end{gather*}

Нетрудно видеть, что операторы $B$ и $E$ --- линейны и непрерывны. Поэтому, согласно теореме \ref{L.L1([a,b],X)->Cs(P,Y)!theorem}, найдутся функции $F:\Gamma\to\mathcal{L}(H,V)$ и
$G:\Gamma\to\mathcal{L}(H,H)$, такие, что

а) при всех фиксированных $t\in[0,T]$ и $h\in H$ измеримы отображения
\begin{gather*}
[0,T]\ni\tau\mapsto[F(t,\tau)h]\in V,\,\,\,[0,T]\ni\tau\mapsto[G(t,\tau)h]\in H;
\end{gather*}

б) конечны величины
$$
\sup\limits_{t\in[0,T]}\vraisup\limits_{\tau\in[0,T]}\|F(t,\tau)\|_{H\to V},\,\,\,\sup\limits_{t\in[0,T]}\vraisup\limits_{\tau\in[0,T]}\|G(t,\tau)\|_{H\to H};
$$

в) при любом $t_0\in[0,T]$
\begin{gather*}
\int\limits_0^TF(t,\tau)f(\tau)d\tau\to\int\limits_0^TF(t_0,\tau)f(\tau)d\tau\,\,\,\text{слабо в $V$ при $t\to t_0$, }\\
\int\limits_0^TG(t,\tau)f(\tau)d\tau\to\int\limits_0^TG(t_0,\tau)f(\tau)d\tau\,\,\,\text{слабо в $H$ при $t\to t_0$, }\forall\,f\in L_1([0,T],H);
\end{gather*}

г) имеют место представления
\begin{gather*}
B[f](t)=\int\limits_a^bF(t,\tau)f(\tau)d\tau,\,\,\,E[f](t)=\int\limits_a^bG(t,\tau)f(\tau)d\tau\,\,\,\text{при п.в. $t\in[0,T]$ }\forall\,f\in L_1([0,T],H);
\end{gather*}

е) справедливы равенства
\begin{gather*}
\|B\|_{L_1([0,T],H)\to C_s([0,T],V)}=\sup\limits_{t\in[0,T]}\vraisup\limits_{\tau\in[0,T]}\|F(t,\tau)\|_{H\to V},\\
\|E\|_{L_1([0,T],H)\to L_\infty([0,T],H)}=\sup\limits_{t\in[0,T]}\vraisup\limits_{\tau\in[0,T]}\|G(t,\tau)\|_{H\to H}.
\end{gather*}

Далее, из определения класса $\textrm{Є}([0,T];V,H)$, операторов $B$ и $E$, и того, что оператор $A$ принимает значения в $\textrm{Є}([0,T];V,H)$, вытекает, что
\begin{gather*}
\int\limits_0^TE[f](t)\varphi(t)dt=-\int\limits_0^TB[f](t)\varphi'(t)dt\,\,\,\forall\,\varphi\in\mathfrak{D}(0,T)\,\,\forall\,f\in L_1([0,T],H).
\end{gather*}
Подставив сюда представления операторов $B$ и $E$, выводим, что
\begin{gather*}
\int\limits_\Gamma G(t,\tau)f(\tau)\varphi(t)dtd\tau=-\int\limits_\Gamma F(t,\tau)f(\tau)\varphi'(t)dtd\tau\,\,\,\forall\,\varphi\in\mathfrak{D}(0,T)\,\,\forall\,f\in L_1([0,T],H),
\end{gather*}
откуда следует, что
\begin{gather*}
\int\limits_0^T\left[\int\limits_0^TG(t,\tau)f(\tau)\varphi(t)dt\right]d\tau=-\int\limits_0^T\left[\int\limits_0^TF(t,\tau)f(\tau)\varphi'(t)dt\right]d\tau\,\,\,
\forall\,\varphi\in\mathfrak{D}(0,T)\,\,\forall\,f\in L_1([0,T],H).
\end{gather*}
Ограничившись в данном соотношении функциями $f$ вида $f(\tau)\equiv h\psi(\tau)$, $\tau\in[0,T]$, где $h\in H$, $\psi\in\mathfrak{D}(0,T)$, заключаем, что
\begin{gather*}
\int\limits_0^T\left[\int\limits_0^TG(t,\tau)h\varphi(t)dt+\int\limits_0^TF(t,\tau)h\varphi'(t)dt\right]\psi(\tau)d\tau=0\,\,\,\forall\,\varphi,\,\,\psi\in\mathfrak{D}(0,T).
\end{gather*}
Это означает, что при почти всех $t\in[0,T]$
\begin{gather*}
\int\limits_0^TG(t,\tau)h\varphi(t)dt=-\int\limits_0^TF(t,\tau)h\varphi'(t)dt\,\,\,\forall\,\varphi\in\mathfrak{D}(0,T).
\end{gather*}
Таким образом, в качестве функции $\Psi$ можно взять функцию $F$.

Докажем теперь неравенство \eqref{L.L1([0,T],H)->En2([0,T];V,H)!norm}. В самом деле, при всех $f\in L_1([0,T],H)$
\begin{align*}
\|A[f]\|_{\textrm{Э}([0,T];V,H)}&=\sup\limits_{t\in[0,T]}[\|B[f](t)\|^2_{V}+\|E[f](t)\|^2_{H}]^{1/2}\leqslant\\
                                &\leqslant[\sup\limits_{t\in[0,T]}\vraisup\limits_{\tau\in[0,T]}\|\Psi(t,\tau)\|^2_{H\to V}+
                                           \sup\limits_{t\in[0,T]}\vraisup\limits_{\tau\in[0,T]}\|\Psi_t(t,\tau)\|^2_{H\to H}]^{1/2}\|f\|_{1,[0,T],H}.
\end{align*}
Перейдя здесь к точной верхней грани по $f\in L_1([0,T],H)$, получаем требуемое неравенство.

2) Утверждение о том, что если функция $\Psi:\Gamma\to\mathcal{L}(V,H)$ обладает свойствами 1)--7), то оператор $A$, задаваемый соотношениями
\eqref{L.L1([0,T],H)->En2([0,T];V,H)!representations}, является линейным ограниченным оператором, действующим из $L_1([0,T],H)$ в $\textrm{Є}([0,T];V,H)$, вытекает непосредственно из
этих свойств.
\end{Proof}

\begin{Theorem}\label{L.V->En2([0,T];V,H)!theorem}
Пусть оператор $A:V\to\textrm{Є}([0,T];V,H)$ --- линеен и непрерывен. Тогда найдётся функция $\Phi:[0,T]\to\mathcal{L}(V,V)$, такая, что

1) при каждом фиксированном $v\in V$ отображение
\begin{gather}\label{L.V->En2([0,T];V,H)!t->Phi(t)v}
[0,T]\ni t\mapsto[\Phi(t)v]\in V;
\end{gather}
непрерывно в слабой топологии пространства $V$, принадлежит классу  $W^1_\infty([0,T],H)$,  и при всех $t\in[0,T]$ имеет место включение  $\Phi'(t)\in\mathcal{L}(V,H)$;

2) при каждом фиксированном $h\in H$ отображение
\begin{gather}\label{L.V->En2([0,T];V,H)!t->Phi.t(t)v}
[0,T]\ni t\mapsto[\Phi'(t)v]\in H
\end{gather}
непрерывно в слабой топологии пространства $H$;

3) конечны величины
\begin{gather}\label{L.V->En2([0,T];V,H)!sup.wrt.t.of.norm(Phi(t))}
\sup\limits_{t\in[0,T]}\|\Phi(t)\|_{V\to V};
\end{gather}
и
\begin{gather}\label{L.V->En2([0,T];V,H)!sup.wrt.t.of.norm(Phi'(t))}
\sup\limits_{t\in[0,T]}\|\Phi'(t)\|_{V\to H};
\end{gather}

4) справедливы представления
\begin{gather}\label{L.V->En2([0,T];V,H)!representations}
A[v](t)=\Phi(t)v,\,\,\,\frac{dA[v](t)}{dt}=\Phi'(t)v\,\,\,\forall\,v\in V;
\end{gather}

5) имеет место неравенство
\begin{gather}\label{L.V->En2([0,T];V,H)!norm}
\|A\|_{V\to\textrm{Є}([0,T];V,H)}\leqslant[\sup\limits_{t\in[0,T]}\|\Phi(t)\|^2_{V\to V}+\sup\limits_{t\in[0,T]}\|\Phi'(t)\|^2_{V\to H}]^{1/2}.
\end{gather}

Обратно, если функция $\Phi:[0,T]\to\mathcal{L}(V,V)$ обладает свойствами 1)--3), то оператор $A$, задаваемый соотношениями~\eqref{L.V->En2([0,T];V,H)!representations}, является линейным ограниченным оператором, действующим из $V$ в $\textrm{Є}([0,T];V,H)$.
\end{Theorem}
\begin{Proof}
1) Пусть оператор $A:V\to\textrm{Є}([0,T];V,H)$ --- линеен и непрерывен.

Введём операторы $B:V\to C_s([0,T],V)$ и $E:V\to C_s([0,T],H)$ равенствами
\begin{gather*}
B[v](t)=A[v](t),\,\,\,E[v](t)=\frac{dA[v](t)}{dt}\,\,\,\forall\,v\in V.
\end{gather*}

Нетрудно видеть, что операторы $B$ и $E$ --- линейны и непрерывны. Поэтому, согласно теореме \ref{L.X->Cs(P,Y)!theorem}, найдутся функции $F:[0,T]\to\mathcal{L}(V,V)$ и $G:[0,T]\to\mathcal{L}(V,H)$, непрерывные в слабых операторных топологиях соответствующих пространств, и такие, что
\begin{gather}
\label{B.V->Cs([0,T],V)!representation}
B[v](t)=F(t)v\,\,\,\forall\,v\in V\,\,\forall\,t\in[0,T];\\
\label{E.V->Cs([0,T],H)!representation}
E[v](t)=G(t)v\,\,\,\forall\,v\in V\,\,\forall\,t\in[0,T];\\
\label{L.V->En2([0,T];V,H)!B.norm}
\|B\|_{V->C_s([0,T],V)}=\sup\limits_{t\in[0,T]}\|F(t)\|_{V\to V};\\
\label{L.V->En2([0,T];V,H)!E.norm}
\|E\|_{V->C_s([0,T],H)}=\sup\limits_{t\in[0,T]}\|G(t)\|_{V\to H}.
\end{gather}

Далее, из определения класса $\textrm{Є}([0,T];V,H)$, операторов $B$ и $E$, и того, что оператор $A$ принимает значения в $\textrm{Є}([0,T];V,H)$, вытекает, что
\begin{gather*}
\int\limits_0^TE[v](t)\varphi(t)dt=-\int\limits_0^TB[v](t)\varphi'(t)dt\,\,\,\forall\,\varphi\in\mathfrak{D}(0,T)\,\,\forall\,v\in V.
\end{gather*}
Подставив сюда представления операторов $B$ и $E$, выводим, что
\begin{gather*}
\int\limits_0^TG(t)v\varphi(t)dt=-\int\limits_0^TF(t)v\varphi'(t)dt\,\,\,\forall\,\varphi\in\mathfrak{D}(0,T)\,\,\forall\,v\in V,
\end{gather*}
откуда следует, что в качестве функции $\Phi$ можно взять функцию $F$. 

Докажем теперь неравенство~\eqref{L.V->En2([0,T];V,H)!norm}. В самом деле,
\begin{align*}
\|A[v]\|_{\textrm{Є}([0,T];V,H)}&=\sup\limits_{t\in[0,T]}[\|B[v](t)\|^2_{V}+\|E[v](t)\|^2_{H}]^{1/2}
\leqslant[\sup\limits_{t\in[0,T]}\|\Phi(t)\|^2_{V\to V}+\sup\limits_{t\in[0,T]}\|\Phi'(t)\|^2_{V\to H}]^{1/2}\|v\|_{V}.
\end{align*}
Перейдя здесь к точной верхней грани по $v\in V$, у которых $\|v\|_V\leqslant1$, получаем требуемое неравенство.

2) Утверждение о том, что если функция $\Phi:[0,T]\to\mathcal{L}(V,V)$ обладает свойствами 1)--3), то оператор $A$, задаваемый соотношениями~\eqref{L.V->En2([0,T];V,H)!representations}, является линейным ограниченным оператором, действующим из $V$ в $\textrm{Є}([0,T];V,H)$, вытекает непосредственно из этих свойств.
\end{Proof}


\begin{Theorem}\label{L.H->En2([0,T];V,H)!theorem}
Пусть оператор $A:H\to\textrm{Є}([0,T];V,H)$ --- линеен и непрерывен. Тогда найдётся функция $\Psi:[0,T]\to\mathcal{L}(H,V)$, такая, что

1) при каждом фиксированном $h\in H$ отображение
\begin{gather}\label{L.H->En2([0,T];V,H)!t->Phi(t)v}
[0,T]\ni t\mapsto[\Psi(t)v]\in V;
\end{gather}
непрерывно в слабой топологии пространства $V$, принадлежит классу  $W^1_\infty([0,T],H)$,  и при всех $t\in[0,T]$ имеет место включение  $\Psi'(t)\in\mathcal{L}(H,H)$;

2) при каждом фиксированном $h\in H$ отображение
\begin{gather}\label{L.H->En2([0,T];V,H)!t->Phi.t(t)v}
[0,T]\ni t\mapsto[\Psi'(t)h]\in H
\end{gather}
непрерывно в слабой топологии пространства $H$;

3) конечны величины
\begin{gather}\label{L.H->En2([0,T];V,H)!sup.wrt.t.of.norm(Phi(t))}
\sup\limits_{t\in[0,T]}\|\Psi(t)\|_{H\to V};
\end{gather}
и
\begin{gather}\label{L.H->En2([0,T];V,H)!sup.wrt.t.of.norm(Phi'(t))}
\sup\limits_{t\in[0,T]}\|\Psi'(t)\|_{H\to H};
\end{gather}

4) справедливы представления
\begin{gather}\label{L.H->En2([0,T];V,H)!representations}
A[h](t)=\Psi(t)h,\,\,\,\frac{dA[h](t)}{dt}=\Psi'(t)h\,\,\,\forall\,h\in H;
\end{gather}

5) имеет место неравенство
\begin{gather}\label{L.H->En2([0,T];V,H)!norm}
\|A\|_{H\to\textrm{Є}([0,T];V,H)}\leqslant[\sup\limits_{t\in[0,T]}\|\Psi(t)\|^2_{H\to V}+\sup\limits_{t\in[0,T]}\|\Psi'(t)\|^2_{H\to H}]^{1/2}.
\end{gather}

Обратно, если функция $\Psi:[0,T]\to\mathcal{L}(H,V)$ обладает свойствами 1)--3), то оператор $A$, задаваемый соотношениями~\eqref{L.H->En2([0,T];V,H)!representations}, является линейным ограниченным оператором, действующим из $H$ в $\textrm{Є}([0,T];V,H)$.
\end{Theorem}
\begin{Proof}
1) Пусть оператор $A:H\to\textrm{Є}([0,T];V,H)$ --- линеен и непрерывен.

Введём операторы $B:H\to C_s([0,T],V)$ и $E:H\to C_s([0,T],H)$ равенствами
\begin{gather*}
B[v](t)=A[v](t),\,\,\,E[v](t)=\frac{dA[v](t)}{dt}\,\,\,\forall\,v\in V.
\end{gather*}

Нетрудно видеть, что операторы $B$ и $E$ --- линейны и непрерывны. Поэтому, согласно теореме \ref{L.X->Cs(P,Y)!theorem}, найдутся функции $F:[0,T]\to\mathcal{L}(H,V)$ и $G:[0,T]\to\mathcal{L}(H,H)$, непрерывные в слабых операторных топологиях соответствующих пространств, и такие, что
\begin{gather}
\label{B.H->Cs([0,T],V)!representation}
B[v](t)=F(t)v\,\,\,\forall\,v\in V\,\,\forall\,t\in[0,T];\\
\label{E.H->Cs([0,T],H)!representation}
E[v](t)=G(t)v\,\,\,\forall\,v\in V\,\,\forall\,t\in[0,T];\\
\label{L.H->En2([0,T];V,H)!B.norm}
\|B\|_{H->C_s([0,T],V)}=\sup\limits_{t\in[0,T]}\|F(t)\|_{H\to V};\\
\label{L.H->En2([0,T];V,H)!E.norm}
\|E\|_{H->C_s([0,T],H)}=\sup\limits_{t\in[0,T]}\|G(t)\|_{H\to H}.
\end{gather}

Далее, из определения класса $\textrm{Є}([0,T];V,H)$, операторов $B$ и $E$, и того, что оператор $A$ принимает значения в $\textrm{Є}([0,T];V,H)$, вытекает, что
\begin{gather*}
\int\limits_0^TE[v](t)\varphi(t)dt=-\int\limits_0^TB[v](t)\varphi'(t)dt\,\,\,\forall\,\varphi\in\mathfrak{D}(0,T)\,\,\forall\,v\in V.
\end{gather*}
Подставив сюда представления операторов $B$ и $E$, выводим, что
\begin{gather*}
\int\limits_0^TG(t)v\varphi(t)dt=-\int\limits_0^TF(t)v\varphi'(t)dt\,\,\,\forall\,\varphi\in\mathfrak{D}(0,T)\,\,\forall\,v\in V,
\end{gather*}
откуда следует, что в качестве функции $\Psi$ можно взять функцию $F$. 

Докажем теперь неравенство~\eqref{L.H->En2([0,T];V,H)!norm}. В самом деле,
\begin{align*}
\|A[v]\|_{\textrm{Є}([0,T];V,H)}&=\sup\limits_{t\in[0,T]}[\|B[v](t)\|^2_{V}+\|E[v](t)\|^2_{H}]^{1/2}
\leqslant[\sup\limits_{t\in[0,T]}\|\Phi(t)\|^2_{H\to V}+\sup\limits_{t\in[0,T]}\|\Phi'(t)\|^2_{H\to H}]^{1/2}\|v\|_{V}.
\end{align*}
Перейдя здесь к точной верхней грани по $v\in V$, у которых $\|v\|_V\leqslant1$, получаем требуемое неравенство.

2) Утверждение о том, что если функция $\Psi:[0,T]\to\mathcal{L}(H,V)$ обладает свойствами 1)--3), то оператор $A$, задаваемый соотношениями~\eqref{L.H->En2([0,T];V,H)!representations}, является линейным ограниченным оператором, действующим из $H$ в $\textrm{Є}([0,T];V,H)$, вытекает непосредственно из этих свойств.
\end{Proof}

Далее под $\mathfrak{R}([0,T]; V,H)$ будем понимать множество функций $\Psi:[0,T]\to\mathcal{L}(V,V)$, таких, что

1) при каждом фиксированном $v\in V$ отображение
\begin{gather*}
[0,T]\ni t\mapsto[\Psi(t)v]\in V;
\end{gather*}
непрерывно в слабой топологии пространства $V$, принадлежит классу  $W^1_\infty([0,T],H)$,  и при всех $t\in[0,T]$ имеет место включение  $\Psi'(t)\in\mathcal{L}(V,H)$;

2) при каждом фиксированном $v\in V$ отображение
\begin{gather*}
[0,T]\ni t\mapsto[\Psi'(t)v]\in H
\end{gather*}
непрерывно в слабой топологии пространства $H$.

\begin{Theorem}\label{L.H->C([0,T],En2([0,T];V,H))!theorem}
Пусть оператор $A:H\to C([0,T],\textrm{Є}([0,T];V,H))$ --- линеен и непрерывен. Тогда найдётся функция $\Psi:\Gamma\to\mathcal{L}(H,V)$, такая, что

1)  при каждом фиксированном $h\in H$ и каждом фиксированном $\tau\in[0,T]$ отображение
\begin{gather}\label{L.H->C([0,T],En2([0,T];V,H))!t->Psi(t,tau)h}
[0,T]\ni t\mapsto[\Psi(t,\tau)h]\in V;
\end{gather}
непрерывно в слабой топологии пространства $V$, принадлежит классу  $W^1_\infty([0,T],H)$,  и при всех $(t,\tau)\in\Gamma$ имеет место включение  $\Psi_t(t,\tau)\in\mathcal{L}(H,H)$; 

2) при каждом фиксированном $h\in H$ и каждом фиксированном $\tau\in[0,T]$ отображение
\begin{gather}\label{L.H->C([0,T],En2([0,T];V,H))!t->Psi.t(t,tau)h}
[0,T]\ni t\mapsto[\Psi_t(t,\tau)h]\in H
\end{gather}
непрерывно в слабой топологии пространства $H$;

3) конечны величины
\begin{gather}\label{L.H->C([0,T],En2([0,T];V,H))!sup.wrt.(t,tau).of.norm(Psi(t,tau))}
\sup\limits_{(t,\tau)\in\Gamma}\|\Psi(t,\tau)\|_{H\to V};
\end{gather}
и
\begin{gather}\label{L.H->C([0,T],En2([0,T];V,H))!sup.wrt.(t,tau).of.norm(Psi.t(t,tau))}
\sup\limits_{(t,\tau)\in\Gamma}\|\Psi_t(t,\tau)\|_{H\to H};
\end{gather}

4) при всех $h\in[0,T]$ и всех $\tau\in[0,T]$ выполнены соотношения
\begin{gather*}
\lim\limits_{\tau'\to\tau}\sup\limits_{t\in[0,T]}\|[\Psi(t,\tau')-\Psi(t,\tau)]h\|_V=0,\,\,\,
\lim\limits_{\tau'\to\tau}\sup\limits_{t\in[0,T]}\|[\Psi_t(t,\tau')-\Psi_t(t,\tau)]h\|_H=0;
\end{gather*}

5) справедливы представления
\begin{gather}\label{L.H->C([0,T],En2([0,T];V,H))!representations}
A[h](\tau)(t)=\Psi(t,\tau)h,\,\,\,\frac{d[A[h](\tau)](t)}{dt}=\Psi_t(t,\tau)h\,\,\,\forall\,h\in H;
\end{gather}

6)  имеет место неравенство
\begin{gather}\label{L.H->C([0,T],En2([0,T];V,H))!A.norm}
\|A\|_{H\to C([0,T],\textrm{Є}([0,T];V,H))}\leqslant[\sup\limits_{(t,\tau)\in\Gamma}\|\Psi(t,\tau)\|^2_{H\to V} +\sup\limits_{(t,\tau)\in\Gamma}\|\Psi_t(t,\tau)\|^2_{H\to H}]^{1/2}.
\end{gather}

Обратно, если функция $\Psi:\Gamma\to\mathcal{L}(H,V)$ обладает свойствами 1)--4), то оператор $A$, задаваемый соотношениями~\eqref{L.H->C([0,T],En2([0,T];V,H))!representations}, является линейным ограниченным оператором, действующим из $H$ в $C([0,T],\textrm{Є}([0,T];V,H))$. 
\end{Theorem}
\begin{Proof}
1) Пусть оператор $A:H\to C([0,T],\textrm{Є}([0,T];V,H))$ --- линеен и непрерывен. Тогда, на основании теоремы~\ref{L.X->C(P,Y)!theorem}, найдётся функция $B:[0,T]\to\mathcal{L}(H,\textrm{Є}([0,T];V,H))$, такая, что
\begin{gather}
\label{L.H->C([0,T],En2([0,T];V,H))!A.representation1}
A[h](\tau)=B(\tau)h,\,\,\,\forall\,h\in H,\,\,\,\tau\in[0,T];\\
\label{L.H->C([0,T],En2([0,T];V,H))!B.continuity}
\lim\limits_{\tau'\to\tau}\|[B(\tau')-B(\tau)]h\|_{\textrm{Є}([0,T];V,H)}=0,\,\,\,\forall\,h\in H,\,\,\,\tau\in[0,T];\\
\label{L.H->C([0,T],En2([0,T];V,H))!A.norm1}
\|A\|_{H\to C([0,T],\textrm{Є}([0,T];V,H))}=\sup\limits_{\tau\in[0,T]}\|B(\tau)\|_{H\to\textrm{Є}([0,T];V,H)}.
\end{gather}

Выберем произвольно $\tau\in[0,T]$ и зафиксируем. Тогда $B(\tau)\in \mathcal{L}(H,\textrm{Є}([0,T];V,H))$. Поэтому, на основании теоремы~\ref{L.H->En2([0,T];V,H)!theorem}, найдётся функция $E(\cdot;\tau):[0,T]\to\mathcal{L}(H,V)$, такая, что


{
\renewcommand{\theenumi}{\asbuk{enumi}}
\renewcommand{\labelenumi}{\theenumi)}
\begin{enumerate}
\item  при каждом фиксированном $h\in H$ отображение
\begin{gather}\label{L.H->C([0,T],En2([0,T];V,H))!t->E(t;tau)h}
[0,T]\ni t\mapsto[E(t;\tau)h]\in V;
\end{gather}
непрерывно в слабой топологии пространства $V$, принадлежит классу  $W^1_\infty([0,T],H)$,  и при всех $t\in[0,T]$ имеет место включение  $E_t(t;\tau)\in\mathcal{L}(H,H)$;

\item при каждом фиксированном $h\in H$ отображение
\begin{gather}\label{L.H->C([0,T],En2([0,T];V,H))!t->E.t(t;tau)h}
[0,T]\ni t\mapsto[E_t(t;\tau)h]\in H
\end{gather}
непрерывно в слабой топологии пространства $H$;

\item конечны величины
\begin{gather}\label{L.H->C([0,T],En2([0,T];V,H))!sup.wrt.t.of.norm(E(t;tau))}
\sup\limits_{t\in[0,T]}\|E(t;\tau)\|_{H\to V};
\end{gather}
и
\begin{gather}\label{L.H->C([0,T],En2([0,T];V,H))!sup.wrt.t.of.norm(E.t(t;tau))}
\sup\limits_{t\in[0,T]}\|E_t(t;\tau)\|_{H\to H};
\end{gather}

\item справедливы представления
\begin{gather}\label{L.H->C([0,T],En2([0,T];V,H))!B(tau).representation}
[B(\tau)h](t)=E(t;\tau)h,\,\,\,\frac{d[B(\tau)h](t)}{dt}=E_t(t;\tau)h,\,\,\,\forall\,h\in H,\,\,\,t\in[0,T];
\end{gather}

\item имеет место неравенство
\begin{gather}\label{L.H->C([0,T],En2([0,T];V,H))!B(tau).norm}
\|B(\tau)\|_{H\to\textrm{Є}([0,T];V,H)}\leqslant[\sup\limits_{t\in[0,T]}\|E(t;\tau)\|^2_{H\to V}+\sup\limits_{t\in[0,T]}\|E_t(t;\tau)\|^2_{H\to H}]^{1/2}.
\end{gather}
\end{enumerate}
}

Положив теперь $\Psi(t,\tau)\equiv E(t;\tau)$, получим требуемое.

2) Утверждение о том, что если функция $\Psi:\Gamma\to\mathcal{L}(H,V)$ обладает свойствами 1)--4), то оператор $A$, определённый соотношениями~\eqref{L.H->C([0,T],En2([0,T];V,H))!representations}, является линейным ограниченным оператором, действующим из $H$ в $C([0,T],\textrm{Є}([0,T];V,H))$, вытекает непосредственно из этих свойств.
\end{Proof}

Далее $\mathfrak{S}(\Gamma;V,H)$ будет обозначать множество функций $\Psi:\Gamma\to\mathcal{L}(H,V)$, таких, что

1)  при каждом фиксированном $h\in H$ и каждом фиксированном $\tau\in[0,T]$ отображение
\begin{gather*}
[0,T]\ni t\mapsto[\Psi(t,\tau)h]\in V;
\end{gather*}
непрерывно в слабой топологии пространства $V$, принадлежит классу  $W^1_\infty([0,T],H)$,  и при всех $(t,\tau)\in\Gamma$ имеет место включение  $\Psi_t(t,\tau)\in\mathcal{L}(H,H)$; 

2) при каждом фиксированном $h\in H$ и каждом фиксированном $\tau\in[0,T]$ отображение
\begin{gather*}
[0,T]\ni t\mapsto[\Psi_t(t,\tau)h]\in H
\end{gather*}
непрерывно в слабой топологии пространства $H$;

3) при всех $h\in[0,T]$ и всех $\tau\in[0,T]$ выполнены соотношения
\begin{gather*}
\lim\limits_{\tau'\to\tau}\sup\limits_{t\in[0,T]}\|[\Psi(t,\tau')-\Psi(t,\tau)]h\|_V=0,\,\,\,
\lim\limits_{\tau'\to\tau}\sup\limits_{t\in[0,T]}\|[\Psi_t(t,\tau')-\Psi_t(t,\tau)]h\|_H=0.
\end{gather*}

        \section{Абстрактное интегро--дифференциальное уравнение}
Пусть $V$ и $H$ --- сепарабельные гильбертовы пространства со скалярными произведениями $\langle\cdot,\cdot\rangle_V$ и $\langle\cdot,\cdot\rangle_H$ соответственно, с соответствующими
нормами $\|\cdot\|_V$ и $\|\cdot\|_H$, $V\subset H$, и это вложение непрерывно. Иными словами, найдётся постоянная $\nu>0$, такая, что
\begin{gather*}
\|v\|_H\leqslant\nu\|v\|_V\,\,\,\forall\,v\in V.
\end{gather*}

Рассмотрим интегро--дифференциальное уравнение
\begin{gather}\label{--integodifferential.equation}
\mathfrak{z}(t)=\omega(t)+\int\limits_0^t\Psi(t,\tau)g(\tau,\mathfrak{z}(\tau),\dot{\mathfrak{z}}(\tau))d\tau,\,\,\,t\in[0,T].
\end{gather}

Считаем, что выполнены следующие условия:
\begin{enumerate}
    \item функция $\omega:[0,T]\to V$ --- элемент класса $\textrm{Є}([0,T];V,H)$;

    \item функция $g:[0,T]\times V\times H\to H$ --- измерима по $t\in[0,T]$ при всех $(v,h)\in V\times H$;

    \item найдётся функция $K_0\in L_1[0,T]$, такая, что
    \begin{gather*}
    \|g(t,v_1,h_1)-g(t,v_2,h_2)\|_H\leqslant K_0(t)\sqrt{\|v_1-v_2\|^2_V+\|h_1-h_2\|^2_H}\,\,\,\forall\,(t,v_i,h_i)\in[0,T]\times V\times H,\,\,i=1,2;
    \end{gather*}

    \item найдётся функция $K_1\in L_1[0,T]$, такая, что
    \begin{gather*}
    \|g(t,0,0)\|_H\leqslant K_1(t)\text{ при п.в. $t\in[0,T]$};
    \end{gather*}

    \item функция $\Psi:\Gamma\to\mathcal{L}(H,H)$ такова, что формула
    \begin{gather*}
    A[f](t)=\int\limits_0^t\Psi(t,\tau)f(\tau)d\tau\,\,\,\text{при всех $t\in[0,T]$ }\forall\,f\in L_1([0,T],H),
    \end{gather*}
    задаёт линейный ограниченный оператор, действующий из $L_1([0,T],H)$ в $\textrm{Є}([0,T];V,H)$, причём
    \begin{gather*}
    \Psi(t,t)=0,\,\,\,\forall\,t\in[0,T].
    \end{gather*}
\end{enumerate}

\begin{Definition}
Функцию $\mathfrak{z}\in\textrm{Є}([0,T];V,H)$ назовём решением интегро--дифференциального уравнения \eqref{--integodifferential.equation}, если функция $\mathfrak{z}$ при всех
$t\in[0,T]$ удовлетворяет этому уравнению.
\end{Definition}

Основным результатом данного раздела является
\begin{Theorem}\label{--integodifferential.equation!theorem}
Существует единственное решение $\mathfrak{z}$ интегро--дифференциальное уравнения \eqref{--integodifferential.equation}, причём найдётся постоянная $B>0$, такая, что
\begin{gather}\label{--integodifferential.equation!apriori.estimate}
\|\mathfrak{z}\|_{\textrm{Є}([0,T];V,H)}\leqslant B[\|\omega\|_{\textrm{Є}([0,T];V,H)}+\|g(\cdot,0,0)\|_{1,[0,T],H}].
\end{gather}
\end{Theorem}
\begin{Proof}
1) Докажем вначале, что решение уравнения~\eqref{--integodifferential.equation} существует и единственно. Введём оператор $\Lambda:\textrm{Є}([0,T];V,H)\to\textrm{Є}([0,T];V,H)$ равенством
\begin{gather*}
\Lambda[\mathfrak{z}](t)=\omega(t)+\int\limits_0^t\Psi(t,\tau)g(\tau,\mathfrak{z}(\tau),\dot{\mathfrak{z}}(\tau))d\tau,\,\,\,t\in[0,T]\,\,\,\forall\,\mathfrak{z}\in\textrm{Є}([0,T];V,H).
\end{gather*}
Для доказательства существования и единственности решения уравнения~\eqref{--integodifferential.equation} нам достаточно показать, что некоторая степень этого оператора является сжимающим
отображением.

Прежде всего заметим, что при всех $\mathfrak{z}^1$, $\mathfrak{z}^2\in\textrm{Є}([0,T];V,H)$
\begin{gather*}
\|\Lambda[\mathfrak{z}^1](t)-\Lambda[\mathfrak{z}^2](t)\|_V\leqslant\int\limits_{0}^t K_0(\xi)M
\sqrt{\|\mathfrak{z}^1(\xi)-\mathfrak{z}^2(\xi)\|^2_V+\|\dot{\mathfrak{z}}^1(\xi)-\dot{\mathfrak{z}}^2(\xi)\|^2_H}d\xi,\\
\left\|\frac{d\Lambda[\mathfrak{z}^1](t)}{dt}-\frac{d\Lambda[\mathfrak{z}^2](t)}{dt}\right\|_H\leqslant\int\limits_{0}^t K_0(\xi)M
\sqrt{\|\mathfrak{z}^1(\xi)-\mathfrak{z}^2(\xi)\|^2_V+\|\dot{\mathfrak{z}}^1(\xi)-\dot{\mathfrak{z}}^2(\xi)\|^2_H}d\xi,\,\,\,\forall\,t\in[0,T],
\end{gather*}
где введено обозначение
$$
M\equiv\max\{\sup\limits_{t\in[0,T]}\vraisup\limits_{\tau\in[0,T]}\|\Psi(t,\tau)\|_{H\to V},
\sup\limits_{t\in[0,T]}\vraisup\limits_{\tau\in[0,T]}\|\Psi_t(t,\tau)\|_{H\to H}\}.
$$
Поэтому
\begin{gather*}
\sqrt{\|\Lambda[\mathfrak{z}^1](t)-\Lambda[\mathfrak{z}^2](t)\|_V^2+ \left\|\frac{d\Lambda[\mathfrak{z}^1](t)}{dt}-\frac{d\Lambda[\mathfrak{z}^2](t)}{dt}\right\|_H^2}\leqslant\\
\leqslant\int\limits_{0}^t2K_0(\xi)M\sqrt{\|\mathfrak{z}^1(\xi)-\mathfrak{z}^2(\xi)\|^2_V+\|\dot{\mathfrak{z}}^1(\xi)-\dot{\mathfrak{z}}^2(\xi)\|^2_H}\,d\xi.
\end{gather*}
Определив функцию $\sigma\colon\textrm{Є}([0,T];V,H)\to BF[0,T]$ равенством $$\sigma[y](t)\equiv\sqrt{\|y(t)\|^2_V+\|\dot y(t)\|^2_H},\,\,\,t\in[0,T],$$ получим, что
$$
\sigma[\Lambda[\mathfrak{z}^1]-\Lambda[\mathfrak{z}^2]](t)\leqslant2\int\limits_{0}^tK_0(\xi)M\sigma[\mathfrak{z}^1-\mathfrak{z}^2](\xi)\,d\xi,\,\,\,\forall\,t\in[0,T].
$$
Следовательно,
\begin{gather*}
\sigma[\Lambda^2[\mathfrak{z}^1]-\Lambda^2[\mathfrak{z}^2]](t)= \sigma[\Lambda[\Lambda_0[\mathfrak{z}^1]]-\Lambda[\Lambda[\mathfrak{z}^2]]](t)\leqslant
\int\limits_{0}^t2K_0(\xi)M\sigma[\Lambda[\mathfrak{z}^1]-\Lambda[\mathfrak{z}^2]](\xi)d\xi\leqslant\\
\leqslant\int\limits_{0}^t2K_0(\xi_1)M\left[\int\limits_{0}^{\xi_1}2K_0(\xi_2)M\sigma[\mathfrak{z}^1-\mathfrak{z}^2](\xi_2)d\xi_2\right]d\xi_1\leqslant\\
\leqslant\sup_{\xi\in[0,T]}\sigma[\mathfrak{z}^1-\mathfrak{z}^2](\xi)\int\limits_{0}^t2K_0(\xi_1)M\left[\int\limits_{0}^{\xi_1}2K_0(\xi_2)Md\xi_2\right]d\xi_1%=\\
=\sup_{\xi\in[0,T]}\sigma[\mathfrak{z}^1-\mathfrak{z}^2](\xi)\frac1{2!}\left[\int\limits_{0}^t2K_0(\xi)Md\xi \right]^2.
\end{gather*}
Итак,
\begin{gather*}
\sigma[\Lambda[\mathfrak{z}^1]-\Lambda[\mathfrak{z}^2]](t)\leqslant\sup_{\xi\in[0,T]}\sigma[\mathfrak{z}^1-\mathfrak{z}^2](\xi)\int\limits_{0}^t2K_0(\xi)Md\xi,\\
\sigma[\Lambda^2[\mathfrak{z}^1]-\Lambda^2[\mathfrak{z}^2]](t)\leqslant\sup_{\xi\in[0,T]}\sigma[\mathfrak{z}^1-\mathfrak{z}^2](\xi)\frac1{2!}
\left[\int\limits_{0}^t2K_0(\xi)Md\xi \right]^2\,\,\, \forall\,t\in[0,T].
\end{gather*}
Пусть для некоторого $m\geqslant1$ доказано, что
\begin{gather*}
\sigma[\Lambda^m[\mathfrak{z}^1]-\Lambda^m[\mathfrak{z}^2]](t)\leqslant
\sup_{\xi\in[0,T]}\sigma[\mathfrak{z}^1-\mathfrak{z}^2](\xi)\frac1{m!}\left[\int\limits_{0}^t2K_0(\xi)Md\xi \right]^m\,\,\,\forall\,t\in[0,T].
\end{gather*}
Тогда при всех $t\in[0,T]$
\begin{gather*}
\sigma[\Lambda^{m+1}[\mathfrak{z}^1]-\Lambda^{m+1}[\mathfrak{z}^2]](t)=\sigma[\Lambda[\Lambda^m[\mathfrak{z}^1]]-\Lambda[\Lambda^m[\mathfrak{z}^2]]](t)\leqslant
\int\limits_{0}^t2K_0(\xi)M\sigma[\Lambda^m[\mathfrak{z}^1]-\Lambda^m[\mathfrak{z}^2]](\xi)d\xi\leqslant\\
\leqslant\int\limits_{0}^t2K_0(\xi_1)M\Biggl[\sup_{\tau\in[0,T]}\sigma[\mathfrak{z}^1-\mathfrak{z}^2](\tau)\frac1{m!}\Biggl[\int\limits_{0}^{\xi_1}2K_0(\xi_2)Md\xi_2\Biggr]^m\Biggr]d\xi_1=\\
=\sup_{\tau\in[0,T]}\sigma[\mathfrak{z}^1-\mathfrak{z}^2](\tau)\frac1{(m+1)!}\left[\int\limits_{0}^t2K_0(\xi)Md\xi \right]^{m+1}.%\,\,\,\forall\,t\in[0,T].
\end{gather*}

Таким образом,
\begin{gather*}
\sigma[\Lambda^m[\mathfrak{z}^1]-\Lambda^m[\mathfrak{z}^2]](t)\leqslant\sup_{\xi\in[0,T]}\sigma[\mathfrak{z}^1-\mathfrak{z}^2](\xi)
\frac1{m!}\left[\int\limits_{0}^t2K_0(\xi)Md\xi\right]^m\,\,\,\forall\,t\in[0,T],\,\,\,m=1,2,\dots.
\end{gather*}
Отсюда выводим, что
\begin{gather*}
\|\Lambda^m[\mathfrak{z}^1]-\Lambda^m[\mathfrak{z}^2]\|_{\textrm{Є}([0,T];V,H)}\leqslant
\frac1{m!}\left[\int\limits_{0}^T2K_0(\xi)Md\xi \right]^m \|\mathfrak{z}^1-\mathfrak{z}^2\|_{\textrm{Є}([0,T];V,H)},\,\,\,m=1,2,\dots.
\end{gather*}
А это и означает, что некоторая степень оператора $\Lambda\colon\textrm{Є}([0,T];V,H)\to\textrm{Є}([0,T];V,H)$ является сжатием, что, в силу принципа неподвижной точки Банаха,
означает существование единственного решения уравнения~\eqref{--integodifferential.equation}.

2) Докажем оценку~(\ref{--integodifferential.equation!apriori.estimate}). В самом деле,
\begin{gather*}
\|\mathfrak{z}(t)\|_V\leqslant\|\omega(t)\|_V+\int\limits_0^t\|\Psi(t,\tau)\|_{H\to V}\|g(\tau,\mathfrak{z}(\tau),\dot{\mathfrak{z}}(\tau))\|_Hd\tau\leqslant
\|\omega\|_{\textrm{Є}([0,T];V,H)}+\int\limits_0^tM\|g(\tau,\mathfrak{z}(\tau),\dot{\mathfrak{z}}(\tau))\|_Hd\tau.
\end{gather*}
Аналогично получаем, что
\begin{gather*}
\|\dot{\mathfrak{z}}(t)\|_H\leqslant\|\dot\omega(t)\|_H+\int\limits_0^t\|\Psi_t(t,\tau)\|_{H\to H}\|g(\tau,\mathfrak{z}(\tau),\dot{\mathfrak{z}}(\tau))\|_Hd\tau\leqslant
\|\omega\|_{\textrm{Є}([0,T];V,H)}+\int\limits_0^tM\|g(\tau,\mathfrak{z}(\tau),\dot{\mathfrak{z}}(\tau))\|_Hd\tau.
\end{gather*}
Следовательно,
\begin{gather*}
\sigma[\mathfrak{z}](t)\leqslant2\left[\|\omega\|_{\textrm{Є}([0,T];V,H)}+\int\limits_0^tM\|g(\tau,\mathfrak{z}(\tau),\dot{\mathfrak{z}}(\tau))\|_Hd\tau\right]
\leqslant2\left[\|\omega\|_{\textrm{Є}([0,T];V,H)}+\int\limits_0^tM\|g(\tau,0,0)\|_Hd\tau\right]+\\
+\int\limits_0^t2M\|g(\tau,\mathfrak{z}(\tau),\dot{\mathfrak{z}}(\tau))-g(\tau,0,0)\|_Hd\tau\leqslant
2\left[\|\omega\|_{\textrm{Є}([0,T];V,H)}+\int\limits_0^TM\|g(\tau,0,0)\|_Hd\tau\right]+\int\limits_0^t2MK_0(\tau)\sigma[\mathfrak{z}](\tau)d\tau.
\end{gather*}
Применяя затем лемму \ref{Gronwall}, заключаем, что
\begin{gather*}
\sigma[\mathfrak{z}](t)\leqslant2\left[\|\omega\|_{\textrm{Є}([0,T];V,H)}+\int\limits_0^TM\|g(\tau,0,0)\|_Hd\tau\right]\exp\left[\int\limits_0^T2K_0(\tau)Md\tau\right]
\,\,\,\forall\,t\in[0,T].
\end{gather*}
Это и означает выполнение оценки \eqref{--integodifferential.equation!apriori.estimate} с $B\equiv 2\exp(\int\limits_{0}^T2K_0(\xi)Md\xi)\max\{1,M\}$.  Теорема полностью доказана.
\end{Proof}


    \chapter{Сведения из теории меры}
        \section{Предельный переход под знаком измеримой функции}
Пусть $(X,\Sigma,\mu)$ --- конечное положительное пространство с мерой, $Y\subset\mathbb{R}^k$ --- некоторое замкнутое множество, $\lambda$ --- мера Лебега на $Y$. Пусть функция
$\Psi\colon X\times Y\to R$ измерима относительно произведения мер $\mu\otimes\lambda$ на $X\times Y$ и непрерывна по $y\in Y$ при $\mu$--п.в. $x\in X$. Пусть, кроме того, для любого
$B>0$ найдётся постоянная $\hat{C}(B)>0$, такая, что $|\Psi(x,y)|\leqslant\hat{C}(B)$ при  $\mu$--п.в. $x\in X$ и при всех $y\in cl(\textrm{Ш}_B^{k}(0))\cap Y$.

Справедлив следующий результат, являющийся обобщением следствия 2.2.6 на стр.142 монографии \cite{BogachyovVI}.
\begin{Lemma}\label{meas:convergence}
Если последовательность $\mu$--измеримых функций $f_i\colon X\to Y$, $i=1,2,\dots,$ сходится к функции $f\colon X\to Y$ по мере $\mu$, то
$$
\Psi(x,f_i(x))\to\Psi(x,f(x)),\,\,\,i\to\infty,
$$
по мере $\mu$.
\end{Lemma}
\begin{Proof}
Предположим, что утверждение данной леммы неверно. Тогда найдутся положительные числа $\sigma_0$ и $\varepsilon_0$, а также подпоследовательность $f_{i_j}$, $j=1,2,\dots$,
последовательности $f_{i}$, $i=1,2,\dots$, такие, что
\begin{equation}\label{fim}
\mu\{x\in X:|\Psi(x,f_{i_j}(x))-\Psi(x,f(x))|\geqslant\sigma_0\}>\varepsilon_0,\,\,\,j=1,2,\dots.
\end{equation}
Поскольку последовательность $f_{i}$, $i=1,2,\dots$, сходится по мере $\mu$ к функции $f$, то и любая её подпоследовательность сходится по мере $\mu$ к той же функции. В частности, этим
свойством обладает и последовательность $f_{i_j}$, $j=1,2,\dots$. Поскольку последовательность $f_{i_j}$, $j=1,2,\dots$, сходится по мере  $\mu$  к функции  $f$, то можно выделить
подпоследовательность $f_{i_{j_l}}$, $l=1,2,\dots$, последовательности   $f_{i_j}$, $j=1,2,\dots$, сходящуюся к функции $f$  $\mu$--почти  всюду на $X$. Следовательно,
$$
\Psi(x,f_{i_{j_l}}(x))\to\Psi(x,f(x)),\,\,\,l\to\infty,
  $$
при  $\mu$--п.в. $x\in X$. В силу данного обстоятельства имеет место следующая сходимость по мере  $\mu$:
$$
\Psi(x,f_{i_{j_l}}(x))\to\Psi(x,f(x)),\,\,\,l\to\infty.
  $$
А это противоречит неравенству (\ref{fim}). Таким образом, лемма доказана.
\end{Proof}

Дадим следующее
\begin{Definition} Пусть $G$ --- множество элементов некоторой природы, и пусть при каждом $g\in G$ заданы $\mu$--измеримые функции $f_i(\cdot,g)$, $f(\cdot,g)$, $i=1,2,\dots$, принимающие
значения в $\mathbb{R}^m$. Будем говорить, что последовательность функций $f_i$, $i=1,2,\dots$, сходится к функции $f$ на $X$ по мере $\mu$ равномерно по $g\in G$ и писать
$f_i\ByMeasureUniformly\limits_{g\in G}^{(X,\Sigma,\mu)}f$, $i\to\infty$, если
\begin{gather*}
\forall\,\sigma>0:\lim\limits_{i\to\infty}\sup\limits_{g\in G}\mu\{x\in X:|f_i(x,g)-f(x,g)|\geqslant\sigma\}=0.
\end{gather*}
\end{Definition}

\begin{Lemma}\label{meas:convergence:uniform} Пусть $G$ --- компактное метрическое пространство с метрикой $d$, и пусть функции $f_i\colon X\times G\to Y$, $f\colon X\times G\to Y$,
$i=1,2,\dots$, таковы, что $f_i$, $f$, $i=1,2,\dots$, измеримы по $x\in X$ при всех $g\in G$ и непрерывны по $g\in G$ при $\mu$--п.в. $x\in X$. Пусть, кроме того, выполнено соотношение
\begin{gather}\label{fi:uniform_convergence_by_measure}
f_i\ByMeasureUniformly\limits_{g\in G}^{(X,\Sigma,\mu)}f,\,\,\,i\to\infty,
\end{gather}
и найдётся функция $K\colon[0,+\infty)\times[0,\diam G]\to[0,+\infty)$, такая, что $\lim\limits_{\delta\to+0}K(\sigma,\delta)=K(\sigma,0)=0$ при всех $\sigma>0$, и
\begin{gather}\label{ravnostepennaya_nepreryvnost'_po_mere_Lebega}
\forall\,\sigma>0\,\,\forall\,g',\,\,g''\in G\,\,\forall\,i=1,2,\dots: \mu\{x\in X:|f_i(x,g')-f_i(x,g'')|\geqslant\sigma\}\leqslant K(\sigma,d(g',g'')).
\end{gather}
Тогда
\begin{gather*}
\Theta_i\ByMeasureUniformly\limits_{g\in G}^{(X,\Sigma,\mu)}\Theta,\,\,\,i\to\infty,
\end{gather*}
где $\Theta_i(x,g)\equiv\Psi(x,f_i(x,g))$, $\Theta(x,g)\equiv\Psi(x,f(x,g))$, $i=1,2,\dots$
\end{Lemma}

\begin{Proof}
Предположим, что утверждение леммы неверно. Тогда найдутся числа $\sigma_0$, $\varepsilon_0>0$, подпоследовательность $i_j$, $j=1,2,\dots$, последовательности $i=1,2,\dots$, и
последовательность $g_j\in G$, $j=1,2,\dots$, такие, что
\begin{gather*}
\mu\{x\in X:|\Psi(x,f_{i_j}(x,g_j))-\Psi(x,f(x,g_j))|\geqslant\sigma_0\}\geqslant\varepsilon_0,\,\,\,j=1,2,\dots
\end{gather*}
Поскольку  $G$ --- компактное метрическое пространство, то найдутся подпоследовательность $j_s$, $s=1,2,\dots$, последовательности $j=1,2,\dots$ и точка $g^*\in G$, такие, что
$g_{j_s}\to g^*$, $s\to\infty$, в $G$. Поэтому
\begin{gather}\label{fim1:uniform}
\mu\{x\in X:|\Psi(x,f_{i_{j_s}}(x,g_{j_s}))-\Psi(x,f(x,g_{j_s}))|\geqslant\sigma_0\}\geqslant\varepsilon_0,\,\,\,s=1,2,\dots
\end{gather}
В силу соотношения (\ref{ravnostepennaya_nepreryvnost'_po_mere_Lebega}) можем записать, что
\begin{gather*}
\mu\{x\in X:|f_{i_{j_s}}(x,g_{j_s})-f_{i_{j_s}}(x,g^*)|\geqslant\sigma\}\leqslant K(\sigma,|g_{j_s}-g^*|),\,\,\,s=1,2,\dots,\,\,\,
\forall\,\sigma>0,
\end{gather*}
откуда следует, что имеет место следующая сходимость по мере $\mu$ на $X$:
\begin{gather*}
f_{i_{j_s}}(x,g_{j_s})-f_{i_{j_s}}(x,g^*)\to0,\,\,\,s\to\infty.
\end{gather*}
Поэтому найдётся подпоследовательность $s_p$, $p=1,2,\dots$, последовательности $s=1,2,\dots$, такая, что
\begin{gather*}
f_{i_{j_{s_p}}}(x,g_{j_{s_p}})-f_{i_{j_{s_p}}}(x,g^*)\to0,\,\,\, f_{i_{j_{s_p}}}(x,g^*)-f(x,g^*)\to0,\,\,\,p\to\infty,
\end{gather*}
при $\mu$--п.в. $x\in X$. Следовательно, при $\mu$--п.в. $x\in X$
\begin{gather*}
|f_{i_{j_{s_p}}}(x,g_{j_{s_p}})-f(x,g^*)|\leqslant|f_{i_{j_{s_p}}}(x,g_{j_{s_p}})-f_{i_{j_{s_p}}}(x,g^*)|+|f_{i_{j_{s_p}}}(x,g^*)-f(x,g^*)|\to0,\,\,\,p\to\infty.
\end{gather*}
Это означает, что $\mu$--п.в. $x\in X$
\begin{gather*}
\Psi(x,f_{i_{j_{s_p}}}(x,g_{j_{s_p}}))-\Psi(x,f(x,g_{j_{s_p}}))\to0,\,\,\,p\to\infty,
\end{gather*}
что противоречит соотношению (\ref{fim1:uniform}). Таким образом, лемма доказана.
\end{Proof}

\begin{Lemma}\label{fi1fi2covergence->Psi(g,fi1)-Psi(g,fi2)convergence!by_measure:::Boundness}
Если последовательности $\mu$--измеримых функций $f_i^1\colon X\to Y$, $f_i^2\colon X\to Y$, $i=1,2,\dots$, таковы, что $f_i^1-f^2_i\to0$, $i\to\infty$, по мере $\mu$, и найдётся
постоянная $K>0$, такая, что при $\mu$--п.в. $x\in X$ и при всех $i=1,2,\dots$
$$
\max\{|f_i^1(x)|,|f_i^2(x)|\}\leqslant K,
$$
то
$$
\Psi(\cdot,f_i^1(\cdot))-\Psi(\cdot,f_i^2(\cdot))\to0,\text{ $i\to\infty$, по мере $\mu$.}
$$
\end{Lemma}
\begin{Proof}
Пусть утверждение леммы неверно.  Тогда найдутся числа $\sigma_0$, $\varepsilon_0>0$, и подпоследовательность $i_j$, $j=1,2,\dots$, последовательности $i=1,2,\dots$, такие, что
\begin{gather}\label{notPsi(g,fi1)-Psi(g,fi2)convergence!by_measure:::Boundness}
\mu\{x\in X:|\Psi(x,f^1_{i_j}(x))-\Psi(x,f^2_{i_j}(x))|\geqslant\sigma_0\}\geqslant\varepsilon_0,\,\,\,j=1,2,\dots
\end{gather}
Поскольку $f_i^1-f_i^2\to0$, $i\to\infty$, по мере $\mu$, то найдётся подпоследовательность $j_p$, $k=1,\dots$, последовательности $j=1,2,\dots$, такая, что
\begin{gather}\label{fijk1-fijk2aeconvergence!by_measure:::Boundness}
f_{i_{j_p}}^1(x)-f_{i_{j_p}}^2(x)\to0,\text{ $p\to\infty$, $\mu$--п.в}.
\end{gather}
Выберем $x\in X$ так, чтобы $\Psi(x,\cdot)$ была непрерывна на $Y\cap cl{{\textrm{Ш}}^{k}_K(0)}$ и выполнялось (\ref{fijk1-fijk2aeconvergence!by_measure:::Boundness}), а затем зафиксируем.
Так как $\Psi(x,\cdot)$ --- непрерывна на $Y\cap cl{{\textrm{Ш}}^{k}_K(0)}$, то
\begin{gather}\label{Psiunicont!by_measure:::Boundness}
\forall\eta>0\,\exists\,\delta=\delta(\eta)>0\,\forall\,y',\,\,\,y''\in Y\cap cl{{\textrm{Ш}}^{k}_K(0)},\,\,\,|y'-y''|<\delta: |\Psi(x,y')-\Psi(x,y'')|<\eta.
\end{gather}
Ввиду (\ref{fijk1-fijk2aeconvergence!by_measure:::Boundness})
\begin{gather}\label{fijk1(g)-fijk2(g)convergence!by_measure:::Boundness}
\forall\,\delta>0\,\exists\,p_0=p_0(\delta)\geqslant1\,\,\forall\,p\geqslant p_0(\delta):|f_{i_{j_p}}^1(x)-f_{i_{j_p}}^2(x)|<\delta.
\end{gather}
Выберем $\eta>0$ и зафиксируем. Подберём по выбранному $\eta>0$ число $\delta(\eta)>0$ согласно (\ref{Psiunicont!by_measure:::Boundness}). По выбранному $\delta=\delta(\eta)>0$ найдём номер
$\tilde{p}_0(\eta)\equiv p_0(\delta(\eta))\geqslant1$ согласно (\ref{fijk1(g)-fijk2(g)convergence!by_measure:::Boundness}). Как следствие,
\begin{gather*}
|\Psi(x,f_{i_{j_p}}^1(x))-\Psi(x,f_{i_{j_p}}^2(x))|<\eta\,\,\,\forall\,p\geqslant\tilde{p}_0(\eta).
\end{gather*}
Иными словами,
\begin{gather*}
\Psi(x,f_{i_{j_p}}^1(x))-\Psi(x,f_{i_{j_p}}^2(x))\to0,\,\,\,p\to\infty.
\end{gather*}
В силу способа выбора точки $x\in X$ это означает, что
\begin{gather*}
\Psi(x,f_{i_{j_p}}^1(x))-\Psi(x,f_{i_{j_p}}^2(x))\to0,\,\,\,p\to\infty,\text{ $\mu$--п.в.,}
\end{gather*}
что противоречит соотношению (\ref{notPsi(g,fi1)-Psi(g,fi2)convergence!by_measure:::Boundness}). Лемма доказана.
\end{Proof}


\begin{Lemma}\label{fi1fi2covergence->Psi(g,fi1)-Psi(g,fi2)convergence!by_measure:::UniformlyContinuous}
Пусть функция $\Psi$ при $\mu$--п.в. $x\in X$ равномерно непрерывна на $Y$. Если последовательности $\mu$--измеримых функций $f_i^1\colon X\to Y$, $f_i^2\colon X\to Y$, $i=1,2,\dots$,
таковы, что $f_i^1-f^2_i\to0$, $i\to\infty$, по мере $\mu$,
то
$$
\Psi(\cdot,f_i^1(\cdot))-\Psi(\cdot,f_i^2(\cdot))\to0,\text{ $i\to\infty$, по мере $\mu$.}
$$
\end{Lemma}
\begin{Proof}
Предположим, что утверждение леммы неверно. Тогда найдутся числа $\sigma_0$, $\varepsilon_0>0$, и подпоследовательность $i_j$, $j=1,2,\dots$, последовательности $i=1,2,\dots$, такие, что
\begin{gather}\label{notPsi(g,fi1)-Psi(g,fi2)convergence!by_measure:::UniformlyContinuous}
\mu\{x\in X:|\Psi(x,f^1_{i_j}(x))-\Psi(x,f^2_{i_j}(x))|\geqslant\sigma_0\}\geqslant\varepsilon_0,\,\,\,j=1,2,\dots
\end{gather}
Поскольку $f_i^1-f_i^2\to0$, $i\to\infty$, по мере $\mu$, то найдётся подпоследовательность $j_l$, $l=1,\dots$, последовательности $j=1,2,\dots$, такая, что
\begin{gather}\label{fijk1-fijk2aeconvergence!by_measure:::UniformlyContinuous}
f_{i_{j_l}}^1(x)-f_{i_{j_l}}^2(x)\to0,\text{ $l\to\infty$, $\mu$--п.в}.
\end{gather}
Выберем $x\in X$ так, чтобы $\Psi(x,\cdot)$ была непрерывна на $Y$ и выполнялось (\ref{fijk1-fijk2aeconvergence!by_measure:::UniformlyContinuous}), а затем зафиксируем. Так как
$\Psi(x,\cdot)$ равномерно непрерывна на $Y$, то
\begin{gather}\label{Psiunicont!by_measure:::UniformlyContinuous}
\forall\eta>0\,\exists\,\delta=\delta(\eta)>0\,\forall\,y',\,\,\,y''\in Y,\,\,\,|y'-y''|<\delta:|\Psi(x,y')-\Psi(x,y'')|<\eta.
\end{gather}
Ввиду (\ref{fijk1-fijk2aeconvergence!by_measure:::UniformlyContinuous})
\begin{gather}\label{fijk1(g)-fijk2(g)convergence!by_measure:::UniformlyContinuous}
\forall\,\delta>0\,\exists\,l_0=l_0(\delta)\geqslant1\,\,\forall\,l\geqslant l_0(\delta):|f_{i_{j_l}}^1(g)-f_{i_{j_l}}^2(g)|<\delta.
\end{gather}
Выберем $\eta>0$ и зафиксируем. Подберём по выбранному $\eta>0$ число $\delta(\eta)>0$ согласно (\ref{Psiunicont!by_measure:::UniformlyContinuous}). По выбранному $\delta=\delta(\eta)>0$
найдём номер $\tilde{l}_0(\eta)\equiv l_0(\delta(\eta))\geqslant1$ согласно (\ref{fijk1(g)-fijk2(g)convergence!by_measure:::UniformlyContinuous}). Как следствие,
\begin{gather*}
|\Psi(x,f_{i_{j_l}}^1(x))-\Psi(x,f_{i_{j_l}}^2(x))|<\eta\,\,\,\forall\,l\geqslant\tilde{l}_0(\eta).
\end{gather*}
Иными словами,
\begin{gather*}
\Psi(x,f_{i_{j_l}}^1(x))-\Psi(x,f_{i_{j_l}}^2(x))\to0,\,\,\,l\to\infty.
\end{gather*}
В силу способа выбора точки $x\in X$ это означает, что
\begin{gather*}
\Psi(x,f_{i_{j_l}}^1(x))-\Psi(x,f_{i_{j_l}}^2(x))\to0,\,\,\,l\to\infty,\text{ $\mu$--п.в.,}
\end{gather*}
что противоречит соотношению (\ref{notPsi(g,fi1)-Psi(g,fi2)convergence!by_measure:::UniformlyContinuous}). Лемма доказана.
\end{Proof}


\begin{Lemma}\label{fi1fi2covergence->Psi(g,fi1)-Psi(g,fi2)convergence}
Пусть $\Pi\subset\mathbb{R}^{m_1}$ --- множество, имеющее конечную положительную меру Лебега; а функция $F\colon \Pi\times\mathbb{R}^{m_2}\to\mathbb{R}^{m_3}$ такова, что $F(\cdot,y)$ ---
измерима по Лебегу при всех $y\in\mathbb{R}^{m_2}$, $F(x,\cdot)$ --- непрерывна при п.в. $x\in \Pi$, и для любого $B>0$ найдётся постоянная $\mathfrak{C}(B)>0$, такая, что
$|F(x,y)|\leqslant\mathfrak{C}(B)$ при п.в. $x\in\Pi$ и при всех $y\in cl{\textrm{Ш}_B^{m_2}(0)}$.

Тогда если последовательности равномерно ограниченных в норме $L_\infty^{m_2}(\Pi)$ функций $f_i^1$, $f^2_i$, $i=1,2,\dots$, таковы, что $f_i^1-f_i^2\to0$, $i\to\infty$, по мере Лебега, то
при всех $p\in[1,+\infty)$
\begin{gather*}
\lim\limits_{i\to\infty}\|F(\cdot,f_i^1(\cdot))-F(\cdot,f_i^2(\cdot))\|_{p,\Pi}=0.
\end{gather*}
\end{Lemma}
\begin{Proof}
Согласно лемме \ref{fi1fi2covergence->Psi(g,fi1)-Psi(g,fi2)convergence!by_measure:::Boundness},
\begin{gather*}
|F(x,f_i^1(x))-F(x,f_i^2(x))|^p\to0,\,\,\,\text{$i\to\infty$, по мере Лебега на $\Pi$}.
\end{gather*}
Из данного соотношения, справедливой при п.в. $x\in \Pi$ и всех $i=1,2,\dots$ оценки
\begin{gather*}
|F(x,f_i^1(x))-F(x,f_i^2(x))|^p\leqslant(2\mathfrak{C}(K))^p
\end{gather*}
и теоремы Лебега о предельном переходе под знаком интеграла Лебега, и следует утверждение настоящей леммы. Лемма доказана.
\end{Proof}


\begin{Lemma}\label{fi1fi2covergence->Psi(g,fi1)-Psi(g,fi2)convergence!by_measure_uniformly:::Boundness}
Пусть $G$ --- топологическое пространство, и пусть функции $f_i^1\colon X\times G\to Y$, $f_i^2\colon X\times G\to Y$, $i=1,2,\dots$, таковы, что $f_i^1$, $f_i^2$, $i=1,2,\dots$, измеримы
по $x\in X$ при всех $g\in G$ и непрерывны по $g\in G$ при $\mu$--п.в. $x\in X$. Пусть, кроме того, выполнено соотношение
\begin{gather}\label{fi1-fi2:uniform_convergence_by_measure}
f_i^1-f_i^2\ByMeasureUniformly\limits_{g\in G}^{(X,\Sigma,\mu)}0,\,\,\,i\to\infty,
\end{gather}
и найдётся постоянная $K>0$, такая, что при $\mu$--п.в. $x\in X$ и при всех $i=1,2,\dots$
$$
\max\limits_{g\in G}\max\{|f_i^1(x,g)|,|f_i^2(x,g)|\}\leqslant K,
$$
Тогда
\begin{gather*}
\Theta_i\ByMeasureUniformly\limits_{g\in G}^{(X,\Sigma,\mu)}0,\,\,\,i\to\infty,
\end{gather*}
где $\Theta_i(x,g)\equiv\Psi(x,f_i^1(x,g))-\Psi(x,f_i^2(x,g))$, $i=1,2,\dots$
\end{Lemma}
\begin{Proof}
Предположим, что утверждение леммы неверно. Тогда найдутся положительные числа $\sigma_0$ и $\varepsilon_0$, а также подпоследовательность $i_j$, $j=1,2,\dots$, последовательности
$j=1,2,\dots$, и последовательность $g_j\in G$, $j=1,2,\dots$, такие, что
\begin{gather}\label{notPsi(g,fi1)-Psi(g,fi2)convergence!by_measure_uniformly:::Boundness}
\mu\{x\in X:|\Psi(x,f_{i_j}^1(x,g_j))-\Psi(x,f_{i_j}^2(x,g_j))|\geqslant\sigma_0\}\geqslant\varepsilon_0,\,\,\,j=1,2,\dots
\end{gather}
Поскольку $f_i^1-f_i^2\ByMeasureUniformly\limits_{g\in G}^{(X,\Sigma,\mu)}0$, $i\to\infty$, то $f_{i_j}^1-f_{i_j}^2\ByMeasureUniformly\limits_{g\in G}^{(X,\Sigma,\mu)}0$, $j\to\infty$,
ввиду чего
\begin{gather*}
\forall\,\delta>0:\mu\{x\in X:|f_{i_j}^1(x,g_j)-f_{i_j}^2(x,g_j)|\geqslant\delta\}\leqslant\sup\limits_{g\in G}\mu\{x\in X:|f_{i_j}^1(x,g)-f_{i_j}^2(x,g)|\geqslant\delta\}\to0,\,\,\,
j\to\infty.
\end{gather*}
Следовательно,
\begin{gather*}
f_{i_j}^1(\cdot,g_j)-f_{i_j}^2(\cdot,g_j)\to0,\,\,\,j\to\infty,\,\,\,\text{по мере $\mu$}.
\end{gather*}
Пользуясь теперь леммой \ref{fi1fi2covergence->Psi(g,fi1)-Psi(g,fi2)convergence!by_measure:::Boundness}, получаем, что
\begin{gather*}
\Theta_{i_j}\ByMeasureUniformly\limits_{g\in G}^{(X,\Sigma,\mu)}0,\,\,\,j\to\infty,
\end{gather*}
а это противоречит соотношению (\ref{notPsi(g,fi1)-Psi(g,fi2)convergence!by_measure_uniformly:::Boundness}). Лемма доказана.
\end{Proof}

\begin{Lemma}\label{fi1fi2covergence->Psi(g,fi1)-Psi(g,fi2)convergence!by_measure_uniformly:::UniformlyContinuous}
Предположим, что функция $\Psi$ при $\mu$--п.в. $x\in X$ равномерно непрерывна на $Y$. Пусть $G$ --- топологическое пространство, и пусть функции $f_i^1\colon X\times G\to Y$,
$f_i^2\colon X\times G\to Y$, $i=1,2,\dots$, таковы, что $f_i^1$, $f_i^2$, $i=1,2,\dots$, измеримы по $x\in X$ при всех $g\in G$ и непрерывны по $g\in G$ при $\mu$--п.в. $x\in X$. Пусть,
кроме того, выполнено соотношение
\begin{gather}\label{fi1-fi2:uniform_convergence_by_measure:::UniformlyContinuous}
f_i^1-f_i^2\ByMeasureUniformly\limits_{g\in G}^{(X,\Sigma,\mu)}0,\,\,\,i\to\infty.
\end{gather}
Тогда
\begin{gather*}
\Theta_i\ByMeasureUniformly\limits_{g\in G}^{(X,\Sigma,\mu)}0,\,\,\,i\to\infty,
\end{gather*}
где $\Theta_i(x,g)\equiv\Psi(x,f_i^1(x,g))-\Psi(x,f_i^2(x,g))$, $i=1,2,\dots$
\end{Lemma}
\begin{Proof}
Предположим, что утверждение леммы неверно. Тогда найдутся положительные числа $\sigma_0$ и $\varepsilon_0$, а также подпоследовательность $i_j$, $j=1,2,\dots$, последовательности
$j=1,2,\dots$, и последовательность $g_j\in G$, $j=1,2,\dots$, такие, что
\begin{gather}\label{notPsi(g,fi1)-Psi(g,fi2)convergence!by_measure_uniformly:::UniformlyContinuous}
\mu\{x\in X:|\Psi(x,f_{i_j}^1(x,g_j))-\Psi(x,f_{i_j}^2(x,g_j))|\geqslant\sigma_0\}\geqslant\varepsilon_0,\,\,\,j=1,2,\dots
\end{gather}
Поскольку $f_i^1-f_i^2\ByMeasureUniformly\limits_{g\in G}^{(X,\Sigma,\mu)}0$, $i\to\infty$, то $f_{i_j}^1-f_{i_j}^2\ByMeasureUniformly\limits_{g\in G}^{(X,\Sigma,\mu)}0$, $j\to\infty$,
ввиду чего
\begin{gather*}
\forall\,\delta>0:\mu\{x\in X:|f_{i_j}^1(x,g_j)-f_{i_j}^2(x,g_j)|\geqslant\delta\}\leqslant\sup\limits_{g\in G}\mu\{x\in X:|f_{i_j}^1(x,g)-f_{i_j}^2(x,g)|\geqslant\delta\}\to0,\,\,\,
j\to\infty.
\end{gather*}
Следовательно,
\begin{gather*}
f_{i_j}^1(\cdot,g_j)-f_{i_j}^2(\cdot,g_j)\to0,\,\,\,j\to\infty,\,\,\,\text{по мере $\mu$}.
\end{gather*}
Пользуясь теперь леммой \ref{fi1fi2covergence->Psi(g,fi1)-Psi(g,fi2)convergence!by_measure:::UniformlyContinuous}, получаем, что
\begin{gather*}
\Theta_{i_j}\ByMeasureUniformly\limits_{g\in G}^{(X,\Sigma,\mu)}0,\,\,\,j\to\infty,
\end{gather*}
а это противоречит соотношению (\ref{notPsi(g,fi1)-Psi(g,fi2)convergence!by_measure_uniformly:::UniformlyContinuous}). Лемма доказана.
\end{Proof}

        \section{Предельный переход под знаком интеграла Лебега}
Пусть $(X,\Sigma,\mu)$ --- конечное положительное пространство с мерой $\mu$.
\begin{Definition} \cite[определение 4.5.1 на стр.310]{BogachyovVI} Множество функций $\mathcal{F}\subset L_1(X,\Sigma,\mu)$ называется равномерно интегрируемым, если
\begin{gather*}
\lim\limits_{\sigma\to+\infty}\sup\limits_{f\in\mathcal{F}}\int\limits_{\{x\in X\,:\,|f(x)|>\sigma\}}|f(x)|\mu(dx)=0.
\end{gather*}
\end{Definition}

\begin{Definition} \cite[определение 4.5.2 на стр.311]{BogachyovVI} Говорят, что множество $\mathcal{F}\subset L_1(X,\Sigma,\mu)$ имеет равномерно абсолютно непрерывные интегралы,
если для всякого $\varepsilon>0$ существует такое $\delta=\delta(\varepsilon)>0$, что для всех $A\in\Sigma$, $\mu(A)<\delta$, выполнено неравенство
\begin{gather*}
\sup\limits_{f\in\mathcal{F}}\int\limits_A|f(x)|\mu(dx)<\varepsilon.
\end{gather*}
\end{Definition}

\begin{Lemma} \cite[предложение 4.5.3 на стр.311]{BogachyovVI} Множество $\mathcal{F}$ $\mu$--интегрируемых функций равномерно интегрируемо в точности тогда, когда оно ограничено в
$L_1(X,\Sigma,\mu)$ и имеет равномерно абсолютно непрерывные интегралы. Если же мера $\mu$ не имеет атомов, то равномерная интегрируемость равносильна равномерной абсолютной непрерывности
интегралов.
\end{Lemma}

Из доказательства предложения 4.5.3 на стр.311 монографии \cite{BogachyovVI} следует
\begin{Lemma}\label{there_is_no_atoms}
Если мера $\mu$ не имеет атомов, то равномерная абсолютная непрерывность интегралов функций семейства $\mathcal{F}\subset L_1(X,\Sigma,\mu)$ влечёт ограниченность функций этого семейства в
норме пространства $L_1(X,\Sigma,\mu)$.
\end{Lemma}

\begin{Lemma}\label{uniformly_integrable} \cite[теорема 4.5.4 на стр.312]{BogachyovVI} (Теорема Лебега--Витали)
Предположим, что $f$ --- $\mu$--измеримая функция, а $f_i$, $i=1,2,\dots$, --- последовательность $\mu$--интегрируемых функций. Тогда следующие утверждения равносильны:
\begin{enumerate}
    \item
последовательность $f_i$, $i=1,2,\dots$, сходится к $f$ по мере $\mu$ и равномерно интегрируема;
    \item
функция $f$ интегрируема и последовательность $f_i$, $i=1,2,\dots$, сходится к $f$ в $L_1(X,\Sigma,\mu)$.
\end{enumerate}
\end{Lemma}


В дальнейшем нам потребуется также следующий результат, являющийся обобщением классической теоремы Лебега о предельном переходе под знаком интеграла Лебега.
\begin{Lemma}\label{mLebesgue:uniform} Пусть $G$ --- секвенциально компактное топологическое пространство. Пусть функции $f_i\colon X\times G\to R^m$, $f\colon X\times G\to R^m$,
$i=1,2,\dots$, таковы, что $f_i(\cdot,g)$, $i=1,2,\dots$, измеримы при всех $g\in G$; $f_i(x,\cdot)$, $f(x,\cdot)$, $i=1,2,\dots$, --- секвенциально непрерывны на $G$; и, кроме того,
\begin{gather*}
f_i\ByMeasureUniformly\limits_{g\in G}^{(X,\Sigma,\mu)}f,\,\,\,i\to\infty
\end{gather*}
Если
\begin{gather}
\label{mLebesgue:uniform:ravnomero_absolyutno_nepr} \forall\,\varepsilon>0\,\,\exists\,\delta=\delta(\varepsilon)>0\,\,\forall\,A\in\Sigma,\,\,\mu(A)<\delta:
\sup\limits_{i\geqslant1}\sup\limits_{g\in G}\int\limits_A|f_i(x,g)|\mu(dx)<\varepsilon,\\
\label{boundness_in_L_1(X,Sigma,mu)} \sup\limits_{i\geqslant1}\sup\limits_{g\in G}\int\limits_X|f_i(x,g)|\mu(dx)\leqslant C,
\end{gather}
для некоторой положительной постоянной $C>0$, то функция $f(\cdot,g)$ $\mu$--интегрируема при всех $g\in G$, причём
\begin{equation}\label{mLebesgueLIM:uniform}
\lim\limits_{i\to\infty}\sup\limits_{g\in G}\left|\int\limits_Xf_i(x,g)\mu(dx)-\int\limits_Xf(x,g)\mu(dx)\right|=0.
\end{equation}
Кроме того, $\lim\limits_{i\to\infty}\sup\limits_{g\in G}\int\limits_X|f(x,g)-f_i(x,g)|\mu(dx)=0.$
\end{Lemma}
\begin{Proof}
Функция $f(\cdot,g)$ $\mu$--измерима при всех $g\in G$, поскольку является пределом сходящейся по мере $\mu$ последовательности $\mu$--измеримых функций. Заметим, что $\mu$--интегрируемость
функции $f(\cdot,g)$ при всех $g\in G$ следует из леммы Фату и оценки (\ref{boundness_in_L_1(X,Sigma,mu)}). При этом справедливо неравенство
\begin{gather*}
\sup\limits_{g\in G}\int\limits_X|f(x,g)|\mu(dx)\leqslant C.
\end{gather*}

Для всех $\sigma>0$, $g\in G$, $i=1,2,\dots$, положим $X^i(\sigma,g)\equiv\{x\in X:|f_i(x,g)-f(x,g)|\geqslant\sigma\}$. Тогда, в силу
условия леммы,
\begin{gather}\label{muXiconvergence}
\forall\,\sigma>0\,\,\forall\,\delta>0\,\,\exists\,i_0=i_0(\delta,\sigma)\geqslant1,\,\forall\,i\geqslant i_0(\delta,\sigma):\sup\limits_{g\in G}\mu(X^i(\sigma,g))\leqslant\delta.
\end{gather}

Покажем, что
\begin{gather}\label{intfmu_ravnomernaya_po_g_absolyutnaya_nepreryvnost'}
\forall\,\varepsilon>0\,\,\exists\,\eta=\eta(\varepsilon)>0\,\,\forall\,A\in\Sigma,\,\,\mu(A)<\eta(\varepsilon):\sup\limits_{g\in G}\int\limits_A|f(x,g)|\mu(dx)<\varepsilon.
\end{gather}
В самом деле, пусть это не так. Тогда
\begin{gather*}
\exists\,\varepsilon_0>0\,\,\forall\,\eta>0\,\,\exists\,A_\eta\in\Sigma,\,\,\mu(A_\eta)<\eta\,\,\exists\,g_\eta\in G:\int\limits_{A_\eta}|f(x,g_\eta)|\mu(dx)\geqslant\varepsilon_0.
\end{gather*}
Пусть $\eta_m>0$, $m=1,2,\dots$, $\eta_m\to0$, $m\to\infty$, --- некоторая последовательность чисел. Тогда получаем, что
\begin{gather*}
\mu(A_{\eta_m})\to0,\,\,\,m\to\infty;\,\,\, \int\limits_{A_{\eta_m}}|f(x,g_{\eta_m})|\mu(dx)\geqslant\varepsilon_0,\,\,\,m=1,2,\dots
\end{gather*}
Поскольку $G$ --- секвенциально компактное топологическое пространство, то найдутся подпоследовательность $m_s$, $s=1,2,\dots$,
последовательности $m=1,2,\dots$ и точка $g^*\in G$, такие, что $g_{\eta_{m_s}}\to g^*$, $s\to\infty$, в $G$. Поэтому
\begin{gather*}
f(x,g_{\eta_{m_s}})\to f(x,g^*),\,\,\,s\to\infty,\,\,\,\mbox{при $\mu$--п.в. $x\in X$,}
\end{gather*}
откуда в силу секвенциальной непрерывности $f$ на $G$ следует, что
\begin{gather}\label{fgetams_convergence}
\int\limits_X|f(x,g_{\eta_{m_s}})-f(x,g^*)|\mu(dx)\to0,\,\,\,s\to\infty.
\end{gather}
Таким образом,
\begin{gather*}
\varepsilon_0\leqslant\int\limits_{A_{\eta_{m_s}}}|f(x,g_{\eta_{m_s}})|\mu(dx)\leqslant
\int\limits_X|f(x,g_{\eta_{m_s}})-f(x,g^*)|\mu(dx)+\int\limits_{A_{\eta_{m_s}}}|f(x,g^*)|\mu(dx).
\end{gather*}
Переходя здесь к пределу при $s\to\infty$ и пользуясь соотношением (\ref{fgetams_convergence}) и абсолютной непрерывностью интеграла Лебега,
заключаем, что $0<\varepsilon_0\leqslant0$, что невозможно. Следовательно, соотношение
(\ref{intfmu_ravnomernaya_po_g_absolyutnaya_nepreryvnost'}) доказано.

Далее,
\begin{gather*}
\sup\limits_{g\in G}\left|\int\limits_Xf_i(x,g)\mu(dx)-\int\limits_Xf(x,g)\mu(dx)\right|\leqslant\sup\limits_{g\in G}\int\limits_X|f_i(x,g)-f(x,g)|\mu(dx)\leqslant\\
\leqslant\sup\limits_{g\in G}\int\limits_{X^i(\sigma,g)}|f_i(x,g)-f(x,g)|\mu(dx)+\sup\limits_{g\in G}\int\limits_{X\setminus X^i(\sigma,g)}|f_i(x,g)-f(x,g)|\mu(dx)\leqslant\\
\leqslant\sup\limits_{g\in G}\int\limits_{X^i(\sigma,g)}|f_i(x,g)|\mu(dx)+\sup\limits_{g\in G}\int\limits_{X^i(\sigma,g)}|f(x,g)|\mu(dx)+\sigma\mu(X).
\end{gather*}
Отсюда
\begin{gather*}
\sup\limits_{g\in G}\left|\int\limits_Xf_i(x,g)\mu(dx)-\int\limits_Xf(x,g)\mu(dx)\right|\leqslant\sup\limits_{g\in G}\int\limits_X|f_i(x,g)-f(x,g)|\mu(dx)\leqslant\\
\leqslant\sup\limits_{g\in G}\int\limits_{X^i(\sigma,g)}|f_i(x,g)|\mu(dx)+\sup\limits_{g\in G}\int\limits_{X^i(\sigma,g)}|f(x,g)|\mu(dx)+\sigma\mu(X).
\end{gather*}

Выберем произвольно $\varepsilon>0$ и зафиксируем. Положим $\sigma=\sigma_0=\frac\varepsilon{3\mu(X)}$ и подберём $\delta=\delta(\frac\varepsilon3)>0$ согласно
(\ref{mLebesgue:uniform:ravnomero_absolyutno_nepr}). Найдём затем $i_1=i_0(\delta(\frac\varepsilon3), \sigma_0)\geqslant1$ в соответствии с (\ref{muXiconvergence}). Тогда получим, что
\begin{gather*}
\sup\limits_{g\in G}\int\limits_{X^i(\sigma,g)}|f_i(x,g)|\mu(dx)\leqslant\frac\varepsilon3\,\,\,\forall\, i\geqslant i_1.
\end{gather*}

Подберём $\eta=\eta(\frac\varepsilon3)>0$ согласно (\ref{intfmu_ravnomernaya_po_g_absolyutnaya_nepreryvnost'}) и найдём $i_2=i_0(\eta(\frac\varepsilon3), \sigma_0)\geqslant1$ из
(\ref{muXiconvergence}). Положив $i^*=\max\{i_1,i_2\}$, будем иметь
\begin{gather*}
\sup\limits_{g\in G}\int\limits_{X^i(\sigma,g)}|f_i(x,g)|\mu(dx)\leqslant\frac\varepsilon3,\,\,\,
\sup\limits_{g\in G}\int\limits_{X^i(\sigma,g)}|f(x,g)|\mu(dx)\leqslant\frac\varepsilon3\,\,\,\forall\,i\geqslant i^*.
\end{gather*}

Как следствие,
\begin{gather*}
\sup\limits_{g\in G}\left|\int\limits_Xf_i(x,g)\mu(dx)-\int\limits_Xf(x,g)\mu(dx)\right|\leqslant\sup\limits_{g\in G}\int\limits_X|f_i(x,g)-f(x,g)|\mu(dx)\leqslant\\
\leqslant\sup\limits_{g\in G}\int\limits_{X^i(\sigma_0,g)}|f_i(x,g)|\mu(dx)+\sup\limits_{g\in G}\int\limits_{X^i(\sigma_0,g)}|f(x,g)|\mu(dx)+\sigma\mu(X)\leqslant\frac\varepsilon3+
\frac\varepsilon3+\frac\varepsilon3=\varepsilon
\end{gather*}
при всех $i\geqslant i^*$. Лемма доказана.
\end{Proof}


\begin{Corrolary}\label{fi1fi2covergence->Psi(g,fi1)-Psi(g,fi2)convergence!by_measure}
Если выполнены условия лемм \ref{fi1fi2covergence->Psi(g,fi1)-Psi(g,fi2)convergence!by_measure:::Boundness} или
\ref{fi1fi2covergence->Psi(g,fi1)-Psi(g,fi2)convergence!by_measure:::UniformlyContinuous}, и, кроме того, семейства функций
\begin{gather*}
X\ni x\mapsto\Psi(x,f^1_i(x)),\,\,\,i=1,2,\dots,\\
X\ni x\mapsto\Psi(x,f^2_i(x)),\,\,i=1,2,\dots,
\end{gather*}
равномерно интегрируемы, то
\begin{gather*}
\lim\limits_{i\to\infty}\left|\int\limits_X\Psi(x,f^1_i(x))\mu(dx)-\int\limits_X\Psi(x,f^2_i(x))\mu(dx)\right|=0.
\end{gather*}
Кроме того,
\begin{gather*}
\lim\limits_{i\to\infty}\int\limits_X|\Psi(x,f^1_i(x))-\Psi(x,f^2_i(x))|\mu(dx)=0.
\end{gather*}
\end{Corrolary}

\begin{Corrolary}\label{fi1fi2covergence->Psi(g,fi1)-Psi(g,fi2)convergence!by_measure_uniformly}
Если выполнены условия лемм \ref{fi1fi2covergence->Psi(g,fi1)-Psi(g,fi2)convergence!by_measure_uniformly:::Boundness}
или \ref{fi1fi2covergence->Psi(g,fi1)-Psi(g,fi2)convergence!by_measure_uniformly:::UniformlyContinuous}, и, кроме того, семейства функций
\begin{gather*}
X\ni x\mapsto\Psi(x,f^1_i(x,g)),\,\,\,g\in G,\,\,\,i=1,2,\dots,\\
X\ni x\mapsto\Psi(x,f^2_i(x,g)),\,\,\,g\in G,\,\,\,i=1,2,\dots,
\end{gather*}
равномерно интегрируемы, то
\begin{gather*}
\lim\limits_{i\to\infty}\sup\limits_{g\in G}\left|\int\limits_X\Psi(x,f^1_i(x,g))\mu(dx)-\int\limits_X\Psi(x,f^2_i(x,g))\mu(dx)\right|=0.
\end{gather*}
Кроме того,
\begin{gather*}
\lim\limits_{i\to\infty}\sup\limits_{g\in G}\int\limits_X|\Psi(x,f^1_i(x,g))-\Psi(x,f^2_i(x,g))|\mu(dx)=0.
\end{gather*}
\end{Corrolary}

\begin{Lemma}
Если последовательность почти всюду (на $[0,T]$) неотрицательных функций $\varphi_j\in L_2[0,T]$, $j=1,2,\dots$, слабо в $L_2[0,T]$ сходится к функции $\varphi\in L_2[0,T]$, то функция
$\varphi$ также почти всюду неотрицательна, причём
\begin{gather}\label{L1norm_phij__tends_to__normL1_phi}
\lim\limits_{j\to\infty}\|\varphi_j\|_{1,[0,T]}=\|\varphi\|_{1,[0,T]}.
\end{gather}
\end{Lemma}
\begin{Proof}
Пусть $\mathcal{A}\subseteq[0,T]$ --- произвольное измеримое по Лебегу множество. Поскольку последовательность $\varphi_j$, $j=1,2,\dots$, слабо в $L_2[0,T]$ сходится к функции $\varphi$,
то
\begin{gather*}
\lim\limits_{j\to\infty}\int\limits_0^T\varphi_j(t)\chi_{\mathcal{A}}(t)dt=\int\limits_0^T\varphi(t)\chi_{\mathcal{A}}(t)dt,
\end{gather*}
где $\chi_{\mathcal{A}}$ --- характеристическая функция множества $\mathcal{A}$. Таким образом,
\begin{gather}\label{intAphij_tends_to_intAphi}
\lim\limits_{j\to\infty}\int\limits_{\mathcal{A}}\varphi_j(t)dt=\int\limits_{\mathcal{A}}\varphi(t)dt\,\,\,\forall\,\mathcal{A}\subseteq[0,T].
\end{gather}
Отсюда, ввиду неотрицательности функций $\varphi_j\in L_2[0,T]$, $j=1,2,\dots$, почти всюду на $[0,T]$, вытекает, что
\begin{gather}\label{intAphi_geq0}
\int\limits_{\mathcal{A}}\varphi(t)dt\geqslant0\,\,\,\forall\,\mathcal{A}\subseteq[0,T].
\end{gather}

Покажем, что функция $\varphi$ неотрицательна почти всюду на $[0,T]$. В самом деле, пусть это не так. Тогда найдутся множество $\mathcal{A}\subseteq[0,T]$, имеющее положительную меру
Лебега, и положительное число $\beta>0$, такие, что
\begin{gather*}
\varphi(t)\leqslant-\beta\,\,\,\text{при п.в. $t\in\mathcal{A}$.}
\end{gather*}
Следовательно,
\begin{gather*}
\int\limits_{\mathcal{A}}\varphi(t)dt\leqslant-\beta\meas\mathcal{A}<0,
\end{gather*}
что противоречит соотношению (\ref{intAphi_geq0}). Итак, неотрицательность функции $\varphi$ почти всюду на $[0,T]$ доказана.

Что же касается предельного соотношения (\ref{L1norm_phij__tends_to__normL1_phi}), то оно является следствием соотношения (\ref{intAphij_tends_to_intAphi}). Лемма полностью доказана.
\end{Proof}

Из данной леммы вытекает
\begin{Corrolary}\label{NonnegFunctionsConvergence:corrolary}
Если последовательность функций $\varphi_j\in L_2^m[0,T]$, $j=1,2,\dots$, все компоненты которых неотрицательны почти всюду на $[0,T]$, слабо в $L_2^m[0,T]$ сходится к функции
$\varphi\in L_2^m[0,T]$, то все компоненты функции $\varphi$ также неотрицательны почти всюду на $[0,T]$, причём
\begin{gather*}
\lim\limits_{j\to\infty}\|\varphi_j\|_{1,[0,T]}=\|\varphi\|_{1,[0,T]}.
\end{gather*}
\end{Corrolary}

\begin{Lemma}\label{min_int=int_min}
Пусть $\Pi\subset\mathbb{R}^{m_1}$ --- ограниченное множество, мера Лебега которого конечна и положительна; $\mathbf{U}\subset\mathbb{R}^{m_2}$ --- компакт; $\mathbf{D}\equiv\{\mathbf{u}\in
L_\infty^{m_2}(\Pi):\mathbf{u}(x)\in\mathbf{U}\text{ при п.в. $x\in\Pi$}\}$. Наконец, пусть функция $\varphi\colon\Pi\times\mathbf{U}\to\mathbb{R}$ измерима (в смысле Лебега) по
совокупности всех своих переменных, при п.в. $x\in\Pi$ функция $\varphi(x,\cdot)$ непрерывна на $\mathbf{U}$, и найдётся постоянная положительная постоянная $K$, такая, что
$|\varphi(x,v)|\leqslant K$ при всех $(x,v)\in\Pi\times\mathbf{U}$. Тогда
\begin{gather*}
\min\limits_{\mathbf{u}(\cdot)\in\mathbf{D}}\int\limits_\Pi\varphi(x,\mathbf{u}(x))dx=\int\limits_\Pi\min\limits_{v\in\mathbf{U}}\varphi(x,v)dx.
\end{gather*}
\end{Lemma}
\begin{Proof}
В самом деле, нетрудно видеть, что при всех $\mathbf{u}(\cdot)\in\mathbf{D}$
\begin{gather*}
\int\limits_\Pi\varphi(x,\mathbf{u}(x))dx\geqslant\int\limits_\Pi\min\limits_{v\in\mathbf{U}}\varphi(x,v)dx.
\end{gather*}
Переходя здесь к точной нижней грани по $\mathbf{u}(\cdot)\in\mathbf{D}$, получаем, что
\begin{gather}\label{min_int=int_min?m1}
\min\limits_{\mathbf{u}(\cdot)\in\mathbf{D}}\int\limits_\Pi\varphi(x,\mathbf{u}(x))dx\geqslant\int\limits_\Pi\min\limits_{v\in\mathbf{U}}\varphi(x,v)dx.
\end{gather}

Обозначим через $\mathbf{u}_{min}(\cdot)$ функцию из $\mathbf{D}$, удовлетворяющую соотношению
\begin{gather*}
\varphi(x,\mathbf{u}_{min}(x))=\min\limits_{v\in\mathbf{U}}\varphi(x,v)\text{ при п.в. $x\in\Pi$}.
\end{gather*}
Тогда из соотношения (\ref{min_int=int_min?m1}) выводим, что
\begin{gather*}
\min\limits_{\mathbf{u}(\cdot)\in\mathbf{D}}\int\limits_\Pi\varphi(x,\mathbf{u}(x))dx\geqslant\int\limits_\Pi\varphi(x,\mathbf{u}_{min}(x))dx,
\end{gather*}
что, ввиду включения $\mathbf{u}_{min}(\cdot)\in\mathbf{D}$, может выполняться лишь в случае, когда
\begin{gather*}
\min\limits_{\mathbf{u}(\cdot)\in\mathbf{D}}\int\limits_\Pi\varphi(x,\mathbf{u}(x))dx=\int\limits_\Pi\varphi(x,\mathbf{u}_{min}(x))dx.
\end{gather*}
Из данного равенства и определения функции $\mathbf{u}_{min}(\cdot)$ и вытекает утверждение леммы. Лемма доказана.
\end{Proof}

\begin{Lemma}\label{integral_min_equiv_pointwise_min}
Пусть $\Pi\subset\mathbb{R}^{m_1}$ --- ограниченное множество, мера Лебега которого конечна и положительна; $\mathbf{U}\subset\mathbb{R}^{m_2}$ --- компакт; $\mathbf{D}\equiv\{\mathbf{u}\in
L_\infty^{m_2}(\Pi):\mathbf{u}(x)\in\mathbf{U}\text{ при п.в. $x\in\Pi$}\}$. Пусть функция $\varphi\colon\Pi\times\mathbf{U}\to\mathbb{R}$ измерима (в смысле Лебега) по совокупности всех
своих переменных, при п.в. $x\in\Pi$ функция $\varphi(x,\cdot)$ непрерывна на $\mathbf{U}$, и найдётся постоянная положительная постоянная $K$, такая, что $|\varphi(x,v)|\leqslant K$ при
всех $(x,v)\in\Pi\times\mathbf{U}$. Наконец, пусть $\mathbf{u}_0(\cdot)\in\mathbf{D}$ --- некоторая функция. Тогда соотношения
\begin{gather}\label{integral_min}
\min\limits_{\mathbf{u}(\cdot)\in\mathbf{D}}\int\limits_\Pi\varphi(x,\mathbf{u}(x))dx=\int\limits_\Pi\varphi(x,\mathbf{u}_0(x))dx
\end{gather}
и
\begin{gather}\label{pointwise_min}
\varphi(x,\mathbf{u}_0(x))=\min\limits_{v\in\mathbf{U}}\varphi(x,v)\text{ при п.в. $x\in\Pi$}.
\end{gather}
эквивалентны.
\end{Lemma}
\begin{Proof}
1) Пусть выполнено соотношение (\ref{integral_min}). Тогда, на основании леммы \ref{min_int=int_min},
\begin{gather*}
\min\limits_{\mathbf{u}(\cdot)\in\mathbf{D}}\int\limits_\Pi\varphi(x,\mathbf{u}(x))dx=\int\limits_\Pi\min\limits_{v\in\mathbf{U}}\varphi(x,v)dx.
\end{gather*}
Следовательно,
\begin{gather*}
\int\limits_\Pi\varphi(x,\mathbf{u}_0(x))dx=\min\limits_{\mathbf{u}(\cdot)\in\mathbf{D}}\int\limits_\Pi\varphi(x,\mathbf{u}(x))dx=\int\limits_\Pi\min\limits_{v\in\mathbf{U}}\varphi(x,v)dx.
\end{gather*}
Иными словами,
\begin{gather*}
\int\limits_\Pi[\varphi(x,\mathbf{u}_0(x))-\min\limits_{v\in\mathbf{U}}\varphi(x,v)]dx=0.
\end{gather*}
А поскольку подынтегральная функция в полученном выражении неотрицательна, то выполнено соотношение (\ref{pointwise_min}). Итак, из выполнения соотношения (\ref{integral_min}) следует
выполнение соотношения (\ref{pointwise_min}).

2) Пусть теперь выполнено соотношение (\ref{pointwise_min}). Интегрируя (\ref{pointwise_min}) по $x\in\Pi$, будем иметь
\begin{gather*}
\int\limits_\Pi\varphi(x,\mathbf{u}_0(x))dx=\int\limits_\Pi\min\limits_{v\in\mathbf{U}}\varphi(x,v)dx.
\end{gather*}
Так как на основании леммы \ref{min_int=int_min}
\begin{gather*}
\min\limits_{\mathbf{u}(\cdot)\in\mathbf{D}}\int\limits_\Pi\varphi(x,\mathbf{u}(x))dx=\int\limits_\Pi\min\limits_{v\in\mathbf{U}}\varphi(x,v)dx,
\end{gather*}
то
\begin{gather*}
\int\limits_\Pi\varphi(x,\mathbf{u}_0(x))dx=\int\limits_\Pi\min\limits_{v\in\mathbf{U}}\varphi(x,v)dx=\min\limits_{\mathbf{u}(\cdot)\in\mathbf{D}}\int\limits_\Pi\varphi(x,\mathbf{u}(x))dx.
\end{gather*}
Иначе говоря, выполнено соотношение (\ref{integral_min}). Таким образом, из выполнения соотношения (\ref{pointwise_min}) следует выполнение соотношения (\ref{integral_min}). Лемма полностью
доказана.
\end{Proof}

        \section{Точки Лебега и максимальные функции}
\begin{Lemma}\label{classic_leb} \cite{Stane}
Если функция $f\colon\mathbb{R}^n\to\mathbb{R}$ --- локально суммируема, то справедливо равенство
\begin{gather*}
\lim\limits_{r\to0}\frac1{\meas \textrm{Ш}_r^n(x)}\int\limits_{\textrm{Ш}_r^n(x)}f(y)dy=f(x)\text{ для п.в. $x\in\mathbb{R}^n$}.
\end{gather*}
\end{Lemma}

\begin{Definition}\cite{Stane}
Пусть дана функция $f\colon\mathbb{R}^n\to\mathbb{R}$. Максимальной функцией $(Mf)$ функции $f$ называется функция
\begin{gather*}
(Mf)(x)\equiv\sup\limits_{r>0}\frac1{\meas \textrm{Ш}_r^n(x)}\int\limits_{\textrm{Ш}_r^n(x)}|f(y)|dy,\,\,\,x\in\mathbb{R}^n.
\end{gather*}
\end{Definition}

Справедлива следующая
\begin{Lemma}\label{maximal.functions!Lemma} \cite{Stane} Пусть дана функция $f\colon\mathbb{R}^n\to\mathbb{R}$.

1) Если $f\in L_p(\mathbb{R}^n)$, где $p\in[1,\infty]$, то $(Mf)$  --- п.в. конечна.

2) Если $f\in L_1(\mathbb{R}^n)$, то
\begin{gather*}
\forall\,\alpha>0:\meas\{x\in\mathbb{R}^n:(Mf)(x)>\alpha\}\leqslant\frac{\tilde{A}}\alpha\int\limits_{\mathbb{R}^n}|f(x)|dx,
\end{gather*}
где $\tilde A>0$ --- константа, зависящая только от размерности $n$ (например, можно взять $\tilde A=5^n$).

3) Если $f\in L_p(\mathbb{R}^n)$, где $p\in(1,\infty]$, то $(Mf)\in L_p(\mathbb{R}^n)$ и
\begin{gather*}
\|Mf\|_{p,\mathbb{R}^n}\leqslant \tilde A_p\|f\|_{p,\mathbb{R}^n},
\end{gather*}
где $\tilde A_p$ зависит лишь от $p$ и $n$.
\end{Lemma}

\begin{Definition}\label{lmleb}\cite{sumin83}--\cite{sumin2000} Пусть $G\subset \mathbb{R}^n$ - открытое множество. Точку $x\in G$ назовем $(l,m)$--точкой Лебега суммируемой функции
$f:\,G\to \mathbb{R}^1$, $1\leqslant l \leqslant m\leqslant n$, если $f(x)\ne\infty$ и
\begin{gather*}
\lim_{h\to 0}\frac 1{(2h)^{m-l+1}}\int\limits_{x_l-h}^{x_l+h}\dots\int\limits_{x_m-h}^{x_m+h}|f(x_1,\dots,x_{l-1},y_1,\dots,y_{m-l+1},x_{m+1},\dots,x_n)-f(x)|\,dy_1\dots dy_{m-l+1}=0.
\end{gather*}
\end{Definition}
\begin{Remark}\label{8r2}
Легко видеть, что $(1,n)$-точка Лебега есть точка Лебега в обычном смысле \cite{Stane}.
\end{Remark}

\begin{Lemma}\label{8l4_1}\cite{sumin83} -- \cite{sumin2000}
При любых фиксированных $l$, $m$, $1\leqslant l\leqslant m\leqslant n$ п.в. точки открытого множества $G$ есть $(l,m)$-точки Лебега
суммируемой функции $f:\,G\to \mathbb{R}^1$.
\end{Lemma}

\begin{Definition}\label{intleb}
Пусть $m_1$, $m_2\geqslant1$ --- натуральные числа, $m=m_1+m_2$; $X\subset\mathbb{R}^{m_1}$, $Y\subset\mathbb{R}^{m_2}$ --- открытые множества, и пусть функция $f\colon X\times Y\to
\mathbb{R}$ --- суммируема по множеству $X\times Y$. Точку $\tilde{y}\in Y$, для которой $\int\limits_X|f(x,\tilde{y})|dx\neq\infty$, назовём интегральной точкой Лебега функции $f$ по
переменной $y$, если
\begin{gather*}
\lim\limits_{h\to0}\frac 1{(2h)^{m_2}}\int\limits_{\tilde{y}_1-h}^{\tilde{y}_1+h}\dots\int\limits_{\tilde{y}_{m_2}-h}^{\tilde{y}_{m_2}+h}dy\int\limits_X|f(x,y)-f(x,\tilde{y})|dx=0.
\end{gather*}
\end{Definition}
\begin{Remark}\label{intleb:remark}
Нетрудно видеть, что интегральная точка Лебега функции $f$ по переменной $y$ --- это в точности точка Лебега функции
\begin{gather*}
Y\ni y\mapsto\int\limits_Xf(x,y)dx.
\end{gather*}
\end{Remark}

Справедлива следующая
\begin{Lemma}\label{intleb:Lemma}
Пусть $m_1$, $m_2\geqslant1$ --- натуральные числа, $m=m_1+m_2$; $X\subset\mathbb{R}^{m_1}$, $Y\subset\mathbb{R}^{m_2}$ --- открытые множества, и пусть функция $f\colon X\times Y\to
\mathbb{R}$ --- суммируема по множеству $X\times Y$. Тогда почти все точки множества $Y$ являются интегральными точками Лебега функции $f$ по переменной $y$.
\end{Lemma}
\begin{Proof}
Поскольку функция $f$ суммируема по множеству $X\times Y$, то, в силу теоремы Фубини, функция
\begin{gather*}
Y\ni y\mapsto\int\limits_Xf(x,y)dx
\end{gather*}
почти всюду конечна и является элементом $L_1(Y)$. Пользуясь затем замечанием \ref{intleb:remark} и леммой \ref{classic_leb}, получаем, что п.в. точки множества $Y$ являются интегральными
точками Лебега функции $f$ по переменной $y$.
\end{Proof}

\begin{Definition}\label{intleb_lm}
Пусть $m_1$, $m_2\geqslant1$ --- натуральные числа, $m=m_1+m_2$; $X\subset\mathbb{R}^{m_1}$, $Y\subset\mathbb{R}^{m_2}$ --- открытые множества, и пусть функция $f\colon X\times Y\to
\mathbb{R}$ --- суммируема по множеству $X\times Y$.Точку $\tilde{x}\in X$, для которой $\int\limits_Y|f(\tilde{x},y)|dy\neq\infty$, назовём интегральной $(l,r)$--точкой Лебега,
$1\leqslant l \leqslant r\leqslant m_1$, функции $f$ по переменной $x$, если точка $\tilde{x}$ является $(l,r)$--точкой Лебега функции
\begin{gather*}
X\ni x\mapsto\int\limits_Yf(x,y)dy.
\end{gather*}
\end{Definition}

Из леммы \ref{lmleb} и определения \ref{intleb_lm} вытекает
\begin{Lemma}\label{intleb_lm:Lemma}
Пусть $m_1$, $m_2\geqslant1$ --- натуральные числа, $m=m_1+m_2$; $X\subset\mathbb{R}^{m_1}$, $Y\subset\mathbb{R}^{m_2}$ --- открытые множества, и пусть функция $f\colon X\times Y\to
\mathbb{R}$ --- суммируема по множеству $X\times Y$. Тогда при любых фиксированных $l$, $r$, $1\leqslant l \leqslant r\leqslant m_1$, почти все точки множества $X$ являются интегральными
$(l,r)$--точками Лебега функции $f$ по переменной $x$.
\end{Lemma}

        \section{Аппроксимация мер Радона, заданных на отрезке числовой оси}
\begin{Lemma}\label{measapprox} Для любой меры $\mu\in \mathbf{M}[0,T]$ найдётся последовательность функций~$\omega^k\in C[0,T]$, $k=1,2,\dots$, такая, что
\begin{equation}\label{measapproxc}
\lim\limits_{k\to\infty}\int\limits_{[0,T]}\zeta(t)\mu^k(dt)=\int\limits_{[0,T]}\zeta(t)\mu(dt), \,\,\,\forall\,\zeta\in C[0,\,T],
\end{equation}
где $\displaystyle\mu^k(E)\equiv\int\limits_E\omega^k(t)dt$, $E\subseteq [0,\,T]$ --- борелевское подмножество отрезка $[0,\,T]$, $k=1,2,\dots$, причём если мера $\mu\in \mathbf{M}[0,\,T]$
--- неотрицательна, то $\omega^k(t)\geqslant0$, $\forall\,t\in[0,\,T]$, $k=1,2,\dots$.
\end{Lemma}
\begin{Proof} Разобьём доказательство на несколько этапов.

1) Покажем вначале, что для любой меры~$\mu\in \mathbf{M}[0,\,T]$ найдётся последовательность мер
\begin{gather*}
\bar{\mu}^m\equiv\sum\limits_{i=1}^{i_m}\lambda_{i,m}\delta_{t_{i,m}},\,\,\,m=1,2,\dots,
\end{gather*}
где $\lambda_{i,m}\in R$, $i= \overline{1,\,i_m}$, $m=1,2,\dots$, --- неотрицательны, если мера $\mu\in \mathbf{M}[0,\,T]$ --- неотрицательна, $t_{i,m}\in[0,\,T]$, $i=\overline{1,\,i_m}$,
$m=1,2,\dots$, такая, что
\begin{equation}\label{fawlc}
\lim\limits_{m\to\infty}\int\limits_{[0,T]}\zeta(t)\bar{\mu}^m(dt)= \int\limits_{[0,T]}\zeta(t){\mu}(dt),\,\,\,\forall\,\zeta\in C[0,\,T].
\end{equation}
В самом деле, согласно~\cite{warga}, $(C[0,\,T])^*$ изометрично изоморфно~$\mathbf{M}[0,\,T]$. С другой стороны, согласно~\cite{KF}, $(C[0,\,T])^*$ изометрично изоморфно
$\mathbf{BV}^0[0,\,T]$. Следовательно, существует изоморфизм $${\cal F}\colon \mathbf{M}[0,\,T]\to\mathbf{BV}^0[0,\,T],$$ такой, что
\begin{equation}\label{isomfv}
\|{\cal F}[\mu]\|_{\mathbf{BV}^0}=\|\mu\|,\,\,\,\forall\,\mu\in \mathbf{M}[0,\,T].
\end{equation}
Пусть функционал~${\cal A}\,:\,C[0,\,T]\to \mathbb{R}$ задаётся формулой
\begin{equation}\label{ffunca}
{\cal A}[\zeta]=\int\limits_{[0,\,T]}\zeta(t)\mu(dt),\,\,\,\forall\,\zeta\in C[0,\,T].
\end{equation}
Тогда, очевидно, ${\cal A}\in(C[0,\,T])^*$, и, стало быть, согласно~\cite{KF},
\begin{equation}\label{azis}
{\cal A}[\zeta]=\int\limits_{[0,T]}\zeta(t)d{\cal F}[\mu](t),\,\,\,\forall\,\zeta \in C[0,\,T],
\end{equation}
где интеграл понимается в смысле интеграла Стильтьеса по отрезку~$[0,\,T]$. Пусть~$\bar{t}_{i,m}= \frac{Ti}{m}$, $i=\overline{0,\,m}$, $\bar{t}_{i,m+1}=t_{i,m}$, $\lambda_{i,m}\equiv
{\cal F}[\mu](\bar{t}_{i,m})-{\cal F}[\mu](\bar{t}_{i-1,m})$, $i= \overline{1,\,m}$, $\lambda_{i,m+1}={\cal F}[\mu](\bar{t}_{i,m})$, $i_m=m+1$, $t_{i,m}= \bar{t}_{i-1,m}$,
$i=\overline{1,\,i_m}$. Тогда, в силу определения интеграла Стильтьеса по отрезку~$[0,\,T]$,
\begin{gather*}
\displaystyle\lim\limits_{m\to\infty}\sum\limits_{i=1}^{i_m}\zeta(t_{i,m})\lambda_{i,m}= \int\limits_{[0,T]}\zeta(t)d{\cal F}[\mu](t),\,\,\,
\forall\,\zeta \in C[0,\,T].
\end{gather*}
Полагая~$\displaystyle\bar{\mu}^m\equiv\sum\limits_{i=1}^{i_m} \lambda_{i,m}\mydelta_{t_{i,m}}$, $m=1,2,\dots$, перепишем последнее неравенство в виде
\begin{gather*}
\lim\limits_{m\to\infty}\int\limits_{[0,T]}\zeta(t)\bar{\mu}^m(dt)= \int\limits_{[0,T]}\zeta(t){\mu}(dt),\,\,\,\forall\,\zeta\in C[0,\,T],
\end{gather*}
что в совокупности с~(\ref{isomfv})--(\ref{azis}) и даёт~(\ref{fawlc}).

Ясно, что если функция~${\cal F}[\mu]$ монотонно не убывает, то построенные меры~$\bar\mu^m$, $m=1,2,\dots$, будут неотрицательными, и, следовательно, мера~$\mu$ также будет
неотрицательной, как $*$--слабый предел таких мер. В силу же~(\ref{isomfv}), неотрицательная мера~$\mu$ может породить лишь монотонно неубывающую функция~${\cal F}[\mu]$. Итак, коэффициенты
мер~$\bar\mu^m$, $m=1,2,\dots$, можно считать неотрицательными, если мера~$\mu$ --- неотрицательна.

2) Докажем существование непрерывных функций, упомянутых в формулировке леммы. Пусть $\varepsilon_s>0$, $s=1,2,\dots$, $\varepsilon_s\to0$, $s\to\infty$, --- некоторая последовательность
чисел, и пусть
\begin{gather*}
\omega^s_{i,m}(t)\equiv \frac{\chi_{(t_{i,m}-\varepsilon_s,t_{i,m}+\varepsilon_s)\cap(0,\,T)}(t)}{\meas\{(t_{i,m}-\varepsilon_s,t_{i,m}+\varepsilon_s)\cap(0,\,T)\}},\,\,\,
i=\overline{1,\,i_m},\\
\bar{\omega}^s_m(t)\equiv\sum\limits_{i=1}^{i_m}\lambda_{i,m}\omega^s_{i,m}(t),\,\,\,m,\,s=1,2,\dots,\,\,\,t\in[0,\,T],\\
\bar{\mu}_s^m(E)\equiv\int\limits_E\bar{\omega}^s_m(t)dt,\,\,\, E\subseteq[0,\,T],\,\,\,m,\,s=1,2,\dots.
\end{gather*}
Тогда, очевидно,
\begin{equation}\label{sawlc}
\lim\limits_{s\to\infty}\int\limits_{[0,T]}\zeta(t)\bar{\mu}^m_s(dt)= \int\limits_{[0,T]}\zeta(t)\bar{\mu}^m(dt),\,\,\,\forall\,\zeta\in C[0,\,T].
\end{equation}
Пусть теперь~$h_p>0$, $p=1,2,\dots$, $h_p\to0$, $p\to\infty$, --- некоторая последовательность чисел, и пусть $\displaystyle\bar{\mu}^m_{s,p}(E)\equiv \int\limits_E\bar{\omega}^{s,p}_m(t)
dt$, $E\subseteq[0,\,T]$, $m,\,s,\,p=1,2,\dots$, где~$\bar{\omega}^{s,p}_m$ --- усреднение с параметром~$h_p>0$ с ядром, независящим от $m,\,s,\,p=1,2,\dots$, функции $\bar{\omega}^{s}_m$.
Пользуясь свойствами средних функций и теоремой Радона--Никодима, заключаем, что
\begin{equation}\label{tawlc}
\lim\limits_{p\to\infty}\int\limits_{[0,T]}\zeta(t)\bar{\mu}^m_{s,p}(dt)= \int\limits_{[0,T]}\zeta(t)\bar{\mu}^m_s(dt),\,\,\,\forall\,\zeta\in C[0,\,T].
\end{equation}
Из~(\ref{fawlc}) следует, что
\begin{gather}\label{weaknc:m}
\lim\limits_{m\to\infty}|\bar{\mu}^m-\mu|_w=0,
\end{gather}
где~$|\cdot|_w$ --- слабая норма~\cite{warga} в~$\mathbf{M}[0,\,T]$.

Выберем произвольно $\alpha>0$ и зафиксируем.

Согласно (\ref{weaknc:m}), найдётся номер $m_0(\alpha)\geqslant1$, такой, что
\begin{gather*}
|\bar{\mu}^{m_0(\alpha)}-\mu|_w\leqslant\frac\alpha3.
\end{gather*}

Далее, согласно (\ref{sawlc}),
\begin{gather*}
\lim\limits_{s\to\infty}|\bar{\mu}^{m_0(\alpha)}_s-\bar\mu^{m_0(\alpha)}|_w=0.
\end{gather*}
Поэтому найдётся номер $s_0(\alpha)\geqslant1$, такой, что
\begin{gather*}
|\bar{\mu}^{m_0(\alpha)}_{s_0(\alpha)}-\bar\mu^{m_0(\alpha)}|_w\leqslant\frac\alpha3.
\end{gather*}

Наконец, согласно (\ref{tawlc}),
\begin{gather*}
\lim\limits_{p\to\infty}|\bar{\mu}^{m_0(\alpha)}_{s_0(\alpha),p}-\bar\mu^{m_0(\alpha)}_{s_0(\alpha)}|_w=0,
\end{gather*}
в силу чего найдётся номер $p_0(\alpha)\geqslant1$, такой, что
\begin{gather*}
|\bar{\mu}^{m_0(\alpha)}_{s_0(\alpha),p_0(\alpha)}-\bar\mu^{m_0(\alpha)}_{s_0(\alpha)}|_w\leqslant\frac\alpha3.
\end{gather*}

Таким образом,
\begin{gather*}
|\bar{\mu}^{m_0(\alpha)}_{s_0(\alpha),p_0(\alpha)}-\mu|_w\leqslant |\bar{\mu}^{m_0(\alpha)}_{s_0(\alpha),p_0(\alpha)}-\bar\mu^{m_0(\alpha)}_{s_0(\alpha)}|_w+
|\bar{\mu}^{m_0(\alpha)}_{s_0(\alpha)}-\bar\mu^{m_0(\alpha)}|_w+|\bar{\mu}^{m_0(\alpha)}-\mu|_w\leqslant \frac\alpha3+\frac\alpha3+\frac\alpha3=\alpha,
\end{gather*}
то есть
\begin{gather}\label{iskomaya_posledovatel'nost'_mer}
|\bar{\mu}^{m_0(\alpha)}_{s_0(\alpha),p_0(\alpha)}-\mu|_w\leqslant\alpha.
\end{gather}
Пусть $\alpha_k>0$, $k=1,2,\dots$, $\alpha_k\to0$, $k\to\infty$, --- некоторая последовательность. Тогда из (\ref{iskomaya_posledovatel'nost'_mer}) следует, что
\begin{gather*}
\bar{\mu}^{m_0(\alpha_k)}_{s_0(\alpha_k),p_0(\alpha_k)}\to\mu,\,\,\,k\to\infty,\text{ $*$--слабо.}
\end{gather*}
Следовательно, в качестве искомой последовательности~$\omega^k$, $k=1,2,\dots$, можно взять последовательность~$\omega^k\equiv\bar\omega_{m_0(\alpha_k)}^{s_0(\alpha_k),p_0(\alpha_k)}$,
$k=1,2,\dots$. Лемма доказана.
\end{Proof}

    \chapter{О некоторых обыкновенных дифференциальных уравнениях}
        \section{Лемма Гронуолла и её следствие}
\begin{Lemma}\label{Gronwall}\cite{ATF}
Пусть функции $\varepsilon(t)$, $\lambda(t)$, $t\in[\alpha,\beta]$, --- измеримые по Лебегу функции, неотрицательные п.в. на отрезке $[\alpha,\beta]$; и пусть произведение $\varepsilon(t)
\lambda(t)$, $t\in[\alpha,\beta]$, --- суммируемо по отрезку $[\alpha,\beta]$. Если для некоторых $b\geqslant0$, $\tau\in[\alpha,\beta]$, и п.в. $t\in[\alpha,\beta]$ справедливо неравенство
\begin{gather*}
\varepsilon(t)\leqslant\left|\int\limits_\tau^t\varepsilon(\xi)\lambda(\xi)d\xi\right|+b,
\end{gather*}
то
\begin{gather*}
\varepsilon(t)\leqslant b\exp\left(\left|\int\limits_\tau^t\lambda(\xi)d\xi\right|\right) \mbox{ при п.в. $t\in[\alpha,\beta]$.}
\end{gather*}
\end{Lemma}

\begin{Lemma}\label{myGronwall}
Пусть $\sigma(t)$, $t\in[\alpha,\beta]$, --- непрерывная неотрицательная на $[\alpha,\beta]$ функция; $\varkappa(t)$, $t\in[\alpha,\beta]$, --- неотрицательная суммируемая по $[\alpha,
\beta]$ функция, и, наконец, пусть $\gamma$ , $\omega\geqslant0$. Тогда если
\begin{gather}\label{sigma:source_inequality}
\sigma(t)\leqslant\gamma[\sigma(\alpha)+\omega\max\limits_{\xi\in[\alpha,t]}\sqrt{\sigma(\xi)}]+\int\limits_\alpha^t\varkappa(\xi)\sigma(\xi)d\xi\,\,\,\forall\,t\in[\alpha,\beta],
\end{gather}
то
\begin{gather}\label{sigma:resulting_inequality}
\max\limits_{t\in[\alpha,\beta]}\sqrt{\sigma(t)}\leqslant\gamma[\sqrt{\sigma(\alpha)}+\omega]\exp\left(\int\limits_\alpha^\beta\varkappa(\xi)d\xi\right).
\end{gather}
\end{Lemma}
\begin{Proof}
Введём функцию $y\colon[\alpha,\beta]\to R$ формулой $y(t)\equiv\max\limits_{\xi\in[\alpha,t]} \sqrt{\sigma(\xi)}$, $t\in[\alpha,\beta]$. Тогда (\ref{sigma:source_inequality}) примет вид
\begin{gather}\label{sigma:source_inequality1}
\sigma(t)\leqslant\gamma[\sigma(\alpha)+\omega y(t)]+ \int\limits_\alpha^t\varkappa(\xi)\sigma(\xi)d\xi\,\,\,\forall\,t\in[\alpha,\beta].
\end{gather}
Отметим, что функция $y$ монотонно не убывает на отрезке $[\alpha,\beta]$ и
\begin{gather}\label{1ymonotonic}
\frac1{y(t_1)}\geqslant\frac1{y(t_2)},\,\,\,t_1\leqslant t_2, \mbox{ если $y(t_1)>0$.}
\end{gather}
Обозначим через $\mathcal{N}$ множество нулей функции $y$. Покажем, что либо $\mathcal{N}= \emptyset$, либо $\mathcal{N}$ представимо в виде $\mathcal{N}=[\alpha,\sup\mathcal{N}]$. В случае
$\mathcal{N}=\{\alpha\}$ последнее очевидно. Поэтому пусть $\mathcal{N}\neq\emptyset$, $\mathcal{N}\neq\{\alpha\}$. Заметим, что в силу неубывания функции $y$
\begin{gather}\label{alphatsubseteq}
\forall\,t\in\mathcal{N}:[\alpha,t]\subseteq\mathcal{N}.
\end{gather}
Пусть $t^*=\sup\mathcal{N}$. На основании определения верхней грани найдётся строго монотонно возрастающая последовательность $t_k\in\mathcal{N}$, $k=1,2,\dots$, такая, что $t_k\to t^*$,
$k\to\infty$. Таким образом, ввиду (\ref{alphatsubseteq}), $[\alpha,t^*)=\bigcup\limits_{k=1}^\infty[\alpha,t_k]\subseteq\mathcal{N}$. Пользуясь теперь непрерывностью функции $\sigma$,
заключаем, что $\mathcal{N}=[\alpha,\sup\mathcal{N}]$.

Ясно, что если $\mathcal{N}=[\alpha,\beta]$, то неравенство (\ref{sigma:resulting_inequality}) справедливо.

Пусть $\mathcal{N}=\emptyset$. Поделив (\ref{sigma:source_inequality1}) на $y(t)$ и воспользовавшись (\ref{1ymonotonic}), получим соотношение
\begin{gather*}
\frac{\sigma(t)}{y(t)}\leqslant\gamma[\sqrt{\sigma(\alpha)}+ \omega]+\int\limits_\alpha^t\varkappa(\xi)\frac{\sigma(\xi)}{y(\xi)}d\xi\,\,\,\forall\,t\in[\alpha,\beta].
\end{gather*}
Применив к нему лемму (\ref{Gronwall}), получим, что
\begin{gather*}
\frac{\sigma(t)}{y(t)}\leqslant\gamma[\sqrt{\sigma(\alpha)}+\omega] \exp\left(\int\limits_\alpha^\beta\varkappa(\xi)d\xi\right)\,\,\,\mbox{ при п.в. $t\in[\alpha,\beta]$}.
\end{gather*}
В частности,
\begin{gather*}
\frac{\sigma(t)}{y(\tau)}\leqslant\gamma[\sqrt{\sigma(\alpha)}+\omega] \exp\left(\int\limits_\alpha^\beta\varkappa(\xi)d\xi\right)\,\,\,
\mbox{ при п.в. $t\in[\alpha,\tau]$ при всех $\tau\in(\alpha,\beta]$}.
\end{gather*}
Переходя здесь к точной верхней грани по $t\in[\alpha,\tau]$, получаем неравенство (\ref{sigma:resulting_inequality}).

Пусть $\mathcal{N}=\{\alpha\}$. Тогда неравенство (\ref{sigma:resulting_inequality}) заведомо выполнено при $t=\alpha$. Предположим теперь, что $t>\alpha$.  Поделив
(\ref{sigma:source_inequality1}) на $y(t)$ и воспользовавшись (\ref{1ymonotonic}), выводим, что
\begin{gather*}
\frac{\sigma(t)}{y(t)}\leqslant\gamma\omega+\int\limits_\alpha^t\varkappa(\xi)\frac{\sigma(\xi)}{y(\xi)}d\xi\,\,\,\forall\,t\in[\alpha,\beta].
\end{gather*}
Применив к данному неравенству лемму (\ref{Gronwall}), получим, что
\begin{gather*}
\frac{\sigma(t)}{y(t)}\leqslant\gamma[\sqrt{\sigma(\alpha)}+\omega] \exp\left(\int\limits_\alpha^\beta\varkappa(\xi)d\xi\right)\,\,\,\mbox{ при п.в. $t\in[\alpha,\beta]$}.
\end{gather*}
В частности,
\begin{gather*}
\frac{\sigma(t)}{y(\tau)}\leqslant\gamma\omega \exp\left(\int\limits_\alpha^\beta\varkappa(\xi)d\xi\right)\,\,\,\mbox{ при п.в. $t\in[\alpha,\tau]$ при всех $\tau\in(\alpha,\beta]$}.
\end{gather*}
Переходя здесь к точной верхней грани по $t\in[\alpha,\tau]$, получаем неравенство (\ref{sigma:resulting_inequality}).

Пусть $\mathcal{N}=[\alpha,t^*]$ и $t^*\in(\alpha,\beta)$. Тогда неравенство (\ref{sigma:resulting_inequality}) заведомо выполнено при $t\in[\alpha,t^*]$. Предположим теперь, что $t>t^*$.
Поделив (\ref{sigma:source_inequality1}) на $y(t)$ и воспользовавшись (\ref{1ymonotonic}), заключаем, что
\begin{gather*}
\frac{\sigma(t)}{y(t)}\leqslant\gamma\omega+\int\limits_{t^*}^t\varkappa(\xi)\frac{\sigma(\xi)}{y(\xi)}d\xi\,\,\, \forall\,t\in[t^*,\beta].
\end{gather*}
Применив к данному неравенству лемму \ref{Gronwall}, получим, что
\begin{gather*}
\frac{\sigma(t)}{y(t)}\leqslant\gamma[\sqrt{\sigma(\alpha)}+\omega] \exp\left(\int\limits_\alpha^\beta\varkappa(\xi)d\xi\right)\,\,\,\mbox{ при п.в. $t\in[t^*,\beta]$}.
\end{gather*}
В частности,
\begin{gather*}
\frac{\sigma(t)}{y(\tau)}\leqslant\gamma\omega \exp\left(\int\limits_{t^*}^\beta\varkappa(\xi)d\xi\right)\,\,\,
\mbox{ при п.в. $t\in[t^*,\tau]$ при всех $\tau\in(t^*,\beta]$}.
\end{gather*}
Переходя здесь к точной верхней грани по $t\in[t^*,\tau]$, получаем неравенство
\begin{gather*}
\max\limits_{t\in[t^*,\beta]}\sqrt{\sigma(t)}\leqslant\gamma\omega \exp\left(\int\limits_{\alpha}^\beta\varkappa(\xi)d\xi\right).
\end{gather*}
А с учётом того, что $\sigma(t)\equiv0$, $t\in[\alpha,t^*]$, будем иметь
\begin{gather*}
\max\limits_{t\in[\alpha,\beta]}\sqrt{\sigma(t)}\leqslant\gamma\omega \exp\left(\int\limits_{\alpha}^\beta\varkappa(\xi)d\xi\right).
\end{gather*}
А это и есть требуемое неравенство (\ref{sigma:resulting_inequality}).
\end{Proof}


        \section{Уравнения первого порядка}
Пусть $\mathfrak{A}\in C([0,T],\mathbb{R}^{m\times m})$, $\mathfrak{B}\in C^1(\Gamma,\mathbb{R}^m)$, $\varphi\in C^1([0,T],\mathbb{R}^m)$.

Введём при каждом $\tau\in[0,T]$ функцию $h(t,\tau)$, $(t,\tau)\in\Gamma$, как решение задачи Коши
\begin{gather}
\label{sec:aux_results:ODE:hCauchy:equation}
h_t(t,\tau)=\mathfrak{A}(t)h(t,\tau)+\mathfrak{B}(t,\tau),\,\,\,(t,\tau)\in\Gamma,\\
\label{sec:aux_results:ODE:hCauchy:initial_cond}
h(\tau,\tau)=\varphi(\tau),\,\,\,\tau\in[0,T].
\end{gather}

Дадим следующее
\begin{Definition}\label{sec:aux_results:ODE:hCauchy:solution_definition}
Функцию $h\in C^1(\Gamma,\mathbb{R}^m)$ назовём решением задачи (\ref{sec:aux_results:ODE:hCauchy:equation})--(\ref{sec:aux_results:ODE:hCauchy:initial_cond}), если она всюду в $\Gamma$
удовлетворяет уравнению (\ref{sec:aux_results:ODE:hCauchy:equation}) и при всех $\tau\in[0,T]$ удовлетворяет начальному условию (\ref{sec:aux_results:ODE:hCauchy:initial_cond}).
\end{Definition}

Покажем, что справедлива следующая
\begin{Lemma}\label{sec:aux_results:ODE:hCauchy:unique_existence_theorem}
Задача Коши (\ref{sec:aux_results:ODE:hCauchy:equation})--(\ref{sec:aux_results:ODE:hCauchy:initial_cond}) имеет единственное решение $h\in C^1(\Gamma,\mathbb{R}^m)$. При этом функция $h$
имеет непрерывные на $\Gamma$ производные $h_{\tau t}$ и $h_{t\tau}$, а функция $h_\tau$ является решением задачи Коши
\begin{gather}\label{sec:aux_results:ODE:h_tau_Cauchy}
h_{\tau t}(t,\tau)=\mathfrak{A}(t)h_\tau(t,\tau)+\mathfrak{B}_\tau(t,\tau),\,\,\,(t,\tau)\in\Gamma;\,\,\,
h_{\tau}(t,\tau)|_{t=\tau}=\varphi'(\tau)-[\mathfrak{A}(\tau)\varphi(\tau)+\mathfrak{B}(\tau,\tau)].
\end{gather}
\end{Lemma}
\begin{Proof}
1) Покажем, что задача Коши (\ref{sec:aux_results:ODE:hCauchy:equation})--(\ref{sec:aux_results:ODE:hCauchy:initial_cond}) эквивалентна некоторому интегральному уравнению.

В самом деле, пусть $h\in C^1(\Gamma,\mathbb{R}^m)$ --- решение задачи Коши (\ref{sec:aux_results:ODE:hCauchy:equation})--(\ref{sec:aux_results:ODE:hCauchy:initial_cond}). Проинтегрировав
уравнение (\ref{sec:aux_results:ODE:hCauchy:equation}) по $t$ от $\tau$ до $\xi$, получим, что $h$ является принадлежащим классу $\mathbb{K}^1_m(\Gamma)$ решением интегрального уравнения
\begin{gather}\label{sec:aux_results:ODE:hCauchy:integral_equation}
h(\xi,\tau)=\varphi(\tau)+\int\limits_\tau^\xi[\mathfrak{A}(t)h(t,\tau)+\mathfrak{B}(t,\tau)]dt,\,\,\,(\xi,\tau)\in\Gamma.
\end{gather}

Обратно, пусть $h$ --- принадлежащее классу $\mathbb{K}^1_m(\Gamma)$ решение интегрального уравнения (\ref{sec:aux_results:ODE:hCauchy:integral_equation}). Тогда она имеет производную по
$t$, при всех $(t,\tau)\in\Gamma$ удовлетворяет уравнению (\ref{sec:aux_results:ODE:hCauchy:equation}) и при всех $\tau\in[0,T]$ удовлетворяет начальному условию
(\ref{sec:aux_results:ODE:hCauchy:initial_cond}).

Итак, любое принадлежащее классу $C^1(\Gamma,\mathbb{R}^m)$ решение задачи Коши (\ref{sec:aux_results:ODE:hCauchy:equation})--(\ref{sec:aux_results:ODE:hCauchy:initial_cond}) является
принадлежащим классу $\mathbb{K}^1_m(\Gamma)$ решением интегрального уравнения (\ref{sec:aux_results:ODE:hCauchy:integral_equation}), и наоборот, любое принадлежащее классу
$\mathbb{K}^1_m(\Gamma)$ решение интегрального уравнения (\ref{sec:aux_results:ODE:hCauchy:integral_equation}) является принадлежащим классу  $C^1(\Gamma,\mathbb{R}^m)$ решением задачи
Коши (\ref{sec:aux_results:ODE:hCauchy:equation})--(\ref{sec:aux_results:ODE:hCauchy:initial_cond}).

Покажем, что интегральное уравнение (\ref{sec:aux_results:ODE:hCauchy:integral_equation}) имеет единственное решение в классе $\mathbb{K}^1_m(\Gamma)$. Введём оператор $\mathfrak{C}\colon
\mathbb{K}^1_m(\Gamma)\to \mathbb{K}^1_m(\Gamma)$ равенством
\begin{gather*}
\mathfrak{C}[h](t,\tau)= \varphi(\tau)+\int\limits_\tau^t[\mathfrak{A}(\xi)h(\xi,\tau)+\mathfrak{B}(\xi,\tau)]d\xi,\,\,\,(t,\tau)\in\Gamma,
\end{gather*}
и покажем, что некоторая степень этого оператора является сжатием.

Заметим прежде всего, что
\begin{gather*}
\frac{\partial}{\partial\tau}\mathfrak{C}[h](t,\tau)=\varphi'(\tau)-[\mathfrak{A}(\tau)h(\tau,\tau)+\mathfrak{B}(\tau,\tau)]+
\int\limits_\tau^t[\mathfrak{A}(\xi)h_\tau(\xi,\tau)+\mathfrak{B}_\tau(\xi,\tau)]d\xi,\,\,\,(t,\tau)\in\Gamma.
\end{gather*}

Пусть теперь $h_1$, $h_2\in \mathbb{K}^1_m(\Gamma)$ --- произвольны. Тогда

\begin{gather}\label{sec:aux_result:ODE:mathfrakC_estimate}
|\mathfrak{C}[h_1](t,\tau)-\mathfrak{C}[h_2](t,\tau)|\leqslant K\left|\int\limits_\tau^t|h_1(\xi,\tau)-h_2(\xi,\tau)|d\xi\right|\leqslant K\pmb{|}h_1-h_2\pmb{|}^{(0)}_{\Gamma,\mathbb{R}^m}
|t-\tau|,\\
\notag \left|\frac{\partial}{\partial\tau}\mathfrak{C}[h_1](t,\tau)-\frac{\partial}{\partial\tau}\mathfrak{C}[h_2](t,\tau)\right|\leqslant K\left|\int\limits_\tau^t
|h_{1\tau}(\xi,\tau)-h_{2\tau}(\xi,\tau)|d\xi\right|\leqslant K\pmb{|}h_{1\tau}-h_{2\tau}\pmb{|}^{(0)}_{\Gamma,\mathbb{R}^m}|t-\tau|,\\
\notag (t,\tau)\in\Gamma,
\end{gather}
где $K=\max\limits_{t\in[0,T]}|\mathfrak{A}(t)|$.

Предположим, что для некоторого $k\geqslant1$ уже доказано, что
\begin{gather}\label{sec:aux_result:ODE:mathfrakCk_estimate}
|\mathfrak{C}^k[h_1](t,\tau)-\mathfrak{C}^k[h_2](t,\tau)|\leqslant\frac{(K|t-\tau|)^k}{k!}\pmb{|}h_1-h_2\pmb{|}^{(0)}_{\Gamma,\mathbb{R}^m},\\
\notag \left|\frac{\partial}{\partial\tau}\mathfrak{C}^k[h_1](t,\tau)-\frac{\partial}{\partial\tau}\mathfrak{C}^k[h_2](t,\tau)
\right|\leqslant \frac{(K|t-\tau|)^k}{k!}\pmb{|}h_{1\tau}-h_{2\tau}\pmb{|}^{(0)}_{\Gamma,\mathbb{R}^m},\,\,\,(t,\tau)\in\Gamma.
\end{gather}
Тогда, согласно (\ref{sec:aux_result:ODE:mathfrakC_estimate}) и предположению индукции,
\begin{gather*}
|\mathfrak{C}^{k+1}[h_1](t,\tau)-\mathfrak{C}^{k+1}[h_2](t,\tau)|=|\mathfrak{C}[\mathfrak{C}^k[h_1]](t,\tau)-\mathfrak{C}[\mathfrak{C}^k[h_2]](t,\tau)|\leqslant\\
\leqslant K\left|\int\limits_\tau^t|\mathfrak{C}^k[h_1](\xi,\tau)-\mathfrak{C}^k[h_2](\xi,\tau)|d\xi\right|\leqslant\frac{K^{k+1}}{k!}\pmb{|}h_1-h_2\pmb{|}^{(0)}_{\Gamma,\mathbb{R}^m}
\left|\int\limits_\tau^t|\xi-\tau|^kd\xi\right|=\frac{(K|t-\tau|)^{k+1}}{(k+1)!}\pmb{|}h_1-h_2\pmb{|}^{(0)}_{\Gamma,\mathbb{R}^m};\\
%
\left|\frac{\partial}{\partial\tau}\mathfrak{C}^{k+1}[h_1](t,\tau)-\frac{\partial}{\partial\tau}\mathfrak{C}^{k+1}[h_2](t,\tau)\right|=\left|\frac{\partial}{\partial\tau}\mathfrak{C}[
\mathfrak{C}^k[h_1]](t,\tau)-\frac{\partial}{\partial\tau}\mathfrak{C}[\mathfrak{C}^k[h_2]](t,\tau)\right|\leqslant\\
\leqslant K\left|\int\limits_\tau^t\left|\frac{\partial}{\partial\tau}\mathfrak{C}^k[h_1](\xi,\tau)-\frac{\partial}{\partial\tau}\mathfrak{C}^k[h_2](\xi,\tau)\right|d\xi\right|\leqslant
\frac{K^{k+1}}{k!}\pmb{|}h_{1\tau}-h_{2\tau}\pmb{|}^{(0)}_{\Gamma,\mathbb{R}^m}\left|\int\limits_\tau^t|\xi-\tau|^kd\xi\right|=\\
=\frac{(K|t-\tau|)^{k+1}}{(k+1)!}\pmb{|}h_{1\tau}-h_{2\tau}\pmb{|}^{(0)}_{\Gamma,\mathbb{R}^m}.
\end{gather*}
Таким образом, оценка (\ref{sec:aux_result:ODE:mathfrakCk_estimate}) имеет место для всех $k\geqslant1$. Поэтому
\begin{gather*}
\pmb{|}\mathfrak{C}^k[h_1]-\mathfrak{C}^k[h_2]\pmb{|}^{(0)}_{\Gamma,\mathbb{R}^m}\leqslant\frac{(KT)^k}{k!}\pmb{|}h_{1}-h_{2}\pmb{|}^{(0)}_{\Gamma,\mathbb{R}^m},\,\,\,
\pmb{\Bigl|}\frac{\partial}{\partial\tau}\mathfrak{C}^k[h_1]- \frac{\partial}{\partial\tau}\mathfrak{C}^k[h_2]{\pmb{\Bigr|}}^{(0)}_{\Gamma,\mathbb{R}^m} \leqslant\frac{(KT)^k}{k!}
\pmb{|}h_{1\tau}-h_{2\tau}\pmb{|}^{(0)}_{\Gamma,\mathbb{R}^m},
\end{gather*}
откуда
\begin{gather*}
\|\mathfrak{C}^k[h_1]-\mathfrak{C}^k[h_2]\|_{\mathbb{K}^1_m(\Gamma)}\leqslant \frac{(KT)^k}{k!}\|h_1-h_2\|_{\mathbb{K}^1_m(\Gamma)}.
\end{gather*}
Следовательно, некоторая степень оператора $\mathfrak{C}$ является сжатием, на основании чего интегральное уравнение (\ref{sec:aux_results:ODE:hCauchy:integral_equation}) имеет единственное
решение в классе $\mathbb{K}^1_m(\Gamma)$. Ввиду доказанной выше эквивалентности задачи Коши (\ref{sec:aux_results:ODE:hCauchy:equation})--(\ref{sec:aux_results:ODE:hCauchy:initial_cond}) и
интегрального уравнения (\ref{sec:aux_results:ODE:hCauchy:integral_equation}) это даёт однозначную разрешимость задачи Коши
(\ref{sec:aux_results:ODE:hCauchy:equation})--(\ref{sec:aux_results:ODE:hCauchy:initial_cond}) в классе $C^1(\Gamma,\mathbb{R}^m)$.

2) Докажем теперь, что функция $h_\tau$ является решением задачи Коши (\ref{sec:aux_results:ODE:h_tau_Cauchy}). В самом деле, продифференцировав интегральное уравнение
(\ref{sec:aux_results:ODE:hCauchy:integral_equation}) по переменной $\tau$, будем иметь
\begin{gather*}
h_\tau(t,\tau)=\varphi'(\tau)-[\mathfrak{A}(\tau)\varphi(\tau)+\mathfrak{B}(\tau,\tau)]+\int\limits_\tau^t[\mathfrak{A}(\xi)h(\xi,\tau)+\mathfrak{B}(\xi,\tau)]d\xi,\,\,\,(t,\tau)\in\Gamma.
\end{gather*}
Дифференцируя данное равенство по $t$, получаем, что функция $h_\tau$ является решением задачи Коши (\ref{sec:aux_results:ODE:h_tau_Cauchy}).

3) Из доказанной ранее непрерывной дифференцируемости функции $h$ и того, что $h_\tau$ --- решение задачи Коши (\ref{sec:aux_results:ODE:h_tau_Cauchy}), следует, что функция  $h_{\tau t}$
непрерывна на $\Gamma$. Дифференцируя теперь по переменной $\tau$ уравнение (\ref{sec:aux_results:ODE:hCauchy:equation}), получим непрерывность на $\Gamma$ функции $h_{t\tau}$.

Лемма полностью доказана.
\end{Proof}


        \section{Уравнения второго порядка}
Пусть $\mathcal{A}$, $\mathcal{B}\in C([0,T],\mathbb{R}^{m\times m})$, $\mathcal{C}\in C^1(\Gamma,\mathbb{R}^{m})$, $\varphi$, $\psi\in C^1([0,T],\mathbb{R}^{m})$. Рассмотрим задачу Коши
\begin{gather}
\label{sec:aux_results:ODE2:yCauchy:equation}
y_{tt}(t,\tau)+\mathcal{A}(t)y_t(t,\tau)+\mathcal{B}(t)y(t,\tau)=\mathcal{C}(t,\tau),\,\,\,(t,\tau)\in\Gamma;\\
\label{sec:aux_results:ODE2:yCauchy:initial_conds} y(t,\tau)|_{t=\tau}=\varphi(\tau),\,\,\,y_t(t,\tau)|_{t=\tau}=\psi(\tau),\,\,\,t\in[0,T].
\end{gather}
Дадим следующее
\begin{Definition}
Функцию $y\in C^1(\Gamma,\mathbb{R}^{m})$, имеющую непрерывные на $\Gamma$ производные $y_{tt}$ , $y_{t\tau}$, $y_{\tau t}$, назовём решением задачи Коши
(\ref{sec:aux_results:ODE2:yCauchy:equation})--(\ref{sec:aux_results:ODE2:yCauchy:initial_conds}), если она при всех $(t,\tau)\in\Gamma$ удовлетворяет уравнению
(\ref{sec:aux_results:ODE2:yCauchy:equation}) и при всех $\tau\in[0,T]$ удовлетворяет начальным условиям (\ref{sec:aux_results:ODE2:yCauchy:initial_conds}).
\end{Definition}

Покажем, что справедлива
\begin{Lemma}\label{sec:aux_results:ODE2:yCauchy:unique_existence}
Задача Коши (\ref{sec:aux_results:ODE2:yCauchy:equation})--(\ref{sec:aux_results:ODE2:yCauchy:initial_conds}) имеет единственное решение $y$, понимаемое в только что определённом смысле,
причём существует непрерывная на  $\Gamma$ производная $y_{\tau tt}$, а функция $y_{\tau}$ является решением задачи Коши
\begin{gather}
\label{sec:aux_results:ODE2:y_tau_Cauchy:equation}
y_{\tau tt}(t,\tau)+\mathcal{A}(t)y_{\tau t}(t,\tau)+\mathcal{B}(t)y_\tau(t,\tau)=\mathcal{C}_\tau(t,\tau),\,\,\, (t,\tau)\in\Gamma;\\
\label{sec:aux_results:ODE2:y_tau_Cauchy:initial_conds} y_\tau(t,\tau)|_{t=\tau}=\varphi'(\tau)-\psi(\tau),\,\,\,
y_{\tau t}(t,\tau)|_{t=\tau}=\psi'(\tau)+\mathcal{A}(\tau)\psi(\tau)+\mathcal{B}(\tau)\varphi(\tau)+\mathcal{C}(\tau,\tau),\,\,\,t\in[0,T].
\end{gather}
\end{Lemma}
\begin{Proof}
Сделаем замену $h_1\equiv y$, $h_2\equiv y_t$, и введём матрицу--функцию $\mathfrak{A}(t)$, $t\in[0,T]$ размерности $2m\times2m$, $2m$--мерные вектор--функции $\mathfrak{B}(t,\tau)$,
$\tilde{\varphi}(\tau)$, $h(t,\tau)$, $(t,\tau)\in\Gamma$, соотношениями
\begin{gather*}
h(t,\tau)=\left[\begin{array}{l}
            h_1(t,\tau)\\
            h_2(t,\tau)
                \end{array}\right],\,\,\,
\tilde{\varphi}(\tau)=\left[\begin{array}{l}
            \varphi(\tau)\\
            \psi(\tau)
                \end{array}\right],\,\,\,
\mathfrak{B}(t,\tau)=\left[\begin{array}{l}
            0\\
            \mathcal{C}(t,\tau)
                \end{array}\right],\,\,\,
\mathfrak{A}(t)=\left[\begin{array}{l|l}
            O_{m\times m}  & E_m\\ \hline
            -\mathcal{B}(t)&-\mathcal{A}(t)
                \end{array}\right],
\end{gather*}
где $O_{m\times m}$ --- нулевая $m\times m$--матрица, $E_m$ --- единичная матрица порядка $m$.

Тогда задача Коши (\ref{sec:aux_results:ODE2:yCauchy:equation})--(\ref{sec:aux_results:ODE2:yCauchy:initial_conds}) примет вид
\begin{gather*}
h_t(t,\tau)=\mathfrak{A}(t)h(t,\tau)+\mathfrak{B}(t,\tau),\,\,\,(t,\tau)\in\Gamma;\,\,\, h(t,\tau)|_{t=\tau}=\tilde{\varphi}(\tau),\,\,\,
\tau\in[0,T].
\end{gather*}
Применив к данной задаче Коши лемму \ref{sec:aux_results:ODE:hCauchy:unique_existence_theorem}, получим утверждения доказываемой леммы.
\end{Proof}

    \chapter{Абстрактная задача Коши и энергетическое расширение}
        \section{Энергетическое расширение}\label{energetic_extension}
Изложение материала настоящего раздела следует \cite[Глава IV, \S2]{osipov.vasilev.potapov}.

Пусть $H$ --- сепарабельное вещественное гильбертово пространство со скалярным произведением $\langle\cdot,\cdot\rangle_H$ и соответствующей нормой $\|\cdot\|_H$. Пусть $A:H\to H$ ---
линейный неограниченный оператор с областью определения $D(A)$, плотной в $H$. Пусть оператор $A$ симметричен, т.е.
\begin{gather}\label{energetic_extension!simmetricity}
\langle Af,\varphi\rangle_H=\langle f,A\varphi\rangle_H\,\,\,\forall\,f,\,\,\varphi\in D(A),
\end{gather}
и положительно определён, т.е.
\begin{gather}\label{energetic_extension!positive}
\langle Af,f\rangle_H\geqslant\mu\|f\|^2_H\,\,\,\forall\,f\in D(A),
\end{gather}
где $\mu>0$ --- константа, не зависящая от выбора $f\in D(A)$.

Опишем процесс энергетического расширения такого оператора.

Обозначим через $H^*$ сопряжённое к $H$ пространство линейных непрерывных функционалов, заданных на $H$. Согласно теореме Рисса, для любого $f\in H^*$ существует, и притом единственный,
элемент $\eta\in H$, такой, что
$$
\langle f,\varphi\rangle=\langle\eta,\varphi\rangle_H\,\,\,\forall\,\varphi\in H;\,\,\,\|f\|_{H^*}=\|\eta\|_H.
$$
Равенство $R_Hf=\eta$ определяет оператор $R_H:H^*\to H$, называемый оператором Рисса. Перечислим некоторые свойства оператора $R_H$.

1) Справедливо равенство
$$
\langle f,\varphi\rangle=\langle R_Hf,\varphi\rangle_H\,\,\,\forall\,f\in H^*,\,\,\varphi\in H.
$$

2) Оператор $R_H$ --- линеен:
\begin{gather*}
R_H(\alpha f+\beta g)=\alpha R_Hf+\beta R_Hg\,\,\,\forall\,f,\,\,g\in H^*,\,\,\,\forall\,\alpha,\,\,\beta\in\mathbb{R}.
\end{gather*}

3) Оператор $R_H$ --- ограничен, с нормой $\|R_H\|_{H^*\to H}=1$, и изометричен, т.е. сохраняет норму:
$$
\|R_Hf\|_H=\|f\|_{H^*}\,\,\,\forall\,f\in H^*.
$$

4) Сопряжённое пространство $H^*$ является гильбертовым, причём скалярное произведение в нём задаётся формулой
\begin{gather}\label{energetic_extension!scalar.product.in.H*}
\langle f,g\rangle_{H_*}=\langle R_Hf,R_Hg\rangle_H\,\,\,\forall\,f,\,\,g\in H^*.
\end{gather}

5) Оператор $R_H$ взаимно однозначно отображает $H^*$ на $H$, норма обратного оператора $R_H^{-1}:H\to H^*$ равна единице, и оператор $R^{-1}_H$ --- изометричен, т.е.
$$
\|R_H^{-1}\eta\|_{H^*}=\|\eta\|_H,\,\,\,\langle R^{-1}_H\eta,\varphi\rangle=\langle\eta,\varphi\rangle_H\,\,\,\forall\,\eta,\,\,\varphi\in H.
$$

6) Справедливо равенство $R_H^*=R_H^{-1}$.

В самом деле, в силу \eqref{energetic_extension!scalar.product.in.H*}
\begin{gather*}
\langle f,g\rangle_{H^*}=\langle R_Hf,R_Hg\rangle_H=\langle R_H^*R_Hf,g\rangle_H\,\,\,\forall\,f,\,\,g\in H^*.
\end{gather*}
Поэтому
\begin{gather*}
f=R_H^*R_Hf\,\,\,\forall\,f\in H^*,
\end{gather*}
то есть
\begin{gather*}
R_H^*R_H=I,
\end{gather*}
где $I$ --- тождественный оператор.

Таким образом, оператор Рисса $R_H$ устанавливает взаимно однозначное изометричное соответствие между пространствами $H$ и $H^*$. Это соответствие обозначают $H\simeq H^*$. В силу
указанного соответствия далее будем отождествлять пространства $H$ и $H^*$ и писать просто $H=H^*$, не различая элементы $R_Hf$ и $f$, $R_H^{-1}\eta$ и $\eta$, и опуская в рассуждениях и
формулах символы $R_H$ и $R_H^{-1}$.

\begin{Definition}
Пусть $V$, $H$ --- гильбертовы пространства. Говорят, что вложение $V\subset H$ плотно и непрерывно, если

1) имеет место поэлементное вложение: $\forall\,v\in V:v\in H$;

2) справедливо условие
\begin{gather*}
\forall\,f\in H\,\forall\,\varepsilon>0\,\exists\,v_\varepsilon\in V:\|f-v_\varepsilon\|_H\leqslant\varepsilon
\end{gather*}
(плотность вложения);

3) выполняется неравенство
\begin{gather}\label{energetic_extension!continuity.of.embedding}
\exists\,C>0\,\forall\,v\in V:\|v\|_H\leqslant C\|v\|_H
\end{gather}
(непрерывность вложения).
\end{Definition}

Пусть $V^*$ --- сопряжённое к $V$ пространство. Если вложение $V\subset H$ непрерывно, то, оказывается, можно говорить о вложении $H\subset V^*$. В самом деле, возьмём произвольный элемент
$f\in H$. Тогда $R_H^{-1}f\in H^*$, причём
\begin{gather*}
\langle R_H^{-1}f,\varphi\rangle=\langle f,\varphi\rangle_H\,\,\,\forall\,f,\,\,\varphi\in H.
\end{gather*}
Так как $V\subset H$, то это равенство выполняется при всех $\varphi\in V$. Отсюда с учётом \eqref{energetic_extension!continuity.of.embedding} имеем
\begin{gather}\label{energetic_extension!RH-1continuity}
|\langle R_H^{-1}f,v\rangle|=|\langle f,v\rangle_H|\leqslant\|f\|_H\|v\|_H\leqslant\|f\|_HC\|v\|_V\,\,\,\forall\, v\in V.
\end{gather}
Это означает, что $R_H^{-1}f$ является линейным непрерывным функционалом на $V$, т.е. $R_H^{-1}f\in V^*$. Поскольку при отождествлении $H=H^*$ элементы $f$ и $R_H^{-1}f$ мы не различаем,
то можно считать, что $f\in V^*$ и
$$
\langle f,\varphi\rangle=\langle f,\varphi\rangle_H\,\,\,\forall\,f,\,\,\varphi\in H.
$$
Таким образом, вложение $H\subset V^*$ действительно имеет смысл, и мы приходим к цепочке вложений
\begin{gather}\label{energetic_extension!embeddings.chain}
V\subset H\simeq H^*\subset V^*.
\end{gather}

\begin{Theorem}\label{VHdense!Th}
Если вложение $V\subset H$ плотно и непрерывно, то и вложение $H\subset V^*$ тоже плотно и непрерывно.
\end{Theorem}
\begin{Proof}
Прежде всего нужно убедиться, что вложение $H\subset V^*$ является поэлементным, т.е. разные элементы $f$, $g\in H$ при описанном вложении порождают разные функционалы из $V^*$. В самом
деле, если два элемента $f$, $g\in H$ порождают один и тот же функционал, то  $\langle f,v\rangle=\langle g,v\rangle$, или, иначе,
\begin{gather*}
\langle R_H^{-1}f,v\rangle_H=\langle R_H^{-1}g,v\rangle_H\,\,\,\forall\,v\in V.
\end{gather*}
Следовательно,
\begin{gather*}
\langle R_H^{-1}(f-g),v\rangle_H\,\,\,\forall\,v\in V.
\end{gather*}
Отсюда и из плотности вложения $V\subset H$ следует, что $R_H^{-1}(f-g)=0$, т.е. $f=g$.

Нетрудно убедиться в том, что вложение $H\subset V^*$ непрерывно. Действительно, из определения нормы функционала
\begin{gather*}
\|f\|_{V^*}=\sup\limits_{v\in V,\,\,\|v\|_V\leqslant1}|\langle f,v\rangle|
\end{gather*}
и соотношений \eqref{energetic_extension!RH-1continuity} следует, что
\begin{gather}\label{energetic_extension!f}
\|f\|_{V^*}=\sup\limits_{v\in V,\,\,\|v\|_V\leqslant1}|\langle R_H^{-1}f,v\rangle|\leqslant C\|f\|_H\,\,\,\forall\,f\in H.
\end{gather}

Докажем, что вложение $V\subset V^*$ --- плотно. Для этого достаточно установить, что $cl(R_H^{-1}(V))=V^*$, где $cl(R_H^{-1}(V))$ --- замыкание множества $R_H^{-1}(V)=\{u\in V^*|
\exists\,v\in V:u=R_H^{-1}v\}$ в норме $V^*$. Предположим противное, т.е. что $cl(R_H^{-1}(V))\neq V^*$. Так как $R_H^{-1}(V)\subset V^*$, то, по теореме Хана--Банаха, найдётся функционал
$f_0\in V^{**}$, такой, что $f_0\neq0$, но $\langle f_0,R_H^{-1}v\rangle=0$ при всех $v\in V$. Пусть $R_V$ и $R_{V^*}$ --- операторы Рисса в пространствах $V$ и $V^*$ соответственно. Тогда
отображение $R_{V^*}^{-1}R_V^{-1}:V\to V^{**}$ является изоморфизмом пространств $V$ и $V^{**}$. Поэтому найдётся элемент $h_0\in V$, такой, что $f_0=R_{V^*}^{-1}R_V^{-1}h_0$. С учётом
свойств операторов Рисса $R_V$ и $R_{V^*}$, для всех $v\in V$ имеем
\begin{gather*}
0=\langle f_0,R_H^{-1}v\rangle=\langle R_{V^*}^{-1}R_V^{-1}h_0,R_H^{-1}v\rangle=\langle R_V^{-1}h_0,R_H^{-1}v\rangle_{V^*}=\\
=\langle h_0,R_VR_H^{-1}v\rangle_{V}=\langle h_0,R_V^{-1}R_VR_H^{-1}v\rangle=\langle h_0,R_H^{-1}v\rangle=\langle h_0,v\rangle_H.
\end{gather*}
Таким образом,
\begin{gather*}
\langle h_0,v\rangle_H=0\,\,\,\forall\,v\in V.
\end{gather*}
Учитывая плотность вложения $V\subset H$, заключаем, что $h_0=0$, а тогда и $f_0=0$, что противоречит выбору элемента $f_0$. Следовательно, $cl(R_H^{-1}(V))=V^*$, т.е. вложение
$V\subset V^*$ --- плотно. Из цепочки вложений \eqref{energetic_extension!embeddings.chain} тогда следует и плотность вложения
\begin{gather*}
H\simeq H^*=R_H^{-1}(H)\subset V^*.
\end{gather*}
Теорема доказана.
\end{Proof}

Заодно выяснилось, что  $V^*$ является пополнением пространств $R_H^{-1}(V)$ и $R_H^{-1}(H)$ в норме $\|f\|_{V^*}=\|R_Hf\|_H$.

Рассмотрим цепочку вложений \eqref{energetic_extension!embeddings.chain} для случая, когда пространство $V$ представляет собой так называемое энергетическое пополнение области определения
$D(A)$ линейного неограниченного оператора $A$, обладающего свойствами \eqref{energetic_extension!simmetricity}, \eqref{energetic_extension!positive} и опишем расширение оператора $A$ на
пространство $V$. Заметим, что линейное многообразие $D(A)$ превращается евклидово пространство, если в нём ввести скалярное произведение по формуле
\begin{gather}\label{energetic_extension!A.norm.in.D(A)}
\langle u,v\rangle_A=\langle Au,v\rangle_H\,\,\,\forall\,u,\,\,v\in D(A).
\end{gather}
Соответствующую норму будем обозначать $\|\cdot\|_A$. Справедливость аксиом скалярного произведения и нормы следует из линейности оператора $A$ и его свойств
\eqref{energetic_extension!simmetricity}, \eqref{energetic_extension!positive}. Пространство $D(A)$ пополним в норме $\|\cdot\|_A$, и получившееся в результате пополнения гильбертово
пространство обозначим через $H_A$. Как известно, пространство $H_A$ состоит из идеальных элементов, представляющих собой классы $\mathcal{U}$ эквивалентных фундаментальных в норме
$\|\cdot\|_A$ последовательностей. Напомним, что последовательность $u_k\in D(A)$, $k=1,2,\dots$, называется фундаментальной в норме $\|\cdot\|_A$, если $\|u_k-u_m\|_A\to0$, $k$,
$m\to\infty$; а две фундаментальные $u_k\in D(A)$, $k=1,2,\dots$, и $v_k\in D(A)$, $k=1,2,\dots$, называются эквивалентными, если $\|u_k-v_k\|_A\to0$, $k\to\infty$.  Операции сложения и
умножения на число определяются так. Пусть $\mathcal{U}$, $\mathcal{V}\in H_A$, $\alpha\in\mathbb{R}$. Выберем произвольные фундаментальные последовательности $\{u_k\}\in\mathcal{U}$,
$\{v_k\}\in\mathcal{V}$, и в качестве $\mathcal{U}+\mathcal{V}$ и $\alpha\mathcal{U}$ примем классы последовательностей, эквивалентных последовательностям $\{u_k+v_k\}$ и $\{\alpha u_k\}$
соответственно. Скалярное произведение и норму в $H_A$ вводится так:
\begin{gather*}
\langle\mathcal{U},\mathcal{V}\rangle_A=\lim\limits_{k\to\infty}\langle u_k,v_k\rangle_A,\,\,\,\|\mathcal{U}\|_A=\sqrt{\langle\mathcal{U},\mathcal{U}\rangle_A}.
\end{gather*}
Исходное подпространство $D(A)$ является подпространством в $H_A$ в следующем смысле. Каждый элемент $u\in D(A)$ порождает класс $\mathcal{U}$ фундаментальных последовательностей,
эквивалентных стационарной последовательности $u_k=u$, $k=1,2,\dots$. Ясно, что такой класс $\mathcal{U}$ состоит из последовательностей $\{u_k\}$, сходящихся к элементу $u$ в норме
$\|\cdot\|_A$. Пусть $\tilde{D}(A)$ --- множество классов фундаментальных последовательностей, эквивалентных какой--либо стационарной последовательности элементов множества $D(A)$. Если
классы $\mathcal{U}$, $\mathcal{V}\in \tilde{D}(A)$ порождены стационарными последовательностями, соответствующими элементам $u$, $v\in D(A)$, то
\begin{gather*}
\langle\mathcal{U},\mathcal{V}\rangle_A=\langle u,v\rangle_A=\langle Au,v\rangle_H.
\end{gather*}
Ясно, что $\tilde{D}(A)$ представляет собой подпространство в $H_A$, изоморфное $D(A)$ и плотное в $H_A$. Пространство $H_A$ описано.

Следует заметить, что с классами фундаментальных последовательностей работать неудобно. Поэтому дадим другое, более удобное, описание пространства $H_A$.

Покажем, что каждому классу $\mathcal{U}\in H_A$ можно следующим образом поставить в соответствие элемент $u\in H$. Возьмём произвольную фундаментальную последовательность $\{u_k\}\in
\mathcal{U}$. Из неравенства \eqref{energetic_extension!positive} следует, что
\begin{gather*}
\|u_k-u_m\|_{H}^2\leqslant\frac1\mu\|u_k-u_m\|^2_A\to0,\,\,\,k,\,\,m\to\infty.
\end{gather*}
Это значит, что последовательность $\{u_k\}$ фундаментальна в $H$, и, в силу полноты $H$, сходится к некоторому элементу $u\in H$:
\begin{gather*}
\lim\limits_{k\to\infty}\|u_k-u\|_H=0.
\end{gather*}
Нетрудно видеть, что элемент $u$ не зависит от выбора последовательности $\{u_k\}\in\mathcal{U}$. Искомый элемент $u\in H$, соответствующий классу $\mathcal{U}$, построен. Это соответствие
кратко будем обозначать так: $\mathcal{U}\Rightarrow u$. Построенное соответствие, очевидно, линейно: если $\mathcal{U}\Rightarrow u$, $\mathcal{V}\Rightarrow v$, $\alpha$, $\beta\in
\mathbb{R}$, то $\mathcal{W}=\alpha\mathcal{U}+\beta\mathcal{V}\Rightarrow\alpha u+\beta v$. Важно убедиться, что разным классам $\mathcal{U}$, $\mathcal{V}\in H_A$ соответствуют разные
элементы $u$, $v\in H$. Допустим, что $u=v$. Тогда, в силу линейности построенного соответствия, классу $\mathcal{W}=\mathcal{U}-\mathcal{V}$ соответствует элемент $w=u-v=0\in H$. Покажем,
что это возможно только в том случае, когда $\mathcal{W}=0$, т.е.  $\mathcal{W}$ --- класс фундаментальных последовательностей, эквивалентных стационарной нулевой последовательности
$u_k=0$, $k=1,2,\dots$. Возьмём произвольный элемент $\mathcal{Z}\in\tilde{D}(A)$. По определению множества $\tilde{D}(A)$, существует элемент $z\in D(A)$, такой, что стационарная
последовательность $z_k=z$, $k=1,2,\dots$, принадлежит классу $\mathcal{Z}$. Пусть $\{w_k\}\in\mathcal{W}$. Так как $\mathcal{W}\Rightarrow w=0$, то $w_k\to0$, $k\to\infty$, в норме
$\|\cdot\|_H$. Тогда с учётом равенства \eqref{energetic_extension!simmetricity} имеем
\begin{gather*}
\langle\mathcal{W},\mathcal{Z}\rangle_A=\lim\limits_{k\to\infty}\langle w_k,z_k\rangle_A=\lim\limits_{k\to\infty}\langle Aw_k,z\rangle_H=
\lim\limits_{k\to\infty}\langle w_k,Az\rangle_H=\langle 0,Az\rangle_H=0.
\end{gather*}
Таким образом,
\begin{gather*}
\langle\mathcal{W},\mathcal{Z}\rangle_A=0\,\,\,\forall\,\mathcal{Z}\in\tilde{D}(A).
\end{gather*}
Так как $\tilde{D}(A)$ плотно в $H_A$, то последнее равенство возможно только при $\mathcal{W}=0$, т.е. при $\mathcal{U}=\mathcal{V}$. Тем самым доказано, что если $\mathcal{U}\neq
\mathcal{V}$, $\mathcal{U}\Rightarrow u$, $\mathcal{V}\Rightarrow v$, то $u\neq v$.

Через $V_A$ обозначим множество, состоящее из тех элементов $u\in H$, которые соответствуют какому--либо классу $\mathcal{U}\in H_A$. Линейность построенного соответствия
$\mathcal{U}\Rightarrow u$ гарантирует, что $V_A$ --- линейное подпространство пространства $H$, с операциями, совпадающими с операциями исходного пространства $H$, причём $V_A$
изоморфно $H_A$. Введём в $V_A$ скалярное произведение по правилу
\begin{gather*}
\langle u,v\rangle_{V_A}=\langle\mathcal{U},\mathcal{V}\rangle_A,\text{ где  $\mathcal{U}$, $\mathcal{V}\in H_A$, $\mathcal{U}\Rightarrow u$, $\mathcal{V}\Rightarrow v$}.
\end{gather*}
В результате линейное пространство $V_A$ превращается в евклидово пространство, изометричное гильбертову пространству $H_A$. Значит, $V_A$ также является полным пространством.

Гильбертово пространство $V_A$ принято называть энергетическим пространством, соответствующим симметричному положительному оператору $A:H\to H$. Так как $\tilde{D}(A)\Rightarrow D(A)$ и
$\tilde{D}(A)$ плотно в $H_A$, то $D(A)$ --- подпространство в $V_A$, плотное в $V_A$. В дальнейшем построенные изометричные гильбертовы пространства $H_A$ и $V_A$ будем отождествлять и
использовать в качестве основного обозначение $V_A$.

Энергетическое пространство $V_A$ по самому построению поэлементно вкладывается в пространство $H$. Это вложение плотно, так как $D(A)\subset V_A\subset H$ и $D(A)$ плотно в $H$ ввиду
определения оператора $A$. Кроме того, это вложение непрерывно, что вытекает из неравенства
\begin{gather}\label{energetic_extension!VAtoHemb.cont}
\|u\|_H\leqslant\frac1{\sqrt\mu}\|u\|_A\,\,\,\forall\,u\in V_A.
\end{gather}
Справедливость этого неравенства для $u\in D(A)$ непосредственно следует из \eqref{energetic_extension!positive}, \eqref{energetic_extension!A.norm.in.D(A)}. Если же $u\in V_A$, $u\not\in
D(A)$, то из построения пространства $V_A$ и плотности $D(A)$ в $V_A$ следует существование последовательности $u_k\in D(A)$, $k=1,2,\dots$, такой, что
\begin{gather*}
\lim\limits_{k\to\infty}\|u_k-u\|_H=0,\,\,\,\lim\limits_{k\to\infty}\|u_k-u\|_A=0.
\end{gather*}
Отсюда, зная, что \eqref{energetic_extension!VAtoHemb.cont} верно для $u=u_k\in D(A)$, $k=1,2,\dots$, и переходя к пределу при $k\to\infty$, с учётом непрерывности норм в $H$ и $V_A$
получим \eqref{energetic_extension!VAtoHemb.cont} для всех $u\in V_A$.

Итак, вложение $V_A\subset H$ двух гильбертовых пространств является поэлементным, плотным, и непрерывным. Тогда спрведлива цепочка вложений \eqref{energetic_extension!embeddings.chain}:
\begin{gather}\label{energetic_extension!embeddings.chain-A}
V=V_A\subset H^*\simeq H\subset V^*_A=V^*,
\end{gather}
в которой, согласно теореме \ref{VHdense!Th}, вложение $H^*\subset V_A^*$ --- также поэлементное, плотное, и непрерывное. Неравенство \eqref{energetic_extension!f} здесь имеет вид
\begin{gather}\label{energetic_extension!f1}
\|u\|_{V^*_A}\leqslant \frac1{\sqrt\mu}\|u\|_H\,\,\,\forall\,u\in H.
\end{gather}

Пользуясь \eqref{energetic_extension!VAtoHemb.cont}--\eqref{energetic_extension!f1}, доопределим оператор $A$ на всё пространство $V_A$. Напомним, что оператор $A$ пока что определён на
линейном многообразии $D(A)$, плотном в $V_A$, его значения $Af$ принадлежат $H$ при всех $f\in D(A)$, и, кроме того, он обладает свойствами \eqref{energetic_extension!simmetricity},
\eqref{energetic_extension!positive}.

Пользуясь приведённой выше трактовкой включения \eqref{energetic_extension!embeddings.chain-A}, будем считать, что
\begin{gather*}
Af\in H\simeq H^*\subset V^*_A,
\end{gather*}
т.е. $Af$ --- линейный непрерывный функционал над $V_A$, определённый так (напоминаем, что элементы $R^{-1}_H(Af)$ и $Af$ отождествляются):
\begin{gather}\label{energetic_extension!Afphi}
\langle Af,\varphi\rangle=\langle Af,\varphi\rangle_H=\langle f,\varphi\rangle_A\,\,\,\forall\,\varphi\in V_A\subset H\,\,\,\forall\,f\in D(A)\subset V_A,
\end{gather}
и, следовательно, оператор $A$ действует из $V_A$ в $V^*_A$. Этот оператор является ограниченным. В самом деле, из \eqref{energetic_extension!Afphi} имеем
\begin{gather*}
|\langle Af,\varphi\rangle|=|\langle f,\varphi\rangle_A|\langle\|f\|_A\|\varphi\|_A\,\,\,\forall\,\varphi\in V_A\,\,\,\forall\,f\in D(A).
\end{gather*}
Это означает, что
\begin{gather}\label{energetic_extension!Afnorm}
\|Af\|_{V^*_A}=\sup\limits_{\varphi\in V_A,\,\,\|\varphi\|_A\leqslant1}\langle Af,\varphi\rangle\leqslant\|f\|_A\,\,\,\forall\,f\in D(A),
\end{gather}
т.е. $\|A\|\leqslant1$. Таким образом, $A\in\mathcal{L}(V_A,V_A^*)$, область определения $D(A)$ --- плотна в $H$. Тогда существует оператор $\tilde{A}$ с областью определения $D(\tilde{A})=
V_A$, областью значений $R(\tilde{A})\subset V^*_A$, являющийся продолжением оператора $A$. Оператор $\tilde{A}$ строится так. Если $f\in D(A)$, то полагаем $\tilde{A}f=Af$. Если
$f\in V_A$, но $f\not\in D(A)$, то, в силу построения пространства $V_A$, существует последовательность $f_k\in D(A)$, $k=1,2,\dots$, такая, что
\begin{gather*}
\lim\limits_{k\to\infty}\|f_k-f\|_H=0,\,\,\,\lim\limits_{k\to\infty}\|f_k-f\|_A=0.
\end{gather*}
Тогда
\begin{gather*}
\|Af_k-Af_m\|_{V^*_A}=\|A(f_k-f_m)\|_{V^*_A}\leqslant\|A\|\|f_k-f_m\|_A\to0
\end{gather*}
при $k$, $m\to\infty$. Это значит, что последовательность $Af_k$, $k=1,2,\dots$, --- фундаментальна в $V^*_A$. Поэтому, в силу полноты пространства $V_A^*$, найдётся элемент $y\in V^*_A$,
такой, что $\lim\limits_{k\to\infty}\|Af_k-y\|_{V^*_A}=0$. Нетрудно видеть, что элемент $y$ зависит лишь от $f$, а не от последовательности $f_k$, $k=1,2,\dots$, аппроксимирующей $f$.
Положим по определению $\tilde{A}f=y$. Построенный оператор $\tilde{A}$ определён на всём пространстве $V_A$ и является линейным. Кроме того, из неравенства
\eqref{energetic_extension!Afnorm}, справедливого при $f=f_k\in D(A)$, $k=1,2,\dots$, предельным переходом по $k\to\infty$ получим
\begin{gather*}
\|\tilde{A}f\|_{V^*_A}\leqslant\|f\|_A\,\,\,\forall\,f\in V_A.
\end{gather*}
Следовательно, $\tilde{A}\in\mathcal{L}(V_A,V^*_A)$, причём $\|\tilde{A}\|\leqslant1$. Устремляя в равенстве \eqref{energetic_extension!Afphi}, записанном для $f=f_k$, $k$ к бесконечности,
получаем, что
\begin{gather}\label{energetic_extension!tildeAfphi}
\langle \tilde{A}f,\varphi\rangle=\langle f,\varphi\rangle_A\,\,\,\forall\,f,\,\,\varphi\in V_A.
\end{gather}
В \eqref{energetic_extension!tildeAfphi} переменные $f$ и $\varphi$ равноправны. Поэтому
\begin{gather*}
\langle \tilde{A}f,\varphi\rangle=\langle \varphi,f\rangle_A.
\end{gather*}
Отсюда и из равенства $\langle f,\varphi\rangle_A=\langle \varphi,f\rangle_A$ следует, что
\begin{gather*}
\langle \tilde{A}f,\varphi\rangle=\langle f,\tilde{A}\varphi\rangle\,\,\,\forall\,f,\,\,\varphi\in V_A,
\end{gather*}
т.е. $\tilde{A}$ --- симметричный оператор (здесь и далее символы $\langle \tilde{A}f,\varphi\rangle$ и $\langle \varphi,\tilde{A}f\rangle$, обозначающие значение функционала $\tilde{A}f\in
V^*_A$ на элементе $\varphi$, считаются равноправными). Далее, из \eqref{energetic_extension!VAtoHemb.cont} и \eqref{energetic_extension!tildeAfphi} вытекает положительная определённость
оператора $\tilde{A}$:
\begin{gather*}
\mu\|f\|^2_H\leqslant\|f\|^2_A=\langle f,f\rangle_A=\langle\tilde{A}f,f\rangle\,\,\,\forall\,f\in V_A.
\end{gather*}
Полученный оператор $\tilde{A}$ часто называют энергетическим расширением оператора $A$.

Заметим, что равенство \eqref{energetic_extension!tildeAfphi} можно интерпретировать как явное определение расширенного оператора $\tilde{A}$, так как оно задаёт правило действия значения
$\tilde{A}f$ оператора $\tilde{A}$ на произвольный элемент $\varphi\in V_A$ при каждом $f\in V_A$. Более того, из \eqref{energetic_extension!tildeAfphi} вытекает тесная связь между
оператором $\tilde{A}$ и оператором Рисса $R_{V_A}:V^*\to V$ пространства $V^*$:
\begin{gather}\label{energetic_extension!tildeA.and.Riesz}
\tilde{A}=R^{-1}_{V_A}.
\end{gather}
В самом деле, для оператора $R_{V_A}$ имеем (свойство 5)):
\begin{gather*}
\langle R_{V_A}^{-1}f,\varphi\rangle=\langle f,\varphi\rangle_A\,\,\,\forall\,f,\,\,\varphi\in V_A.
\end{gather*}
Отсюда и из \eqref{energetic_extension!tildeAfphi} получаем
\begin{gather*}
\langle \tilde{A}f,\varphi\rangle=\langle R_{V_A}^{-1}f,\varphi\rangle\,\,\,\forall\,f,\,\,\varphi\in V_A,
\end{gather*}
что равносильно \eqref{energetic_extension!tildeA.and.Riesz}. Из \eqref{energetic_extension!tildeA.and.Riesz} и свойств оператора Рисса следует, что область значений $R(\tilde{A})$
оператора $\tilde{A}$ совпадает с $V^*_A$, оператор $\tilde{A}$ взаимно однозначно отображает $V_A$ на $V_A^*$, обратный оператор $\tilde{A}{}^{-1}=R_{V_A}\in\mathcal{L}(V^*_A,V_A)$,
$\|\tilde{A}\|=\|\tilde{A}{}^{-1}\|=1$, а скалярное произведение в $V_A^*$ с учётом формул \eqref{energetic_extension!scalar.product.in.H*}, \eqref{energetic_extension!tildeA.and.Riesz}
можно записать в виде
\begin{gather*}
\langle f,g\rangle_{V^*_A}=\langle R_{V_A}f,R_{V_A}g\rangle_{V_A}=\langle\tilde{A}{}^{-1}f,\tilde{A}{}^{-1}g\rangle_{V_A}\,\,\,\forall\,f,\,\,g\in V_A.
\end{gather*}

Исследуем теперь существование счётной системы собственных элементов оператора $\tilde{A}$. Для удобства изложения оператор $\tilde{A}$ переобозначим через $A$; а область его определения,
$V_A$, будем обозначать просто через $V$. Иначе говоря, будем предполагать, что описанные выше процедуры расширения уже проведены, и сам оператор $A$ является энергетическим расширением
некоторого линейного, неограниченного, симметричного, положительно определённого оператора с областью определения, плотной в вещественном сепарабельном гильбертовом пространстве $H$. Таким
образом $A\in\mathcal{L}(V,V^*)$, где $V$ --- энергетическое гильбертово пространство, $V^*$ --- сопряжённое к $V$ пространство, $D(A)=V$ --- область определения оператора $A$, $R(A)=V^*$
--- область значений оператора $A$. Оператор $A$ симметричен:
\begin{gather}\label{energetic_extension!enextAsimm}
\langle Au,v\rangle=\langle u,Av\rangle\,\,\,\forall\,u,\,\,v\in V,
\end{gather}
положительно определён:
\begin{gather}\label{energetic_extension!enextAposdef}
\exists\,\mu>0\,\,\forall\,u\in V:\langle Au,u\rangle\geqslant\mu\|u\|^2_H,
\end{gather}
и осуществляет взаимно однозначное отображение пространства $V$ на пространство $V^*$; обратный оператор $A^{-1}$ принадлежит $\mathcal{L}(V^*,V)$, $\|A\|=\|A^{-1}\|=1$. Скалярное
произведение и норма в $V$ равны соответственно $\langle u,v\rangle_V=\langle Au,v\rangle$ и $\|u\|_V=\langle Au,u\rangle^{1/2}$, а скалярное произведение и норма в $V^*$ определяются как
\begin{gather*}
\langle f,g\rangle_{V^*}=\langle A{}^{-1}f,{A}{}^{-1}g\rangle_V=\langle f,{A}{}^{-1}g\rangle\,\,\,\forall\,f,\,\,g\in V.
\end{gather*}
Имеют место вложения
\begin{gather} \label{energetic_extension!embeddings.chain2}
V\subset H\simeq H^*\subset V^*,
\end{gather}
причём эти вложения являются плотными и непрерывными, т.е.
\begin{gather*}
\|f\|_H\leqslant C\|f\|_V\,\,\,\forall\,f\in V
\end{gather*}
и
\begin{gather*}
\|f\|_{V^*}\leqslant C\|f\|_H\,\,\,\forall\,f\in H,
\end{gather*}
где $C=\mu^{-1/2}$. Напоминаем также, что в \eqref{energetic_extension!enextAsimm}, \eqref{energetic_extension!enextAposdef}, и далее запись $\langle f,v\rangle$ означает результат
применения функционала $f\in V^*$ к элементу $v\in V$; то же означает записть $\langle v,f\rangle$. Если о функционале $f\in V^*$ дополнительно известно, что $f\in H$ или $f\in V$,
то, как следует из определения вложения \eqref{energetic_extension!embeddings.chain2}, $\langle f,v\rangle=\langle f,v\rangle_H$ для любых $f\in H$ и, тем более, для любых $f\in V$.

От оператора $A$ дополнительно будем требовать, чтобы порождаемое им энергетическое пространство $V$ вкладывалось в пространство $H$ компактно.

\begin{Definition}
Собственным элементом оператора $A$, соответствующим собственному числу $\lambda$, называется элемент $e\in V$, $e\neq0$, такой, что
\begin{gather}\label{energetic_extension!eigenvalue}
Ae=\lambda e.
\end{gather}
Поскольку $Ae\in V^*$ и (в силу \eqref{energetic_extension!embeddings.chain2}) $V\subset V^*$, то равенство \eqref{energetic_extension!eigenvalue} понимается как равенство двух
линейных функционалов над $V$, т.е. как
\begin{gather*}
\langle Ae,\varphi\rangle=\lambda\langle e,\varphi\rangle\,\,\,\forall\,\varphi\in V.
\end{gather*}
\end{Definition}

\begin{Theorem} Пусть $A\in\mathcal{L}(V,V^*)$ --- энергетическое расширение линейного, неограниченного, симметричного, положительно определённого оператора с областью определения, плотной
в $H$, и пусть вложение $V\subset H$ компактно. Тогда оператор $A$ обладает счётной системой собственных чисел $\lambda_k$,
\begin{gather*}
0<\lambda_1\leqslant\lambda_2\leqslant\dots\leqslant\lambda_k\leqslant\dots,\,\,\,\lim\limits_{k\to\infty}\lambda_k=+\infty,
\end{gather*}
а соответствующая система $e_k\in V$, $k=1,2,\dots$, собственных элементов образует ортонормированный базис в $H$. При этом система $\frac{e_k}{\sqrt{\lambda_k}}$, $k=1,2,\dots$, является
ортонормированным базисом в $V$, а система $e_k\sqrt{\lambda_k}$, $k=1,2,\dots$, --- ортонормированным базисом в $V^*$. Для элементов и их норм в пространствах $H$, $V$, $V^*$ имеют место
представления
\begin{gather}
\label{energetic_extension!ekprop.v}
v=\sum\limits_{j=1}^\infty v_je_j,\,\,\,v_j=\langle v,\,e_j\rangle_H,\,\,\,j=1,2,\dots,\,\,\,\|v\|^2_V=\sum\limits_{j=1}^\infty\omega_j^2|v_j|^2,\,\,\, \forall\,v\in V;\\
\label{energetic_extension!ekprop.h}
h=\sum\limits_{j=1}^\infty h_je_j,\,\,\,h_j=\langle h,\,e_j\rangle_H,\,\,\,j=1,2,\dots,\,\,\,\|h\|^2_H=\sum\limits_{j=1}^\infty|h_j|^2, \,\,\, \forall\,h\in H;\\
\label{energetic_extension!ekprop.v*}
v^*=\sum\limits_{j=1}^\infty v^*_je_j,\,\,\,v^*_j=\langle v^*,\,e_j\rangle,\,\,\,j=1,2,\dots,\,\,\,\|v^*\|^2_{V^*}=\sum\limits_{j=1}^\infty\frac{|v^*_j|^2}{\omega_j^2}, \,\,\,
\forall\,v^*\in V^*;
\end{gather}
где
$$
\omega_j\equiv\sqrt{\lambda_j},\,\,\,j=1,2,\dots
$$
\end{Theorem}
\begin{Proof}
Рассмотрим вспомогательную задачу минимизации
\begin{gather}\label{energetic_extension!aux.problem}
J(u)\equiv\langle Au,u\rangle=\|u\|_V^2\to\inf,\,\,\,u\in U_1\equiv\{u\in V:\|u\|_H=1\}.
\end{gather}
Из \eqref{energetic_extension!enextAposdef} следует, что
\begin{gather*}
\lambda_1=\inf\limits_{u\in U_1}J(u)\geqslant\mu>0.
\end{gather*}

Пусть $u_n$, $n=1,2,\dots$, --- произвольная минимизирующая последовательность задачи \eqref{energetic_extension!aux.problem}, т.е.
\begin{gather*}
u_n\in V,\,\,\,\|u_n\|_H=1,\,\,\,n=1,2,\dots;\,\,\,\lim\limits_{n\to\infty}J(u_n)=\lim\limits_{n\to\infty}\|u_n\|^2_V=\lambda_1.
\end{gather*}
Поскольку числовая последовательность $\|u_n\|^2_V$, $n=1,2,\dots$, сходится, то она ограничена. Это означает, что последовательность $u_n$, $n=1,2,\dots$, ограничена в норме пространства
$V$. В силу компактности вложения $V\subset H$ из последовательности $u_n$, $n=1,2,\dots$, можно выделить сходящуюся в норме пространства $H$ к некоторому элементу $e_1\in H$
подпоследовательность. Без ограничения общности можем считать, что сама последовательность $u_n$, $n=1,2,\dots$, сходится к $e_1$ сильно в $H$. Из того, что $\|u_n\|_H=1$, $n=1,2,\dots$,
следует, что $\|e_1\|_H=1$.

Покажем, что $e_1\in V$ и $\|u_n-e_1\|_V\to0$ при $n\to\infty$. Для этого сначала установим, что последовательность $u_n$, $n=1,2,\dots$, фундаментальна в $V$. С этой целью возьмём
произвольный элемент $v\in V$ и положим
\begin{gather*}
w_n=(u_n+tv)\|u_n+tv\|_H^{-1},\,\,\,n=1,2,\dots,\,\,\,t\in\mathbb{R}.
\end{gather*}
Так как $w_n\in U_1$, то $J(w_n)\geqslant\lambda_1$, что равносильно неравенству
\begin{gather*}
\|u_n+tv\|^2_V\geqslant\lambda_1\|u_n+tv\|_H^2\,\,\,\forall\,t\in\mathbb{R},
\end{gather*}
или, что то же, неравенству
\begin{gather}\label{energetic_extension!ineq}
t^2[\|v\|^2_V-\lambda_1\|v\|^2H]+2t[\langle u_n,v\rangle_V-\lambda_1\langle u_n,v\rangle_H]+J(u_n)-\lambda_1\geqslant0\,\,\,\forall\,t\in\mathbb{R}\,\,\,\forall\,v\in V.
\end{gather}
Поделив данное неравенство на $t^2$ и перейдя затем к пределу при $t\to\infty$, получим, что
\begin{gather}\label{energetic_extension!vVH.ineq}
\|v\|^2_V-\lambda_1\|v\|^2H\geqslant0\,\,\,\forall\,v\in V.
\end{gather}
Если коэффициент при $t^2$ в \eqref{energetic_extension!ineq} отличен от нуля, то для выполнения неравенства \eqref{energetic_extension!ineq} при всех $t\in\mathbb{R}$ необходимо, чтобы
\begin{gather}\label{energetic_extension!ineq2}
(\langle u_n,v\rangle_V-\lambda_1\langle u_n,v\rangle_H)^2-(\|v\|^2_V-\lambda_1\|v\|^2H)(J(u_n)-\lambda_1)\leqslant0\,\,\,\forall\,v\in V.
\end{gather}
Если коэффициент при $t^2$ в \eqref{energetic_extension!ineq} равен нулю, то \eqref{energetic_extension!ineq} имеет место лишь тогда, когда коэффициент при $t$ также равен нулю, что снова
приводит к \eqref{energetic_extension!ineq2}. Таким образом, неравенство \eqref{energetic_extension!ineq2} верно во всех случаях, и выполняется при всех $v\in V$.

Возьмём в \eqref{energetic_extension!vVH.ineq}, \eqref{energetic_extension!ineq2} $v=u_n-u_m\in V$. Учитывая, что $J(u_n)\geqslant\lambda_1>0$, получим
\begin{gather*}
|\langle u_n,u_n-u_m\rangle_V-\lambda_1\langle u_n,u_n-u_m\rangle_H|\leqslant(\|u_n-u_m\|^2_V-\lambda_1\|u_n-u_m\|^2H)^{1/2}(J(u_n)-\lambda_1)^{1/2}\leqslant\\
\leqslant\|u_n-u_m\|_V(J(u_n)-\lambda_1)^{1/2}\leqslant C_0(J(u_n)-\lambda_1)^{1/2},\,\,\,m,\,n=1,2,\dots,
\end{gather*}
где $C_0$ --- некоторая константа. Поменяв ролями $n$ и $m$, выводим, что
\begin{gather*}
|\langle u_m,u_m-u_n\rangle_V-\lambda_1\langle u_m,u_m-u_n\rangle_H|\leqslant C_0(J(u_m)-\lambda_1)^{1/2}.
\end{gather*}
Поэтому
\begin{gather*}
0\leqslant\|u_n-u_m\|^2_V-\lambda_1\|u_n-u_m\|^2H=(\langle u_n,u_n-u_m\rangle_V-\lambda_1\langle u_n,u_n-u_m\rangle_H)-\\
-(\langle u_m,u_m-u_n\rangle_V-\lambda_1\langle u_m,u_m-u_n\rangle_H)\leqslant C_0[(J(u_n)-\lambda_1)^{1/2}+(J(u_m)-\lambda_1)^{1/2}],
\end{gather*}
или, иначе,
\begin{gather*}
\|u_n-u_m\|^2_V\leqslant\lambda_1\|u_n-u_m\|^2H+C_0[(J(u_n)-\lambda_1)^{1/2}+(J(u_m)-\lambda_1)^{1/2}],\,\,\,m,\,n=1,2,\dots
\end{gather*}
Правая часть этого неравенства при $n$, $m\to\infty$ стремится к нулю, ибо последовательность $u_n$, $n=1,2,\dots$, фундаментальна в $H$ и минимизирует функционал $J$ на $U_1$. Как
следствие, последовательность $u_n$, $n=1,2,\dots$, --- фундаментальна в $V$, и потому сильно в $V$ сходится к некоторому элементу $\tilde{e}_1\in V$. В силу
\eqref{energetic_extension!enextAposdef}, из сходимости в $V$ следует сходимость в $H$, что означает совпадение элементов $e_1$ $\tilde{e}_1$. Таким образом, $u_n\to e_1$, $n\to\infty$,
в норме пространства $V$, в силу чего
\begin{gather*}
J(u_n)=\|u_n\|^2_V\to\|e_1\|_V^2=J(e_1)=\lambda_1,\,\,\,n\to\infty;\,\,\,e_1\in U_1,\,\,\,J(e_1)=\lambda_1=\|e_1\|_V^2;
\end{gather*}
т.е. $e_1$ --- решение задачи \eqref{energetic_extension!aux.problem}. Покажем, что $e_1$ --- собственный вектор оператора $A$, соответствующий собственному числу $\lambda_1$. С этой целью
совершим в \eqref{energetic_extension!ineq} предельный переход при $n\to\infty$. С учётом равенства $\|e_1\|^2_V=\lambda_1$ получим неравенство
\begin{gather*}
t^2[\|v\|^2_V-\lambda_1\|v\|^2H]+2t[\langle e_1,v\rangle_V-\lambda_1\langle e_1,v\rangle_H]\geqslant0\,\,\,\forall\,t\in\mathbb{R}\,\,\,\forall\,v\in V,
\end{gather*}
выполнение которого при всех $t\in\mathbb{R}$ возможно только тогда, когда коэффициент при $t$ равен нулю:
\begin{gather}\label{energetic_extension!eigen.e1}
\lambda_1\langle e_1,v\rangle_H=\langle e_1,v\rangle_V=\langle Ae_1,v\rangle\,\,\,\forall\,v\in V.
\end{gather}
Это означает, что $\lambda_1$ --- собственное число оператора $A$, а $e_1$ --- соответствующий этому числу собственный элемент.

Для доказательства существования следующего собственного числа $\lambda_2\geqslant\lambda_1\geqslant\mu>0$ и соответствующего ему собственного элемента $e_2$ в $V$, возьмём подпространство
\begin{gather*}
V^1=\{v\in V:\langle e_1,v\rangle_V=0\}
\end{gather*}
и рассмотрим в $V^1$ вспомогательную задачу минимизации, аналогичную задаче \eqref{energetic_extension!aux.problem}:
\begin{gather}\label{energetic_extension!aux.problem.V1}
J(u)\equiv\langle Au,u\rangle=\|u\|_V^2\to\inf,\,\,\,u\in U_2\equiv\{u\in V^1:\|u\|_H=1\}.
\end{gather}

Полезно заметить, что $V^1=V\cap H^1$, где $H^1=\{v\in H:\langle e_1,v\rangle_H=0\}$. В самом деле, если $v\in V$, то как видно из \eqref{energetic_extension!eigen.e1}, равенство
$\langle e_1,v\rangle_V=0$ имеет место тогда и только тогда, когда $\langle e_1,v\rangle_H=0$.

Так как $U_2\subset U_1$, то $\lambda_2=\inf\limits_{u\in U_2}J(u)\geqslant\lambda_1\geqslant\mu>0$. Рассуждая затем так же, как при исследовании задачи
\eqref{energetic_extension!aux.problem}, с заменой $V$ на $V^1$, $\lambda_1$ на $\lambda_2$, устанавливем, что существует элемент $e_2\in U_2$, такой, что $J(e_2)=\lambda_2$; и что
$\lambda_2$ --- собственное число оператора $A$, а $e_2$ --- соответствующий собственный элемент, причём
\begin{gather*}
\langle e_1,e_2\rangle_V=0,\,\,\,\langle e_1,e_2\rangle_H=0,\,\,\,\|e_2\|_H=1,\,\,\,\|e_2\|_V^2=\lambda_2.
\end{gather*}

Далее сделаем индуктивное предположение: пусть уже построены собственные элементы $e_1$,\dots,$e_k$, соответствующие собственным числам
\begin{gather*}
\mu\leqslant\lambda_1\leqslant\lambda_2\leqslant\dots\leqslant\lambda_{k-1}\leqslant\lambda_k,
\end{gather*}
такие, что
\begin{gather*}
e_i\in V,\,\,\,\langle e_i,e_j\rangle_H=0,\,\,\,\langle e_i,e_j\rangle_V=0,\,\,\,i\neq j,\,\,\,\|e_i\|_H=1,\,\,\,\|e_i\|_V^2=\lambda_i,\,\,\,i,\,j=\overline{1,k}.
\end{gather*}
Тогда вводим в $V$ подпространство
\begin{gather*}
V^k=\{v\in V:\langle e_1,v\rangle_V=0,\,\,\,\langle e_k,v\rangle_V=0\}
\end{gather*}
и рассматриваем задачу минимизации
\begin{gather}\label{energetic_extension!aux.problem.Vk}
J(u)\equiv\langle Au,u\rangle=\|u\|_V^2\to\inf,\,\,\,u\in U_{k+1}\equiv\{u\in V^k:\|u\|_H=1\}.
\end{gather}
Аналогично \eqref{energetic_extension!aux.problem}, \eqref{energetic_extension!aux.problem.V1} доказываем существование элемента
\begin{gather*}
e_{k+1}\in U_{k+1},\,\,\,J(e_{k+1})=\lambda_{k+1}=\inf\limits_{u\in U_{k+1}}J(u)\geqslant\lambda_k,
\end{gather*}
где $\lambda_k$ --- собственное число оператора $A$, а $e_{k+1}$ --- соответствующий собственный элемент, причём
\begin{gather*}
\langle e_i,e_{k+1}\rangle_V=0,\,\,\,\langle e_i,e_{k+1}\rangle_H=0,\,\,\,\|e_{k+1}\|_H=1,\,\,\,\|e_{k+1}\|_V^2=\lambda_{k+1}\,\,\,\forall\,i=\overline{1,k}.
\end{gather*}
В бесконечномерном пространстве $V$ этот процесс построения собственных чисел может быть продолжен неограниченно, и в результате мы получим последовтельность собственных чисел оператора
$A$,
\begin{gather*}
0<\mu\leqslant\lambda_1\leqslant\lambda_2\leqslant\dots\lambda_k\leqslant\dots,
\end{gather*}
и последовательность соответствующих им собственных элементов $e_1,\dots,e_k,\dots$, причём
\begin{gather*}
e_k\in V,\,\,\,\|e_k\|_H=1,\,\,\,\|e_k\|^2_V=\lambda_k,\,\,\,\langle e_k,e_j\rangle_V=0,\,\,\,\langle e_k,e_j\rangle_H=0,\,\,\,k\neq j,\,\,\,k,\,j=1,2,\dots,
\end{gather*}
т.е. система $e_k$, $k=1,2,\dots$, является ортонормированной системой в $H$, а система $\frac{e_k}{\sqrt{\lambda_k}}$, $k=1,2,\dots$, --- ортонормированной системой в $V$.

Убедимся в том, что система $e_k\sqrt{\lambda_k}$, $k=1,2,\dots$, является ортонормированной системой в $V^*$. В самом деле, поскольку
\begin{gather*}
\langle f,g\rangle_{V^*}=\langle A^{-1}f,A^{-1}g\rangle_V\,\,\,\forall\,f,\,g\in V^*,\,\,\,A^{-1}e_k=\frac1{\lambda_k}e_k,
\end{gather*}
то при $i\neq k$
\begin{gather*}
\langle e_i,e_k\rangle_{V^*}=\langle A^{-1}e_i,A^{-1}e_k\rangle_V=\frac1{\lambda_i\lambda_k}\langle e_i,e_k\rangle_V=0,
\end{gather*}
а при $i=k$ отсюда же имеем
\begin{gather*}
\|e_k\|^2_{V^*}=\frac1{\lambda_k^2}\|e_k\|^2_V=\frac{\lambda_k}{\lambda_k^2}=\frac1{\lambda_k},\,\,\,k=1,2,\dots
\end{gather*}

Докажем, что на самом деле система $e_k$, $k=1,2,\dots$, образует ортонормированный базис в пространстве $H$, система $\frac{e_k}{\sqrt{\lambda_k}}$, $k=1,2,\dots$, --- ортонормированный
базис в пространстве $V$, а система ${e_k}{\sqrt{\lambda_k}}$, $k=1,2,\dots$, --- ортонормированный базис в $V^*$. Для этого сначала покажем, что $\lambda_k\to\infty$, $k\to\infty$.
В самом деле, если бы неубывающая последовательность $\lambda_k$, $k=1,2,\dots$, имела бы конечный предел $\lambda$, то из равенств $\|e_k\|^2_V=\lambda_k$, $k=1,2,\dots$, следовало бы, что
$\|e_k\|^2_V\to\lambda<\infty$, $k\to\infty$, т.е. последовательность $e_k$, $k=1,2,\dots$, была бы ограниченной в норме пространства $V$. По условию теоремы, вложение $V\subset H$ ---
компактно, и поэтому из последовательности $e_k$, $k=1,2,\dots$, можно было бы выделить подпоследовательность $e_{k_m}$, $m=1,2,\dots$, сильно сходящуюся в норме пространства $H$. Однако
последовательность $e_k$, $k=1,2,\dots$, --- ортонормированная система в $H$, и потому $\|e_{k_m}-e_{k_n}\|^2_H=\|e_{k_m}\|^2_H-2\langle e_{k_m},e_{k_n}\rangle_H+\|e_{k_n}\|^2_H=2$ для
всех $m$, $n=1,2,\dots$, т.е. последовательность $e_{k_m}$, $m=1,2,\dots$, фундаментальной в норме пространства $H$ быть не может. Полученное противоречие означает, что на самом деле
$\lambda_k\to\infty$, $k\to\infty$. Далее докажем, что система $e_k$, $k=1,2,\dots$, полна в $V$, т.е. если для некоторого $v\in V$ выполнено равенство $\langle e_k,v\rangle_V=0$, $k=1,2,
\dots$, то $v=0$. Предположим противное: пусть существует элемент $e\in V$, $e\neq0$, такой, что  $\langle e_k,e\rangle_V=0$, $k=1,2,\dots$ Тогда
\begin{gather*}
e\in V^\infty=\{v\in V:\langle e_k,v\rangle_V=0,\,\,\,k=1,2,\dots\}.
\end{gather*}
где $V^\infty$ --- подпространство в $V$. По аналогии с \eqref{energetic_extension!aux.problem}, \eqref{energetic_extension!aux.problem.V1}, \eqref{energetic_extension!aux.problem.Vk}
рассмотрим задачу минимизации
\begin{gather*}
J(u)=\|u\|^2_V\to\inf,\,\,\,u\in U_\infty=\{u'\in V^\infty:\|u'\|_H=1\}.
\end{gather*}
Рассуждая так же, как и выше, показываем, что существует собственный элемент $e_\infty\in U_\infty$ оператора $A$, отвечающий собственному числу
\begin{gather*}
\lambda_\infty=J(u_\infty)=\|u_\infty\|^2_V=\inf\limits_{u\in U_\infty}J(u).
\end{gather*}
Так как $U_\infty\subset U_k$ при всех $k=1,2,\dots$, то $\lambda_k\leqslant\lambda_\infty<\infty$, $k=1,2,\dots$, а это противоречит уже установленному соотношению
\begin{gather*}
\lim\limits_{k\to\infty}\lambda_k=+\infty.
\end{gather*}
Полнота системы $e_k$, $k=1,2,\dots$, в пространстве $V$ доказана. Это означает, что линейная оболочка элементов $e_k$, $k=1,2,\dots$, плотна в $V$, и, в силу плотности вложений
$V\subset H\simeq H^*\subset V^*$, эта линейная оболочка будет плотна также и в пространствах $H$ и $V^*$. Отсюда следует, что ортонормированная система $\frac{e_k}{\sqrt{\lambda_k}}$,
$k=1,2,\dots$, --- полна в $V$, ортонормированная система $e_k$, $k=1,2,\dots$, --- полна в $H$, а ортонормированная система ${e_k}{\sqrt{\lambda_k}}$, $k=1,2,\dots$, --- полна в $V^*$.
Поэтому система $\frac{e_k}{\sqrt{\lambda_k}}$, $k=1,2,\dots$, --- ортонормированный базис в пространстве $V$, система $e_k$, $k=1,2,\dots$, --- ортонормированный базис в пространстве $H$,
система ${e_k}{\sqrt{\lambda_k}}$, $k=1,2,\dots$, --- ортонормированный базис в пространстве $V^*$, а любой элемент из пространств $V$, $H$, $V^*$ разлагается в сильно сходящийся ряд Фурье
по соответствующей системе. Так, если $u\in H$, то
\begin{gather*}
u=\sum\limits_{k=1}^\infty u_ke_k,\,\,\,u_k=\langle u,e_k\rangle_H,
\end{gather*}
а равенство
\begin{gather*}
\|u\|^2_H=\sum\limits_{k=1}^\infty u^2_k
\end{gather*}
представляет собой обычное равенство Парсеваля--Стеклова. Если $u\in V$, то
\begin{gather*}
u=\sum\limits_{k=1}^\infty v_k\left[\frac{e_k}{\sqrt{\lambda_k}}\right],\,\,\,
v_k=\left\langle u,\frac{e_k}{\sqrt{\lambda_k}}\right\rangle_V=\frac1{\sqrt{\lambda_k}}\langle Ae_k,u\rangle=\sqrt{\lambda_k}\langle u,e_k\rangle_H=\sqrt{\lambda_k}u_k,
\end{gather*}
а равенство Парсеваля--Стеклова в пространстве $V$ записывается в виде
\begin{gather*}
\|u\|^2_V=\sum\limits_{k=1}^\infty v^2_k=\sum\limits_{k=1}^\infty\lambda_ku_k^2.
\end{gather*}
Наконец, если $u\in V^*$, то
\begin{gather*}
u=\sum\limits_{k=1}^\infty w_k[e_k\sqrt{\lambda_k}],\,\,\,
w_k=\langle u,e_k\sqrt{\lambda_k}\rangle_{V^*}=\langle u,A^{-1}(e_k\sqrt{\lambda_k})\rangle=\left\langle u,\frac1{\sqrt{\lambda_k}}e_k\right\rangle=\frac1{\sqrt{\lambda_k}}
\langle u,e_k\rangle=\frac1{\sqrt{\lambda_k}}u_k,
\end{gather*}
а равенство Парсеваля--Стеклова принимает вид
\begin{gather*}
\|u\|^2_V=\sum\limits_{k=1}^\infty w^2_k=\sum\limits_{k=1}^\infty\frac1{\lambda_k}u^2_k.
\end{gather*}
Теорема полностью доказана.
\end{Proof}

    \section{Абстрактная задача Коши с автономной главной частью}
Пусть $H$ --- сепарабельное гильбертово пространство со скалярным произведением $\langle\cdot,\cdot\rangle_H$ и соответствующей нормой $\|\cdot\|_H$. Пусть
$\mathfrak{A}\in\mathcal{L}(V,V^*)$ --- энергетическое расширение некоторого линейного неограниченного симметричного положительно опеределённого оператора с плотной в $H$ областью
определения, $V$ --- энергетическое пространство. Оператор $\mathfrak{A}$ симметричен
\begin{gather*}
\langle \mathfrak{A}f,g\rangle=\langle f,\mathfrak{A}g\rangle_H\,\,\,\forall\,f,\,\,g\in V;
\end{gather*}
положительно определён
\begin{gather*}
\langle \mathfrak{A}f,f\rangle\geqslant \mu\|f\|^2_H\,\,\,\forall\,f\in V,
\end{gather*}
где $\mu>0$ --- некоторая постоянная, не зависящая от выбора $f\in V$; и осуществляет взаимно однозначное отображение $V$ на $V^*$; обратный оператор $\mathfrak{A}^{-1}\in
\mathcal{L}(V^*,V)$, $\|\mathfrak{A}\|=\|\mathfrak{A}^{-1}\|=1$. Скалярное произведение и норма в $V$ равны соответственно $\langle u,v\rangle_V=\langle \mathfrak{A}u,v\rangle$ и
$\|u\|_V=\sqrt{\langle \mathfrak{A}u,u\rangle}$, а скалярное произведение и норма в $V^*$ определяются как $\langle f,g\rangle_{V^*}=\langle \mathfrak{A}^{-1}f,\mathfrak{A}^{-1}g\rangle_V=
\langle f,\mathfrak{A}^{-1}g\rangle$ и $\|f\|_{V^*}= \|\mathfrak{A}^{-1}f\|_V$. Согласно разделу
\ref{energetic_extension}, имеют место вложения
\begin{gather*}
V\subset H\simeq H^*\subset V^*,
\end{gather*}
причём эти вложения плотны и непрерывны, т.е.
\begin{gather*}
\|f\|_H\leqslant C\|f\|_V\,\,\,\forall\,f\in V;\,\,\,\|f\|_{V^*}\leqslant C\|f\|_H\,\,\,\forall\,f\in H;
\end{gather*}
где $C=\mu^{-1/2}$. Пусть, кроме того, вложение $V\subset H$ --- компактно.

Пусть отображение $\beta\colon\textbf{Э}_1([0, T]; V,H)\to L_1([0,T],H)$ таково, что для некоторой функции $K_0\in L_1[0,T]$
\begin{gather}\label{betalip!more.more}
\|\beta[\mathfrak{z}_1](t)-\beta[\mathfrak{z}_2](t)\|_H\leqslant K_0(t)\sqrt{\|\mathfrak{z}_1(t)-\mathfrak{z}_2(t)\|^2_V+\|\dot{\mathfrak{z}}_1(t)-\dot{\mathfrak{z}}_2(t)\|_H^2}\\
\notag\forall\,(t,\mathfrak{z}_i)\in[0,T]\times\textbf{Э}_1([0, T]; V,H),\,\,\,i=1,2.
\end{gather}

Наконец, пусть $\varphi\in V$, $\psi\in H$, $P\in L_1(\Gamma,\mathcal{L}(H,H))$.

Рассмотрим задачу Коши
\begin{gather}\label{abstrChauchyprobeq:more.more.abstract}
\ddot{\mathfrak{z}}(t)+\mathfrak{A}{\mathfrak{z}}(t)=\beta[{\mathfrak{z}}](t)+\int\limits_0^tP(t,\tau)\dot{\mathfrak{z}}(\tau)d\tau,\,\,\,t\in[0,T],\\
\label{abstrChauchyprobinitcond:more.more.abstract}
{\mathfrak{z}}(0)=\varphi,\,\,\,\dot{\mathfrak{z}}(0)=\psi,
\end{gather}
и дадим следующее
\begin{Definition}\label{SolutionDef1:more.more.abstract} Функцию ${\mathfrak{z}}\in{\textbf{Э}}_1([0,T];V,H)$ назовём решением задачи Коши (\ref{abstrChauchyprobeq:more.more.abstract}),
(\ref{abstrChauchyprobinitcond:more.more.abstract}), если
\begin{gather}\label{defsolabstrChauchyprob1:more.more.abstract}
\int\limits_0^T[-\langle\dot{\mathfrak{z}}(t),\dot{\eta}(t)\rangle_H+\langle \mathfrak{A}{\mathfrak{z}}(t),\eta(t)\rangle]dt=\langle\psi,\eta(0)\rangle+
\int\limits_0^T\langle\theta[t,\mathfrak{z}],\eta(t)\rangle_Hdt\,\,\,\forall\,{\eta}\in\hat{\textbf{Э}}{}_1([0,T];V,H);\\
\notag \mathfrak{z}(0)=\varphi.
\end{gather}
\end{Definition}
Под $\hat{\textbf{Э}}{}_1([0,T];V,H)$ мы в данном определении понимаем множество $\{\mathfrak{z}\in{\textbf{Э}}_1([0,T];V,H):\mathfrak{z}(T)=0\}$, а под $\theta[t,\mathfrak{z}]$ ---
выражение
$$
\beta[\mathfrak{z}](t)+\int\limits_0^tP(t,\tau)\dot{\mathfrak{z}}(\tau)d\tau,\,\,\,t\in[0,T].
$$

Начиная с этого момента здесь и всюду ниже через $e_j$, $j=1,2,\dots$, мы обозначаем последовательность элементов $V$, таких, что
\begin{gather}\label{ekdef1:more.more.abstract}
\langle e_i,\,e_j\rangle_H=\delta^i_j,\,\,\,\langle e_i,\,e_j\rangle_V=\delta^i_j\lambda_j,\,\,\,Ae_j=\lambda_je_j,\,\,\,i,\,\,j=1,2,\dots,\\
\notag 0<\lambda_1\leqslant\lambda_2\leqslant\dots\leqslant\lambda_j\leqslant\lambda_{j+1}\leqslant\dots,\,\,\, \lim_{j\to\infty}\lambda_j=+\infty.
\end{gather}
Согласно разделу \ref{energetic_extension},
\begin{gather}
\label{ekprop1:more.more.abstract}
v=\sum\limits_{j=1}^\infty v_je_j,\,\,\,v_j=\langle v,\,e_j\rangle_H,\,\,\,j=1,2,\dots,\,\,\,\|v\|^2_V=\sum\limits_{j=1}^\infty\omega_j^2|v_j|^2,\,\,\, \forall\,v\in V;\\
\label{ekprop1.h:more.more.abstract}
h=\sum\limits_{j=1}^\infty h_je_j,\,\,\,h_j=\langle h,\,e_j\rangle_H,\,\,\,j=1,2,\dots,\,\,\,\|h\|^2_H=\sum\limits_{j=1}^\infty|h_j|^2, \,\,\, \forall\,h\in H;\\
\label{ekprop1.v*:more.more.abstract}
v^*=\sum\limits_{j=1}^\infty v^*_je_j,\,\,\,v^*_j=\langle v^*,\,e_j\rangle,\,\,\,j=1,2,\dots,\,\,\,\|v^*\|^2_{V^*}=\sum\limits_{j=1}^\infty\frac{|v^*_j|^2}{\omega_j^2},\,\,\,
\forall\,v^*\in V^*;
\end{gather}
где
$$
\omega_j\equiv\sqrt{\lambda_j},\,\,\,j=1,2,\dots
$$

Дадим ещё одно определение решения задачи Коши  (\ref{abstrChauchyprobeq:more.more.abstract}), (\ref{abstrChauchyprobinitcond:more.more.abstract}).
\begin{Definition}\label{SolutionDef2:more.more.abstract} Функцию ${\mathfrak{z}}\in{\textbf{Э}}_2([0,T];V,H)$ назовём решением задачи Коши (\ref{abstrChauchyprobeq:more.more.abstract}),
(\ref{abstrChauchyprobinitcond:more.more.abstract}), если
\begin{gather}\label{defsolabstrChauchyprob2:more.more.abstract}
\langle\ddot{\mathfrak{z}}(t),v\rangle+\langle\mathfrak{A}{\mathfrak{z}}(t), v\rangle=\langle\theta[t,{\mathfrak{z}}], v\rangle\,\,\,\mbox{ при п.в. }t\in[0,T]\,\,\,\text{$\forall v\in V$,}\\
\notag {\mathfrak{z}}(0)=\varphi,\,\,\,\dot{\mathfrak{z}}(0)=\psi.
\end{gather}
\end{Definition}

Пусть ${\mathfrak{M}}^N\equiv\{\sum\limits_{j=1}^N\zeta_je_j\,:\,\zeta_j\in W^1_2[0,T],\,\,\,\zeta_j(T)=0,\,\,\,j=\overline{1,\,N}\}$,
${\mathfrak{M}}\equiv \bigcup\limits_{N=1}^\infty{\mathfrak{M}}^N$.

Докажем, что справедлива следующая
\begin{Lemma}\label{SolutionDefsEquivalence:more.more.abstract}
Определения \ref{SolutionDef1:more.more.abstract} и \ref{SolutionDef2:more.more.abstract} --- эквивалентны.
\end{Lemma}
\begin{Proof}
1) Докажем, что если функция  ${\mathfrak{z}}\in{\textbf{Э}}_2([0,T];V,H)$ является решением в смысле определения \ref{SolutionDef2:more.more.abstract}, то она является и решением в смысле
определения \ref{SolutionDef1:more.more.abstract}.

В самом деле, пусть  ${\mathfrak{z}}\in{\textbf{Э}}_2([0,T];V,H)$ --- решение в смысле определения \ref{SolutionDef2:more.more.abstract}.

Поскольку, согласно лемме \ref{approx:abstract}, ${\mathfrak{M}}\equiv \bigcup\limits_{N=1}^\infty{\mathfrak{M}}^N$ плотно в $\hat{\textbf{Э}}{}_1([0,T];V,H)$, то нам достаточно
доказать, что тождество \eqref{defsolabstrChauchyprob1:more.more.abstract} справедливо для функций $\eta$, имеющих вид $\eta(t)\equiv\zeta(t)e_j$, $t\in[0,T]$, где $\zeta\in W^1_2[0,T]$,
$\zeta(T)=0$.

Действительно, применив взяв в равенстве \eqref{defsolabstrChauchyprob2:more.more.abstract} $v=\zeta(t)e_j$, $t\in[0,T]$, и проинтегрировав результат по $t\in[0,T]$, будем иметь
\begin{gather*}
\int\limits_0^T\langle\ddot{\mathfrak{z}}(t),\zeta(t)e_j\rangle dt+\int\limits_0^T\langle\mathfrak{A}{\mathfrak{z}}(t), \zeta(t)e_j\rangle dt=
\int\limits_0^T\langle \theta[t,{\mathfrak{z}}], \zeta(t)e_j\rangle dt.
\end{gather*}
Взяв первый из стоящих слева интегралов по частям, получим справедливость тождества \eqref{defsolabstrChauchyprob1:more.more.abstract} для функций $\eta$, имеющих вид
$\eta(t)\equiv\zeta(t)e_j$, $t\in[0,T]$, где $\zeta\in W^1_2[0,T]$, $\zeta(T)=0$.

Таким образом, мы доказали, что еслия  ${\mathfrak{z}}\in{\textbf{Э}}_2([0,T];V,H)$ является решением в смысле определения \ref{SolutionDef2:more.more.abstract}, то она является и решением
в смысле определения \ref{SolutionDef1:more.more.abstract}.

2) Докажем теперь, что если функция  ${\mathfrak{z}}\in{\textbf{Э}}_1([0,T];V,H)$ является решением в смысле определения \ref{SolutionDef1:more.more.abstract}, то она является и решением в
смысле определения \ref{SolutionDef2:more.more.abstract}.

Подставляя в интегральное тождество~\eqref{defsolabstrChauchyprob1:more.more.abstract} $\eta(t)\equiv\zeta(t)v$, $t\in[0,T]$, где $\zeta\in W^1_2[0,T]$, $\zeta(T)=0$, $v\in V$, заключаем,
что
\begin{gather}\label{idetaej1:more.more.abstract}
\int\limits_0^T\langle-\dot{\mathfrak{z}}(t),\zeta'(t)v\rangle dt+\int\limits_0^T\langle\mathfrak{A}{\mathfrak{z}}(t), \zeta(t)v\rangle dt=
\int\limits_0^T\langle \theta[t,{\mathfrak{z}}], \zeta(t)v\rangle dt+\langle\psi,\zeta(0)v\rangle.
\end{gather}
В частности, для всех $\zeta\in\mathfrak{D}(0,T)$
\begin{gather}\label{idetaej2:more.more.abstract}
\int\limits_0^T\langle-\dot{\mathfrak{z}}(t),\zeta'(t)v\rangle dt+\int\limits_0^T\langle\mathfrak{A}{\mathfrak{z}}(t), \zeta(t)v\rangle dt=
\int\limits_0^T\langle \theta[t,{\mathfrak{z}}], \zeta(t)v\rangle dt+\langle\psi,\zeta(0)v\rangle.
\end{gather}
Положив затем $\tilde{\theta}(\zeta)\equiv\int\limits_{0}^{T}\theta[t,{\mathfrak{z}}](t)\zeta(t)\,dt$, ${\mathfrak{Z}}(\zeta)\equiv\int\limits_{0}^{T}{\mathfrak{z}}(t)\zeta(t)\,dt$, и
замечая, что ${\mathfrak{Z}}'(\zeta)\equiv-\int\limits_{0}^{T}{\mathfrak{z}}(t)\dot\zeta(t)\,dt$, из тождества \eqref{idetaej2:more.more.abstract} получаем, что
$$
-{\mathfrak{Z}}'(\zeta')+\mathfrak{A}{\mathfrak{Z}}(\zeta)=\tilde{\theta}(\zeta),\,\,\, \forall\,\zeta\in {\mathfrak{D}}(0,T),
$$
или, иначе,
$$
{\mathfrak{Z}}''(\zeta)+\mathfrak{A}{\mathfrak{Z}}(\zeta)=\tilde{\theta}(\zeta)\,\,\, \forall\,\zeta\in {\mathfrak{D}}(0,T).
$$
Последнее означает, что ${\mathfrak{Z}}''$ --- регулярна, и лежит в $L_1([0,T],\,V^*)$. Поэтому ${\mathfrak{z}}\in{\textbf{Э}}_2([0,T];V,H)$, и
\begin{equation}\label{sfzideq:more.more.abstract}
\langle\ddot{\mathfrak{z}}(t)+\mathfrak{A}{\mathfrak{z}}(t)-\theta[t,{\mathfrak{z}}],\,v\rangle=0,\,\,\,\forall\,v\in V.
\end{equation}
Взяв по частям первый из интегралов, стоящих в левой части тождества \ref{idetaej1:more.more.abstract}, выводим, что
\begin{gather*}
\int\limits_0^T\langle\ddot{\mathfrak{z}}(t)+\mathfrak{A}{\mathfrak{z}}(t)-\theta[t,{\mathfrak{z}}],\zeta(t)v\rangle dt+\langle\dot{\mathfrak{z}}(0),\zeta(0)v\rangle=
\langle\psi,\zeta(0)v\rangle.
\end{gather*}
Учтя здесь равенство \eqref{sfzideq:more.more.abstract}, получим, что
\begin{gather*}
\langle\dot{\mathfrak{z}}(0)-\psi,v\rangle=0\,\,\,\forall\,v\in V.
\end{gather*}
Иными словами,
$$
\dot{\mathfrak{z}}(0)=\psi.
$$
Из этого соотношения и соотношения \eqref{sfzideq:more.more.abstract} и вытекает, что $\mathfrak{z}$ является решением в смысле определения \ref{SolutionDef2:more.more.abstract}.

Лемма полностью доказана.
\end{Proof}



Покажем, что задача Коши (\ref{abstrChauchyprobeq:more.more.abstract}), (\ref{abstrChauchyprobinitcond:more.more.abstract}) эквивалентна некоторому интегро--дифференциальному уравнению в
пространстве ${\textbf{Э}}_1([0,T];V,H)$. Для этого нам потребуется ввести ряд обозначений и доказать ряд результатов.

Прежде всего для любого $h\in H$ положим
\begin{gather*}
\Pi_1(t)h=\sum\limits_{m=1}^\infty\cos(\omega_mt)h_me_m,\,\,\,\Pi_2(t,\xi)h=\sum\limits_{m=1}^\infty\frac{\sin(\omega_m(t-\xi))}{\omega_m}h_me_m,\,\,\,(t,\xi)\in\Gamma.
\end{gather*}

Справедлива следующая
\begin{Lemma}\label{Pi1Pi2_properties:more.more.abstract}
1) При всех $(t,\xi)\in\Gamma$ справедливы включения $\Pi_1(t)\in\mathcal{L}(H,H)$, $\Pi_1(t)\in\mathcal{L}(V,V)$, $\Pi_2(t,\xi)\in\mathcal{L}(H,V)$, причём при всех $(t,\xi)\in\Gamma$
имеют место неравенства
\begin{gather*}
\|\Pi_1(t)\|_{H\to H}\leqslant1,\,\,\,\|\Pi_1(t)\|_{V\to V}\leqslant1,\,\,\,\|\Pi_2(t,\xi)\|_{H\to V}\leqslant1.
\end{gather*}

2) При каждом $h\in H$ ряд для $\Pi_1(t)h$ сходится в норме $H$ равномерно по $t\in[0,T]$.

3) При каждом $v\in V$ ряд для $\Pi_1(t)v$ сходится в норме $V$ равномерно по $t\in[0,T]$.

4) При каждом $h\in H$ ряд для $\Pi_2(t,\xi)h$ сходится в норме $V$ равномерно по $(t,\xi)\in\Gamma$.

5) При каждом $h\in H$ функция $[0,T]\ni t\mapsto\Pi_1(t)h$ принадлежит $C([0,T],H)$.

6) При каждом $v\in V$ функция $[0,T]\ni t\mapsto\Pi_1(t)v$ --- элемент пространства $C([0,T],V)$.

7) При каждом $h\in H$ функция $\Gamma\ni(t,\xi)\mapsto\Pi_2(t,\xi)h$ принадлежит $C(\Gamma,V)$.

8) Для любой функции $y\in C([0,T],H)$ функция $\Gamma\ni (t,\xi)\mapsto \Pi_2(t,\xi)y(\xi)$ --- элемент $C(\Gamma,V)$.
\end{Lemma}
\begin{Proof}
1) Докажем первое утверждение леммы. В самом деле, пусть $h\in H$, $v\in V$, $(t,\xi)\in\Gamma$, --- произвольны. Тогда
\begin{gather*}
\|\Pi_1(t)h\|_H^2=\sum\limits_{m=1}^\infty\cos^2(\omega_mt)h_m^2\leqslant\sum\limits_{m=1}^\infty h_m^2=\|h\|^2_H,\,\,\,
\|\Pi_1(t)v\|_V^2=\sum\limits_{m=1}^\infty\cos^2(\omega_mt)v_m^2\omega_m^2\leqslant\sum\limits_{m=1}^\infty v_m^2\omega_m^2=\|v\|^2_V,\\
\|\Pi_2(t,\xi)h\|_V^2=\sum\limits_{m=1}^\infty\left|\frac{\sin(\omega_m(t-\xi))}{\omega_m}h_m\right|^2\omega_m^2=
\sum\limits_{m=1}^\infty\sin^2(\omega_m(t-\xi))h_m^2\leqslant\sum\limits_{m=1}^\infty h_m^2=\|h\|^2_H.
\end{gather*}
Таким образом, при всех $h\in H$, $v\in V$, $(t,\xi)\in\Gamma$ имеют место соотношения
\begin{gather*}
\|\Pi_1(t)v\|_V\leqslant\|h\|_H,\,\,\,\|\Pi_1(t)v\|_V\leqslant\|v\|_V,\,\,\,\|\Pi_2(t,\xi)h\|_V\leqslant\|h\|_H,
\end{gather*}
которые и доказывают первое утверждение леммы.

2) Докажем остальные утверждения леммы.  Пусть $h\in H$, $v\in V$ --- произвольны. Во--первых, заметим, что $[0,T]$ и $\Gamma$, рассматриваемые со стандартной топологией, --- компактные
топологические пространства. При этом функции
\begin{gather*}
[0,T]\ni t\mapsto\cos(\omega_mt)h_m\in\mathbb{R},\,\,\,[0,T]\ni t\mapsto\cos(\omega_mt)v_m\in\mathbb{R},\,\,\,m=1,2,\dots,
\end{gather*}
непрерывны на $[0,T]$, а  функции
\begin{gather*}
\Gamma\ni(t,\xi)\mapsto\frac{\sin(\omega_m(t-\xi))}{\omega_m}h_m\in\mathbb{R},\,\,\,m=1,2,\dots,
\end{gather*}
непрерывны на $\Gamma$.

Во--вторых, при всех  $(t,\xi)\in\Gamma$ имеют место оценки
\begin{gather*}
|\cos(\omega_mt)h_m|\leqslant|h_m|,\,\,\,|\cos(\omega_mt)v_m|\leqslant|v_m|,\,\,\, \left|\frac{\sin(\omega_m(t-\xi))}{\omega_m}h_m\right|\leqslant\frac1{\omega_m}|h_m|,\,\,\,m=1,2,\dots,
\end{gather*}
из которых следует, что
\begin{gather*}
|\cos(\omega_mt)h_m|^2\|e_m\|^2_H\leqslant|h_m|^2,\,\,\,|\cos(\omega_mt)v_m|^2\|e_m\|^2_V\leqslant|v_m|^2\omega_m^2,\\
\left|\frac{\sin(\omega_m(t-\xi))}{\omega_m}h_m\right|^2\|e_m\|^2_V\leqslant|h_m|^2,\,\,\,m=1,2,\dots
\end{gather*}
Пользуясь теперь теоремой \ref{priznak:uniform_convergence_of_functional_sequence}, получаем второе, третье и четвёртое утверждения леммы.

Пятое, шестое и седьмое утверждения являются следствиями второго, третьего и четвёртого утверждений и следствия \ref{consequence_of_uniform_convergence_of_functional_series}.

Восьмое утверждение вытекает из включения $y\in C([0,T],H)$, седьмого утверждения и леммы \ref{continuity_Pi(t,xi)z(xi)}. Лемма полностью доказана.
\end{Proof}

Из лемм \ref{parametric_integral_1} и \ref{Pi1Pi2_properties:more.more.abstract} вытекает
\begin{Lemma}\label{integral_Pi2ydxi:properties:more.more.abstract}
Для любой функции $y\in L_1([0,T],H)$ функция
\begin{gather*}
[0,T]\ni t\mapsto\int\limits_0^t\Pi_2(t,\xi)y(\xi)d\xi
\end{gather*}
непрерывна по $t\in[0,T]$ в норме $V$.
\end{Lemma}

Для каждого $y\in{\textbf{Э}}_1([0,T];V,H)$ положим
\begin{equation}\label{lambdaopdef:more.more.abstract}
\Lambda_0[y](t)\equiv\Pi_1(t)\varphi+\Pi_2(t,0)\psi+ \int\limits_{0}^{t}\Pi_2(t,\xi)\theta[\xi,y]\,d\xi,\,\,\,t\in[0,\,T].
\end{equation}
Из лемм \ref{Pi1Pi2_properties:more.more.abstract} и \ref{integral_Pi2ydxi:properties:more.more.abstract} следует, что
\begin{gather}\label{lambdaop:1:more.more.abstract}
\Lambda_0[y]\in C([0,T],V)\,\,\,\forall\,y\in{\textbf{Э}}_1([0,T];V,H).
\end{gather}

Изучим теперь дифференциальные свойства функции $\Lambda_0[y]$. Для этого нам потребуется ввести два линейных оператора.

Для любых $v\in V$ и $h\in H$ положим
\begin{gather*}
\Pi_3(t)v=\sum\limits_{m=1}^\infty[-\omega_m\sin(\omega_mt)]v_me_m,\,\,\,\Pi_4(t,\xi)h=\sum\limits_{m=1}^\infty\cos(\omega_m(t-\xi))h_me_m,\,\,\,(t,\xi)\in\Gamma.
\end{gather*}


\begin{Lemma}\label{Pi3Pi4_properties:more.more.abstract}
1) При всех $(t,\xi)\in\Gamma$ справедливы включения $\Pi_3(t)\in\mathcal{L}(V,H)$, $\Pi_4(t,\xi)\in\mathcal{L}(H,H)$, причём при всех $(t,\xi)\in\Gamma$ имеют место неравенства
\begin{gather*}
\|\Pi_3(t)\|_{V\to H}\leqslant1,\,\,\,\|\Pi_4(t,\xi)\|_{H\to H}\leqslant1.
\end{gather*}

2) При каждом $v\in V$ ряд для $\Pi_3(t)v$ сходится в норме $H$ равномерно по $t\in[0,T]$.

3) При каждом $h\in H$ ряд для $\Pi_4(t,\xi)h$ сходится в норме $H$ равномерно по $(t,\xi)\in\Gamma$.

4) При каждом $v\in V$ функция $[0,T]\ni t\mapsto\Pi_3(t)v$ --- элемент пространства $C([0,T],H)$.

5) При каждом $h\in H$ функция $\Gamma\ni(t,\xi)\mapsto\Pi_4(t,\xi)h$ принадлежит $C(\Gamma,H)$.
\end{Lemma}
\begin{Proof}
1) Докажем первое утверждение леммы. В самом деле, пусть $h\in H$, $v\in V$, $(t,\xi)\in\Gamma$, --- произвольны. Тогда
\begin{gather*}
\|\Pi_3(t)v\|_H^2=\sum\limits_{m=1}^\infty\omega_m^2\sin^2(\omega_mt)v_m^2\leqslant\sum\limits_{m=1}^\infty\omega_m^2v_m^2=\|v\|^2_V,\\
\|\Pi_4(t,\xi)h\|_H^2=\sum\limits_{m=1}^\infty\cos^2(\omega_m(t-\xi))h_m^2\leqslant\sum\limits_{m=1}^\infty h_m^2=\|h\|^2_H.
\end{gather*}
Таким образом, при всех $h\in H$, $v\in V$, $(t,\xi)\in\Gamma$ имеют место соотношения
\begin{gather*}
\|\Pi_3(t)v\|_H\leqslant\|v\|_V,\,\,\,\|\Pi_4(t,\xi)h\|_H\leqslant\|h\|_H,
\end{gather*}
которые и доказывают первое утверждение леммы.

2) Докажем остальные утверждения леммы.  Пусть $h\in H$, $v\in V$ --- произвольны. Во--первых, заметим, что $[0,T]$ и $\Gamma$, рассматриваемые со стандартной топологией, --- компактные
топологические пространства. При этом функции
\begin{gather*}
[0,T]\ni t\mapsto-\omega_m\sin(\omega_mt)v_m\in\mathbb{R},\,\,\,m=1,2,\dots,
\end{gather*}
непрерывны на $[0,T]$, а  функции
\begin{gather*}
\Gamma\ni(t,\xi)\mapsto\cos(\omega_m(t-\xi))h_m\in\mathbb{R},\,\,\,m=1,2,\dots,
\end{gather*}
непрерывны на $\Gamma$.

Во--вторых, при всех  $(t,\xi)\in\Gamma$ имеют место оценки
\begin{gather*}
|-\omega_m\sin(\omega_mt)v_m|\leqslant\omega_m|v_m|,\,\,\,|\cos(\omega_m(t-\xi))h_m|\leqslant|h_m|,\,\,\,m=1,2,\dots,
\end{gather*}
из которых следует, что
\begin{gather*}
|-\omega_m\sin(\omega_mt)v_m|^2\|e_m\|^2_H\leqslant\omega_m^2|v_m|^2,\,\,\, |\cos(\omega_m(t-\xi))h_m|^2\|e_m\|^2_H\leqslant|h_m|^2,\,\,\,
m=1,2,\dots
\end{gather*}
Пользуясь теперь теоремой \ref{priznak:uniform_convergence_of_functional_sequence}, получаем второе и третье утверждения леммы.

Четвёртое и пятое утверждения вытекают из второго и третьего и из следствия \ref{consequence_of_uniform_convergence_of_functional_series}. Лемма полностью доказана.
\end{Proof}

\begin{Lemma}\label{Pi1Pi3_and_Pi2Pi4_interconnection:more.more.abstract}
При всех  $h\in H$ и $v\in V$, $(t,\xi)\in\Gamma$ справедливы равенства
\begin{gather*}
\lim\limits_{\Delta t\to0}\left\|\left[\frac{\Pi_1(t+\Delta t)-\Pi_1(t)}{\Delta t}-\Pi_3(t)\right]v\right\|_H=0,\,\,\,
\lim\limits_{\Delta t\to0}\left\|\left[\frac{\Pi_2(t+\Delta t,\xi)-\Pi_2(t,\xi)}{\Delta t}-\Pi_4(t,\xi)\right]h\right\|_H=0.
\end{gather*}
\end{Lemma}
\begin{Proof}
1) Предельное соотношение
\begin{gather*}
\lim\limits_{\Delta t\to0}\left\|\left[\frac{\Pi_1(t+\Delta t)-\Pi_1(t)}{\Delta t}-\Pi_3(t)\right]v\right\|_H=0
\end{gather*}
вытекает из непрерывности вложения $V\subset H$, третьего утверждения леммы \ref{Pi1Pi2_properties:more.more.abstract}, второго утверждения
леммы \ref{Pi3Pi4_properties:more.more.abstract} и следствия \ref{consequence::differetiability_of_functional_sequence}.

2) Равенство
\begin{gather*}
\lim\limits_{\Delta t\to0}\left\|\left[\frac{\Pi_2(t+\Delta t,\xi)-\Pi_2(t,\xi)}{\Delta t}-\Pi_4(t,\xi)\right]h\right\|_H=0
\end{gather*}
вытекает из непрерывности вложения $V\subset H$, четвёртого утверждения леммы \ref{Pi1Pi2_properties:more.more.abstract}, третьего утверждения
леммы \ref{Pi3Pi4_properties:more.more.abstract} и следствия \ref{consequence::differetiability_of_functional_sequence:Delta}.
\end{Proof}

Из лемм \ref{parametric_integral_1} и \ref{Pi3Pi4_properties:more.more.abstract} вытекает
\begin{Lemma}\label{integral_Pi4ydxi:properties:more.more.abstract}
Для любой функции $y\in L_1([0,T],H)$ функция
\begin{gather*}
[0,T]\ni t\mapsto\int\limits_0^t\Pi_4(t,\xi)y(\xi)d\xi
\end{gather*}
непрерывна по $t\in[0,T]$ в норме $H$.
\end{Lemma}

Из лемм \ref{Pi1Pi2_properties:more.more.abstract} и \ref{integral_Pi4ydxi:properties:more.more.abstract} и теоремы \ref{parametric_integral_2} вытекает
\begin{Lemma}\label{derivative_of_int0tPi2ydxi::more.more.abstract}
Для любой функции $y\in C([0,T],H)$ функция
\begin{gather*}
[0,T]\ni t\mapsto\int\limits_0^t\Pi_2(t,\xi)y(\xi)d\xi
\end{gather*}
непрерывно дифференцируема на $[0,T]$ в норме пространства $H$, причём
\begin{gather*}
\frac{d}{dt}\int\limits_0^t\Pi_2(t,\xi)y(\xi)d\xi=\int\limits_0^t\Pi_4(t,\xi)y(\xi)d\xi\,\,\,\forall\,t\in[0,T].
\end{gather*}
\end{Lemma}

\begin{Lemma}\label{derivative_of_int0tPi2ydxi:more.more.abstract}
Для любой функции $y\in L_1([0,T],H)$ функция
\begin{gather*}
[0,T]\ni t\mapsto\int\limits_0^t\Pi_2(t,\xi)y(\xi)d\xi
\end{gather*}
непрерывно дифференцируема на $[0,T]$ в норме пространства $H$, причём
\begin{gather*}
\frac{d}{dt}\int\limits_0^t\Pi_2(t,\xi)y(\xi)d\xi=\int\limits_0^t\Pi_4(t,\xi)y(\xi)d\xi\,\,\,\forall\,t\in[0,T].
\end{gather*}
\end{Lemma}
\begin{Proof}
Для всех $y\in L_1([0,T],H)$ положим
\begin{gather*}
\Xi_1[y](t)=\int\limits_0^t\Pi_2(t,\xi)y(\xi)d\xi,\,\,\,\Xi_2[y](t)=\int\limits_0^t\Pi_4(t,\xi)y(\xi)d\xi,\,\,\,t\in[0,T].
\end{gather*}
Предположим сначала, что $y\in C([0,T],H)$. Тогда, согласно лемме \ref{derivative_of_int0tPi2ydxi::more.more.abstract}, для всех $h\in H$ выполнено тождество
\begin{gather*}%\label{pi2pi4yh}
\left\langle\Xi_1[y](t)-\Xi_1[y](0)-\int\limits_0^t\Xi_2[y](\eta)d\eta,h\right\rangle_H=0\,\,\,\forall\,t\in[0,T].
\end{gather*}

Пусть теперь $y\in L_1([0,T],H)$. Тогда найдётся последовательность $y_j\in C([0,T],H)$, $j=1,2,\dots$, такая, что $\lim\limits_{j\to\infty}\|y_j-y\|_{1,[0,T],H}=0$. Для каждого
$j=1,2,\dots$ можем записать тождество
\begin{gather*}%\label{pi2pi4yh}
\left\langle\Xi_1[y_j](t)-\Xi_1[y_j](0)-\int\limits_0^t\Xi_2[y_j](\eta)d\eta,h\right\rangle_H=0\,\,\,\forall\,t\in[0,T].
\end{gather*}
Перейдя затем к пределу при $j\to\infty$, получаем требуемое равенство для случая $y\in L_1([0,T],H)$. Лемма полностью доказана.
\end{Proof}

Для каждого $y\in{\textbf{Э}}_1([0,T];V,H)$ положим
\begin{equation*}
\Lambda_1[y](t)\equiv\Pi_3(t)\varphi+\Pi_4(t,0)\psi+ \int\limits_{0}^{t}\Pi_4(t,\xi)\theta[\xi,y]\,d\xi,\,\,\,t\in[0,\,T].
\end{equation*}

Из лемм \ref{Pi3Pi4_properties:more.more.abstract} и \ref{integral_Pi4ydxi:properties:more.more.abstract} следует, что
\begin{gather*}
\Lambda_1[y]\in C([0,T],H)\,\,\,\forall\,y\in{\textbf{Э}}_1([0,T];V,H).
\end{gather*}

Кроме того, согласно леммам \ref{Pi1Pi3_and_Pi2Pi4_interconnection:more.more.abstract} и \ref{derivative_of_int0tPi2ydxi:more.more.abstract}, при каждом $y\in{\textbf{Э}}_1([0,T];V,H)$
функция $\Lambda_0[y]$ непрерывно дифференцируема на $[0,T]$ в норме $H$, причём
\begin{gather*}
\frac{d}{dt}\Lambda_0[y](t)=\Lambda_1[y](t)\,\,\,\forall\,t\in[0,T].
\end{gather*}

Итак, мы доказали, что $\Lambda_0$ можно рассматривать как оператор, переводящий пространство ${\textbf{Э}}_1([0,T];V,H)$ в себя.

Изучим теперь возможность дальнейшего дифференцирования функции $\Lambda_0[y]$, или, что то же самое, дифференциальные свойства функции  $\Lambda_1[y]$.

Для всех $h\in H$, $v\in V$ положим
\begin{gather*}
\Pi_5(t)v=\sum\limits_{m=1}^\infty[-\omega_m^2\cos(\omega_mt)]v_me_m,\,\,\,\Pi_6(t,\xi)h=\sum\limits_{m=1}^\infty[-\omega_m\sin(\omega_m(t-\xi))]h_me_m,\,\,\,(t,\xi)\in\Gamma.
\end{gather*}

\begin{Lemma}\label{Pi5Pi6_properties:more.more.abstract}
1) При всех $(t,\xi)\in\Gamma$ справедливы включения $\Pi_5(t)\in\mathcal{L}(V,V^*)$, $\Pi_6(t,\xi)\in\mathcal{L}(H,V^*)$, причём при всех $(t,\xi)\in\Gamma$ имеют место неравенства
\begin{gather*}
\|\Pi_5(t)\|_{V\to V^*}\leqslant1,\,\,\,\|\Pi_6(t,\xi)\|_{H\to V^*}\leqslant1.
\end{gather*}

2) При каждом $v\in V$ ряд для $\Pi_5(t)v$ сходится в норме $V^*$ равномерно по $t\in[0,T]$.

3) При каждом $h\in H$ ряд для $\Pi_6(t,\xi)h$ сходится в норме $V^*$ равномерно по $(t,\xi)\in\Gamma$.

4) При каждом $v\in V$ функция $[0,T]\ni t\mapsto\Pi_5(t)v$ --- элемент пространства  $C([0,T],V^*)$.

5) При каждом $h\in H$ функция $\Gamma\ni(t,\xi)\mapsto\Pi_6(t,\xi)h$ принадлежит  $C(\Gamma,V^*)$.
\end{Lemma}
\begin{Proof}
1) Докажем первое утверждение леммы. В самом деле, пусть $h\in H$, $v\in V$, $(t,\xi)\in\Gamma$, --- произвольны. Тогда
\begin{gather*}
\|\Pi_5(t)v\|_{V^*}^2=\sum\limits_{m=1}^\infty\frac1{\omega_m^2}|-\omega_m^2\cos(\omega_mt)|^2v_m^2\leqslant\sum\limits_{m=1}^\infty\omega_m^2v_m^2=\|v\|^2_V,\\
\|\Pi_6(t,\xi)h\|_{V^*}^2=\sum\limits_{m=1}^\infty\frac1{\omega_m^2} |-\omega_m\sin(\omega_m(t-\xi))|^2h_m^2\leqslant\sum\limits_{m=1}^\infty h_m^2=\|h\|^2_H.
\end{gather*}
Таким образом, при всех $h\in H$, $v\in V$, $(t,\xi)\in\Gamma$ имеют место соотношения
\begin{gather*}
\|\Pi_5(t)v\|_{V^*}\leqslant\|v\|_V,\,\,\,\|\Pi_6(t,\xi)h\|_{V^*}\leqslant\|h\|_H,
\end{gather*}
которые и доказывают первое утверждение леммы.

2) Докажем остальные утверждения леммы.  Пусть $h\in H$, $v\in V$ --- произвольны. Во--первых, заметим, что $[0,T]$ и $\Gamma$, рассматриваемые со стандартной топологией, --- компактные
топологические пространства. При этом функции
\begin{gather*}
[0,T]\ni t\mapsto-\omega_m^2\cos(\omega_mt)v_m\in\mathbb{R},\,\,\,m=1,2,\dots,
\end{gather*}
непрерывны на $[0,T]$, а  функции
\begin{gather*}
\Gamma\ni(t,\xi)\mapsto-\omega_m\sin(\omega_m(t-\xi))h_m\in\mathbb{R},\,\,\,m=1,2,\dots,
\end{gather*}
непрерывны на $\Gamma$.

Во--вторых, при всех  $(t,\xi)\in\Gamma$ имеют место оценки
\begin{gather*}
|-\omega_m^2\cos(\omega_mt)v_m|\leqslant\omega_m^2|v_m|,\,\,\,|-\omega_m\sin(\omega_m(t-\xi))h_m|\leqslant\omega_m|h_m|,\,\,\,m=1,2,\dots,
\end{gather*}
из которых следует, что
\begin{gather*}
|-\omega_m^2\cos(\omega_mt)v_m|^2\|e_m\|^2_{V^*}\leqslant\omega_m^2|v_m|^2,\,\,\, |-\omega_m\sin(\omega_m(t-\xi))h_m|^2\|e_m\|^2_{V^*}\leqslant|h_m|^2,\,\,\,m=1,2,\dots
\end{gather*}
Пользуясь теперь теоремой \ref{priznak:uniform_convergence_of_functional_sequence}, получаем второе и третье утверждения леммы.

Четвёртое и пятое утверждения вытекают из второго и третьего утверждений и из следствия \ref{consequence_of_uniform_convergence_of_functional_series}. Лемма полностью доказана.
\end{Proof}

\begin{Lemma}\label{Pi3Pi5_and_Pi4Pi6_interconnection:more.more.abstract}
При всех  $h\in H$ и $v\in V$, $(t,\xi)\in\Gamma$ справедливы равенства
\begin{gather*}
\lim\limits_{\Delta t\to0}\left\|\left[\frac{\Pi_3(t+\Delta t)-\Pi_3(t)}{\Delta t}-\Pi_5(t)\right]v\right\|_{V^*}=0,\,\,\,
\lim\limits_{\Delta t\to0}\left\|\left[\frac{\Pi_4(t+\Delta t,\xi)-\Pi_4(t,\xi)}{\Delta t}-\Pi_6(t,\xi)\right]h\right\|_{V^*}=0.
\end{gather*}
\end{Lemma}
\begin{Proof}
1) Предельное соотношение
\begin{gather*}
\lim\limits_{\Delta t\to0}\left\|\left[\frac{\Pi_3(t+\Delta t)-\Pi_3(t)}{\Delta t}-\Pi_5(t)\right]v\right\|_{V^*}=0
\end{gather*}
следует из непрерывности вложений $V\subset H\cong H^*\subset V^*$, второго утверждения леммы \ref{Pi3Pi4_properties:more.more.abstract}, второго
утверждения леммы \ref{Pi5Pi6_properties:more.more.abstract} и следствия \ref{consequence::differetiability_of_functional_sequence}.

2) Равенство
\begin{gather*}
\lim\limits_{\Delta t\to0}\left\|\left[\frac{\Pi_4(t+\Delta t,\xi)-\Pi_4(t,\xi)}{\Delta t}-\Pi_6(t,\xi)\right]h\right\|_{V^*}=0
\end{gather*}
вытекает из непрерывности вложений $V\subset H\cong H^*\subset V^*$, третьего утверждения леммы \ref{Pi3Pi4_properties:more.more.abstract}, третьего
утверждения леммы \ref{Pi5Pi6_properties:more.more.abstract} и следствия \ref{consequence::differetiability_of_functional_sequence:Delta}.
\end{Proof}

Из лемм \ref{parametric_integral_1} и \ref{Pi5Pi6_properties:more.more.abstract} вытекает
\begin{Lemma}\label{integral_Pi6ydxi:properties:more.more.abstract}
Для любой функции $y\in L_1([0,T],H)$ функция
\begin{gather*}
[0,T]\ni t\mapsto\int\limits_0^t\Pi_6(t,\xi)y(\xi)d\xi
\end{gather*}
непрерывна по $t\in[0,T]$ в норме $V^*$.
\end{Lemma}

Из лемм \ref{Pi3Pi4_properties:more.more.abstract} и \ref{integral_Pi6ydxi:properties:more.more.abstract} и теоремы \ref{parametric_integral_2} вытекает
\begin{Lemma}\label{derivative_of_int0tPi4ydxi::more.more.abstract}
Для любой функции $y\in C([0,T],H)$ функция
\begin{gather*}
[0,T]\ni t\mapsto\int\limits_0^t\Pi_4(t,\xi)y(\xi)d\xi
\end{gather*}
непрерывно дифференцируема на $[0,T]$ в норме пространства $V^*$, причём
\begin{gather*}
\frac{d}{dt}\int\limits_0^t\Pi_4(t,\xi)y(\xi)d\xi=\int\limits_0^t\Pi_6(t,\xi)y(\xi)d\xi\,\,\,\forall\,t\in[0,T].
\end{gather*}
\end{Lemma}

\begin{Lemma}\label{derivative_of_int0tPi4ydxi:more.more.abstract}
Для любой функции $y\in L_1([0,T],H)$ функция
\begin{gather*}
[0,T]\ni t\mapsto\int\limits_0^t\Pi_4(t,\xi)y(\xi)d\xi
\end{gather*}
абсолютно непрерывна на $[0,T]$ в норме пространства $V^*$, причём её обобщённая производная имеет вид
\begin{gather*}
\frac{d}{dt}\int\limits_0^t\Pi_4(t,\xi)y(\xi)d\xi=y(t)+\int\limits_0^t\Pi_6(t,\xi)y(\xi)d\xi\,\,\,\forall\,t\in[0,T].
\end{gather*}
\end{Lemma}
\begin{Proof}
Для всех $y\in L_1([0,T],H)$ положим
\begin{gather*}
\Xi_1[y](t)=\int\limits_0^t\Pi_4(t,\xi)y(\xi)d\xi,\,\,\,\Xi_2[y](t)=y(t)+\int\limits_0^t\Pi_6(t,\xi)y(\xi)d\xi,\,\,\,t\in[0,T].
\end{gather*}
Предположим сначала, что $y\in C([0,T],H)$. Тогда, согласно лемме \ref{derivative_of_int0tPi4ydxi::more.more.abstract}, для всех $v\in V$ выполнено тождество
\begin{gather*}%\label{pi2pi4yh}
\left\langle\Xi_1[y](t)-\Xi_1[y](0)-\int\limits_0^t\Xi_2[y](\eta)d\eta,v\right\rangle=0\,\,\,\forall\,t\in[0,T].
\end{gather*}

Пусть теперь $y\in L_1([0,T],H)$. Тогда найдётся последовательность $y_j\in C([0,T],H)$, $j=1,2,\dots$, такая, что $\lim\limits_{j\to\infty}\|y_j-y\|_{1,[0,T],H}=0$. Для каждого
$j=1,2,\dots$ можем записать тождество
\begin{gather*}%\label{pi2pi4yh}
\left\langle\Xi_1[y_j](t)-\Xi_1[y_j](0)-\int\limits_0^t\Xi_2[y_j](\eta)d\eta,v\right\rangle=0\,\,\,\forall\,t\in[0,T].
\end{gather*}
Перейдя затем к пределу при $j\to\infty$, получаем требуемое равенство для случая $y\in L_1([0,T],H)$. Лемма полностью доказана.
\end{Proof}

Для каждого $y\in{\textbf{Э}}_1([0,T];V,H)$ положим
\begin{equation*}
\Lambda_2[y](t)\equiv\Pi_5(t)\varphi+\Pi_6(t,0)\psi+ \int\limits_{0}^{t}\Pi_6(t,\xi)\theta[\xi,y]\,d\xi,\,\,\, t\in[0,\,T].
\end{equation*}

Из лемм \ref{Pi5Pi6_properties:more.more.abstract}, \ref{Pi3Pi5_and_Pi4Pi6_interconnection:more.more.abstract}, \ref{integral_Pi6ydxi:properties:more.more.abstract} и
\ref{derivative_of_int0tPi4ydxi:more.more.abstract} следует, что
\begin{gather*}
\Lambda_1[y]\in W^1_1([0,T],V^*)\,\,\,\forall\,y\in{\textbf{Э}}_1([0,T];V,H),
\end{gather*}
причём
\begin{gather*}
\frac{d}{dt}\Lambda_1[y](t)=\Lambda_2[y](t)\,\,\,\forall\,t\in[0,T].
\end{gather*}

Итак, мы доказали, что $\Lambda_0$ можно рассматривать как оператор, переводящий элементы пространства ${\textbf{Э}}_1([0,T];V,H)$ в элементы пространства ${\textbf{Э}}_2([0,T];V,H)$.

Заметим также, что из определения операторов $\Lambda_0$ и $\Lambda_2$ следует, что
\begin{gather}\label{Lambda2representation:more.more.abstract}
\Lambda_2[y](t)=-\mathfrak{A}\Lambda_0[y](t)+\theta[t,y],\,\,\,\forall\,t\in[0,\,T],\,\,\,y\in{\textbf{Э}}_1([0,T];V,H).
\end{gather}

Рассмотрим теперь интегро--дифференциальное уравнение
\begin{equation}\label{lambdafixedpoint:more.more.abstract}
y(t)=\Lambda_0[y](t),\,\,\,t\in[0,T].
\end{equation}

Связь между решениями из ${\textbf{Э}}_1([0,T];V,H)$ уравнения~(\ref{lambdafixedpoint:more.more.abstract}) и решениями из ${\textbf{Э}}_2([0,T];V,H)$ задачи Коши
(\ref{abstrChauchyprobeq:more.more.abstract})--(\ref{abstrChauchyprobinitcond:more.more.abstract}) даёт следующая
\begin{Lemma}\label{abstrcpequivabstrintegrdeq:more.more.abstract} Всякое решение $y\in{\textbf{Э}}_1([0,T];V,H)$ уравнения~(\ref{lambdafixedpoint:more.more.abstract}) является одновременно
и решением из ${\textbf{Э}}_2([0,T];V,H)$ задачи Коши~(\ref{abstrChauchyprobeq:more.more.abstract})--(\ref{abstrChauchyprobinitcond:more.more.abstract}). Обратно, всякое решение
$y\in{\textbf{Э}}_2([0,T];V,H)$ задачи Коши~(\ref{abstrChauchyprobeq:more.more.abstract})--(\ref{abstrChauchyprobinitcond:more.more.abstract}) является решением
уравнения~(\ref{lambdafixedpoint:more.more.abstract}) в классе ${\textbf{Э}}_1([0,T];V,H)$.
\end{Lemma}

\begin{Proof}
1) Пусть $y\in{\textbf{Э}}_1([0,T];V,H)$ --- решение уравнения~(\ref{lambdafixedpoint:more.more.abstract}). Как было доказано выше, ряд для $\Lambda_1[y](t)$ можно почленно
дифференцировать, и полученный дифференцированием ряд сходится равномерно по $t\in[0,T]$ в норме $H$ к непрерывной в норме $H$ функции. Поэтому
\begin{equation}\label{dotyeq:more.more.abstract}
\dot y(t)=\Lambda_1[y](t),\,\,\,\forall\,t\in[0,T].
\end{equation}
Как показано выше, функция $\Lambda_1[y]$ абсолютно непрерывна по $t\in[0,T]$ в норме пространства $V^*$. Дифференцируя равенство (\ref{dotyeq:more.more.abstract}) и пользуясь равенством
(\ref{Lambda2representation:more.more.abstract}), получаем, что
\begin{gather*}
\ddot y(t)=-\mathfrak{A}\Lambda_0[y](t)+\theta[t,y]\mbox{ при п.в. $t\in[0,T]$.}
\end{gather*}
Поскольку $y\in{\textbf{Э}}_1([0,T];V,H)$ --- решение уравнения~(\ref{lambdafixedpoint:more.more.abstract}), то только что полученное равенство можно переписать в виде
\begin{gather*}
\ddot y(t)+\mathfrak{A}y(t)=\theta[t,y]\mbox{ при п.в. $t\in[0,T]$.}
\end{gather*}
Итак, если $y\in{\textbf{Э}}_1([0,T];V,H)$ --- решение уравнения~(\ref{lambdafixedpoint:more.more.abstract}), то $y\in{\textbf{Э}}_2([0,T];V,H)$ и при п.в. $t\in[0,T]$ удовлетворяет
уравнению (\ref{abstrChauchyprobeq:more.more.abstract}). Покажем, что $y$ удовлетворяет начальным условиям (\ref{abstrChauchyprobinitcond:more.more.abstract}).

В самом деле, т.к. $y$ --- решение уравнения~(\ref{lambdafixedpoint:more.more.abstract}), то $y(0)=\Lambda[y](0)=\Pi_1(0)\varphi=\varphi$. Кроме того, в силу
(\ref{dotyeq:more.more.abstract}), $\dot y(0)=\tilde\Lambda[y](0)=\Pi_4(0,0)\psi=\psi$. Итак, функция $y$ удовлетворяет начальным
условиям (\ref{abstrChauchyprobinitcond:more.more.abstract}).

Следовательно,  если $y\in{\textbf{Э}}_1([0,T];V,H)$ --- решение уравнения~(\ref{lambdafixedpoint:more.more.abstract}), то $y\in{\textbf{Э}}_2([0,T];V,H)$ и является решением из
${\textbf{Э}}_2([0,T];V,H)$ задачи Коши~(\ref{abstrChauchyprobeq:more.more.abstract})--(\ref{abstrChauchyprobinitcond:more.more.abstract}).

2) Покажем, что всякое решение $y\in{\textbf{Э}}_2([0,T];V,H)$ задачи Коши~(\ref{abstrChauchyprobeq:more.more.abstract})--(\ref{abstrChauchyprobinitcond:more.more.abstract}) является
решением уравнения (\ref{lambdafixedpoint:more.more.abstract}) в классе ${\textbf{Э}}_1([0,T];V,H)$. В самом деле, пусть $y\in{\textbf{Э}}_2([0,T];V,H)$ --- решение задачи
Коши~(\ref{abstrChauchyprobeq:more.more.abstract})--(\ref{abstrChauchyprobinitcond:more.more.abstract}). Тогда
$$
\langle\ddot y(t),e_j\rangle+\langle \mathfrak{A}y(t),e_j\rangle=\langle\theta[t,y],e_j\rangle,
$$
т.е.
$$
\ddot y_j(t)+\omega_k^2y_j(t)=[\theta[t,y]]_j,\,\,\,k=1,2,\dots
$$
Следовательно,
$$
y_j(t)=\cos(\omega_jt)\varphi_j+\frac{\sin(\omega_jt)}{\omega_j}\psi_j+\int\limits_{0}^t\frac{\sin(\omega_j(t-\xi))}{\omega_j}[\theta[\xi,y]]_j\,d\xi,
$$
что, в силу определения оператора $\Lambda_0$, означает, что
$$
y(t)=\Lambda_0[y](t),\,\,\,t\in[0,T].
$$
Итак, всякое решение $y\in{\textbf{Э}}_2([0,T];V,H)$ задачи Коши~(\ref{abstrChauchyprobeq:more.more.abstract})--(\ref{abstrChauchyprobinitcond:more.more.abstract}) является решением
уравнения~(\ref{lambdafixedpoint:more.more.abstract}) в классе ${\textbf{Э}}_1([0,T];V,H)$. Лемма доказана.
\end{Proof}

Прежде чем доказывать существование и единственность решения уравнения~({\ref{lambdafixedpoint:more.more.abstract}}), преобразуем выражения для $\Lambda_0$ и  $\Lambda_1$, используя
определение величины $\theta[t,y]$:
\begin{gather*}
\Lambda_0[y](t)\equiv\Pi_1(t)\varphi+\Pi_2(t,0)\psi+ \int\limits_{0}^{t}\Pi_2(t,\xi)\theta[\xi,y]\,d\xi=\Pi_1(t)\varphi+\Pi_2(t,0)\psi+\\
+ \int\limits_{0}^{t}\Pi_2(t,\xi)\beta[y](\xi)\,d\xi+\int\limits_{0}^{t}\Pi_2(t,\xi)\left[\int\limits_0^\xi P(\xi,\tau)\dot{y}(\tau)d\tau\right]\,d\xi,\,\,\,
t\in[0,T],\,\,\,y\in{\textbf{Э}}_1([0,T];V,H);\\
\Lambda_1[y](t)\equiv\Pi_3(t)\varphi+\Pi_4(t,0)\psi+ \int\limits_{0}^{t}\Pi_4(t,\xi)\theta[\xi,y]\,d\xi=\Pi_3(t)\varphi+\Pi_3(t,0)\psi+\\
+\int\limits_{0}^{t}\Pi_3(t,\xi)\beta[y](\xi)\,d\xi+\int\limits_{0}^{t}\Pi_3(t,\xi)\left[\int\limits_0^\xi P(\xi,\tau)\dot{y}(\tau)d\tau\right]\,d\xi,\,\,\,
t\in[0,T],\,\,\,y\in{\textbf{Э}}_1([0,T];V,H).
\end{gather*}
Поменяв порядок интегрирования в последнем интеграле в формуле для $\Lambda_0$ и в последнем интеграле в формуле для $\Lambda_1$, получим, что
\begin{gather*}
\Lambda_0[y](t)\equiv\Pi_1(t)\varphi+\Pi_2(t,0)\psi+\int\limits_{0}^{t}\Pi_2(t,\xi)\beta[y](\xi)\,d\xi+\int\limits_{0}^{t}\tilde{\Pi}_2(t,\tau)\dot{y}(\tau)\,d\tau,\\
\Lambda_1[y](t)\equiv\Pi_3(t)\varphi+\Pi_4(t,0)\psi+\int\limits_{0}^{t}\Pi_4(t,\xi)\beta[y](\xi)\,d\xi+\int\limits_{0}^{t}\tilde{\Pi}_4(t,\tau)\dot{y}(\tau)\,d\tau,\\
t\in[0,T],\,\,\,y\in{\textbf{Э}}_1([0,T];V,H),
\end{gather*}
где введены обозначения
\begin{gather*}
\tilde{\Pi}_2(t,\tau)=\int\limits_\tau^t\Pi_2(t,\xi)P(\xi,\tau)d\xi,\,\,\,\tilde{\Pi}_4(t,\tau)=\int\limits_\tau^t\Pi_4(t,\xi)P(\xi,\tau)d\xi,\,\,\,(t,\tau)\in\Gamma.
,\,\,\,y\in{\textbf{Э}}_1([0,T];V,H).
\end{gather*}
При этом заметим, что
\begin{gather*}
\|\tilde{\Pi}_2(t,\tau)h\|_V\leqslant\int\limits_\tau^t\|\Pi_2(t,\xi)\|_{H\to V}\|P(\xi,\tau)\|_{H\to H}\|h\|_Hd\xi\leqslant\|h\|_H\int\limits_0^T\|P(\xi,\tau)\|_{H\to H}d\xi,\\
\|\tilde{\Pi}_4(t,\tau)h\|_H\leqslant\int\limits_\tau^t\|\Pi_4(t,\xi)\|_{H\to H}\|P(\xi,\tau)\|_{H\to H}\|h\|_Hd\xi\leqslant\|h\|_H\int\limits_0^T\|P(\xi,\tau)\|_{H\to H}d\xi,\\
(t,\tau)\in\Gamma,\,\,\,h\in H.
\end{gather*}
Введя обозначения
\begin{gather*}
\eta_1[y](t,\tau)\equiv\Pi_2(t,\tau)\beta[y](\tau)+\tilde{\Pi}_2(t,\tau)\dot{y}(\tau),\,\,\,\eta_2[y](t,\tau)\equiv\Pi_4(t,\tau)\beta[y](\tau)+\tilde{\Pi}_4(t,\tau)\dot{y}(\tau),\\
(t,\tau)\in\Gamma,\,\,\,y\in{\textbf{Э}}_1([0,T];V,H),
\end{gather*}
получим, что выражения для $\Lambda_0$ и  $\Lambda_1$ можно переписать в виде
\begin{gather*}
\Lambda_0[y](t)\equiv\Pi_1(t)\varphi+\Pi_2(t,0)\psi+\int\limits_{0}^{t}\eta_1[y](t,\tau)d\tau,\,\,\,
\Lambda_1[y](t)\equiv\Pi_3(t)\varphi+\Pi_4(t,0)\psi+\int\limits_{0}^{t}\eta_2[y](t,\tau)d\tau,\\
t\in[0,T],\,\,\,y\in{\textbf{Э}}_1([0,T];V,H).
\end{gather*}

Далее, пусть $y_1$, $y_2\in{\textbf{Э}}_1([0,T];V,H)$ --- произвольны. Тогда при всех $(t,\tau)\in\Gamma$
\begin{gather*}
\|\eta_1[y_1](t,\tau)-\eta_1[y_2](t,\tau)\|_V\leqslant\|\Pi_2(t,\tau)[\beta[y_1](\tau)-\beta[y_2](\tau)]\|_V+\|\tilde{\Pi}_2(t,\tau)[\dot{y}_1(\tau)-\dot{y}_1(\tau)]\|_V\leqslant\\
\leqslant\|\Pi_2(t,\tau)\|_{H\to V}\|\beta[y_1](\tau)-\beta[y_2](\tau)\|_H+\|\tilde{\Pi}_2(t,\tau)\|_{H\to V}\|\dot{y}_1(\tau)-\dot{y}_1(\tau)\|_H\leqslant
\|\beta[y_1](\tau)-\beta[y_2](\tau)\|_H+\\
+\|\dot{y}_1(\tau)-\dot{y}_1(\tau)\|_H\int\limits_0^T\|P(\xi,\tau)\|_{H\to H}d\xi\leqslant K_0(\tau)\sqrt{\|y_1(t)-y_2(t)\|^2_V+\|\dot{y}_1(t)-\dot{y}_2(t)\|_H^2}+\\
+\sqrt{\|y_1(t)-y_2(t)\|^2_V+\|\dot{y}_1(t)-\dot{y}_2(t)\|_H^2}\int\limits_0^T\|P(\xi,\tau)\|_{H\to H}d\xi\equiv
\tilde{K}_0(\tau)\sqrt{\|y_1(t)-y_2(t)\|^2_V+\|\dot{y}_1(t)-\dot{y}_2(t)\|_H^2}\,,\\
%%%%%%%%%%%%%%%%%%%%%%%%%%%%%%%%%%%%%%%%%%%%%%
\|\eta_2[y_1](t,\tau)-\eta_2[y_2](t,\tau)\|_H\leqslant\|\Pi_4(t,\tau)[\beta[y_1](\tau)-\beta[y_2](\tau)]\|_H+\|\tilde{\Pi}_2(t,\tau)[\dot{y}_1(\tau)-\dot{y}_1(\tau)]\|_H\leqslant\\
\leqslant\|\Pi_2(t,\tau)\|_{H\to H}\|\beta[y_1](\tau)-\beta[y_2](\tau)\|_H+\|\tilde{\Pi}_2(t,\tau)\|_{H\to H}\|\dot{y}_1(\tau)-\dot{y}_1(\tau)\|_H\leqslant
\|\beta[y_1](\tau)-\beta[y_2](\tau)\|_H+\\
+\|\dot{y}_1(\tau)-\dot{y}_1(\tau)\|_H\int\limits_0^T\|P(\xi,\tau)\|_{H\to H}d\xi\leqslant K_0(\tau)\sqrt{\|y_1(t)-y_2(t)\|^2_V+\|\dot{y}_1(t)-\dot{y}_2(t)\|_H^2}+\\
+\sqrt{\|y_1(t)-y_2(t)\|^2_V+\|\dot{y}_1(t)-\dot{y}_2(t)\|_H^2}\int\limits_0^T\|P(\xi,\tau)\|_{H\to H}d\xi\equiv
\tilde{K}_0(\tau)\sqrt{\|y_1(t)-y_2(t)\|^2_V+\|\dot{y}_1(t)-\dot{y}_2(t)\|_H^2}\,,
\end{gather*}
где введено обозначение
\begin{gather*}
\tilde{K}_0(\tau)\equiv K_0(\tau)+\int\limits_0^T\|P(\xi,\tau)\|_{H\to H}d\xi,\,\,\,\tau\in[0,T].
\end{gather*}


Иначе говоря, при всех $y_1$, $y_2\in{\textbf{Э}}_1([0,T];V,H)$ и всех $(t,\tau)\in\Gamma$
\begin{gather*}
\|\eta_1[y_1](t,\tau)-\eta_1[y_2](t,\tau)\|_V\leqslant\tilde{K}_0(\tau)\sqrt{\|y_1(t)-y_2(t)\|^2_V+\|\dot{y}_1(t)-\dot{y}_2(t)\|_H^2}\,,\\
\|\eta_2[y_1](t,\tau)-\eta_2[y_2](t,\tau)\|_H\leqslant\tilde{K}_0(\tau)\sqrt{\|y_1(t)-y_2(t)\|^2_V+\|\dot{y}_1(t)-\dot{y}_2(t)\|_H^2}\,.
\end{gather*}


Докажем теперь следующий результат о существовании и единственности решения уравнения~({\ref{lambdafixedpoint:more.more.abstract}}).
\begin{Lemma}\label{uniquenesslambdafixedpoint:more.more.abstract} Уравнение~(\ref{lambdafixedpoint:more.more.abstract}) имеет единственное решение $y\in{\textbf{Э}}_1([0,T];V,H)$. Более
того, существует постоянная $\tilde{c}>0$, определяемая лишь функцией $\tilde{K}_0\in L_1[0,T]$, такая, что
\begin{equation}\label{yestim:more.more.abstract}
\|y\|_{{\textbf{Э}}_1([0,T];V,H)}\leqslant\tilde c\left[\sqrt{\|\varphi\|^2_V+\|\psi\|_H^2}+ \int\limits_{0}^{T}\|\beta[0](t)\|_H\,dt\right].
\end{equation}
\end{Lemma}
\begin{Proof}
1) Докажем вначале, что решение уравнения~({\ref{lambdafixedpoint:more.more.abstract}}) существует и единственно. Для этого нам достаточно показать, что некоторая степень оператора
$\Lambda_0\colon{\textbf{Э}}_1([0,T];V,H)\to{\textbf{Э}}_1([0,T];V,H)$ является сжимающим отображением. В силу~(\ref{betalip!more.more}),
\begin{gather*}
\|\Lambda_0[y^1](t)-\Lambda_0[y^2](t)\|_V\leqslant\int\limits_{0}^t\tilde{K}_0(\xi) \sqrt{\|y^1(\xi)-y^2(\xi)\|^2_V+\|\dot y^1(\xi)-\dot y^2(\xi)\|^2_H}\,d\xi,\\
\left\|\frac{d\Lambda_0[y^1](t)}{dt}-\frac{d\Lambda_0[y^2](t)}{dt}\right\|_H\leqslant\int\limits_{0}^t\tilde{K}_0(\xi)\sqrt{\|y^1(\xi)-y^2(\xi)\|^2_V+\|\dot y^1(\xi)-\dot y^2(\xi)\|^2_H}
\,d\xi,\,\,\,\forall\,t\in[0,T].
\end{gather*}
Поэтому
\begin{gather*}
\sqrt{\|\Lambda_0[y^1](t)-\Lambda_0[y^2](t)\|_V^2+ \left\|\frac{d\Lambda_0[y^1](t)}{dt}-\frac{d\Lambda_0[y^2](t)}{dt}\right\|_H^2}\leqslant\\
\leqslant2\int\limits_{0}^t\tilde{K}_0(\xi)\sqrt{\|y^1(\xi)-y^2(\xi)\|^2_V+\|\dot y^1(\xi)-\dot y^2(\xi)\|^2_H}\,d\xi.
\end{gather*}
Определив функцию $\sigma\colon{\textbf{Э}}_1([0,T];V,H)\to C[0,T]$ равенством $\sigma[y](t)\equiv\sqrt{\|y(t)\|^2_V+\|\dot y(t)\|^2_H}$, $t\in[0,T]$, получим, что
$$
\sigma[\Lambda_0[y^1]-\Lambda_0[y^2]](t)\leqslant2\int\limits_{0}^t\tilde K_0(\xi) \sigma[y^1-y^2](\xi)\,d\xi,\,\,\,\forall\,t\in[0,T].
$$
Следовательно,
\begin{gather*}
\sigma[\Lambda^2_0[y^1]-\Lambda^2_0[y^2]](t)= \sigma[\Lambda_0[\Lambda_0[y^1]]-\Lambda_0[\Lambda_0[y^2]]](t)\leqslant
2\int\limits_{0}^t\tilde{K}_0(\xi)\sigma[\Lambda_0[y^1]-\Lambda_0[y^2]](\xi)\,d\xi\leqslant\\
\leqslant\int\limits_{0}^t2\tilde{K}_0(\xi_1)\left[\int\limits_{0}^{\xi_1}2\tilde{K}_0(\xi_2)\sigma[y^1-y^2](\xi_2)d\xi_2\right]d\xi_1\leqslant\max_{\xi\in[0,T]}\sigma[y^1-y^2](\xi)
\int\limits_{0}^t2\tilde{K}_0(\xi_1)\left[\int\limits_{0}^{\xi_1}2\tilde{K}_0(\xi_2)d\xi_2\right]d\xi_1=\\
=\max_{\xi\in[0,T]}\sigma[y^1-y^2](\xi)\frac1{2!}\left[\int\limits_{0}^t2\tilde{K}_0(\xi)\,d\xi \right]^2.
\end{gather*}
Итак,
\begin{gather*}
\sigma[\Lambda_0[y^1]-\Lambda_0[y^2]](t)\leqslant\max_{\xi\in[0,T]}\sigma[y^1-y^2](\xi)\int\limits_{0}^t2\tilde{K}_0(\xi)\,d\xi,\\
\sigma[\Lambda^2_0[y^1]-\Lambda^2_0[y^2]](t)\leqslant\max_{\xi\in[0,T]}\sigma[y^1-y^2](\xi)\frac1{2!}\left[\int\limits_{0}^t2\tilde{K}_0(\xi)\,d\xi \right]^2\,\,\, \forall\,t\in[0,T].
\end{gather*}
Пусть для некоторого $m\geqslant1$ доказано, что
\begin{gather*}
\sigma[\Lambda^m_0[y^1]-\Lambda^m_0[y^2]](t)\leqslant\max_{\xi\in[0,T]}\sigma[y^1-y^2](\xi)\frac1{m!}\left[\int\limits_{0}^t2\tilde K_0(\xi)\,d\xi \right]^m\,\,\,\forall\,t\in[0,T].
\end{gather*}
Тогда при всех $t\in[0,T]$
\begin{gather*}
\sigma[\Lambda^{m+1}_0[y^1]-\Lambda^{m+1}_0[y^2]](t)=\sigma[\Lambda_0[\Lambda^m_0[y^1]]-\Lambda_0[\Lambda^m_0[y^2]]](t)\leqslant
\int\limits_{0}^t2\tilde{K}_0(\xi)\sigma[\Lambda^m_0[y^1]-\Lambda^m_0[y^2]](\xi)\,d\xi\leqslant\\
\leqslant\int\limits_{0}^t2\tilde{K}_0(\xi_1)\Biggl[\max_{\tau\in[0,T]}\sigma[y^1-y^2](\tau)\frac1{m!}\Biggl[\int\limits_{0}^{\xi_1}2\tilde \tilde{K}_0(\xi_2)d\xi_2\Biggr]^m\Biggr]d\xi_1=\\
=\max_{\tau\in[0,T]}\sigma[y^1-y^2](\tau)\frac1{(m+1)!}\left[\int\limits_{0}^t2\tilde{K}_0(\xi)\,d\xi \right]^{m+1}.%\,\,\,\forall\,t\in[0,T].
\end{gather*}
Таким образом,
\begin{gather*}
\sigma[\Lambda^m_0[y^1]-\Lambda^m_0[y^2]](t)\leqslant\max_{\xi\in[0,T]}\sigma[y^1-y^2](\xi)\frac1{m!}\left[\int\limits_{0}^t2\tilde{K}_0(\xi)\,d\xi \right]^m\,\,\,\forall\,t\in[0,T],\,\,\,
m=1,2,\dots.
\end{gather*}
Отсюда выводим, что
\begin{gather*}
\|\Lambda^m_0[y^1]-\Lambda^m_0[y^2]\|_{{\textbf{Э}}_1([0,T];V,H)}\leqslant\frac1{m!}\left[\int\limits_{0}^T2\tilde{K}_0(\xi)\,d\xi \right]^m \|y^1-y^2\|_{{\textbf{Э}}_1([0,T];V,H)},\,\,\,
m=1,2,\dots.
\end{gather*}
А это и означает, что некоторая степень оператора $\Lambda_0\colon{\textbf{Э}}_1([0,T];V,H)\to{\textbf{Э}}_1([0,T];V,H)$ является сжатием, что, в силу принципа неподвижной точки Банаха,
означает существование единственного решения уравнения~(\ref{lambdafixedpoint:more.more.abstract}).

2) Докажем оценку~(\ref{yestim:more.more.abstract}). В самом деле,
\begin{gather*}
\|y(t)\|_V=\|\Lambda_0[y](t)\|_V= \left\|\Pi_1(t)\varphi+\Pi_2(t,0)\psi+\int\limits_0^t\eta_1[y](t,\xi)d\xi\right\|_V\leqslant\|\Pi_1(t)\varphi\|_V+\\
+\|\Pi_2(t,0)\psi\|_V+\int\limits_0^t\|\eta_1[y](t,\xi)\|_Vd\xi\leqslant \|\varphi\|_V+\|\psi\|_H+\int\limits_0^t\|\eta_1[y](t,\xi)\|_Hd\xi\leqslant\\
\leqslant\|\varphi\|_V+\|\psi\|_H+\int\limits_0^t\|\eta_1[0](t,\xi)\|_Hd\xi+\int\limits_0^t\|\eta_1[y](t,\xi)-\eta_1[0](t,\xi)\|_Hd\xi\leqslant\\
\leqslant\|\varphi\|_V+\|\psi\|_H+\int\limits_0^t\|\beta[0](\xi)\|_Hd\xi+\int\limits_0^t\tilde{K}_0(\tau)\sigma[y](\tau)\,d\tau.
\end{gather*}
Аналогично получаем, что
\begin{gather*}
\|\dot y(t)\|_H\leqslant\|\varphi\|_V+\|\psi\|_H+\int\limits_0^t\|\beta[0](\xi)\|_Hd\xi+\int\limits_0^t\tilde{K}_0(\tau)\sigma[y](\tau)\,d\tau.
\end{gather*}
Следовательно,
\begin{gather*}
\sigma[y](t)\leqslant{2}\left[\|\varphi\|_V+\|\psi\|_H+\int\limits_0^t\|\beta[0](\xi)\|_Hd\xi+\int\limits_0^t\tilde{K}_0(\tau)\sigma[y](\tau)\,d\tau\right].
\end{gather*}
Окончательно выводим, что
\begin{gather*}
\sigma[y](t)\leqslant 2\left[\|\varphi\|_V+\|\psi\|_H+\int\limits_0^T\|\beta[0](\xi)\|_Hd\xi\right]+\int\limits_0^t2\tilde K_0(\xi)\sigma[y](\xi)\,d\xi \,\,\,\forall\,t\in[0,T].
\end{gather*}
Применяя лемму \ref{Gronwall}, заключаем, что
\begin{gather*}
\sigma[y](t)\leqslant2\sqrt{2}\exp\left[\int\limits_{0}^T2\tilde{K}_0(\xi)\,d\xi\right] \left[\sqrt{\|\varphi\|_V^2+\|\psi\|_H^2}+\int\limits_0^t\|\beta[0](\xi)\|_Hd\xi\right]\,\,\,
\forall\,t\in[0,T].
\end{gather*}
Это и означает выполнение оценки \eqref{yestim:more.more.abstract} с $\tilde c\equiv 2\sqrt2\exp(\int\limits_{0}^T2\tilde{K}_0(\xi)\,d\xi)$.  Лемма полностью доказана.
\end{Proof}


\begin{Theorem}\label{abstract_Cauchy_problem:unique_existence_theorem:more.more.abstract}
Задача Коши (\ref{abstrChauchyprobeq:more.more.abstract})--(\ref{abstrChauchyprobinitcond:more.more.abstract}) имеет единственное решение ${\mathfrak{z}}\in{\textbf{Э}}_2([0,T];V,H)$,
причём для этого решения выполнена оценка
\begin{equation}\label{zestim:more.more.abstract}
\|{\mathfrak{z}}\|_{{\textbf{Э}}_1([0,T];V,H)}\leqslant\tilde c\left[\sqrt{\|\varphi\|^2_V+\|\psi\|_H^2}+ \int\limits_{0}^{T}\|\beta[0](\xi)\|_H\,dt\right],
\end{equation}
где постоянная $\tilde c>0$ --- та же, что и в лемме \ref{uniquenesslambdafixedpoint:more.more.abstract}.
\end{Theorem}
\begin{Proof}
Пусть ${\mathfrak{z}}\in{\textbf{Э}}_1([0,T];V,H)$ --- решение уравнения~({\ref{lambdafixedpoint:more.more.abstract}}) в классе ${\textbf{Э}}_1([0,T];V,H)$, которое существует и единственно
в силу леммы~\ref{uniquenesslambdafixedpoint:more.more.abstract}, причём для этого решения справедлива оценка (\ref{zestim:more.more.abstract}) с постоянной $\tilde{c}>0$, определяемой лишь
функцией $\tilde{K}_0\in L_1[0,T]$. Согласно же лемме~\ref{abstrcpequivabstrintegrdeq:more.more.abstract}, ${\mathfrak{z}}\in{\textbf{Э}}_2([0,T];V,H)$ и является решением задачи Коши
(\ref{abstrChauchyprobeq:more.more.abstract})--(\ref{abstrChauchyprobinitcond:more.more.abstract}). Теорема доказана.
\end{Proof}

    \section{Абстрактная задача Коши с неавтономной главной частью}
Пусть $V$, $H$, $Z$ --- сепарабельные гильбертовы пространства со скалярными произведениями $\langle\cdot,\cdot\rangle_V$,  $\langle\cdot,\cdot\rangle_H$  и  $\langle\cdot,\cdot\rangle_Z$, и с соответствующими нормами $\|\cdot\|_V$, $\|\cdot\|_H$ и $\|\cdot\|_Z$;
имеет место вложение $V\subset H$, и это вложение непрерывно и компактно. Иными словами, найдётся постоянная $c_0>0$, такая, что
\begin{gather*}
\|v\|_H\leqslant c_0\|v\|_V\,\,\,\forall\,v\in V,
\end{gather*}
причём любое ограниченное в норме $V$ множество предкомпактно в норме $H$. Сопряжённое к $Z$ пространство отождествляем с $Z$.

Кроме того, пусть $T>0$ --- некоторое число, $Y$ --- рефлексивное банахово пространство с нормой $\|\cdot\|_Y$, и пусть функции $\mathfrak{A}:[0,T]\to \mathcal{L}(V,V^*)$, $\mathfrak{B}:[0,T]\to \mathcal{L}(H,V^*)$, $\mathfrak{F}:[0,T]\to \mathcal{L}(Z,Z)$, и операторы $\mathfrak{C}\in\mathcal{L}(V, Y)$, $\mathfrak{G}\in\mathcal{L}(V,Z)$ таковы, что
\begin{enumerate}
    \item при всех $v\in V$, $h\in H$, $z\in Z$ отображения
\begin{gather*}
[0,T]\ni t\mapsto\mathfrak{A}(t)v\in V^*,\,\,\,[0,T]\ni t\mapsto\mathfrak{B}(t)h\in V^*,\,\,\,[0,T]\ni t\mapsto\mathfrak{F}(t)z\in Z,
\end{gather*}
измеримы в смысле Бохнера и абсолютно непрерывны;

	\item найдутся постоянные $c_1$, $c_2>0$, такие, что
\begin{gather*}
\langle \mathfrak{A}(t)v,v\rangle+c_1\|v\|^2_H\geqslant c_2\|v\|^2_H\,\,\,\forall\,v\in V;
\end{gather*}

	\item при каждом фиксированном $t\in[0,T]$ оператор $\mathfrak{F}(t)$ --- самосопряжён;
	\item справедливо равенство
\begin{gather*}
\langle \mathfrak{A}(t)v,w\rangle=\langle \mathfrak{A}(t)w,v\rangle\,\,\forall\,v,\,\,w\in V\,\,\text{при всех $t\in[0,T]$;}
\end{gather*}

    \item операторы $\mathfrak{C}$ и $\mathfrak{G}$ --- компактны;
	\item найдётся постоянная $c_3>0$, такая, что
\begin{gather*}
\max\limits_{t\in[0,T]}\|\mathfrak{A}(t)\|_{V\to V^*}+\max\limits_{t\in[0,T]}\|\mathfrak{B}(t)\|_{H\to V^*}+
\vraisup\limits_{t\in[0,T]}\|\mathfrak{A}'(t)\|_{V\to V^*}+\vraisup\limits_{t\in[0,T]}\|\mathfrak{B}'(t)\|_{H\to V^*}+\\
+\max\limits_{t\in[0,T]}\|\mathfrak{F}(t)\|_{Z\to Z}+\vraisup\limits_{t\in[0,T]}\|\mathfrak{F}'(t)\|_{Z\to Z}\leqslant c_3;
\end{gather*}
	\item для каждого $\varepsilon>0$ найдётся число $c_4=c_4(\varepsilon)>0$, такое, что
\begin{gather*}
\|\mathfrak{G}v\|^2_Z\leqslant\varepsilon\|v\|^2_V+c_4(\varepsilon)\|v\|^2_H\,\,\,\forall\,v\in V.
\end{gather*}
\end{enumerate}

Пусть  $e_j$, $j=1,2,\dots$, --- последовательность элементов $V$, являющихся ортонормированным базисом в $H$, ортогональным базисом в $V$ и ортогональным базисом в $V^*$, причём
\begin{gather}
\label{ekprop1:nonaut}
v=\sum\limits_{j=1}^\infty v_je_j,\,\,\,v_j=\langle v,\,e_j\rangle_H,\,\,\,j=1,2,\dots,\,\,\,\|v\|^2_V=\sum\limits_{j=1}^\infty\|e_j\|^2_V|v_j|^2,\,\,\, \forall\,v\in V;\\
\label{ekprop1.h:nonaut}
h=\sum\limits_{j=1}^\infty h_je_j,\,\,\,h_j=\langle h,\,e_j\rangle_H,\,\,\,j=1,2,\dots,\,\,\,\|h\|^2_H=\sum\limits_{j=1}^\infty|h_j|^2, \,\,\, \forall\,h\in H;\\
\label{ekprop1.v*:nonaut}
v^*=\sum\limits_{j=1}^\infty v^*_je_j,\,\,\,v^*_j=\langle v^*,\,e_j\rangle,\,\,\,j=1,2,\dots,\,\,\,\|v^*\|^2_{V^*}=\sum\limits_{j=1}^\infty\|e_j\|^2_{V^*}|v^*_j|^2,\,\,\,
\forall\,v^*\in V^*.
\end{gather}

	    \subsection{Линейное уравнение без меры Радона в правой части}
Пусть $\varphi\in V$, $\psi\in H$, $f\in L_1([0,T],H)$, $\mathfrak{g}\in W^1_1([0,T], Y^*)$ --- фиксированы.

Рассмотрим задачу Коши
\begin{gather}\label{abstrChauchyprobeq:nonaut}
\ddot{\mathfrak{z}}(t)+\mathfrak{A}(t){\mathfrak{z}}(t)+\mathfrak{B}(t){\mathfrak{z}}(t) + \mathfrak{G}^*\mathfrak{F}(t)\mathfrak{G}{\mathfrak{z}}(t) = f(t) + \mathfrak{C}^*\mathfrak{g}(t),\,\,\,t\in[0,T],\\
\label{abstrChauchyprobinitcond:nonaut}
{\mathfrak{z}}(0)=\varphi,\,\,\,\dot{\mathfrak{z}}(0)=\psi,
\end{gather}
и дадим следующее
\begin{Definition}\label{SolutionDef1:nonaut} Функцию ${\mathfrak{z}}\in{\textrm{Э}}([0,T];V,H)$ назовём решением задачи Коши (\ref{abstrChauchyprobeq:nonaut}),
(\ref{abstrChauchyprobinitcond:nonaut}), если
\begin{gather}\label{defsolabstrChauchyprob1:nonaut}
\int\limits_0^T[-\langle\dot{\mathfrak{z}}(t),\dot{\eta}(t)\rangle_H+\langle \mathfrak{A}(t){\mathfrak{z}}(t),\eta(t)\rangle+\langle \mathfrak{B}(t){\mathfrak{z}}(t),\eta(t)\rangle +\langle \mathfrak{F}(t)\mathfrak{G}{\mathfrak{z}}(t),\mathfrak{G}\eta(t)\rangle_Z]dt=\\
=\notag\langle\psi,\eta(0)\rangle+\int\limits_0^T\langle f(t),\eta(t)\rangle_Hdt +\int\limits_0^T\langle{\mathfrak{g}}(t), \mathfrak{C}\eta(t)\rangle\,dt\,\,\,\forall\,{\eta}\in\hat{\textrm{Э}}{}([0,T];V,H);\\
\notag \mathfrak{z}(0)=\varphi.
\end{gather}
\end{Definition}
Под $\hat{\textrm{Э}}([0,T];V,H)$ мы в данном определении понимаем множество $\{\mathfrak{z}\in{\textrm{Э}}([0,T];V,H):\mathfrak{z}(T)=0\}$.

Далее под ${\textrm{Э}}_2([0,T];V,H)$ понимается $\{\mathfrak{z}\in{\textrm{Э}}([0,T];V,H):\ddot{\mathfrak{z}}\in L_1([0,T], V^*)\}$.

Дадим ещё одно определение решения задачи Коши  (\ref{abstrChauchyprobeq:nonaut}), (\ref{abstrChauchyprobinitcond:nonaut}).
\begin{Definition}\label{SolutionDef2:nonaut} Функцию ${\mathfrak{z}}\in{\textrm{Э}}_2([0,T];V,H)$ назовём решением задачи Коши (\ref{abstrChauchyprobeq:nonaut}),
(\ref{abstrChauchyprobinitcond:nonaut}), если
\begin{gather}\label{defsolabstrChauchyprob2:nonaut}
\langle\ddot{\mathfrak{z}}(t),v\rangle+\langle\mathfrak{A}(t){\mathfrak{z}}(t)+\mathfrak{B}(t){\mathfrak{z}}(t) +  \mathfrak{G}^*\mathfrak{F}(t)\mathfrak{G}{\mathfrak{z}}(t), v\rangle=\langle f(t), v\rangle+\langle{\mathfrak{g}}(t), \mathfrak{C}\eta(t)\rangle\,\,\,
\mbox{ при п.в. }t\in[0,T]\,\,\,\text{$\forall v\in V$,}\\
\notag {\mathfrak{z}}(0)=\varphi,\,\,\,\dot{\mathfrak{z}}(0)=\psi.
\end{gather}
\end{Definition}
Пусть ${\mathfrak{M}}^N\equiv\{\sum\limits_{j=1}^N\zeta_je_j\,:\,\zeta_j\in W^1_2[0,T],\,\,\,\zeta_j(T)=0,\,\,\,j=\overline{1,\,N}\}$,
${\mathfrak{M}}\equiv \bigcup\limits_{N=1}^\infty{\mathfrak{M}}^N$.

Покажем, что справедлива следующая
\begin{Lemma}\label{SolutionDefsEquivalence:nonaut}
Определения \ref{SolutionDef1:nonaut} и \ref{SolutionDef2:nonaut} --- эквивалентны.
\end{Lemma}
\begin{Proof}
1) Докажем, что если функция  ${\mathfrak{z}}\in{\textrm{Э}}_2([0,T];V,H)$ является решением в смысле определения \ref{SolutionDef2:nonaut}, то она является и решением в смысле
определения \ref{SolutionDef1:nonaut}.

В самом деле, пусть  ${\mathfrak{z}}\in{\textrm{Э}}_2([0,T];V,H)$ --- решение в смысле определения \ref{SolutionDef2:nonaut}.

Поскольку, согласно лемме~\ref{approx:w.abstract}, множество ${\mathfrak{M}}\equiv \bigcup\limits_{N=1}^\infty{\mathfrak{M}}^N$ плотно в $\hat{\mathcal{W}}{}^1_2([0,T];V,H)$, то нам достаточно
доказать, что тождество \eqref{defsolabstrChauchyprob1:nonaut} справедливо для функций $\eta$, имеющих вид $\eta(t)\equiv\zeta(t)e_j$, $t\in[0,T]$, где $\zeta\in W^1_2[0,T]$,
$\zeta(T)=0$.

Действительно, взяв в равенстве \eqref{defsolabstrChauchyprob2:nonaut} $v=\zeta(t)e_j$, $t\in[0,T]$, и проинтегрировав результат по $t\in[0,T]$, будем иметь
\begin{gather*}
\int\limits_0^T\langle\ddot{\mathfrak{z}}(t),\zeta(t)e_j\rangle dt+\int\limits_0^T\langle\mathfrak{A}(t){\mathfrak{z}}(t)+\mathfrak{B}(t){\mathfrak{z}}(t)+ \mathfrak{G}^*\mathfrak{F}(t)\mathfrak{G}{\mathfrak{z}}(t), \zeta(t)e_j\rangle dt=
\int\limits_0^T\langle f(t), \zeta(t)e_j\rangle dt+\int\limits_0^T\langle{\mathfrak{g}}(t), \mathfrak{C}\eta(t)\rangle\,dt.
\end{gather*}
Взяв первый из стоящих слева интегралов по частям, получим справедливость тождества \eqref{defsolabstrChauchyprob1:nonaut} для функций $\eta$, имеющих вид $\eta(t)\equiv\zeta(t)e_j$, $t\in[0,T]$, где $\zeta\in W^1_2[0,T]$, $\zeta(T)=0$.

Таким образом, мы доказали, что если  ${\mathfrak{z}}\in{\textrm{Э}}_2([0,T];V,H)$ является решением в смысле определения \ref{SolutionDef2:nonaut}, то она является и решением в смысле определения \ref{SolutionDef1:nonaut}.

2) Докажем теперь, что если функция  ${\mathfrak{z}}\in{\textrm{Э}}([0,T];V,H)$ является решением в смысле определения \ref{SolutionDef1:nonaut}, то она является и решением в
смысле определения \ref{SolutionDef2:nonaut}.

Подставляя в интегральное тождество~\eqref{defsolabstrChauchyprob1:nonaut} $\eta(t)\equiv\zeta(t)v$, $t\in[0,T]$, где $\zeta\in W^1_2[0,T]$, $\zeta(T)=0$, $v\in V$, заключаем, что
\begin{gather}\label{idetaej1:nonaut}
\int\limits_0^T\langle-\dot{\mathfrak{z}}(t),\zeta'(t)v\rangle dt+\int\limits_0^T\langle\mathfrak{A}(t){\mathfrak{z}}(t)+\mathfrak{B}(t){\mathfrak{z}}(t) + \mathfrak{G}^*\mathfrak{F}(t)\mathfrak{G}{\mathfrak{z}}(t), \zeta(t)v\rangle dt=\\
\notag=\int\limits_0^T\langle f(t), \zeta(t)v\rangle dt+\langle\psi,\zeta(0)v\rangle+\int\limits_0^T\langle{\mathfrak{g}}(t), \mathfrak{C}[v\zeta(t)]\rangle\,dt.
\end{gather}
В частности, для всех $\zeta\in\mathfrak{D}(0,T)$
\begin{gather}\label{idetaej2:nonaut}
\int\limits_0^T\langle-\dot{\mathfrak{z}}(t),\zeta'(t)v\rangle dt+\int\limits_0^T\langle\mathfrak{A}(t){\mathfrak{z}}(t)+\mathfrak{B}(t){\mathfrak{z}}(t) +\mathfrak{G}^*\mathfrak{F}(t)\mathfrak{G}{\mathfrak{z}}(t), \zeta(t)v\rangle dt=\\
\notag=\int\limits_0^T\langle f(t), \zeta(t)v\rangle dt+\int\limits_0^T\langle\mathfrak{C}^*{\mathfrak{g}}(t), \zeta(t)v\rangle\,dt.
\end{gather}
Положив затем $\tilde{f}(\zeta)\equiv\int\limits_{0}^{T}f(t)\zeta(t)\,dt$, ${\mathfrak{Z}}(\zeta)\equiv\int\limits_{0}^{T}{\mathfrak{z}}(t)\zeta(t)\,dt$,
$\tilde{\mathfrak{A}}(\zeta)\equiv\int\limits_{0}^{T}\mathfrak{A}(t){\mathfrak{z}}(t)\zeta(t)\,dt$,
$\tilde{\mathfrak{B}}(\zeta)\equiv\int\limits_{0}^{T}\mathfrak{B}(t){\mathfrak{z}}(t)\zeta(t)\,dt$, 
$\tilde{\mathfrak{F}}(\zeta)\equiv\int\limits_{0}^{T}\mathfrak{G}^*\mathfrak{F}(t)\mathfrak{G}{\mathfrak{z}}(t)\zeta(t)\,dt$, 
$\tilde{\mathfrak{g}}(\zeta)\equiv\int\limits_{0}^{T}\mathfrak{C}^*{\mathfrak{g}}(t)\zeta(t)\,dt$ и
замечая, что ${\mathfrak{Z}}'(\zeta)\equiv-\int\limits_{0}^{T}\dot{\mathfrak{z}}(t)\zeta(t)\,dt$, из тождества \eqref{idetaej2:nonaut} получаем, что
$$
-{\mathfrak{Z}}'(\zeta')+\tilde{\mathfrak{A}}(\zeta)+\tilde{\mathfrak{B}}(\zeta)+\tilde{\mathfrak{F}}(\zeta)=\tilde{f}(\zeta)+\tilde{\mathfrak{g}}(\zeta),\,\,\, \forall\,\zeta\in {\mathfrak{D}}(0,T),
$$
или, иначе,
$$
{\mathfrak{Z}}''(\zeta)+\tilde{\mathfrak{A}}(\zeta)+\tilde{\mathfrak{B}}(\zeta)+\tilde{\mathfrak{F}}(\zeta)=\tilde{f}(\zeta)+\tilde{\mathfrak{g}}(\zeta),\,\,\, \forall\,\zeta\in {\mathfrak{D}}(0,T),
$$
Последнее означает, что ${\mathfrak{Z}}''$ --- регулярна, и лежит в $L_1([0,T],\,V^*)$. Поэтому ${\mathfrak{z}}\in{\textrm{Э}}_2([0,T];V,H)$, и
\begin{equation}\label{sfzideq:nonaut}
\langle\ddot{\mathfrak{z}}(t)+\mathfrak{A}(t){\mathfrak{z}}(t)+\mathfrak{B}(t){\mathfrak{z}}(t)+\mathfrak{G}^*\mathfrak{F}(t)\mathfrak{G}{\mathfrak{z}}(t)-f(t)-\mathfrak{C}^*{\mathfrak{g}}(t),\,v\rangle=0,\,\,\,\forall\,v\in V.
\end{equation}
Взяв по частям первый из интегралов, стоящих в левой части тождества \ref{idetaej1:nonaut}, выводим, что
\begin{gather*}
\int\limits_0^T\langle\ddot{\mathfrak{z}}(t)+\mathfrak{A}(t){\mathfrak{z}}(t)+\mathfrak{B}(t){\mathfrak{z}}(t)+\mathfrak{G}^*\mathfrak{F}(t)\mathfrak{G}{\mathfrak{z}}(t)-f(t)-\mathfrak{C}^*{\mathfrak{g}}(t),\zeta(t)v\rangle dt+\langle\dot{\mathfrak{z}}(0),\zeta(0)v\rangle=\langle\psi,\zeta(0)v\rangle.
\end{gather*}
Учтя здесь равенство \eqref{sfzideq:nonaut}, получим, что
\begin{gather*}
\langle\dot{\mathfrak{z}}(0)-\psi,v\rangle=0\,\,\,\forall\,v\in V.
\end{gather*}
Иными словами,
$$
\dot{\mathfrak{z}}(0)=\psi.
$$

Из этого соотношения и соотношения \eqref{sfzideq:nonaut} и вытекает, что $\mathfrak{z}$ является решением в смысле определения \ref{SolutionDef2:nonaut}.

Лемма полностью доказана.
\end{Proof}

Покажем, что имеет место
\begin{Theorem}\label{unique_existence_theorem:abstrChauchyprobeq:nonaut}
Задача  Коши (\ref{abstrChauchyprobeq:nonaut}), (\ref{abstrChauchyprobinitcond:nonaut}) имеет единственное решение $\mathfrak{z}$ в смысле определения \ref{SolutionDef1:nonaut},
причём найдётся постоянная $\varkappa_1>0$, зависящая лишь от $T$, $c_1$, $c_2$, $c_3>0$ и от $\|\mathfrak{C}\|_{V\to Y}$,  $\|\mathfrak{G}\|_{V\to Z}$, такая, что
\begin{gather}\label{abstrChauchyprobeq:nonaut:a_priori_solution_estimate}
\|\mathfrak{z}\|_{{\textrm{Э}}([0,T];V,H)}\leqslant \varkappa_1[\sqrt{\|\varphi\|^2_V+\|\psi\|_H^2}+ \|f\|_{1,[0,T],H}+\|\mathfrak{g}\|^{(1)}_{1,[0,T],Y^*}].
\end{gather}
\end{Theorem}
\begin{Proof}
Доказательство разобьём на три части.

1) Докажем сначала единственность решения. В самом деле, пусть $\mathfrak{z}_1$, $\mathfrak{z}_2\in{\textrm{Э}}([0,T];V,H)$ --- решения задачи Коши (\ref{abstrChauchyprobeq:nonaut}), (\ref{abstrChauchyprobinitcond:nonaut}) в смысле определения \ref{SolutionDef1:nonaut}, и пусть $\mathfrak{w}\equiv\mathfrak{z}_1-\mathfrak{z}_2$. Тогда $\mathfrak{w}\in{\textrm{Э}}([0,T];V,H)$ и удовлетворяет интегральному тождеству
\begin{gather}\label{nonaut:integral_identity:w}
\int\limits_0^T[-\langle\dot{\mathfrak{w}}(t),\dot{\eta}(t)\rangle_H+\langle \mathfrak{A}(t){\mathfrak{w}}(t),\eta(t)\rangle+\langle \mathfrak{B}(t){\mathfrak{w}}(t),\eta(t)\rangle+\langle \mathfrak{F}(t)\mathfrak{G}{\mathfrak{w}}(t),\mathfrak{G}\eta(t)\rangle_Z]dt=0\\
\notag\forall\,{\eta}\in\hat{\textrm{Э}}{}([0,T];V,H);\,\,\,\notag\,\,\,\mathfrak{w}(0)=0.
\end{gather}
Введём функции $\eta^\alpha:[0,T]\to V$ ($\alpha\in[0,T]$ --- параметр) и $\beta:[0,T]\to V$ равенствами
\begin{gather*}
\eta^\alpha(t)=-\chi_{[0,\alpha]}(t)\int\limits_t^\alpha \mathfrak{w}(\xi)d\xi,\,\,\,\beta(t)=\int\limits^t_0 \mathfrak{w}(\xi)d\xi,\,\,\,t\in[0,T].
\end{gather*}
Можно показать, что $\eta^\alpha\in{\textrm{Э}}([0,T];V,H)$, $\dot{\eta}^\alpha\in L_\infty([0,T],V)\cap C([0,\alpha],V)$, $\ddot{\eta}^\alpha\in C([0,\alpha],H)$, причём
\begin{gather*}
\dot{\eta}^\alpha(t)=\chi_{[0,\alpha]}(t)\mathfrak{w}(t),\,\,\,t\in[0,T];\,\,\,\ddot{\eta}^\alpha(t)=\dot{\mathfrak{w}}(t),\,\,\,t\in[0,\alpha].
\end{gather*}
Полагая в \eqref{nonaut:integral_identity:w} $\eta=\eta^\alpha$, получаем, что для всех $\alpha\in[0,T]$
\begin{gather*}
\int\limits_0^\alpha[-\langle\ddot{\eta}^\alpha(t),\dot{\eta}^\alpha(t)\rangle_H+\langle \mathfrak{A}(t)\dot{\eta}^\alpha(t),\eta^\alpha(t)\rangle+
\langle \mathfrak{B}(t)\dot{\eta}^\alpha(t),\eta^\alpha(t)\rangle+\langle \mathfrak{F}(t)\mathfrak{G}\dot{\eta}^\alpha(t),\mathfrak{G}\eta^\alpha(t)\rangle]dt=0.
\end{gather*}
Интегрируя это соотношение по частям, выводим, что
\begin{gather*}
-\frac12[\|\mathfrak{w}(\alpha)\|^2_H+\langle\mathfrak{A}(0)\eta^\alpha(0),\eta^\alpha(0)\rangle]-\frac12\langle\mathfrak{F}(0)\mathfrak{G}\eta^\alpha(0),\mathfrak{G}\eta^\alpha(0)\rangle_Z-\\
-\int\limits_0^\alpha\left[\frac12\langle\mathfrak{A}'(t)\eta^\alpha(t),\eta^\alpha(t)\rangle-\langle\mathfrak{B}(t)\dot{\eta}^\alpha(t),\eta^\alpha(t)\rangle+\frac12\langle\mathfrak{F}'(t)\mathfrak{G}\eta^\alpha(t),\mathfrak{G}\eta^\alpha(t)\rangle_Z\right]dt=0.
\end{gather*}
Поэтому
\begin{gather*}
\frac12[\|\mathfrak{w}(\alpha)\|^2_H+\langle\mathfrak{A}(0)\eta^\alpha(0),\eta^\alpha(0)\rangle]=-\frac12\langle\mathfrak{F}(0)\mathfrak{G}\eta^\alpha(0),\mathfrak{G}\eta^\alpha(0)\rangle_Z-\int\limits_0^\alpha\biggl[\frac12\langle\mathfrak{A}'(t)\eta^\alpha(t),\eta^\alpha(t)\rangle-\langle\mathfrak{B}(t)\dot{\eta}^\alpha(t),\eta^\alpha(t)\rangle+ \\
+\frac12\langle\mathfrak{F}'(t)\mathfrak{G}{\eta}^\alpha(t),\mathfrak{G}\eta^\alpha(t)\rangle_Z\biggr]dt\leqslant \frac{c_3}2\|\mathfrak{G}\eta^\alpha(0)\|_Z^2+\int\limits_0^\alpha\biggl[\frac12\|\mathfrak{A}'(t)\|_{V\to V^*}\|\eta^\alpha(t)\|^2_V+\|\mathfrak{B}(t)\|_{H\to V^*}\|\mathfrak{w}(t)\|_H\|\eta^\alpha(t)\|_V+\\
+\frac12\|\mathfrak{F}'(t)\|_{Z\to Z}\|\mathfrak{G}\eta^\alpha(t)\|^2_Z\biggr]dt\leqslant\frac{c_3}2\|\mathfrak{G}\eta^\alpha(0)\|_Z^2+\int\limits_0^\alpha\biggl[\frac12\|\mathfrak{A}'(t)\|_{V\to V^*}\|\eta^\alpha(t)\|^2_V+\frac12\|\mathfrak{B}(t)\|_{H\to V^*}\|\mathfrak{w}(t)\|_H^2+ \\
+\frac12\|\mathfrak{B}(t)\|_{H\to V^*}\|\eta^\alpha(t)\|_V^2+\frac{c_3}2\|\mathfrak{G}\|^2_{V\to Z}\|\eta^\alpha(t)\|^2_V\biggr]dt\leqslant\frac{c_3}2\|\mathfrak{G}\eta^\alpha(0)\|_Z^2+\\
+\int\limits_{0}^{\alpha}c_3\biggl[1+\frac12\|\mathfrak{G}\|^2_{V\to Z}\biggr][\|\mathfrak{w}(t)\|_H^2+\|\eta^\alpha(t)\|_V^2]dt\leqslant
\frac{c_3}2\varepsilon\|\eta^\alpha(0)\|_V^2+\frac{c_3c_4(\varepsilon)}2\|\eta^\alpha(0)\|_H^2+\\
+\int\limits_{0}^{\alpha}c_3\biggl[1+\frac12\|\mathfrak{G}\|^2_{V\to Z}\biggr][\|\mathfrak{w}(t)\|_H^2+\|\eta^\alpha(t)\|_V^2]dt=
\frac{c_3}2\varepsilon\|\eta^\alpha(0)\|_V^2+\frac{c_3c_4(\varepsilon)}2\left\|\int\limits_0^\alpha \mathfrak{w}(\xi)d\xi\right\|_H^2+\\
+\int\limits_{0}^{\alpha}c_3\biggl[1+\frac12\|\mathfrak{G}\|^2_{V\to Z}\biggr][\|\mathfrak{w}(t)\|_H^2+\|\eta^\alpha(t)\|_V^2]dt\leqslant
\frac{c_3}2\varepsilon\|\eta^\alpha(0)\|_V^2+\frac{c_3c_4(\varepsilon)}2\left[\int\limits_0^\alpha1\cdot\|\mathfrak{w}(\xi)\|_Hd\xi\right]^2+\\
+\int\limits_{0}^{\alpha}c_3\biggl[1+\frac12\|\mathfrak{G}\|^2_{V\to Z}\biggr][\|\mathfrak{w}(t)\|_H^2+\|\eta^\alpha(t)\|_V^2]dt\leqslant
\frac{c_3}2\varepsilon\|\eta^\alpha(0)\|_V^2+\frac{c_3c_4(\varepsilon)}2\alpha\int\limits_0^\alpha\|\mathfrak{w}(\xi)\|^2_Hd\xi+\\
+\int\limits_{0}^{\alpha}c_3\biggl[1+\frac12\|\mathfrak{G}\|^2_{V\to Z}\biggr][\|\mathfrak{w}(t)\|_H^2+\|\eta^\alpha(t)\|_V^2]dt\leqslant
\frac{c_3}2\varepsilon\|\eta^\alpha(0)\|_V^2+\\
+\int\limits_{0}^{\alpha}c_3\biggl[1+\frac12\|\mathfrak{G}\|^2_{V\to Z}+\frac{c_4(\varepsilon)}2T\biggr][\|\mathfrak{w}(t)\|_H^2+\|\eta^\alpha(t)\|_V^2]dt.
\end{gather*}
Итак,
\begin{gather*}
\|\mathfrak{w}(\alpha)\|^2_H+\langle\mathfrak{A}(0)\eta^\alpha(0),\eta^\alpha(0)\rangle-c_3\varepsilon\|\eta^\alpha(0)\|_V^2\leqslant
\int\limits_{0}^{\alpha}\tilde{c}_0(\varepsilon)[\|\mathfrak{w}(t)\|_H^2+\|\eta^\alpha(t)\|_V^2]dt,
\end{gather*}
где введено обозначение $\tilde{c}_0(\varepsilon)=c_3[2+\|\mathfrak{G}\|^2_{V\to Z}+c_4(\varepsilon)T]$.

Добавив к обеим частям этого неравенства величину $c_1\|\eta^\alpha(0)\|^2_H$, получим, что
\begin{gather*}
\|\mathfrak{w}(\alpha)\|^2_H+(c_2-c_3\varepsilon)\|\eta^\alpha(0)\|^2_V\leqslant\int\limits_{0}^{\alpha}\tilde{c}_0(\varepsilon)[\|\mathfrak{w}(t)\|_H^2+\|\eta^\alpha(t)\|_V^2]dt+c_1\left\|\int\limits_0^\alpha \mathfrak{w}(\xi)d\xi\right\|^2_H\leqslant\\
\leqslant\int\limits_{0}^{\alpha}\tilde{c}_0(\varepsilon)[\|\mathfrak{w}(t)\|_H^2+\|\eta^\alpha(t)\|_V^2]dt+
c_1c_0^2\left[\int\limits_0^\alpha\|\mathfrak{w}(\xi)\|_Vd\xi\right]^2\leqslant\int\limits_{0}^{\alpha}\tilde{c}_1(\varepsilon)[\|\mathfrak{w}(t)\|_H^2+\|\eta^\alpha(t)\|_V^2]dt,
\end{gather*}
где $\tilde{c}_1(\varepsilon)=\tilde{c}_0(\varepsilon)+c_1c_0^2T$.

Таким образом,
\begin{gather*}
\|\mathfrak{w}(\alpha)\|^2_H+(c_2-c_3\varepsilon)\|\eta^\alpha(0)\|^2_V\leqslant\int\limits_{0}^{\alpha}\tilde{c}_1(\varepsilon)[\|\mathfrak{w}(t)\|_H^2+\|\eta^\alpha(t)\|_V^2]dt.
\end{gather*}
Взяв здесь $\varepsilon=\frac{c_2}{2c_3}$ и введя обозначение $\tilde{c}_2\equiv \tilde{c}_1\left(\frac{c_2}{2c_3}\right)$, получим, что
\begin{gather*}
\|\mathfrak{w}(\alpha)\|^2_H+\frac{c_2}2\|\eta^\alpha(0)\|^2_V\leqslant\int\limits_{0}^{\alpha}\tilde{c}_2[\|\mathfrak{w}(t)\|_H^2+\|\eta^\alpha(t)\|_V^2]dt.
\end{gather*}
С помощью функции $\beta$ это можно переписать в виде
\begin{gather*}
\|\mathfrak{w}(\alpha)\|^2_H+\frac{c_2}2\|\beta(\alpha)\|^2_V\leqslant\int\limits_0^\alpha\tilde{c}_2[\|\mathfrak{w}(t)\|_H^2+\|\beta(t)-\beta(\alpha)\|_V^2]dt=\int\limits_0^\alpha\tilde{c}_2[\|\mathfrak{w}(t)\|_H^2+\|\beta(t)\|^2_V+\|\beta(\alpha)\|^2_V-\\
-2\langle\beta(t),\beta(\alpha)\rangle_V]dt\leqslant\int\limits_0^\alpha\tilde{c}_2[\|\mathfrak{w}(t)\|_H^2+2\|\beta(t)\|^2_V+2\|\beta(\alpha)\|^2_V]dt=\int\limits_0^\alpha\tilde{c}_2[\|\mathfrak{w}(t)\|_H^2+2\|\beta(t)\|^2_V]dt+\\
+2\tilde{c}_2\alpha\|\beta(\alpha)\|^2_V\leqslant\int\limits_0^\alpha2\tilde{c}_2[\|\mathfrak{w}(t)\|_H^2+\|\beta(t)\|^2_V]dt+2\tilde{c}_2\alpha\|\beta(\alpha)\|^2_V.
\end{gather*}
Отсюда вытекает, что
\begin{gather*}
\|\mathfrak{w}(\alpha)\|^2_H+\left(\frac{c_2}{2}-2\tilde{c}_2\alpha\right)\|\beta(\alpha)\|^2_V\leqslant\int\limits_0^\alpha2\tilde{c}_2[\|\mathfrak{w}(t)\|_H^2+\|\beta(t)\|^2_V]dt.
\end{gather*}
Ограничившись в данном неравенстве числами $\alpha\in[0,\alpha_0]$, где $\alpha_0\equiv\displaystyle\frac{c_2}{8\tilde{c}_2}$, получим, что для всех $\alpha\in[0,\alpha_0]$
\begin{gather*}
\|\mathfrak{w}(\alpha)\|^2_H+\|\beta(\alpha)\|^2_V\leqslant\int\limits_0^\alpha\frac{2\tilde{c}_0}{\min\{1,\frac{c_2}4\}}[\|\mathfrak{w}(t)\|_H^2+\|\beta(t)\|^2_V]dt.
\end{gather*}
Применив к данному неравенству лемму Гронуолла, выводим, что
\begin{gather*}
\|\mathfrak{w}(\alpha)\|^2_H+\|\beta(\alpha)\|^2_V=0,\,\,\,\alpha\in[0,\alpha_0].
\end{gather*}
Рассуждая аналогичным образом, за конечное число шагов получим, что
\begin{gather*}
\|\mathfrak{w}(\alpha)\|^2_H+\|\beta(\alpha)\|^2_V=0,\,\,\,\alpha\in[0,T].
\end{gather*}
Вспоминая теперь определение функции $\mathfrak{w}$, заключаем, что единственность решения доказана.

2) Докажем существование решения.

Будем искать приближённое решение $\mathfrak{z}^N$ задачи Коши (\ref{abstrChauchyprobeq:nonaut}), (\ref{abstrChauchyprobinitcond:nonaut}) в виде
$\mathfrak{z}^N(t)\equiv\sum\limits_{m=1}^Nh^N_m(t)e_m$, где набор функций $h^N_m\in W^1_1[0,T]$, $m=\overline{1,N}$, --- единственное решение задачи Коши
\begin{gather}
\label{hNCauchyProblem:equation}
\ddot{h}^N_k(t)+\sum\limits_{m=1}^{N}\mathfrak{a}_{km}(t)h^N_m(t)=f_k(t)+(\mathfrak{C}^*\mathfrak{g}(t))_k,\,\,\,t\in[0,T],\\
\label{hNCauchyProblem:initial.conditions}
h^N_k(0)=\varphi_k,\,\,\,\dot{h}^N_k(0)=\psi_k,\,\,\,k=\overline{1,N}.
\end{gather}
Здесь введены следующие обозначения:
\begin{gather*}
\mathfrak{a}_{km}(t)\equiv\langle\mathfrak{A}(t)e_m+\mathfrak{B}(t)e_m,e_k\rangle+\langle\mathfrak{F}(t)\mathfrak{G}e_m,\mathfrak{G}e_k\rangle_Z,\,\,\,
f_k(t)\equiv\langle f(t),e_k\rangle_H,\,\,\, (\mathfrak{C}^*\mathfrak{g}(t))_k\equiv\langle\mathfrak{C}^*\mathfrak{g}(t), e_k\rangle,\\
\varphi_k\equiv\langle\varphi,e_k\rangle_H,\,\,\,\psi_k\equiv\langle\psi,e_k\rangle_H,\,\,\,m,\,\,k=1,2,\dots,\,\,\,t\in[0,T].
\end{gather*}
Умножив $k$--е уравнение (\ref{hNCauchyProblem:equation}) на $\dot{h}^N_k(t)$, сложив все получившиеся уравнения, и проинтегрировав результат по $t\in[0,\tau]$, выводим, что
\begin{gather*}
\frac12[\|\dot{\mathfrak{z}}^N(\tau)\|^2_H+\langle\mathfrak{A}(\tau)\mathfrak{z}^N(\tau),\mathfrak{z}^N(\tau)\rangle]-
\frac12[\|\dot{\mathfrak{z}}^N(0)\|^2_H+\langle\mathfrak{A}(0)\mathfrak{z}^N(0),\mathfrak{z}^N(0)\rangle]+\Biggl[
\langle\mathfrak{B}(t)\mathfrak{z}^N(t),\mathfrak{z}^N(t)\rangle+\\
+\frac12\langle\mathfrak{F}(t)\mathfrak{G}\mathfrak{z}^N(t),\mathfrak{G}\mathfrak{z}^N(t)\rangle_Z-\langle\mathfrak{g}(t),\mathfrak{C}\mathfrak{z}^N(t)\rangle\Biggr]\Biggl|_{t=0}^{t=\tau}-\int\limits_0^\tau\Biggl[\frac12\langle\mathfrak{A}'(t)\mathfrak{z}^N(t),\mathfrak{z}^N(t)\rangle+
\langle\mathfrak{B}'(t)\mathfrak{z}^N(t),\mathfrak{z}^N(t)\rangle+\\
+\langle\mathfrak{B}(t)\dot{\mathfrak{z}}^N(t),\mathfrak{z}^N(t)\rangle\Biggr]dt-\frac12\int\limits_0^\tau\langle\mathfrak{F}'(t)\mathfrak{G}\mathfrak{z}^N(t),\mathfrak{G}\mathfrak{z}^N(t)\rangle_Z\,dt=\int\limits_0^\tau\langle f(t),\dot{\mathfrak{z}}^N(t)\rangle_H\,dt-\int\limits_0^\tau
\langle\mathfrak{g}'(t),\mathfrak{C}\mathfrak{z}^N(t)\rangle\,dt.
\end{gather*}
Поэтому
\begin{gather*}
\frac12[\|\dot{\mathfrak{z}}^N(\tau)\|^2_H+\langle\mathfrak{A}(\tau)\mathfrak{z}^N(\tau),\mathfrak{z}^N(\tau)\rangle]\leqslant
\frac12[\|\dot{\mathfrak{z}}^N(0)\|^2_H+\langle\mathfrak{A}(0)\mathfrak{z}^N(0),\mathfrak{z}^N(0)\rangle]+\Biggl[
-\langle\mathfrak{B}(t)\mathfrak{z}^N(t),\mathfrak{z}^N(t)\rangle-\\
-\frac12\langle\mathfrak{F}(t)\mathfrak{G}\mathfrak{z}^N(t),\mathfrak{G}\mathfrak{z}^N(t)\rangle_Z+\langle\mathfrak{g}(t),\mathfrak{C}\mathfrak{z}^N(t)\rangle\Biggr]\Biggl|_{t=0}^{t=\tau}+\int\limits_0^\tau\Biggl[\frac{c_3}2\|\mathfrak{z}^N(t)\|^2_V+c_3\|\mathfrak{z}^N(t)\|_H\|\mathfrak{z}^N(t)\|_V+\\
+c_3\|\dot{\mathfrak{z}}^N(t)\|_H\|\mathfrak{z}^N(t)\|_V\Biggr]\,dt+\frac{c_3}2\int\limits_0^\tau\|\mathfrak{G}\mathfrak{z}^N(t)\|^2_Zdt+\\
+[\|f\|_{1,[0,T],H}+\|\mathfrak{C}\|_{V\to Y}\|\mathfrak{g}\|^{(1)}_{1,[0,T],Y^*}]\max\limits_{t\in[0,\tau]}\sqrt{\|\dot{\mathfrak{z}}^N(t)\|^2_H+\|{\mathfrak{z}}^N(t)\|^2_V}\,
\leqslant\frac{\max\{1,c_3\}}2[\|\dot{\mathfrak{z}}^N(0)\|^2_H+\|\mathfrak{z}^N(0)\|_V^2]+\\
+c_3\|\mathfrak{z}^N(\tau)\|_H\|\mathfrak{z}^N(\tau)\|_V+c_3\|\mathfrak{z}^N(0)\|_H\|\mathfrak{z}^N(0)\|_V+\frac{c_3}{2}\|\mathfrak{G}\mathfrak{z}^N(\tau)\|_Z^2+\frac{c_3}{2}\|\mathfrak{G}\|_{V\to Z}^2\|\mathfrak{z}^N(0)\|_V^2+\\
+\|\mathfrak{g}(\tau)\|_{Y^*}\|\mathfrak{C}\|_{V\to Y}\|\mathfrak{z}^N(\tau)\|_V+\|\mathfrak{g}(0)\|_{Y^*}\|\mathfrak{C}\|_{V\to Y}\|\mathfrak{z}^N(0)\|_V+\int\limits_0^\tau c_3(1+c_0)[\|\mathfrak{z}^N(t)\|^2_V+\|\dot{\mathfrak{z}}^N(t)\|^2_H]\,dt+\\
+\frac{c_3\|\mathfrak{G}\|_{V\to Z}^2}2\int\limits_0^\tau\|\mathfrak{z}^N(t)\|^2_Vdt+[\|f\|_{1,[0,T],H}+\|\mathfrak{C}\|_{V\to Y}\|\mathfrak{g}\|^{(1)}_{1,[0,T],Y^*}]\max\limits_{t\in[0,\tau]}\sqrt{\|\dot{\mathfrak{z}}^N(t)\|^2_H+\|{\mathfrak{z}}^N(t)\|^2_V}\,.
\end{gather*}
Таким образом,
\begin{gather*}
\frac12[\|\dot{\mathfrak{z}}^N(\tau)\|^2_H+\langle\mathfrak{A}(\tau)\mathfrak{z}^N(\tau),\mathfrak{z}^N(\tau)\rangle]\leqslant\rho_1[\|\dot{\mathfrak{z}}^N(0)\|^2_H+\|\mathfrak{z}^N(0)\|_V^2]+\rho_1\int\limits_0^\tau[\|\mathfrak{z}^N(t)\|^2_V+\|\dot{\mathfrak{z}}^N(t)\|^2_H]\,dt+\\
+c_3\|\mathfrak{z}^N(\tau)\|_H\|\mathfrak{z}^N(\tau)\|_V+\frac{c_3}{2}\|\mathfrak{G}\mathfrak{z}^N(\tau)\|_Z^2+[\|\mathfrak{C}\|_{V\to Y} [\|\mathfrak{g}(\tau)\|_{Y^*}+\|\mathfrak{g}(0)\|_{Y^*}+\|\mathfrak{g}\|^{(1)}_{1,[0,T],Y^*}]+\\
+\|f\|_{1,[0,T],H}]\max\limits_{t\in[0,\tau]}\sqrt{\|\dot{\mathfrak{z}}^N(t)\|^2_H+\|{\mathfrak{z}}^N(t)\|^2_V}\,,
\end{gather*}
где $\rho_1\equiv\max\{1,c_3\}+c_3c_0+\frac{c_3\|\mathfrak{G}\|_{V\to Z}^2}2$. 

Применяя к данному неравенству теорему \ref{W1p([0,T],X)_embedding}, будем иметь
\begin{gather*}
\frac12[\|\dot{\mathfrak{z}}^N(\tau)\|^2_H+\langle\mathfrak{A}(\tau)\mathfrak{z}^N(\tau),\mathfrak{z}^N(\tau)\rangle]\leqslant\rho_1[\|\dot{\mathfrak{z}}^N(0)\|^2_H+\|\mathfrak{z}^N(0)\|_V^2]+\rho_1\int\limits_0^\tau[\|\mathfrak{z}^N(t)\|^2_V+\|\dot{\mathfrak{z}}^N(t)\|^2_H]\,dt+\\
+c_3\|\mathfrak{z}^N(\tau)\|_H\|\mathfrak{z}^N(\tau)\|_V+\frac{c_3}{2}\|\mathfrak{G}\mathfrak{z}^N(\tau)\|_Z^2+ \rho_2[\|\mathfrak{g}\|^{(1)}_{1,[0,T],Y^*}+\|f\|_{1,[0,T],H}]\max\limits_{t\in[0,\tau]}\sqrt{\|\dot{\mathfrak{z}}^N(t)\|^2_H+\|{\mathfrak{z}}^N(t)\|^2_V}\,,
\end{gather*}
где $\rho_2\equiv[2A_1+1]\|\mathfrak{C}\|_{V\to Y}$.

Применяя к слагаемому $\|\mathfrak{z}^N(\tau)\|_H\|\mathfrak{z}^N(\tau)\|_V$ неравенство Коши с $\varepsilon$, заключаем, что
\begin{gather*}
\frac12[\|\dot{\mathfrak{z}}^N(\tau)\|^2_H+\langle\mathfrak{A}(\tau)\mathfrak{z}^N(\tau),\mathfrak{z}^N(\tau)\rangle]\leqslant\rho_1[\|\dot{\mathfrak{z}}^N(0)\|^2_H+\|\mathfrak{z}^N(0)\|_V^2]+\rho_1\int\limits_0^\tau[\|\mathfrak{z}^N(t)\|^2_V+\|\dot{\mathfrak{z}}^N(t)\|^2_H]\,dt+\\
+c_3\biggl[\frac1{2\varepsilon}\|\mathfrak{z}^N(\tau)\|^2_H+\frac{\varepsilon}{2}\|\mathfrak{z}^N(\tau)\|^2_V\biggr]+\frac{c_3}{2}\|\mathfrak{G}\mathfrak{z}^N(\tau)\|_Z^2+ \rho_2[\|\mathfrak{g}\|^{(1)}_{1,[0,T],Y^*}+\|f\|_{1,[0,T],H}]\max\limits_{t\in[0,\tau]}\sqrt{\|\dot{\mathfrak{z}}^N(t)\|^2_H+\|{\mathfrak{z}}^N(t)\|^2_V}\,.
\end{gather*}
Учитывая здесь условия на оператор $\mathfrak{G}$, выводим, что
\begin{gather*}
\frac12[\|\dot{\mathfrak{z}}^N(\tau)\|^2_H+\langle\mathfrak{A}(\tau)\mathfrak{z}^N(\tau),\mathfrak{z}^N(\tau)\rangle]\leqslant\rho_1[\|\dot{\mathfrak{z}}^N(0)\|^2_H+\|\mathfrak{z}^N(0)\|_V^2]+\rho_1\int\limits_0^\tau[\|\mathfrak{z}^N(t)\|^2_V+\|\dot{\mathfrak{z}}^N(t)\|^2_H]\,dt+\\
+c_3\biggl[\frac1{2\varepsilon}\|\mathfrak{z}^N(\tau)\|^2_H+\frac{\varepsilon}{2}\|\mathfrak{z}^N(\tau)\|^2_V\biggr]+
\frac{c_3}{2}\biggl[\varepsilon\|\mathfrak{z}^N(\tau)\|_V^2+c_4(\varepsilon)\|\mathfrak{z}^N(\tau)\|_H^2\biggr]+\\ +\rho_2[\|\mathfrak{g}\|^{(1)}_{1,[0,T],Y^*}+\|f\|_{1,[0,T],H}]\max\limits_{t\in[0,\tau]}\sqrt{\|\dot{\mathfrak{z}}^N(t)\|^2_H+\|{\mathfrak{z}}^N(t)\|^2_V}\,.
\end{gather*}
Отсюда вытекает, что
\begin{gather*}
\|\dot{\mathfrak{z}}^N(\tau)\|^2_H+\langle\mathfrak{A}(\tau)\mathfrak{z}^N(\tau),\mathfrak{z}^N(\tau)\rangle-2c_3\varepsilon\|\mathfrak{z}^N(\tau)\|_V^2\leqslant2\rho_1[\|\dot{\mathfrak{z}}^N(0)\|^2_H+\|\mathfrak{z}^N(0)\|_V^2]+\\
+2\rho_1\int\limits_0^\tau[\|\mathfrak{z}^N(t)\|^2_V+\|\dot{\mathfrak{z}}^N(t)\|^2_H]\,dt+\rho_3(\varepsilon)\|\mathfrak{z}^N(\tau)\|^2_H+2\rho_2[\|\mathfrak{g}\|^{(1)}_{1,[0,T],Y^*}+\|f\|_{1,[0,T],H}]\max\limits_{t\in[0,\tau]}\sqrt{\|\dot{\mathfrak{z}}^N(t)\|^2_H+\|{\mathfrak{z}}^N(t)\|^2_V}\,,
\end{gather*}
где $\rho_3(\varepsilon)\equiv c_3(\varepsilon^{-1}+c_4(\varepsilon))$.

Прибавляя к обеим частям данного неравенства слагаемое $c_1\|\mathfrak{z}^N(\tau)\|^2_H$, получим, что
\begin{gather*}
\|\dot{\mathfrak{z}}^N(\tau)\|^2_H+[c_2-2c_3\varepsilon]\|\mathfrak{z}^N(\tau)\|_V^2\leqslant2\rho_1[\|\dot{\mathfrak{z}}^N(0)\|^2_H+\|\mathfrak{z}^N(0)\|_V^2]+2\rho_1\int\limits_0^\tau[\|\mathfrak{z}^N(t)\|^2_V+\|\dot{\mathfrak{z}}^N(t)\|^2_H]\,dt+\\
+[\rho_3(\varepsilon)+c_1]\|\mathfrak{z}^N(\tau)\|^2_H+2\rho_2[\|\mathfrak{g}\|^{(1)}_{1,[0,T],Y^*}+\|f\|_{1,[0,T],H}]\max\limits_{t\in[0,\tau]}\sqrt{\|\dot{\mathfrak{z}}^N(t)\|^2_H+\|{\mathfrak{z}}^N(t)\|^2_V}\,.
\end{gather*}
Однако,
\begin{gather*}
\|\mathfrak{z}^N(\tau)\|^2_H=\left\|\mathfrak{z}^N(0)+\int\limits_0^\tau\dot{\mathfrak{z}}^N(t)dt\right\|^2_H\leqslant2\|\mathfrak{z}^N(0)\|^2_H+2\left\|\int\limits_0^\tau\dot{\mathfrak{z}}^N(t)dt\right\|^2_H\leqslant2c_0^2\|\mathfrak{z}^N(0)\|^2_V+2\tau\int\limits_0^\tau\|\dot{\mathfrak{z}}^N(t)\|^2_Hdt\leqslant\\
\leqslant2c_0^2\|\mathfrak{z}^N(0)\|^2_V+2T\int\limits_0^\tau\|\dot{\mathfrak{z}}^N(t)\|^2_Hdt.
\end{gather*}
Поэтому
\begin{gather*}
\|\dot{\mathfrak{z}}^N(\tau)\|^2_H+[c_2-2c_3\varepsilon]\|\mathfrak{z}^N(\tau)\|_V^2\leqslant\rho_4(\varepsilon)[\|\dot{\mathfrak{z}}^N(0)\|^2_H+\|\mathfrak{z}^N(0)\|_V^2]+\rho_5(\varepsilon)\int\limits_0^\tau[\|\mathfrak{z}^N(t)\|^2_V+\|\dot{\mathfrak{z}}^N(t)\|^2_H]\,dt+\\
+2\rho_2[\|\mathfrak{g}\|^{(1)}_{1,[0,T],Y^*}+\|f\|_{1,[0,T],H}]\max\limits_{t\in[0,\tau]}\sqrt{\|\dot{\mathfrak{z}}^N(t)\|^2_H+\|{\mathfrak{z}}^N(t)\|^2_V}\,,
\end{gather*}
где $\rho_4(\varepsilon)\equiv2\rho_1+2c_0^2[\rho_3(\varepsilon)+c_1]$,  $\rho_5(\varepsilon)\equiv2\rho_1+2T[\rho_3(\varepsilon)+c_1]$.

Взяв здесь $\varepsilon=\varepsilon_0\equiv\frac{c_2}{4c_3}$, будем иметь
\begin{gather*}
\|\mathfrak{z}^N(\tau)\|_V^2+\|\dot{\mathfrak{z}}^N(\tau)\|^2_H\leqslant\rho_6[[\|\mathfrak{z}^N(0)\|_V^2+\|\dot{\mathfrak{z}}^N(0)\|^2_H]+
[\|\mathfrak{g}\|^{(1)}_{1,[0,T],Y^*}+\|f\|_{1,[0,T],H}]\max\limits_{t\in[0,\tau]}\sqrt{\|\dot{\mathfrak{z}}^N(t)\|^2_H+\|{\mathfrak{z}}^N(t)\|^2_V}]+\\
+\rho_7\int\limits_0^\tau[\|\mathfrak{z}^N(t)\|^2_V+\|\dot{\mathfrak{z}}^N(t)\|^2_H]\,dt,
\end{gather*}
где $\rho_6\equiv\frac{\max\{\rho_4(\varepsilon_0),2\rho_2\}}{\min\{1,\frac{c_2}{2}\}}$, $\rho_7\equiv\frac{\rho_5(\varepsilon)}{\min\{1,\frac{c_2}{2}\}}$.

Введя обозначение
$$
\mathfrak{y}^N(\tau)\equiv\|\mathfrak{z}^N(\tau)\|_V^2+\|\dot{\mathfrak{z}}^N(\tau)\|^2_H,
$$
можем записать
\begin{gather*}
\mathfrak{y}^N(\tau)\leqslant\rho_6[\mathfrak{y}^N(0)+[\|\mathfrak{g}\|^{(1)}_{1,[0,T],Y^*}+\|f\|_{1,[0,T],H}]\max\limits_{t\in[0,\tau]}\sqrt{\mathfrak{y}^N(t)}]+\rho_7\int\limits_0^\tau\mathfrak{y}^N(t)dt\,\,\,\forall\,\tau\in[0,T].
\end{gather*}
Применив к данному неравенству лемму \ref{myGronwall}, получим, что
\begin{gather}\label{sec:more.more.abstract:yN_inequality1}
\max\limits_{t\in[0,T]}\sqrt{\mathfrak{y}^N(t)}\leqslant\rho_8\left[\sqrt{\mathfrak{y}^N(0)}+\|\mathfrak{g}\|^{(1)}_{1,[0,T],Y^*}+\|f\|_{1,[0,T],H}\right],
\end{gather}
где $\rho_8\equiv\rho_6\exp[\rho_7T]$.

Заметим, что из (\ref{sec:more.more.abstract:yN_inequality1}) следует
\begin{gather}\label{sec:more.more.abstract:zNnorm_estimate}
\|\mathfrak{z}^N\|_{\textrm{Э}([0,T]; V,H)}\leqslant2\rho_8\left[\sqrt{\|\varphi^N\|^2_V+\|\psi^N\|^2_H}+\|\mathfrak{g}\|^{(1)}_{1,[0,T],Y^*}+\|f\|_{1,[0,T],H}\right],
\end{gather}
где $\varphi^N\equiv\sum\limits_{m=1}^N\varphi_me_m$, $\psi^N\equiv\sum\limits_{m=1}^N\psi_me_m$.

Поскольку
\begin{gather}\label{sec:more.more.abstract:phipsi}
\lim\limits_{N\to\infty}\|\varphi^N-\varphi\|_V=0,\,\,\, \lim\limits_{N\to\infty}\|\psi^N-\psi\|_H=0,
\end{gather}
то справедливо предельное соотношение
\begin{gather}\label{sec:semilinear:problem:third2[n>1]:phipsi1}
\lim\limits_{N\to\infty}[\|\varphi^N\|_V+\|\psi^N\|_H]= \|\varphi\|_V+\|\psi\|_H.
\end{gather}
Поэтому найдётся константа $\rho_9>0$, такая, что
\begin{gather}\label{sec:more.more.abstract:zNnorm_boundness}
\|\mathfrak{z}^N\|_{\textrm{Э}([0,T]; V,H)}\leqslant\rho_9\,\,\,\forall\,N=1,2,\dots.
\end{gather}

На основании \eqref{sec:more.more.abstract:zNnorm_boundness} и теоремы \ref{V1120QTcompactness:abstract} заключаем, что найдутся подпоследовательность $\mathfrak{z}^{N_m}$, $m=1,2,\dots$, последовательности $\mathfrak{z}^N$, $N=1,2,\dots$, и функция $\mathfrak{z}\in \textrm{Э}([0,T];V,H)$, такие, что 
\begin{gather}\label{sec:more.more.abstract:zNWconvergence}
\mathfrak{z}^{N_m}\to\mathfrak{z},\,\,\,m\to\infty,\,\,\,\mbox{слабо в $\mathcal{W}^1_2([0,T];V,H)$},\,\,\,
\lim\limits_{m\to\infty}\max\limits_{t\in[0,T]}\|\mathfrak{z}^{N_m}(t)-\mathfrak{z}(t)\|_{H}=0;\\
\notag \mathfrak{z}^{N_m}\to\mathfrak{z},\,\,\,m\to\infty,\,\,\,\mbox{$*$--слабо в $L_\infty([0,T],V)$};\,\,\,
\dot{\mathfrak{z}}^{N_m}\to\dot{\mathfrak{z}},\,\,\,m\to\infty,\,\,\,\mbox{$*$--слабо в $L_\infty([0,T],H)$};\\
\notag\mathfrak{z}^{N_m}\to\mathfrak{z},\,\,\,m\to\infty,\,\,\,\mbox{в $V^*$--топологии пространства $C_s([0,T],V)$};\\
\notag\lim\limits_{m\to\infty}\max\limits_{t\in[0,T]}\|\mathfrak{G}[\mathfrak{z}^{N_m}(t)]-\mathfrak{G}[\mathfrak{z}(t)]\|_{Z}=0,\,\,\,
\lim\limits_{m\to\infty}\max\limits_{t\in[0,T]}\|\mathfrak{C}[\mathfrak{z}^{N_m}(t)]-\mathfrak{C}[\mathfrak{z}(t)]\|_{Y}=0.
\end{gather}

Перепишем задачу Коши \eqref{hNCauchyProblem:equation}, \eqref{hNCauchyProblem:initial.conditions} для $N=N_m$ в виде
\begin{gather}\label{sec:more.more.abstract:hNk_Cauchy:rewritten}
\langle\ddot{\mathfrak{z}}^{N_m}(t),e_k\rangle+\langle\mathfrak{A}(t)\mathfrak{z}^{N_m}(t)+\mathfrak{B}(t)\mathfrak{z}^{N_m}(t),e_k\rangle+
\langle\mathfrak{F}(t) \mathfrak{G}\mathfrak{z}^{N_m}(t), \mathfrak{G}e_k\rangle_Z=\langle f(t), e_k\rangle_H+
\langle\mathfrak{g}(t),\mathfrak{C}e_k\rangle,\\
\notag\mathfrak{z}^{N_m}(0)=\varphi^{N_m},\,\,\,\dot{\mathfrak{z}}^{N_m}(0)=\psi^{N_m},\,\,\,k=\overline{1,N_m},\,\,\,t\in[0,T].
\end{gather}
Пусть $\eta\in\hat{\textrm{Э}}([0,T]; V,H)$ --- произвольна. Тогда $\eta\in\mathcal{W}^1_2([0,T];V,H)$ и $\eta(T)=0$. Согласно лемме~\ref{approx:w.abstract}, существует последовательность функций $\eta^N\in\mathfrak{M}^N_T$, $N=1,2,\dots$, такая, что
$\lim\limits_{N\to\infty} \|\eta^N-\eta\|_{\mathcal{W}^1_2([0,T];V,H)}=0$. Следовательно,
\begin{gather}\label{sec:more.more.abstract:etaNm_convergence}
\lim\limits_{m\to\infty}\|\eta^{N_m}-\eta\|_{\mathcal{W}^1_2([0,T];V,H)}=0.
\end{gather}
Из (\ref{sec:more.more.abstract:hNk_Cauchy:rewritten}) выводим, что
\begin{gather*}
\int\limits_0^T[\langle\ddot{\mathfrak{z}}^{N_m}(t),\eta^{N_m}(t)\rangle+\langle\mathfrak{A}(t)\mathfrak{z}^{N_m}(t),\eta^{N_m}(t)\rangle+
\langle\mathfrak{B}(t)\mathfrak{z}^{N_m}(t),\eta^{N_m}(t)\rangle+\langle\mathfrak{F}(t) \mathfrak{G}\mathfrak{z}^{N_m}(t), \mathfrak{G}\eta^{N_m}(t)\rangle_Z]dt=\\
=\int\limits_0^T\langle f(t),\eta^{N_m}(t)\rangle_Hdt+\int\limits_0^T\langle\mathfrak{g}(t),\mathfrak{C}\eta^{N_m}(t)\rangle dt,\,\,\,
\mathfrak{z}^{N_m}(0)=\varphi^{N_m},\,\,\,\dot{\mathfrak{z}}^{N_m}(0)=\psi^{N_m}.
\end{gather*}
Интегрируя по частям, заключаем, что
\begin{gather}\label{sec:more.more.abstract:zNm_identity}
\int\limits_0^T[-\langle\dot{\mathfrak{z}}^{N_m}(t),\dot{\eta}^{N_m}(t)\rangle+\langle\mathfrak{A}(t)\mathfrak{z}^{N_m}(t),\eta^{N_m}(t)\rangle+
\langle\mathfrak{B}(t)\mathfrak{z}^{N_m}(t),\eta^{N_m}(t)\rangle+\langle\mathfrak{F}(t) \mathfrak{G}\mathfrak{z}^{N_m}(t), \mathfrak{G}\eta^{N_m}(t)\rangle_Z]dt=\\
\notag=\langle\psi^{N_m},\eta^{N_m}(0)\rangle+\int\limits_0^T\langle f(t),\eta^{N_m}(t)\rangle_Hdt+\int\limits_0^T\langle\mathfrak{g}(t),\mathfrak{C}\eta^{N_m}(t)\rangle dt,\,\,\,
\mathfrak{z}^{N_m}(0)=\varphi^{N_m}.
\end{gather}
Переходя в данном соотношении к пределу при $m\to\infty$, на основании (\ref{sec:more.more.abstract:zNWconvergence}) и (\ref{sec:more.more.abstract:etaNm_convergence}) заключаем, что
\begin{gather*}
\int\limits_0^T[-\langle\dot{\mathfrak{z}}(t),\dot{\eta}(t)\rangle+\langle\mathfrak{A}(t)\mathfrak{z}(t),\eta(t)\rangle+\langle\mathfrak{B}(t)\mathfrak{z}(t),\eta(t)\rangle+\langle\mathfrak{F}(t) \mathfrak{G}\mathfrak{z}(t), \mathfrak{G}\eta(t)\rangle_Z]dt=\langle\psi,\eta(0)\rangle+\int\limits_0^T\langle f(t),\eta(t)\rangle_Hdt+\\
+\int\limits_0^T\langle\mathfrak{g}(t),\mathfrak{C}\eta(t)\rangle dt,\,\,\,\forall\,\eta\in\hat{\textrm{Э}}([0,T]; V,H);\,\,\,\mathfrak{z}(0)=\varphi.
\end{gather*}
Это означает, что $\mathfrak{z}\in{\textrm{Э}}([0,T]; V,H)$ --- решение задачи Коши (\ref{abstrChauchyprobeq:nonaut}), (\ref{abstrChauchyprobinitcond:nonaut}) в смысле определения \ref{SolutionDef1:nonaut}.

Итак, существование решения задачи Коши (\ref{abstrChauchyprobeq:nonaut}), (\ref{abstrChauchyprobinitcond:nonaut}) в смысле определения \ref{SolutionDef1:nonaut} --- доказано.

3) Докажем априорную оценку \eqref{abstrChauchyprobeq:nonaut:a_priori_solution_estimate}.  В самом деле, на основании~\eqref{sec:semilinear:problem:third2[n>1]:phipsi1}, для каждого $\varepsilon>0$ найдётся номер $m_0(\varepsilon)\geqslant1$, такой, что
при всех $m\geqslant m_0(\varepsilon)$ справедливо неравенство
\begin{gather}\label{sec:more.more.abstract:phipsi_inequality}
\sqrt{\|\varphi^{N_{m}}\|^2_V+\|\psi^{N_{m}}\|^2_H}\leqslant \sqrt{\|\varphi\|^2_V+\|\psi\|^2_H}+\varepsilon.
\end{gather}
Далее, из \eqref{sec:more.more.abstract:zNnorm_estimate} следует, что
\begin{gather*}
\|\mathfrak{z}^{N_m}\|_{\textrm{Э}([0,T]; V,H)}\leqslant2\rho_8\left[\sqrt{\|\varphi\|^2_V+\|\psi\|^2_H}+\|\mathfrak{g}\|^{(1)}_{1,[0,T],Y^*}+\|f\|_{1,[0,T],H}+\varepsilon\right],\,\,\,\forall\,m\geqslant m_0(\varepsilon).
\end{gather*}
Из  теоремы \ref{V1120QTcompactness:abstract} и соотношений \eqref{sec:more.more.abstract:zNWconvergence} следует, что
\begin{gather*}
\|\mathfrak{z}\|_{\textrm{Э}([0,T]; V,H)}\leqslant2\rho_8\left[\sqrt{\|\varphi\|^2_V+\|\psi\|^2_H}+\|\mathfrak{g}\|^{(1)}_{1,[0,T],Y^*}+\|f\|_{1,[0,T],H}+\varepsilon\right].
\end{gather*}
Устремляя затем $\varepsilon$ к нулю, получаем оценку~\eqref{abstrChauchyprobeq:nonaut:a_priori_solution_estimate} с $\varkappa_1=2\rho_8$. Теорема полностью доказана.
\end{Proof}

\begin{Theorem}
Пусть $\mathfrak{z}$ --- решение задачи Коши (\ref{abstrChauchyprobeq:nonaut}), (\ref{abstrChauchyprobinitcond:nonaut}) в смысле определения \ref{SolutionDef1:nonaut}. Тогда $\dot{\mathfrak{z}}\in C_s([0,T],H)$.
\end{Theorem}
\begin{Proof}
Пусть $\mathfrak{z}$ --- решение задачи Коши (\ref{abstrChauchyprobeq:nonaut}), (\ref{abstrChauchyprobinitcond:nonaut}) в смысле определения \ref{SolutionDef1:nonaut}. Тогда, на основании леммы~\ref{SolutionDefsEquivalence:nonaut}, $\ddot{\mathfrak{z}}\in L_1([0,T],V^*)$. Поэтому $\dot{\mathfrak{z}}\in W^1_1([0,T],V^*)$. На основании теоремы \ref{W1p([0,T],X)_embedding} отсюда следует, что $z_t\in C_s([0,T],V^*)$. Пользуясь теперь включением $z_t\in L_\infty([0,T],H)$ и леммой \ref{LionsMajenesXY}, получаем требуемое утверждение.
\end{Proof}


\begin{Theorem}\label{unique_existence_theorem:nonaut..utochnenie}
Пусть $\mathfrak{z}$ --- решение задачи Коши (\ref{abstrChauchyprobeq:nonaut}), (\ref{abstrChauchyprobinitcond:nonaut}) в смысле определения \ref{SolutionDef1:nonaut}. Тогда $\mathfrak{z}\in\textrm{Є}([0,T];V,H)$, причём найдётся постоянная $\varkappa_2>0$, зависящая лишь от $T$, $c_1$, $c_2$, $c_3>0$ и от $\|\mathfrak{C}\|_{V\to Y}$,  $\|\mathfrak{G}\|_{V\to Z}$, такая, что
\begin{gather}\label{abstrChauchyprobeq:nonaut:a_priori_solution_estimate1}
\|\mathfrak{z}\|_{{\textrm{Є}}([0,T];V,H)}\leqslant \varkappa_2[\sqrt{\|\varphi\|^2_V+\|\psi\|_H^2}+ \|f\|_{1,[0,T],H}+\|\mathfrak{g}\|^{(1)}_{1,[0,T],Y^*}].
\end{gather}
\end{Theorem}
\begin{Proof}
Включение $\mathfrak{z}\in\textrm{Є}([0,T];V,H)$ вытекает из предыдущей теоремы и определения класса $\textrm{Є}([0,T];V,H)$. Поэтому достаточно
доказать лишь оценку (\ref{abstrChauchyprobeq:nonaut:a_priori_solution_estimate1}).

В самом деле, из оценки (\ref{abstrChauchyprobeq:nonaut:a_priori_solution_estimate}) следует, что
\begin{gather*}
\sup\limits_{t\in[0,T]}\|\mathfrak{z}(t)\|_V+\vraisup\limits_{t\in[0,T]}\|\dot{\mathfrak{z}}(t)\|_H\leqslant\varkappa_1[\sqrt{\|\varphi\|^2_V+\|\psi\|_H^2}+ \|f\|_{1,[0,T],H}+\|\mathfrak{g}\|^{(1)}_{1,[0,T],Y^*}].
\end{gather*}
На основании леммы \ref{Cs[0,T]X:sup::vraisup} и включения $\mathfrak{z}\in\textrm{Є}([0,T];V,H)$ из данного неравенства выводим, что
\begin{gather*}
\sup\limits_{t\in[0,T]}\|\mathfrak{z}(t)\|_V+\sup\limits_{t\in[0,T]}\|\dot{\mathfrak{z}}(t)\|_H\leqslant\varkappa_1[\sqrt{\|\varphi\|^2_V+\|\psi\|_H^2}+ \|f\|_{1,[0,T],H}+\|\mathfrak{g}\|^{(1)}_{1,[0,T],Y^*}].
\end{gather*}
Как следствие,
\begin{gather*}
\|\mathfrak{z}\|_{{\textrm{Є}}([0,T];V,H)}\equiv\sup\limits_{t\in[0,T]}\sqrt{\|\mathfrak{z}(t)\|_V^2+\|\dot{\mathfrak{z}}(t)\|_H^2}\leqslant\sup\limits_{t\in[0,T]}\|\mathfrak{z}(t)\|_V+\sup\limits_{t\in[0,T]}\|\dot{\mathfrak{z}}(t)\|_H\leqslant\\
\leqslant\varkappa_1[\sqrt{\|\varphi\|^2_V+\|\psi\|_H^2}+ \|f\|_{1,[0,T],H}+\|\mathfrak{g}\|^{(1)}_{1,[0,T],Y^*}].
\end{gather*}
Итак, оценка (\ref{abstrChauchyprobeq:nonaut:a_priori_solution_estimate1}) доказана, причём можно взять  $\varkappa_2=\varkappa_1$.
\end{Proof}

	    \subsection{Параметрическая задача Коши для однородного уравнения}
При каждом $\psi\in H$ определим функцию $\mathfrak{y}[\psi](t,\tau)$, $t\in[0,T]$, $\tau\in[0,T]$, при $t\in[0,\tau]$ как решение задачи Коши
\begin{gather}\label{abstrChauchyprobeq:nonaut:parametric:homogenious.eq!t.less.than.tau}
{\mathfrak{y}}_{tt}+\mathfrak{A}(t){\mathfrak{y}}+\mathfrak{B}(t){\mathfrak{y}} + \mathfrak{G}^*\mathfrak{F}(t)\mathfrak{G}{\mathfrak{y}} = 0,\,\,\,t\in[0,\tau],\\
\label{abstrChauchyprobinitcond:nonaut:parametric:homogenious.eq!t.less.than.tau}
{\mathfrak{y}}|_{t=\tau}=0,\,\,\,{\mathfrak{y}}_t|_{t=\tau}=\psi,
\end{gather}
а при $t\in[0,T]$ --- как решение задачи Коши
\begin{gather}\label{abstrChauchyprobeq:nonaut:parametric:homogenious.eq!t.greater.than.tau}
{\mathfrak{y}}_{tt}+\mathfrak{A}(t){\mathfrak{y}}+\mathfrak{B}(t){\mathfrak{y}} + \mathfrak{G}^*\mathfrak{F}(t)\mathfrak{G}{\mathfrak{y}} = 0,\,\,\,t\in[\tau,T],\\
\label{abstrChauchyprobinitcond:nonaut:parametric:homogenious.eq!t.greater.than.tau}
{\mathfrak{y}}|_{t=\tau}=0,\,\,\,{\mathfrak{y}}_t|_{t=\tau}=\psi.
\end{gather}
	    
Покажем, что справедлива
\begin{Theorem}\label{abstrChauchyprobeq:nonaut:parametric:homogenious.eq!theorem.y.cont}
Справедливо включение $\mathfrak{y}[\psi]\in C([0,T],{\textrm{Є}}([0,T];V,H))$, понимаемое в том смысле, что функция
\begin{gather*}
[0,T]\ni \tau\mapsto\mathfrak{y}[\psi](\cdot,\tau)\in{\textrm{Є}}([0,T];V,H)
\end{gather*}
непрерывна на отрезке $[0,T]$ в смысле нормы пространства ${\textrm{Є}}([0,T];V,H)$. При этом найдётся постоянная $\varkappa_3>0$, зависящая лишь от $T$, $c_1$, $c_2$, $c_3>0$ и от $\|\mathfrak{G}\|_{V\to Z}$, такая, что
\begin{gather}\label{abstrChauchyprobeq:nonaut:parametric:homogenious.eq!theorem.y.cont!apriori.estimate}
\max\limits_{\tau\in[0,T]}\|\mathfrak{y}[\psi](\cdot,\tau)\|_{{\textrm{Є}}([0,T];V,H)}\leqslant\varkappa_3\|\psi\|_H.
\end{gather}
\end{Theorem}
\begin{Proof}
1) Прежде всего отметим, что, в силу теоремы~\ref{unique_existence_theorem:abstrChauchyprobeq:nonaut}, функция $\mathfrak{y}[\psi](\cdot,\tau)$ 
однозначно определяется на $[0,T]$, $\mathfrak{y}[\psi](\cdot,\tau)\in{\textrm{Э}}([0,T];V,H)$ при всех $\tau\in[0,T]$, и найдётся константа 
$\varkappa_1>0$,, зависящая лишь от $T$, $c_1$, $c_2$, $c_3>0$ и от $\|\mathfrak{C}\|_{V\to Y}$,  $\|\mathfrak{G}\|_{V\to Z}$, такая, что
\begin{gather*}
\sup\limits_{t\in[0,\tau]}\|\mathfrak{y}[\psi](t,\tau)\|_{V}+\vraisup\limits_{t\in[0,\tau]}\|\mathfrak{y}_t[\psi](t,\tau)\|_{H}\leqslant\varkappa_1\|\psi\|_H;\\
\sup\limits_{t\in[\tau,T]}\|\mathfrak{y}[\psi](t,\tau)\|_{V}+\vraisup\limits_{t\in[\tau,T]}\|\mathfrak{y}_t[\psi](t,\tau)\|_{H}\leqslant\varkappa_1\|\psi\|_H.
\end{gather*}
Следовательно,
\begin{gather}\label{abstrChauchyprobeq:nonaut:parametric:homogenious.eq!theorem.y.cont:auxiliary_a_priori_estimate}
\sup\limits_{\tau\in[0,T]}\|\mathfrak{y}[\psi](\cdot,\tau)\|_{\textrm{Э}([0,T];V,H)}\leqslant2\varkappa_1\|\psi\|_H.
\end{gather}
Заметим также, что $\mathfrak{y}[\psi]$ линейно зависит от $\psi\in H$.

2) Докажем теперь включение $\mathfrak{y}[\psi]\in C([0,T],\textrm{Э}([0,T];V,H))$. Для этого воспользуемся методом Галёркина. Пусть
\begin{gather*}
\psi^N\equiv\sum\limits_{m=1}^N\psi_me_m,\,\,\psi_j\equiv\langle\psi,e_m\rangle_H,\,\,\,j,\,N=1,2,\dots.
\end{gather*}
Тогда
\begin{gather}\label{psiN:nonaut:parametric:convergence}
\lim\limits_{N\to\infty}\|\psi^N-\psi\|_H=0.
\end{gather}
Будем искать приближение $\mathfrak{y}^N[\psi]$ к функции $\mathfrak{y}[\psi]$ в виде
\begin{gather*}
\mathfrak{y}^N[\psi](t,\tau)\equiv\sum\limits_{k=1}^Nh^N_k(t,\tau)e_k,
\end{gather*}
где набор функций $h^N_k$, $k=\overline{1,N}$, является решением задачи Коши
\begin{gather}\label{hNkttau_Cauchy:nonaut:parametric:convergence}
h^N_{ktt}(t,\tau)+\sum\limits_{m=1}^Nq_{km}(t)h^N_{m}(t,\tau)=0,\\
\notag h^N_k(t,\tau)|_{t=\tau}=0,\,\,\,h^N_{kt}(t,\tau)|_{t=\tau}=\psi_k,\,\,\,k=\overline{1,N},
\end{gather}
в которой
\begin{gather*}
q_{km}(t)\equiv\langle\mathfrak{A}(t)e_m,e_k\rangle+\langle\mathfrak{B}(t)e_m,e_k\rangle + \langle\mathfrak{F}(t)\mathfrak{G}e_m, \mathfrak{G}e_k\rangle_Z.
\end{gather*}
Согласно лемме \ref{sec:aux_results:ODE2:yCauchy:unique_existence}, существует единственный набор функций $h_k^N$, $k=\overline{1,N}$, непрерывных на $\Gamma$ и имеющих на $\Gamma$
непрерывные производные $h^N_{kt}$, $h^N_{k\tau}$, $h^N_{ktt}$, $h^N_{kt\tau}$, $h^N_{k\tau t}$, $k=\overline{1,N}$, являющийся решением задачи Коши (\ref{hNkttau_Cauchy:nonaut:parametric:convergence}).
Кроме того, производные $h^N_{k\tau tt}$, $k=\overline{1,N}$, также существуют и непрерывны на $\Gamma$, а набор функций $h_{k\tau}^N$, $k=\overline{1,N}$, является решением задачи Коши
\begin{gather}\label{hN_tau_kttau_Cauchy:nonaut:parametric:convergence}
h^N_{k\tau tt}(t,\tau)+\sum\limits_{m=1}^Nq_{km}(t)h^N_{m\tau}(t,\tau)=0,\\
\notag h^N_{k\tau}(t,\tau)|_{t=\tau}=-\psi_k,\,\,\,h^N_{k\tau t}(t,\tau)|_{t=\tau}=0,\,\,\,k=\overline{1,N}.
\end{gather}
Как следствие, для функций $\mathfrak{y}^N[\psi]$ и $\xi^N[\psi]\equiv\mathfrak{y}^N_\tau[\psi]$ справедливы тождества
\begin{gather}
\label{thetaN_identity:nonaut:parametric}
\langle\mathfrak{y}^{N}_{tt}[\psi](t,\tau),e_k\rangle + \langle\mathfrak{A}(t)\mathfrak{y}^{N}[\psi](t,\tau)+\mathfrak{B}(t)\mathfrak{y}^{N}[\psi](t,\tau),e_k\rangle+\langle\mathfrak{F}(t) \mathfrak{G}\mathfrak{y}^{N}[\psi](t,\tau), \mathfrak{G}e_k\rangle_Z=0,\\
\notag\mathfrak{y}^{N}[\psi]|_{t=\tau}=0,\,\,\,\mathfrak{y}^{N}_t[\psi]|_{t=\tau}=\psi^{N},\,\,\,k=\overline{1,N},\,\,\,(t,\tau)\in\Gamma;\\
\label{xiN_identity:nonaut:parametric}
\langle\xi^{N}_{tt}[\psi](t,\tau),e_k\rangle + \langle\mathfrak{A}(t)\xi^N[\psi](t,\tau)+\mathfrak{B}(t)\xi^{N}[\psi](t,\tau),e_k\rangle+\langle\mathfrak{F}(t) \mathfrak{G}\xi^{N}[\psi](t,\tau), \mathfrak{G}e_k\rangle_Z=0,\\
\notag\xi^N[\psi]|_{t=\tau}=-\psi^N,\,\,\,\xi^N_t[\psi]|_{t=\tau}=0,\,\,\,k=\overline{1,N},\,\,\,(t,\tau)\in\Gamma.
\end{gather}
Рассуждая затем подобно тому, как это делалось при выводе оценки~\eqref{sec:more.more.abstract:zNnorm_estimate}, заключаем, что
\begin{gather*}
\|\mathfrak{y}^{N}[\psi](\cdot,\tau)\|_{\textrm{Э}([0,T]; V,H)}\leqslant C\|\psi\|_H,\,\,\,
\|\xi^N[\psi](\cdot,\tau)\|_{\textrm{Э}([0,T]; V,H)}\leqslant C\|\psi\|_H,
\end{gather*}
где постоянная $C>0$ определяется лишь числами $T$, $c_1$, $c_2$, $c_3>0$ и $\|\mathfrak{C}\|_{V\to Y}$,  $\|\mathfrak{G}\|_{V\to Z}$.

Следовательно, при всех $\tau_1$, $\tau_2\in[0,T]$
\begin{gather*}
\|\mathfrak{y}^N[\psi](\cdot,\tau_1)-\mathfrak{y}^N[\psi](\cdot,\tau_2)\|_{\textrm{Э}([0,T]; V,H)}= {\biggl\|}\sum\limits_{m=1}^Nh^N_m(\cdot,\tau_1)e_m-\sum\limits_{m=1}^Nh^N_m(\cdot,\tau_2)e_m{\biggr\|}_{\textrm{Э}([0,T]; V,H)}=\\
=\biggl\|\sum\limits_{m=1}^N[h^N_m(\cdot,\tau_1)-h^N_m(\cdot,\tau_2)]e_m{\biggr\|}_{\textrm{Э}([0,T]; V,H)} =
\biggl\|\sum\limits_{m=1}^N\int\limits_{\tau_2}^{\tau_1}h^N_{m\tau}(\cdot,\tau)e_md\tau{\biggr\|}_{\textrm{Э}([0,T]; V,H)}=\\
=\biggl\|\int\limits_{\tau_2}^{\tau_1}\sum\limits_{m=1}^Nh^N_{m\tau}(\cdot,\tau)e_md\tau{\biggr\|}_{\textrm{Э}([0,T]; V,H)}=
{\biggl\|}\int\limits_{\tau_2}^{\tau_1}\xi^N[\psi](\cdot,\tau)d\tau{\biggr\|}_{\textrm{Э}([0,T]; V,H)}\leqslant\\
\leqslant\left|\int\limits_{\tau_2}^{\tau_1}{\|}\xi^N[\psi](\cdot,\tau){\|}_{\textrm{Э}([0,T]; V,H)}d\tau\right|\leqslant C|\tau_1-\tau_2|\|\psi\|_H.
\end{gather*}
Таким образом, при всех $\tau$, $\tau_1$, $\tau_2\in[0,T]$,
\begin{gather}
\label{thetaN_energetic_estimate:nonaut:parametric}
\|\mathfrak{y}^{N}[\psi](\cdot,\tau)\|_{\textrm{Э}([0,T]; V,H)}\leqslant C\|\psi\|_H,\\
\label{thetaN_difference_energetic_estimate:nonaut:parametric}
\|\mathfrak{y}^N[\psi](\cdot,\tau_1)-\mathfrak{y}^N[\psi](\cdot,\tau_2)\|_{\textrm{Э}([0,T]; V,H)}\leqslant C|\tau_1-\tau_2|\|\psi\|_H.
\end{gather}
Рассуждая как при получении тождества~\eqref{sec:more.more.abstract:zNm_identity}, заключаем, что
\begin{gather}
\label{thetaN_identity1:nonaut:parametric}
\int\limits_0^\tau[-\langle\mathfrak{y}^{N}_t[\psi](t,\tau),{\eta}^{N}(t)\rangle+\langle\mathfrak{A}(t)\mathfrak{y}^{N}[\psi](t,\tau),\eta^{N}(t)\rangle+
\langle\mathfrak{B}(t)\mathfrak{y}^{N}[\psi](t,\tau),\eta^{N}(t)\rangle+\\
\notag+\langle\mathfrak{F}(t) \mathfrak{G}\mathfrak{y}^{N}[\psi](t,\tau), \mathfrak{G}\eta^{N}(t)\rangle_Z]dt=-\langle\psi^{N},\eta^{N}(\tau)\rangle,\,\,\,\forall\,\eta^N\in\mathfrak{K}^N_{0}[0,\tau];\,\,\,
\mathfrak{y}^{N}[\psi]|_{t=\tau}=0;\\
%%%%%%%%%%%%%%%%%%%%%%%%%%%%%%%%%%%%%%%%%%%%%%%%%%%%%
\label{thetaN_identity2:nonaut:parametric}
\int\limits_\tau^T[-\langle\mathfrak{y}^{N}_t[\psi](t,\tau),{\eta}^{N}(t)\rangle+\langle\mathfrak{A}(t)\mathfrak{y}^{N}[\psi](t,\tau),\eta^{N}(t)\rangle+
\langle\mathfrak{B}(t)\mathfrak{y}^{N}[\psi](t,\tau),\eta^{N}(t)\rangle+\\
\notag+\langle\mathfrak{F}(t) \mathfrak{G}\mathfrak{y}^{N}[\psi](t,\tau), \mathfrak{G}\eta^{N}(t)\rangle_Z]dt=\langle\psi^{N},\eta^{N}(\tau)\rangle,\,\,\,\forall\,\eta^N\in\mathfrak{K}^T_{T}[\tau,T];\,\,\,
\mathfrak{y}^{N}[\psi]|_{t=\tau}=0.
\end{gather}
Произвольно выберем и зафиксируем $\tau_1$, $\tau_2\in[0,T]$. Ввиду неравенства~\eqref{thetaN_energetic_estimate:nonaut:parametric} и  теоремы~\ref{V1120QTcompactness:abstract} заключаем, что найдутся подпоследовательность ${N_m}$, $m=1,2,\dots$, последовательности $N=1,2,\dots$, и функции $\mathfrak{y}^-(\cdot,\tau_i)\in \textrm{Э}([0,\tau_i];V,H)$, $\mathfrak{y}^+(\cdot,\tau_i)\in \textrm{Э}([\tau_i,T];V,H)$,  $\mathfrak{y}^*(\cdot,\tau_i)\in \textrm{Э}([0,T];V,H)$, $i=1,2$, такие, что
\begin{gather}
\label{thetaN_convergence:nonaut:parametric}
\lim\limits_{m\to\infty}\max\limits_{t\in[0,\tau_i]}\|\mathfrak{y}^{N_m}[\psi](t,\tau_i)-\mathfrak{y}^-(t,\tau_i)\|_{H}=0,\,\,\,
\lim\limits_{m\to\infty}\max\limits_{t\in[\tau_i,T]}\|\mathfrak{y}^{N_m}[\psi](t,\tau_i)-\mathfrak{y}^+(t,\tau_i)\|_{H}=0, \\
\notag\mathfrak{y}^{N_m}[\psi](\cdot,\tau_i)\to\mathfrak{y}^-(\cdot,\tau_i),\,\,\,m\to\infty,\,\,\,\mbox{слабо в $\mathcal{W}^1_2([0,\tau_i];V,H)$},\\
\notag\mathfrak{y}^{N_m}[\psi](\cdot,\tau_i)\to\mathfrak{y}^+(\cdot,\tau_i),\,\,\,m\to\infty,\,\,\,\mbox{слабо в $\mathcal{W}^1_2([\tau_i,T];V,H)$},\\
\notag\mathfrak{y}^{N_m}[\psi](\cdot,\tau_i)\to\mathfrak{y}^-(\cdot,\tau_i),\,\,\,m\to\infty,\,\,\,\mbox{$*$--слабо в $L_\infty([0,\tau_i],V)$},\\
\notag\mathfrak{y}^{N_m}[\psi](\cdot,\tau_i)\to\mathfrak{y}^+(\cdot,\tau_i),\,\,\,m\to\infty,\,\,\,\mbox{$*$--слабо в $L_\infty([\tau_i,T],V)$},\\
\notag\mathfrak{y}^{N_m}_t[\psi](\cdot,\tau_i)\to\mathfrak{y}^-_t(\cdot,\tau_i),\,\,\,m\to\infty,\,\,\,\mbox{$*$--слабо в $L_\infty([0,\tau_i],H)$},\\
\notag\mathfrak{y}^{N_m}_t[\psi](\cdot,\tau_i)\to\mathfrak{y}^+_t(\cdot,\tau_i),\,\,\,m\to\infty,\,\,\,\mbox{$*$--слабо в $L_\infty([\tau_i,T],H)$},\\
\notag\mathfrak{y}^{N_m}[\psi](\cdot,\tau_i)\to\mathfrak{y}^*(\cdot,\tau_i),\,\,\,m\to\infty,\,\,\,\mbox{слабо в $\mathcal{W}^1_2([0,T];V,H)$}, \\
\notag\lim\limits_{m\to\infty}\max\limits_{t\in[0,\tau_i]}
\|\mathfrak{G}[\mathfrak{y}^{N_m}[\psi](\cdot,\tau_i)]-\mathfrak{G}[\mathfrak{y}^-(t,\tau_i)]\|_{Z}=0,\,\,\,
\notag\lim\limits_{m\to\infty}\max\limits_{t\in[\tau_i,T]}
\|\mathfrak{G}[\mathfrak{y}^{N_m}[\psi](\cdot,\tau_i)]-\mathfrak{G}[\mathfrak{y}^+(t,\tau_i)]\|_{Z}=0,\,\,\,\notag i=1,\,2.
\end{gather}
Переходя к пределу при $m\to\infty$ в (\ref{thetaN_identity1:nonaut:parametric}) и (\ref{thetaN_identity2:nonaut:parametric}) с $N=N_m$, $\tau=\tau_i$, $i=1,2$, получим, что $\mathfrak{y}^*(\cdot,\tau_i)\equiv\mathfrak{y}[\psi](\cdot,\tau_i)$, 
$\mathfrak{y}^-(\cdot,\tau_i)\equiv\mathfrak{y}[\psi](\cdot,\tau_i)|_{[0,\tau_i]}$, 
$\mathfrak{y}^+(\cdot,\tau_i)\equiv\mathfrak{y}[\psi](\cdot,\tau_i)|_{[\tau_i,T]}$, $i=1,\,2$.

Используя предельные соотношения~(\ref{thetaN_convergence:nonaut:parametric}), теорему~\ref{V1120QTcompactness:abstract}, и неравенство~\eqref{thetaN_difference_energetic_estimate:nonaut:parametric},  заключаем, что
\begin{gather}
\label{theta_difference_energetic_estimate:nonaut:parametric}
\|\mathfrak{y}[\psi](\cdot,\tau_1)-\mathfrak{y}[\psi](\cdot,\tau_2)\|_{\textrm{Э}([0,T]; V,H)}\leqslant C|\tau_1-\tau_2|\|\psi\|_H\,\,\,\forall\,\tau_1,\,\,\,\tau_2\in[0,T].
\end{gather}

Таким образом, включение $\mathfrak{y}[\psi]\in C([0,T],\textrm{Э}([0,T];V,H))$ доказано.

3) Докажем включение $\mathfrak{y}[\psi]\in C([0,T],\textrm{Є}([0,T];V,H))$. В самом деле, на основании теоремы~\ref{unique_existence_theorem:nonaut..utochnenie}, при всех $\tau\in[0,T]$ справедливо включение $\mathfrak{y}[\psi](\cdot,\tau)\in \textrm{Є}([0,T];V,H))$. Кроме того, из доказанного во второй части данного доказательства включения $\mathfrak{y}[\psi]\in C([0,T],\textrm{Э}([0,T];V,H))$ следует, что для всех $\tau$ и $\tau'\in[0,T]$
\begin{gather*}
\|\mathfrak{y}[\psi](\cdot,\tau')-\mathfrak{y}[\psi](\cdot,\tau)\|_{\textrm{Є}([0,T];V,H)}=\sup\limits_{t\in[0,T]}\biggl[
\|\mathfrak{y}[\psi](t,\tau')-\mathfrak{y}[\psi](t,\tau)\|_V^2+\|\mathfrak{y}_t[\psi](t,\tau')-\mathfrak{y}_t[\psi](t,\tau)\|_H^2\biggr]^{\frac12}\leqslant\\
\leqslant\|\mathfrak{y}[\psi](\cdot,\tau')-\mathfrak{y}[\psi](\cdot,\tau)\|_{\textrm{Э}([0,T];V,H)}.
\end{gather*}
Таким образом,
\begin{gather}\label{EErevnorms:nonaut:parametric}
\|\mathfrak{y}[\psi](\cdot,\tau')-\mathfrak{y}[\psi](\cdot,\tau)\|_{\textrm{Є}([0,T];V,H)}
\leqslant\|\mathfrak{y}[\psi](\cdot,\tau')-\mathfrak{y}[\psi](\cdot,\tau)\|_{\textrm{Э}([0,T];V,H)}.
\end{gather}
Из данного неравенства и ранее доказанного включения $\mathfrak{y}[\psi]\in C([0,T],\textrm{Э}([0,T];V,H))$ и следует включение $\mathfrak{y}[\psi]\in C([0,T],\textrm{Є}([0,T];V,H))$.

4) Априорная оценка~\eqref{abstrChauchyprobeq:nonaut:parametric:homogenious.eq!theorem.y.cont!apriori.estimate} является следствием неравенств 
\eqref{abstrChauchyprobeq:nonaut:parametric:homogenious.eq!theorem.y.cont:auxiliary_a_priori_estimate} и \eqref{EErevnorms:nonaut:parametric}. Теорема полностью доказана.
\end{Proof}

\begin{Theorem}\label{abstrChauchyprobeq:nonaut:parametric:homogenious.eq!theorem.y.representation}
Найдётся функция $\Psi\in\mathfrak{S}(\Gamma;V,H)$, такая, что
\begin{gather*}
\mathfrak{y}[h](t,\tau)=\Psi(t,\tau)h\,\,\,\forall\,(t,\tau)\in\Gamma,\,\,\,h\in H.
\end{gather*}
\end{Theorem}
\begin{Proof}
В самом деле, на основании теоремы~\ref{abstrChauchyprobeq:nonaut:parametric:homogenious.eq!theorem.y.cont}, отображение
\begin{gather*}
H\ni h\mapsto \mathfrak{y}[h]\in C([0,T],{\textrm{Є}}([0,T];V,H))
\end{gather*}
представляет собой линейный непрерывный оператор. Поэтому, согласно теореме~\ref{L.H->C([0,T],En2([0,T];V,H))!theorem}, найдётся функция $\Psi\in\mathfrak{S}(\Gamma;V,H)$, такая, что
\begin{gather*}
\mathfrak{y}[h](t,\tau)=\Psi(t,\tau)h\,\,\,\forall\,(t,\tau)\in\Gamma,\,\,\,h\in H.
\end{gather*}
Теорема доказана.
\end{Proof}


	    \subsection{Представление решения линейного уравнения} 
В данном разделе мы получим представление решения задачи Коши (\ref{abstrChauchyprobeq:nonaut}), (\ref{abstrChauchyprobinitcond:nonaut}). 

Прежде всего определим гильбертово пространство $\mathfrak{V}$ как множество пар $\mathfrak{v}\equiv(v,h)\in V\times H$, наделённое скалярным произведением 
$$
\langle(v_1,h_1),(v_2,h_2)\rangle_{\mathfrak{V}}\equiv\langle v_1,v_2\rangle_V+\langle h_1,h_2\rangle_H.
$$
Далее, введём банахово пространство $\mathfrak{W}$ как множество четвёрок $\mathfrak{w}\equiv(v,h,\mathfrak{f},{g})\in V\times H\times  L_1([0,T],H)\times W^1_1([0,T], Y^*)$,
 наделённое нормой
$$
\|\mathfrak{w}\|_{\mathfrak{W}}\equiv\|(v,h)\|_{\mathfrak{V}}+\|\mathfrak{f}\|_{1,[0,T],H}+\|{g}\|^{(1)}_{1,[0,T],Y^*}.
$$
Наконец, через $\Lambda$ обозначим оператор, ставящий в соответствие каждой четвёрке $\mathfrak{w}\equiv(v,h,\mathfrak{f},{g})\in\mathfrak{W}$ решение задачи Коши (\ref{abstrChauchyprobeq:nonaut}), (\ref{abstrChauchyprobinitcond:nonaut}) с $\varphi=v$, $\psi=h$, $f=\mathfrak{f}$, $\mathfrak{g}=g$.

Из теоремы \ref{unique_existence_theorem:nonaut..utochnenie} следует, что этот оператор принимает значения в пространстве $\textrm{Є}([0,T];V,H)$.
Кроме того, нетрудно видеть, что этот оператор линеен. Из теоремы \ref{unique_existence_theorem:nonaut..utochnenie} также следует, что оператор $\Lambda$ --- ограничен. 

Таким образом, решение задачи Коши (\ref{abstrChauchyprobeq:nonaut}), (\ref{abstrChauchyprobinitcond:nonaut}) можно записать в виде 
\begin{gather*}
\mathfrak{z}(t)=\Lambda[\varphi,\psi,f,\mathfrak{g}](t),\,\,\,t\in[0,T].
\end{gather*}

\begin{Theorem}\label{abstrChauchyprob:nonaut:solution.representation}
Найдутся оператор $\Theta\in\mathcal{L}(W^1_1([0,T],Y^*),\textrm{Є}([0,T];V,H))$ и функции $\Phi\in\mathfrak{R}([0,T]; V,H)$, $\Psi\in\mathfrak{S}(\Gamma; V,H)$, такие, что при всех $v\in V$, $h\in H$
\begin{gather}\label{Lambda.repres}
\Lambda[\varphi,\psi,f,\mathfrak{g}](t)=\Phi(t)\varphi+\Psi(t,0)\psi+\int\limits_0^t\Psi(t,\xi)f(\xi)d\xi+\Theta[\mathfrak{g}](t)\,\,\,\forall\,t\in[0,T].
\end{gather}
\end{Theorem}	
\begin{Proof}
Введём операторы $\Lambda_1:V\to \textrm{Є}([0,T];V,H)$, $\Lambda_2:H\to\textrm{Є}([0,T];V,H)$, $\Lambda_3:L_1([0,T],H)\to \textrm{Є}([0,T];V,H)$, $\Lambda_4:W^1_1([0,T], Y^*)\to\textrm{Є}([0,T];V,H)$  по формулам
\begin{gather*}
\Lambda_1[v]\equiv\Lambda[v,0,0,0],\,\,\,\Lambda_2[h]\equiv\Lambda[0,h,0,0]\,\,\,\forall\,(v,h)\in V\times H;\\
\Lambda_3[\mathfrak{f}]\equiv\Lambda[0,0,\mathfrak{f},0],\,\,\,\Lambda_4[g]\equiv\Lambda[0,0,0,g]\,\,\,\forall\,\mathfrak{f}\in L_1([0,T],H)\,\,\,\forall\,g\in W^1_1([0,T], Y^*).
\end{gather*}
Тогда решение задачи Коши (\ref{abstrChauchyprobeq:nonaut}), (\ref{abstrChauchyprobinitcond:nonaut}) можно записать в виде
\begin{gather*}
\mathfrak{z}(t)=\Lambda_1[\varphi](t)+\Lambda_2[\psi](t)+\Lambda_3[f](t)+\Lambda_4[\mathfrak{g}](t),\,\,\,t\in[0,T].
\end{gather*}
Положив $\Theta\equiv\Lambda_4$, получим, что
\begin{gather*}
\mathfrak{z}(t)=\Lambda_1[\varphi](t)+\Lambda_2[\psi](t)+\Lambda_3[f](t)+\Theta[\mathfrak{g}](t),\,\,\,t\in[0,T].
\end{gather*}

Далее, на основании теоремы~\ref{L.V->En2([0,T];V,H)!theorem}, найдётся функция $\Phi\in\mathfrak{R}([0,T]; V,H)$, такая, что
\begin{gather*}
\Lambda_1[\varphi](t)\equiv\Phi(t)\varphi,\,\,\,t\in[0,T].
\end{gather*}
Как следствие,
\begin{gather}\label{z.repres!Phi.Theta}
\mathfrak{z}(t)=\Phi(t)\varphi+\Lambda_2[\psi](t)+\Lambda_3[f](t)+\Theta[\mathfrak{g}](t),\,\,\,t\in[0,T].
\end{gather}

Нетрудно видеть, что
\begin{gather}\label{Lambda.2.repres}
\Lambda_2[\psi](t)\equiv\mathfrak{y}[\psi](t,0).\,\,\,t\in[0,T].
\end{gather}

Покажем, что
\begin{gather}\label{Lambda.3.repres}
\Lambda_3[f](t)=\int\limits_0^t\mathfrak{e}[f](t,\tau)d\tau,\,\,\,t\in[0,T],
\end{gather}
где $\mathfrak{e}[f]$ ---  при $t\in[0,\tau]$ решение задачи Коши
\begin{gather}\label{abstrChauchyprobeq:nonaut:parametric:homogenious.eq.e!t.less.than.tau}
{\mathfrak{e}}_{tt}+\mathfrak{A}(t){\mathfrak{e}}+\mathfrak{B}(t){\mathfrak{e}} + \mathfrak{G}^*\mathfrak{F}(t)\mathfrak{G}{\mathfrak{e}} = 0,\,\,\,t\in[0,\tau],\\
\label{abstrChauchyprobinitcond:nonaut:parametric:homogenious.eq.e!t.less.than.tau}
{\mathfrak{e}}|_{t=\tau}=0,\,\,\,{\mathfrak{e}}_t|_{t=\tau}=f(\tau),
\end{gather}
а при $t\in[0,T]$ --- решение задачи Коши
\begin{gather}\label{abstrChauchyprobeq:nonaut:parametric:homogenious.eq.e!t.greater.than.tau}
{\mathfrak{e}}_{tt}+\mathfrak{A}(t){\mathfrak{e}}+\mathfrak{B}(t){\mathfrak{e}} + \mathfrak{G}^*\mathfrak{F}(t)\mathfrak{G}{\mathfrak{e}} = 0,\,\,\,t\in[\tau,T],\\
\label{abstrChauchyprobinitcond:nonaut:parametric:homogenious.eq.e!t.greater.than.tau}
{\mathfrak{e}}|_{t=\tau}=0,\,\,\,{\mathfrak{e}}_t|_{t=\tau}=f(\tau).
\end{gather}

В самом деле, введём функцию $\mathfrak{w}:[0,T]\to V$ равенством
\begin{gather*}
\mathfrak{w}(t)=\int\limits_0^t\mathfrak{e}[f](t,\tau)d\tau,\,\,\,t\in[0,T].
\end{gather*}
Дифференцируя эту функцию дважды, получим, что
\begin{gather*}
\dot{\mathfrak{w}}(t)=\mathfrak{e}[f](t,t)+\int\limits_0^t\mathfrak{e}_t[f](t,\tau)d\tau,\,\,\,
\notag\ddot{\mathfrak{w}}(t)=\mathfrak{e}_t[f](t,t)+\int\limits_0^t\mathfrak{e}_{tt}[f](t,\tau)d\tau,\,\,\,t\in[0,T].
\end{gather*}
Пользуясь затем определением функции $\mathfrak{e}[f]$, выводим, что
\begin{gather}\label{w.derivatives1}
\dot{\mathfrak{w}}(t)=\int\limits_0^t\mathfrak{e}_t[f](t,\tau)d\tau,\,\,\,t\in[0,T];\,\,\,\ddot{\mathfrak{w}}(t)=f(t)+\int\limits_0^t\mathfrak{e}_{tt}[f](t,\tau)d\tau,\,\,\,t\in[0,T].
\end{gather}

Из~\eqref{w.derivatives1} вытекает, что
\begin{gather}\label{w.initial.conds}
\mathfrak{w}(0)=0,\,\,\,\dot{\mathfrak{w}}(0)=0.
\end{gather}

Далее, в силу~\eqref{w.derivatives1}, \eqref{abstrChauchyprobeq:nonaut:parametric:homogenious.eq.e!t.less.than.tau}, \eqref{abstrChauchyprobeq:nonaut:parametric:homogenious.eq.e!t.greater.than.tau},
\begin{gather*}
\ddot{\mathfrak{w}}(t)+\mathfrak{A}(t){\mathfrak{w}}(t)+\mathfrak{B}(t){\mathfrak{w}}(t) + \mathfrak{G}^*\mathfrak{F}(t)\mathfrak{G}{\mathfrak{w}}(t)=\\
=f(t)+\int\limits_0^t[\mathfrak{e}_{tt}[f](t,\tau)+\mathfrak{A}(t)\mathfrak{e}[f](t,\tau)+\mathfrak{B}(t)\mathfrak{e}[f](t,\tau) + \mathfrak{G}^*\mathfrak{F}(t)\mathfrak{G}\mathfrak{e}[f](t,\tau)]d\tau=f(t).
\end{gather*}

Иными словами, 
\begin{gather}\label{w.equation}
\ddot{\mathfrak{w}}(t)+\mathfrak{A}(t){\mathfrak{w}}(t)+\mathfrak{B}(t){\mathfrak{w}}(t) + \mathfrak{G}^*\mathfrak{F}(t)\mathfrak{G}{\mathfrak{w}}(t)=f(t),\,\,\,t\in[0,T].
\end{gather}
Из соотношений~\eqref{w.initial.conds} и \eqref{w.equation} и следует равенство~\eqref{Lambda.3.repres}.

Из теоремы~\ref{abstrChauchyprobeq:nonaut:parametric:homogenious.eq!theorem.y.representation} и равенств~\eqref{Lambda.2.repres}, \eqref{Lambda.3.repres} вытекает, что найдётся функция $\Psi\in\mathfrak{S}(\Gamma; V,H)$,
такая, что
\begin{gather*}
\Lambda_2[\psi](t)=\Psi(t,0)\psi,\,\,\,\Lambda_3[f](t)=\int\limits_0^t\Psi(t,\tau)f(\tau)d\tau,\,\,\,t\in[0,T].
\end{gather*}
Подставляя эти формулы в соотношение~\eqref{z.repres!Phi.Theta}, получим формулу~\eqref{Lambda.repres}. Теорема полностью доказана.
\end{Proof}    

	    \subsection{Нелинейное уравнение}
Пусть $\varphi\in V$, $\psi\in H$, $f\in L_1([0,T],H)$, $\mathfrak{g}\in W^1_1([0,T], Y^*)$ --- фиксированы.

Рассмотрим задачу Коши
\begin{gather}\label{abstrChauchyprobeq:nonaut:nonlinear}
\ddot{\mathfrak{z}}(t)+\mathfrak{A}(t){\mathfrak{z}}(t)+\mathfrak{B}(t){\mathfrak{z}}(t) + \mathfrak{G}^*\mathfrak{F}(t)\mathfrak{G}{\mathfrak{z}}(t) = \mathfrak{b}(t,\mathfrak{z}(t),\dot{\mathfrak{z}}(t)) + \mathfrak{C}^*\mathfrak{g}(t),\,\,\,t\in[0,T],\\
\label{abstrChauchyprobinitcond:nonaut:nonlinear}
{\mathfrak{z}}(0)=\varphi,\,\,\,\dot{\mathfrak{z}}(0)=\psi,
\end{gather}
где функция $\mathfrak{b}:[0,T]\times V\times H\to H$ такова, что
\begin{enumerate}
\item функция $[0,T]\ni t\mapsto\mathfrak{b}(t,v,h)$ --- сильно измерима при всех $v\in V$, $h\in H$;
\item найдётся функция $\mathbf{K}_0\in L_1[0,T]$, такая, что
\begin{gather*}
\|\mathfrak{b}(t,v_1,h_1)-\mathfrak{b}(t,v_2,h_2)\|_H\leqslant \mathbf{K}_0(t)\sqrt{\|v_1-v_2\|^2_V+\|h_1-h_2\|^2_H}\,\,\,\forall\,(t,v_i,h_i)\in[0,T]\times V\times H,\,\,\,i=1,2;
\end{gather*}
\item найдётся функция $\mathbf{K}_1\in L_1([0,T],H)$, такая, что
\begin{gather*}
\|\mathfrak{b}(t,0,0)\|_H\leqslant\|\mathbf{K}_1(t)\|_H\,\,\,\forall\,t\in[0,T].
\end{gather*}
\end{enumerate}
	
Дадим следующее
\begin{Definition}\label{SolutionDef1:nonaut:nonlinear} Функцию ${\mathfrak{z}}\in{\textrm{Э}}([0,T];V,H)$ назовём решением задачи Коши (\ref{abstrChauchyprobeq:nonaut:nonlinear}),
(\ref{abstrChauchyprobinitcond:nonaut:nonlinear}), если
\begin{gather}\label{defsolabstrChauchyprob1:nonaut:nonlinear}
\int\limits_0^T[-\langle\dot{\mathfrak{z}}(t),\dot{\eta}(t)\rangle_H+\langle \mathfrak{A}(t){\mathfrak{z}}(t),\eta(t)\rangle+\langle \mathfrak{B}(t){\mathfrak{z}}(t),\eta(t)\rangle +\langle \mathfrak{F}(t)\mathfrak{G}{\mathfrak{z}}(t),\mathfrak{G}\eta(t)\rangle_Z]dt=\\
=\notag\langle\psi,\eta(0)\rangle+\int\limits_0^T\langle \mathfrak{b}(t,\mathfrak{z}(t),\dot{\mathfrak{z}}(t)),\eta(t)\rangle_Hdt +\int\limits_0^T\langle{\mathfrak{g}}(t), \mathfrak{C}\eta(t)\rangle\,dt\,\,\,\forall\,{\eta}\in\hat{\textrm{Э}}{}([0,T];V,H);\\
\notag \mathfrak{z}(0)=\varphi.
\end{gather}
\end{Definition}

Дадим ещё одно определение решения задачи Коши  (\ref{abstrChauchyprobeq:nonaut:nonlinear}), (\ref{abstrChauchyprobinitcond:nonaut:nonlinear}).
\begin{Definition}\label{SolutionDef2:nonaut:nonlinear} Функцию ${\mathfrak{z}}\in{\textrm{Э}}_2([0,T];V,H)$ назовём решением задачи Коши (\ref{abstrChauchyprobeq:nonaut:nonlinear}),
(\ref{abstrChauchyprobinitcond:nonaut:nonlinear}), если
\begin{gather}\label{defsolabstrChauchyprob2:nonaut:nonlinear}
\langle\ddot{\mathfrak{z}}(t),v\rangle+\langle\mathfrak{A}(t){\mathfrak{z}}(t)+\mathfrak{B}(t){\mathfrak{z}}(t) +  \mathfrak{G}^*\mathfrak{F}(t)\mathfrak{G}{\mathfrak{z}}(t), v\rangle=\\
\notag=\langle \mathfrak{b}(t,\mathfrak{z}(t),\dot{\mathfrak{z}}(t)), v\rangle_H+\langle{\mathfrak{g}}(t), \mathfrak{C}\eta(t)\rangle\,\,\,\mbox{ при п.в. }t\in[0,T]\,\,\,\text{$\forall v\in V$,}\\
\notag {\mathfrak{z}}(0)=\varphi,\,\,\,\dot{\mathfrak{z}}(0)=\psi.
\end{gather}
\end{Definition}	

Эквивалентность этих двух определений доказывается аналогично тому, как доказывалась эквивалентность определений  \ref{SolutionDef1:nonaut} и \ref{SolutionDef2:nonaut}.    

\begin{Theorem}\label{abstrChauchyprobeq:nonaut:nonlinear!theorem}
Задача Коши  (\ref{abstrChauchyprobeq:nonaut:nonlinear}), (\ref{abstrChauchyprobinitcond:nonaut:nonlinear}) имеет единственное решение $\mathfrak{z}$ в классе ${\textrm{Э}}([0,T];V,H)$, это решение является элементом пространства $\textrm{Є}([0,T];V,H)$, и найдётся постоянная $\varkappa_4>0$, зависящая лишь от чисел $T$, $c_1$, $c_2$, $c_3>0$, от $\|\mathfrak{C}\|_{V\to Y}$,  $\|\mathfrak{G}\|_{V\to Z}$, и функции 
$\mathbf{K}_0\in L_1[0,T]$, такая, что
\begin{gather}\label{abstrChauchyprobeq:nonaut:a_priori_solution_estimate1:nonlinear}
\|\mathfrak{z}\|_{{\textrm{Є}}([0,T];V,H)}\leqslant \varkappa_2[\sqrt{\|\varphi\|^2_V+\|\psi\|_H^2}+ \|\mathfrak{b}(t,0,0)\|_{1,[0,T],H}+\|\mathfrak{g}\|^{(1)}_{1,[0,T],Y^*}].
\end{gather}
\end{Theorem}
\begin{Proof}
Положив $F(t)\equiv\mathfrak{b}(t,\mathfrak{z}(t),\dot{\mathfrak{z}}(t))$, $t\in[0,T]$, получим, что решение $\mathfrak{z}$ задачи Коши~(\ref{abstrChauchyprobeq:nonaut:nonlinear}), (\ref{abstrChauchyprobinitcond:nonaut:nonlinear}) является решением задачи Коши
\begin{gather*}
\ddot{\mathfrak{z}}(t)+\mathfrak{A}(t){\mathfrak{z}}(t)+\mathfrak{B}(t){\mathfrak{z}}(t) + \mathfrak{G}^*\mathfrak{F}(t)\mathfrak{G}{\mathfrak{z}}(t) = F(t) + \mathfrak{C}^*\mathfrak{g}(t),\,\,\,t\in[0,T],\\
{\mathfrak{z}}(0)=\varphi,\,\,\,\dot{\mathfrak{z}}(0)=\psi,
\end{gather*}
причём, в силу условий на функцию $\mathfrak{b}$, справедливо включение $F\in L_1([0,T],H)$.

Поэтому, на основании теоремы~\ref{abstrChauchyprob:nonaut:solution.representation}, найдутся оператор $\Theta\in\mathcal{L}(W^1_1([0,T],Y^*),\textrm{Є}([0,T];V,H))$ и функции $\Phi\in\mathfrak{R}([0,T]; V,H)$, $\Psi\in\mathfrak{S}(\Gamma; V,H)$, такие, что при всех $v\in V$, $h\in H$
\begin{gather}\label{nonlinear.equation.solution.repres1}
\mathfrak{z}(t)=\Phi(t)\varphi+\Psi(t,0)\psi+\int\limits_0^t\Psi(t,\xi)F(\xi)d\xi+\Theta[\mathfrak{g}](t)\,\,\,\forall\,t\in[0,T].
\end{gather}
Следовательно, ввиду определения функции $F$,
\begin{gather}\label{nonlinear.equation.solution.repres2}
\mathfrak{z}(t)=\Phi(t)\varphi+\Psi(t,0)\psi+\Theta[\mathfrak{g}](t)+\int\limits_0^t\Psi(t,\xi)\mathfrak{b}(\xi,\mathfrak{z}(\xi),\dot{\mathfrak{z}}(\xi))d\xi\,\,\,\forall\,t\in[0,T].
\end{gather}
Положив здесь $\omega(t)\equiv\Phi(t)\varphi+\Psi(t,0)\psi+\Theta[\mathfrak{g}](t)$, $t\in[0,T]$, получим, что
\begin{gather}\label{nonlinear.equation.solution.repres3}
\mathfrak{z}(t)=\omega(t)+\int\limits_0^t\Psi(t,\xi)\mathfrak{b}(\xi,\mathfrak{z}(\xi),\dot{\mathfrak{z}}(\xi))d\xi\,\,\,\forall\,t\in[0,T].
\end{gather}

Иными словами, отыскание решения задачи Коши~(\ref{abstrChauchyprobeq:nonaut:nonlinear}), (\ref{abstrChauchyprobinitcond:nonaut:nonlinear}) в классе ${\textrm{Э}}([0,T];V,H)$ эквивалентно отысканию решения интегро--дифференциального уравнения~\eqref{nonlinear.equation.solution.repres3} в классе $\textrm{Є}([0,T];V,H)$. Применяя затем теорему~\ref{--integodifferential.equation!theorem}, заключаем, что уравнение~\ref{nonlinear.equation.solution.repres3} имеет единственное решение в классе $\textrm{Є}([0,T];V,H)$, причём найдётся постоянная $\varkappa^*>0$, такая, что
\begin{gather*}
\|\mathfrak{z}\|_{\textrm{Є}([0,T];V,H)}\leqslant\varkappa^*[\|\omega\|_{\textrm{Є}([0,T];V,H)}+\|\mathfrak{b}(\cdot,0,0)\|_{1,[0,T],H}].
\end{gather*}
Ввиду определения функции $\omega$, найдутся постоянные $\varkappa^*_1$, $\varkappa^*_2$, $\varkappa^*_3>0$, такие, что
\begin{gather*}
\|\omega\|_{\textrm{Є}([0,T];V,H)}\leqslant\varkappa^*_1\|\varphi\|_V+\varkappa^*_2\|\psi\|_H+\varkappa^*_3\|\mathfrak{g}\|^{(1)}_{1,[0,T],Y^*}\leqslant
\sqrt{(\varkappa^*_1)^2+(\varkappa^*_2)^2}\sqrt{\|\varphi\|_V^2+\|\psi\|_H^2}+\varkappa^*_3\|\mathfrak{g}\|^{(1)}_{1,[0,T],Y^*}\leqslant\\
\leqslant\varkappa_4^*[\sqrt{\|\varphi\|_V^2+\|\psi\|_H^2}+\|\mathfrak{g}\|^{(1)}_{1,[0,T],Y^*}],
\end{gather*}
где $\varkappa_4^*\equiv\max\{\sqrt{(\varkappa^*_1)^2+(\varkappa^*_2)^2},\varkappa^*_3\}$. Таким образом,
\begin{gather*}
\|\mathfrak{z}\|_{\textrm{Є}([0,T];V,H)}\leqslant\varkappa^*[\varkappa_4^*[\sqrt{\|\varphi\|_V^2+\|\psi\|_H^2}+\|\mathfrak{g}\|^{(1)}_{1,[0,T],Y^*}]+\|\mathfrak{b}(\cdot,0,0)\|_{1,[0,T],H}]\leqslant\\
\leqslant\varkappa_5^*[\sqrt{\|\varphi\|_V^2+\|\psi\|_H^2}+\|\mathfrak{g}\|^{(1)}_{1,[0,T],Y^*}+\|\mathfrak{b}(\cdot,0,0)\|_{1,[0,T],H}],
\end{gather*}
где $\varkappa_5^*\equiv\max\{\varkappa_4^*,1\}$.

Итак, интегро--дифференциальное уравнение~\ref{nonlinear.equation.solution.repres3} имеет единственное решение $\mathfrak{z}\in\textrm{Є}([0,T];V,H)$, причём найдётся постоянная $\varkappa^*_5>0$, такая, что
\begin{gather*}
\|\mathfrak{z}\|_{\textrm{Є}([0,T];V,H)}\leqslant\varkappa_5^*[\sqrt{\|\varphi\|_V^2+\|\psi\|_H^2}+\|\mathfrak{g}\|^{(1)}_{1,[0,T],Y^*}+\|\mathfrak{b}(\cdot,0,0)\|_{1,[0,T],H}],
\end{gather*}

Ввиду упомянутой выше эквивалентности интегро--дифференциального уравнения~\ref{nonlinear.equation.solution.repres3} и задачи Коши~(\ref{abstrChauchyprobeq:nonaut:nonlinear}), (\ref{abstrChauchyprobinitcond:nonaut:nonlinear}), это означает, что задача Коши~(\ref{abstrChauchyprobeq:nonaut:nonlinear}), (\ref{abstrChauchyprobinitcond:nonaut:nonlinear}) имеет единственное решение в смысле определения~\ref{defsolabstrChauchyprob1:nonaut:nonlinear}, и справедлива оценка~\eqref{abstrChauchyprobeq:nonaut:a_priori_solution_estimate1:nonlinear} с $\varkappa_3\equiv\varkappa_5^*$. Теорема полностью доказана.
\end{Proof}

	    \subsection{Линейное уравнение с мерой Радона в правой части}
	    
	    \subsection{Параметрическая задача Коши с ненулевой правой частью}




\part{Гиперболические уравнения дивергентного вида}

    \chapter{Уравнения с главной частью второго порядка ($n=1$)}

    \chapter{Уравнения с главной частью второго порядка ($n>1$)}

    \chapter{Уравнения с главной частью четвёртого порядка $(n=1)$}

    \chapter{Уравнения с главной частью четвёртого порядка $(n>1)$}


\begin{thebibliography}{99}\addcontentsline{toc}{part}{Литература}

\bibitem{BalesLasiecka} Bales L., Lasiecka I. Negative norm estimates for fully discrete finite element approximations to the wave equation with nonhomogeneous $L_2$ Dirichlet boundary
data // Mathematics of computation. 1995. V.64. No.209. P.89--115.

\bibitem{ekeland}{ Ekeland I.} On the variational principle //J. Math. Anal. Appl. 1974. V.47. \No.2. P.324-353.

\bibitem{KaraStavr} Karachalios N., Stavrakakis N. Asymptotic behavior of solutions of some nonlinearly damped equations on $\mathbb{R}^N$ // Topological Methods in Nonlinear Analysis.
2001. V.18. P.73--87.

\bibitem{LasiekaLionsTriggiani}Lasiecka I., Lions J.-L., Triggiani R. Nonhomogeneous boundary value problems for second-order hyperbolic operators // J. Mat. Pures Appl. 1986. V.65. No.2.
P.149–-192.

\bibitem{LasiekaSokolowski}Lasiecka I., Sokolowski J. Regularity and strong convergence of a variational approximation to a nonhomogeneous Dirichlet hyperbolic boundary problem //
SIAM J. Math. Anal. 1988. V.19. P.528–540.

\bibitem{LasiekaTriggiani1}Lasiecka I., Triggiani R. Sharp regularity theory for second order hyperbolic equations of Neumann type, I: $L_2$ nonhomogeneous data // Ann. Mat. Pura Appl.
1990. V.157. P.285–-367.

\bibitem{LasiekaTriggiani2}Lasiecka I., Triggiani R. Regularity theory of hyperbolic equations with non-homogeneous Neumann boundary conditions, II: General boundary data // J. Diff. Eq.
1991. V.94. P.112–-164.

\bibitem {mord1} { Mordukhovich B.S.} Variational Analysis and Generalized Differentiation, I: Basic Theory. Springer: Berlin, 2006.

\bibitem {mordukh} {Mordukhovich B.S., Shao Y.} Nonsmooth sequential analysis in asplund spaces // Trans. Amer. Math. Soc. 1996. V.346. No.4. P.1235-1280.

\bibitem {ward} {Ward A.L. }Differentiability of vector monotone functions // Proc. London Math. Soc. 1935. V.32. No.2. P.339-362.

\bibitem{ATF} Алексеев В.М., Тихомиров В.М., Фомин С.В. Оптимальное управление. --- М.: Наука, 1979.

\bibitem{BogachyovVI} Богачёв В.И. Основы теории меры. Том I. --- Москва--Ижевск:\ НИЦ \glqq Регулярная и хаотическая динамика\grqq, 2003.

\bibitem{ZhelBuk2} Букесова Н.Н., Железовский С.Е. О скорости сходимости метода Галеркина для одного класса квазилинейных операторных дифференциальных уравнений // Журн. вычисл. матем. и
матем. физ. 1999. Т.39. \No.9. С.1519--1531.

\bibitem{warga} Варга Дж. Оптимальное управление дифференциальными и функциональными  уравнениями. --- М.: Наука, 1977.

\bibitem{Vorovich} Ворович И.И. О некоторых прямых методах в нелинейной теории колебаний пологих оболочек // Изв. АН СССР. Сер. Математическая. 1957. Т.21. С.747--784.

\bibitem{zhvmmf}{ Гаврилов В.С., Сумин М.И.} Параметрическая оптимизация нелинейных систем Гурса--Дарбу с фазовыми ограничениями // Журн. вычисл. матем. и матем. физ. 2004. Т.44. \No\,6.
С.1002--1022.

\bibitem{izvvuz}{Гаврилов В.С., Сумин М.И.} Параметрическая задача субоптимального управления системой Гурса-Дарбу с поточечным фазовым ограничением // Известия вузов. Математика. 2005.
\No\,6. С.40--52.

\bibitem{Gaevskij:Greger:Zaharias} Гаевский Х., Грегёр К., Захариас К. Нелинейные операторные уравнения и операторные дифференциальные уравнения. --- М.: Наука, 1978.

\bibitem{Del'Santo.Mitidieri} Дель Санто Д., Митидиери Э. Разрушение решений гиперболической системы: критический случай // Дифф. уравнения. 1998. Т.34. \No9. С.1155--1161.

\bibitem{Zhel1995} Железовский С.Е. Метод Бубнова--Галеркина для абстрактной квазилинейной задачи о стационарном действии // Дифф. уравнения. 1995. Т.31. \No7. С.1222--1231.

\bibitem{Zhel2001} Железовский С.Е. Оценки скорости сходимости метода Галёркина для абстрактного гиперболического уравнения // Матем. заметки. 2001. Т.69. Вып.2. С.223--234.

\bibitem{Zhel2002} Железовский С.Е. Оценки скорости сходимости проекционно--разностного метода для гиперболических уравнений // Изв. вузов. Математика. 2002. \No.1. С.21--30.

\bibitem{Zhel2005b} Железовский С.Е. К оценкам погрешности метода Галёркина для гиперболических уравнений // Сибирский матем. журн. 2005. Т.46. \No.2. С.374--389.

\bibitem{Zhel2007a} Железовский С.Е. К обоснованию метода Галеркина для гиперболических уравнений // Дифф. уравнения. 2007. Т.43. \No.3. С.402--410.

\bibitem{Zhel2007b} Железовский С.Е. К исследованию сходимости проекционно--разностного метода для гиперболических уравнений // Сибирский матем. журн. 2007. Т.48. \No.1. С.93--102.

\bibitem{ZhelBuk} Железовский С.Е., Букесова Н.Н. Оценки погрешности проекционного метода для абстрактного квазилинейного гиперболического уравнения // Изв. вузов. Математика.
1999. \No.5. С.94--96.

\bibitem{ZhelLyashko} Железовский С.Е., Ляшко А.Д. Оценки погрешности метода Галеркина для квазилинейных гиперболических уравнений // Дифф. уравнения. 2001. Т.37. \No7. С.941--949.

\bibitem{ZorichTomI}Зорич В.А. Математический анализ. Часть I. --- изд.2--е. --- М.: ФАЗИС, 1997.

\bibitem{Il'inKuleshov2012a}Ильин В.А., Кулешов А.А. О некоторых свойствах обобщённых решений волнового уравнения из классов $L_p$ и $W^1_p$ при $p\geq1$
// Дифф. уравнения. 2012. Т.48. \No11. С.1493--1500.

\bibitem{Il'inKuleshov2012b}Ильин В.А., Кулешов А.А. Необходимое и достаточное условие принадлежности классу $L_p$ при $p\geq1$ обобщенного решения смешанной задачи для волнового уравнения
// Дифф. уравнения. 2012. Т.48. \No12. С.1607--1611.

\bibitem{Il'inPoznyak} Ильин В.А., Позняк Э.Г.  Основы математического анализа. В 2 ч. Часть II. --- М.: Наука, Физматлит, 2000.

\bibitem{Ishmuhametov} Ишмухаметов А.З. Об аппроксимации гиперболических дифференциально--операторных уравнений второго порядка // Журн. вычисл. матем. и матем. физ. 1987. Т.27. \No.8.
С.1154--1165.

\bibitem{iosida} Иосида К. Функциональный анализ. --- М.: Мир, 1967.

\bibitem{clarke} Кларк Ф. { Оптимизация и негладкий анализ.} М.: Наука, 1988.

\bibitem{KozhanovLar'kin} Кожанов А.И., Ларькин Н.А. О разрешимости краевых задач для волнового уравнения с нелинейной диссипацией в неоднородных областях // Сибирский матем. журн.
2001. Т.42. \No6. С.1275--1299.

\bibitem{KolesovRozov} Колесов А.Ю., Розов Н.Х. Многочастотный параметрический резонанс в нелинейном волновом уравнении // Изв. РАН. Сер. математическая. 2002. Т.66. \No.6. С.49--64.

\bibitem{KF} Колмогоров А.Ф., Фомин С.В. Элементы теории функций и функционального анализа. --- изд. 6-е. --- М.: Наука, 1988.

\bibitem{KudryavtsevT1}Кудрявцев Л.Д. Курс математического анализа. Т1. --- М.: Наука, 2003.

\bibitem{lad1953} Ладыженская О.А. Смешанная задача для гиперболических уравнений. --- М.: Гостехиздат, 1953.

\bibitem{Lad1954} Ладыженская О.А. О разрешимости основных краевых задач для уравнений параболического и гиперболического типов // ДАН СССР. 1954. Т.97. \No.3. С.395--398.

\bibitem{Lad1956} Ладыженская О.А. О решении нестационарных операторных уравнений // Матем. сб. 1956. Т.39. \No.4. С.491--524.

\bibitem{Lad1958} Ладыженская О.А. О нестационарных операторных уравнениях и их приложениях к линейным задачам математической физики // Матем. сб. 1958. Т.45. \No.2. С.123--158.

\bibitem{lad} Ладыженская О.А. Краевые задачи математической физики. --- М.: Наука, 1973.

\bibitem{ladsolur}  Ладыженская О.А., Солонников В.А., Уральцева Н.Н. Линейные и квазилинейные уравнения параболического типа. --- М.: Наука, 1967.

\bibitem{Lions1972} Лионс Ж.--Л. Некоторые методы решения нелинейных краевых задач. --- М.: Мир, 1972.

\bibitem{LionsMajenes} Лионс Ж.--Л., Мадженес Э. Неоднородные граничные задачи и их приложения. --- М.: Мир, 1971.

\bibitem{Lomovcev} Ломовцев Ф.Е. Гиперболические дифференциальные уравнения второго порядка с разрывными операторными коэффициентами // Дифф. уравнения. 1997. Т.33. \No10. С.1394--1403.

\bibitem{Nikitin} Никитин А.А. О смешанной задаче для волнового уравнения с третьим и первым краевым условиями // Дифф. уравнения. 2007. Т.43. \No12. С.1692--1699.

\bibitem{Aubin} Обэн Ж.--П. Нелинейный анализ и его экономические приложения. --- М.: Мир, 1988.

\bibitem{osipov.vasilev.potapov} Осипов Ю.С., Васильев Ф.П., Потапов М.М. Основы метода динамической регуляризации. --- М.: Изд--во МГУ, 1999. --- 237с.

\bibitem{Pohozhaev.Mitidieri} Митидиери Э., Похожаев С.И. Априорные оценки и отсутствие решений нелинейных уравнений и неравенств в частных производных. // Тр. Математического института
им. В.А.~Стеклова. 2001. Т.234.

\bibitem{Ramm} Рамм А.Г. О поведении решения краевой задачи для гиперболического уравнения при $t\to\infty$ // Изв. вузов. Математика. 1966. \No.1. С.124--138.

\bibitem{smirnov} {Смирнов В.И.} Курс высшей математики. Том V. М.: ГИФМЛ, 1959.

\bibitem{Stane}{Стейн М. И.} Сингулярные интегралы и дифференциальные свойства функций. М.: Мир, 1973.

\bibitem{sumin83} {Сумин М.И.} Дисс. ... канд. физ.-мат. наук, Горький: Горьковский гос. ун-т, 1983.

\bibitem{Sumin1991b}{Сумин М.И.} О первой вариации в теории оптимального управления системами с распределенными параметрами // Дифференц. уравнения. 1991.  Т.27. \No\,12.  С.2179--2181.

\bibitem {sumin97a}{ Сумин М.И.} Субоптимальное управление системами с распределенными параметрами: минимизирующие последовательности,
функция значений // Журн. вычисл. матем. и матем. физ. 1997. Т.37. \No\,1. С.23-41.

\bibitem{sumin00b}Сумин М.И. Субоптимальное управление полулинейными эллиптическими уравнениями с фазовыми ограничениями, I: принцип
максимума для минимизирующих последовательностей, нормальность. // Изв. вузов. Математика. 2000. \No\,6. C.33--44.

\bibitem{sumin00c}Сумин М.И. Субоптимальное управление полулинейными эллиптическими уравнениями с фазовыми ограничениями, II:
чувствительность, типичность регулярного принципа максимума. // Изв.вузов. Математика. 2000. \No\,8. C.52--63.

\bibitem{sumin2000}{Сумин М.И.}  Дисс. ... докт. физ.-мат. наук. Н. Новгород: Нижегородский гос. ун-т, 2000.

\bibitem{SuminMmetodichka} Сумин М.И. Элементы математической теории оптимального управления. Часть I. Принцип максимума Л.С.Понтрягина
в задаче с нефиксированным временем и функциональными ограничениями. Методическая разработка. Нижний Новгород: Изд-во ННГУ. 2001. 48с.

\bibitem{variation_2008} {Сумин М.И.} Первая вариация и принцип максимума Понтрягина в оптимальном управлении для уравнений в
частных производных // Журн. вычисл. матем. и матем. физ. 2009. Т.49. \No.\,6. С.998--1020.

\bibitem{Trenogin}Треногин В.А. Функциональный анализ. --- М.: Наука, 1980.

\bibitem{Edwards} Эдвардс Р. Функциональный анализ. --- М.: Мир, 1969.

\bibitem{Yakubov} Якубов С.Я. Равномерно корректная задача Коши для абстрактных гиперболических уравнений // Изв. вузов. Математика. 1970. \No.12. С.108--113.


\bibitem{GavrilovSumin2008}Гаврилов В.С., Сумин М.И. Принцип максимума Понтрягина в параметрической задаче субоптимального управления для дивергентного гиперболического уравнения с фазовым
ограничением // В кн. <<Международная конференция \glqq Дифференциальные уравнения и топология\grqq, посвященная 100--летию Л.С. Понтрягина. Тезисы докладов. Москва,
17-–22 июня 2008 г.>>. М.: Издательский отдел факультета ВМиК МГУ им. М.В. Ломоносова; МАКС Пресс, 2008. С.329–-330.

\bibitem{GavrilovSumin2013-1}Гаврилов В.С., Сумин М.И. Параметрическая оптимизация для гиперболического уравнения дивергентного вида с поточечным фазовым ограничением. I //
Дифференциальные уравнения. 2011. Т. 47. \No4. С.550–562.

\bibitem{GavrilovSumin2013-2}Гаврилов В.С., Сумин М.И. Параметрическая оптимизация для гиперболического уравнения дивергентного вида с поточечным фазовым ограничением. II //
Дифференциальные уравнения. 2011. Т.47. \No5. С.724–-735.

\bibitem{MR1}{ Mordukhovich B.S., Raymond J.--P.} Dirichlet boundary control of hyperbolic equations in the presence of state constraints // Appl.  Math.  Optim.  2004.  V.49.  P.145-157.

\bibitem{MR2}{ Mordukhovich B.S., Raymond J.--P.} Neumann boundary control of hyperbolic equations with pointwise state constraints // SIAM J. Control Optim. V.43. No.4. 2005.  P. 135-137.

%   \bibitem{Bourbaki} Бурбаки Н. Интегрирование. Векторное интегрирование. Мера Хаара. Свёртка и %представления. --- М.: Наука, 1970.
\end{thebibliography}
\end{document}
